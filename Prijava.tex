\documentclass[11pt]{article}

\newcommand{\lat}{}
\usepackage[utf8]{inputenc}
\usepackage[T2A]{fontenc}
\usepackage[serbian]{babel}
\usepackage{amsmath}
\usepackage{amsthm}
\usepackage{graphicx}
\usepackage{float}
\usepackage{pgf}
\usepackage{wrapfig}
\usepackage{gclc}

\usepackage{fullpage}
\usepackage[unicode,hidelinks, breaklinks=true]{hyperref}
\usepackage{enumitem}

\usepackage{amssymb}
\usepackage{stmaryrd}
\usepackage{tikz}

\begin{document}
       УНИВЕРЗИТЕТ У БЕОГРАДУ  
       
       МАТЕМАТИЧКИ ФАКУЛТЕТ
 

\vspace{1cm}
\begin{center}
 {\largeКатедри за рачунарство и информатику } \\[5mm]
 
 {\large\bf Пријава теме за израду докторске дисертације}
\end{center}

\vspace{1cm}
Молим да ми одобрите израду докторске дисертације под насловом
\textit{,,Формализација ра\-зли\-чи\-тих модела геометрије и примене у верификацији аутоматских доказивача 
теорема''}. Тему
предлажем у договору са ментором др Филипом Марићем.
  
  \vspace{1cm}
  
  \hfill{Подносилац пријаве} 
  
  \vspace{0.5cm}
  
  \hfill{\underline{\hspace{7cm}}} 
  
  \hfill{Данијела Симић} 
  
  \vspace{1cm}
  
  \hfill{Ментор:} 
  
  \hfill{др Филип Марић}

Београд, новембар 2015.
\thispagestyle{empty}

  
  \newpage
\setcounter{page}{1}  
\begin{center}
{\bf Наставно-научном већу Математичког факултета\\
Универзитета у Београду}

\vspace{1cm}

{\bf Пријава теме за израду докторске тезе \\
     кандидата Данијеле Симић (девојачко Петровић):}

\vspace{0.5cm}     

{\bf Формализација раличитих модела геометрије и примене у формализацији аутоматских доказивача теорема}
\end{center}

\section{Подаци о кандидату}
\subsection{Биографија}
   Данијела Симић је рођена 26.09.1986. године у Ваљеву. Основну школу 
   Жикица Јовановић Шпанац завршила је као ђак генерације и вуковац. 
   Потом је уписала Ваљевску гимназију, коју је такође завршила као ђак 
   генерације и вуковац. Током школовања учествовала је на бројним такмичењима
   при чему се посебно истичу резултати и награде са републичких и савезних 
   такмичења из физике и програмирања. Године 2005. уписала је Математички
   факултет, уневерзитета у Београду, смер Рачунарство и Информатика. 
   Студије је завршила 2009. године са просечном оценом 9.86. Исте године
   уписала је докторске студије на смеру Информатика. Положила је све испите 
   на докторским студијама са просечном оценом 10.
   
   Запослена је на Математичком факултету, Универзитета у Београду од 2009.
   године и држала је вежбе из следећих предмета:
   \begin{itemize}
   \item Програмирање 1
   \item Програмирање 2
   \item Увод у организацију рачунара
   \item Вештачка интелигенција
   \end{itemize}
   
\subsection{Списак научних радова}

\paragraph{Објављени радови}
\begin{itemize}
\item {\lat Filip Marić, Ivan Petrović, Danijela Petrović, Predrag
    Janičić:
    \href{http://eptcs.web.cse.unsw.edu.au/paper.cgi?THedu11.4}{{\em Forma\-lization
      and Implementation of Algebraic Methods in Geometry.}}
    \href{http://about.eptcs.org/}{Electronic Proceedings in
      Theoretical Computer Science} 79, pp. 63–81.}
\item {\lat Filip Marić, Danijela Petrović:
    \href{http://argo.matf.bg.ac.rs/publications/2014/moebius.pdf}{{\em Formalizing
      Complex Plane Geometry.}}
    \href{http://link.springer.com/article/10.1007/s10472-014-9436-4}{Annals
      of Mathematics and Artificial Intelligence}, November 2014}
\item {\lat Danijela Simić:
    \href{http://argo.matf.bg.ac.rs/publications/2014/small-step-verification.pdf}{{\em Using
      Small-Step Refinement for Algorithm Verification in Computer
      Science Education.}} Accepted for publication in
    \href{https://www.fose1.plymouth.ac.uk/mathematics\_education/field\%20of\%20work/IJTME/}{The
      International Journal for Technology in Mathematics Education},
    Volume 22, Number 4 (December 2015)}
\end{itemize}

\paragraph{Саопштења на научним скуповима}
\begin{itemize}
\item {\lat Danijela Petrović, Filip Marić:
    \href{http://poincare.matf.bg.ac.rs/~danijela/publications/ADG2012.pdf}{{\em Formalizing
      Analytic Geometries}}.
    \href{http://dream.inf.ed.ac.uk/events/adg2012/}{Automated
      Deduction in Geometry 2012}}, Edinburgh, UK, 2012.
\item {\lat Danijela Petrović:
    \href{http://poincare.matf.bg.ac.rs/~danijela/publications/text.pdf}{{\em Using
      Small-Step Refinement for Algorithm Verification in Computer
      Science Education.}}
    \href{http://www.uc.pt/en/congressos/thedu/thedu14}{ThEdu'14}}
\end{itemize}

\subsection{Конференције и летње школе}
\begin{itemize}
\item {\lat Automated Deduction in Geometry, Automated Deduction in
    Geometry} 2012, Единбург, Велика Британија, 17.09. -- 19.09.2012.
\item {\lat European Summer School in Logic, Language and
    Information}, Љубљана, Словенија, 01.08. -- 12.08.2011
\item Више пута презентовала и учествовала у организацији радионица
  ARGO групе
\end{itemize}

\section{Преглед области тезе и постојећих резултата}

\subsection{Формално доказивање теорема геометрије}

У класичној математици постоји много различитих геометријских теорија. Такође,
разли\-чи\-та су и гледишта шта се сматра стандардном (Еуклидском)
геометријом. Пoнекад, геоме\-три\-ја се дефинише као независна формална
теорија, а понекад као специфични модел. Наравно, везе између
различтих заснивања геометрије су јаке. На пример, може се показати да
Дека\-рто\-ва раван представља модел формалних теорија геометрије.

Традиционална Еукидска (синетичка) геометрија је још од античке Грчке
заснована на често малом скупу основних појмова (на пример, тачке,
праве, основних геометријских релација итд.) и на скупу аксиома које
имплицитно дефинишу ове основне појмове. Иако су Еуклидови
,,Елементи'' један од најутицајних радова из математике, поставило се
озбиљно питање да ли систем аксиома, теорема и лема којима се
геометрија описује заиста прецизан. Испоставило се да су нађене грешке
у доказима, а и да су неки докази били непотпуни јер су имали
имплицитне претпоставке настале због погрешне интуиције или погрешног позивања на
слике. Ове празнине су утицале на појаву других аксиоматских система
чији је циљ био да дају формалну аксиоматизацију Еуклидове
геометрије. Најважнији су Хилбертов систем аксиома, систем аксиома
Тарског и најмодернија варијанта -- Авигадов систем аксиома.

Хилбертов систем се састоји из три основна појма (тачка, права и
раван), 6 предиката и 20 аксиома подељених по групама. Хилберт није
желео ништа да остави интуицији. Овакав приступ је повећао ниво
ригорозности не само у геометрији, него у целој математици.

Систем Тарског је мањи, састоји се од једног основног појма (тачка), 2
предиката и 11 аксиома и његова основна предност у односу на Хилбертов
систем је у његовој једноставности. Са друге стране, систем Тарског
уводи појам праве као скупове тачака што доста отежава резоновање јер
захтева да се у доказима користи теорија скупова.

Једнo од најзначајнијих открића у математици, које датира из {\lat
  XVII} века, јесте Декартово откриће координатног система и оно је
омогућило да се алгебарским изразима представе геометријске фигуре. То
је довело до рада на новој математичкој области која је названа
аналитичка геометрија. 

Иако се појам сферне геометрије појавио још у старој Грчкој, озбиљније
истраживање неeуклидске геометрије је започето 1829.~године са радом
Лобачевског. Ипак, са интензивнијим истраживањем нееуклидских
геометрија се почело тек после пола века. Оно што је највише утицало на
ову промену јесте откриће комплексних бројева крајем XVIII
века. Комплексни бројеви су представљали значајану алатку за
истраживање особина објеката у различитим геометријама. Заменом
Декартове координатне равни комплексном равни добијају се
је\-дно\-став\-ни\-је формуле које описују геометријске објекте. Након
Гаусовe теорије о закривљеним површинама и Римановог рада о
многострукостима, геометрија Лобачевског добија на значају. Ипак,
највећи утицај има рад Белтрамија који показује да дводимензионална
неeуклидска геометрија је ништа друго до изучавање одговарајуће површи
константне негативне кривине. Уводи и појам пројективног диск модела
који је касније популаризован од стране Клајна. Поинкаре посматра
модел полуравни који су предложили Лиувил и Белтрами и пре свега
изучава изометрије хиперболичке равни које чувају оријентацију. Данас
се те трансформацију најчешће називају Мебијусове трансформације.

\begin{itemize}

\item David Hilbert. {\em Grundlagen der Geometrie.}
 Leipzig, B.G. Teubner, 1903.

\item W.~Schwabhäuser, W.~Szmielew, and A.~Tarski.
{\em Metamathematische Methoden in der Geometrie}.
Springer-Verlag, 1983.

\item {\lat Tristan Needham. {\em Visual Complex Analysis.} Oxford
    University Press, 1998.}

\item {\lat Hans Schwerdtfeger. {\em Geometry of Complex Numbers.}
    Dover Books on Mathematics. Dover Publications, 1979.}
\end{itemize}

\subsection{Интерактивно доказивање теорема уз помоћ рачунара}

Потреба за ригорозним заснивањем математике постоји веома дуго и са
развојем ма\-те\-ма\-ти\-ке повећавао се и степен ригорозности. Међу
наукама, математика се издваја својим прецизним језиком и јасним
правилима за изношење аргумената. Ова чињеница омогућава да се
ма\-те\-ма\-ти\-чки докази моделују као формална аксиоматска
извођења. Још у седамнаестом веку, постојала је идеја да мора
постајати неки заједнички језик којим би се могла записати
ма\-те\-ма\-ти\-чка тврђења и заједнички систем правила за
извођење. Један од најзначајних напретка у математици почетком
двадесетог века био је у открићу да се математички аргументи могу
представити у формалним аксиоматским системима на такав начин да
њихова исправност се може једноставно испитати коришћењем једноставних
механичких правила. Генерално, математика се могла формализовати
коришћењем аксиоматске теорије скупова и теорије
ти\-по\-ва. Математички доказ је ригорозан ако може бити записан у
логици првог реда као низ закључака који су изведени применом
дефинисаних правила из аксиоматске теорије скупова.

Често, механички проверени докази попуњавају празнине које постоје у
дефиницијама и доказима и упућују на дубљу анализу теме која се
изучава. У историји математематике постоји пуно контроверзи око
исправности математичких аргумената. Године 1935.~Лекат је објавио
књигу о грешкама које су до 1900.~године направили познати
математичари. Поред грешака, често се дешавало да математичари нису умели
да одреде да ли је неки доказ исправан или не и дешавало се да
се у потпуности верује да је доказ тачан ако га је објавио познати
математичар, као Гаус или Коши, и њихови докази нису подлегали дубљој
критици. У девентанестом веку докази постају све комплекснији и
математичари почињу да све више истичу важност ригорозности
доказа. Математичари се свакодневно сусрећу са прескоченим корацима у
доказима, са непрецизним дефиницијама, са хипотезама и претпоставкама
које недостају. Понекад грешке у доказима не буду примећене јако
дуго. На пример, први доказ теореме o обојивости графа са четири боје
је имао грешку која је уочена тек десет година касније. Грешке је
углавном лако исправити, али има случајева када је то јако тешко. На
пример, 1980.~године објављено је да је завршена класификација
једноставних коначних група, али је примећено да постоји пропуст у
једној од класа и исправка тог пропуста обајвљења је тек
2001.~године, а доказ је имао 1221 страну.  Додатно, често се дешава да
се одређени делови доказа никада не покажу, често уз реченицу
,,специјалан случај се тривијално показује'' при чему се дешава да за
тај специјалан случај тврђење не важи или га није тривијално
показати. Поред овога, понекад је потребно много времена да би се неки
доказ проверио. На пример, доказ Томаса Хејлса Кеплерове хипотезе
има 300 страна и 12 рецензената су провели четири године у анализи
доказа и коначно су написали да су 99\% сигурни да је доказ исправан.

Mноги научници су сматрали да потпуна формализација математике је
недостижни идеал. Са појавом рачунара настала је могућност генерисања машински
проверивих доказа. Тако су се појавили системи за формално доказивање
теорема. Постоје системи који омогућавају потпуно ау\-то\-мат\-ску
конструкцију доказа и они најчешће користе SAT решаваче или технике
презаписивања. Иако је систем за потпуно аутоматско доказивање теорема
важан подухват, постоје, за сада, мале реалне могућности да такав
систем заиста доказује компликована ма\-те\-ма\-ти\-чка тврђења.

Зато, посебан акценат је на системима који се заснивају на интеракцији
корисника и рачунара. Системи који се заснивају на логикама вишег реда
су полуаутоматски и у процесу формалног доказивања теорема од стране
корисника (често програмер и/или математичар) помажу тако што
контролишу исправност доказа и, колико је то могуће, проналазе
аутоматске доказе. Ови {\em интерактивни доказивачи} се називају и
{\em асистенти за доказивање теорема}. Данас постоји много
интерактивних доказивача: {\lat Isabelle, Isabelle/HOL, Coq, HOL
  Light, PVS} и други. Посебно се истичу {\lat Isabelle/HOL} и {\lat
  Coq} као системи са већим бројем корисника који су током година
развили велики скуп библиотека са формално доказаним теоријама које је
могуће даље надограђивати. Асистенти за доказивање теорема се користе
у различитим областима. Пре свега, истиче се примена у
образовању. Поред тога, могу се користити и за формалну верификацију
рачунарских програма. Помажу развој и продубљивање математичког знања.

У последњих десет година направљени су значајни резултати у формализацији
математике ко\-ри\-шће\-њем интерактивних доказивача. Формално је
доказана Броверова теорема фиксне тачке, основна теорема алгебре,
Годелова теорема некомплетности, реална анализа и многи други велики
фрагменти математике. У најзначајније постигнуте резултате до данас
могу се убројати формализација теореме о простим бројевима коју је
урадио Џереми Авигад, затим формални доказ о обојивости графа са
четири боје. Важно је поменути и пројекат Flyspeck који је покрену
Томас Хејлс да би могао формално да докаже Кеплерову хипотезу. У
оквиру овог пројекта формално је показано много математичких тврђења и
он има велику базу математичког знања која може да послужи у неким
новим формализацијама. Поред фор\-ма\-ли\-за\-ци\-је математике,
коришћењем асистената за доказивање теорема рађена је и верификација
софтвера. Значајан резултат је CompCert, формално верификован
компилатор и L4, формално верификован оперативни систем.

Постоји велики број формализација фрагмената различитих геометрија у
оквиру инте\-ра\-кти\-вних доказивача теорема. Делови Хилберове књиге
,,Основи геометрије'' су формализовани у {\lat Isabelle}-у и {\lat
  Coq}-у. У оквиру система {\lat Coq} је формализована геометрија
Тарског, конструктивна ге\-о\-мет\-ри\-ја, пројективна геометрија,
геометрија лењира и шестара и друге.

Неки од најзначајних радова из формализације геометрије су:
\begin{itemize}
\item {\lat Laura Meikle and Jacques Fleuriot. {\em Formalizing
      Hilberts Grundlagen in Isabelle/Isar.}  In Theorem Proving in
    Higher Order Logics, volume 2758 of Lecture Notes in Computer
    Science. Springer, 2003.}

\item {\lat Phil Scott. {\em Mechanising Hilberts Foundations of
      Geometry in Isabelle.} Master’s thesis, University of Edinburgh,
    2008.}

\item {\lat Julien Narboux. {\em Mechanical Theorem Proving in
      Tarski’s Geometry.} In Automated Deduction in Geometry, volume
    4869 of Lecture Notes in Computer Science. Springer, 2007.}

\item {\lat Gilles Kahn. {\em Constructive geometry according to Jan
      von Plato.} Coq contribution, Coq V5.10, 1995.}

\item {\lat Fr\'ed\'erique Guilhot. {\em Formalisation en Coq et
      visualisation d’un cours de g\'eom\'etrie pour le lyc\'ee.}
    Technique et Science Informatiques, 24(9), 2005.}

\item Nicolas Magaud, Julien Narboux, and Pascal Schreck. {\em
    Formalizing Projective Plane geometry in Coq.} In Automated
  Deduction in Geometry, volume 6301 of Lecture Notes in Computer
  Science. Springer, 2011.

\item Stojanović, Sana, Vesna Pavlović, and Predrag Janičić. {\em A coherent logic based geometry theorem prover capable of producing formal and readable proofs.} Automated Deduction in Geometry. Springer Berlin Heidelberg, 2011. 201-220.
\end{itemize}

Формализација система Тарски за неeуклидске геометрије
(Клајн-Белтрами модел) урађена је у раду:
\begin{itemize}
\item {\lat Timothy James McKenzie Makarios. {\em A mechanical verification of the independence of
Tarski’s Euclidean axiom.} Master’s thesis, Victoria University of Wellington, 2012.}
\end{itemize}

Формализација комплексне анализе може се видети у радовима:
\begin{itemize}
\item {\lat Robert Milewski. {\em Fundamental theorem of algebra.} Formalized Mathematics, 9(3), 2001.}

\item {\lat Herman Geuvers, Freek Wiedijk, and Jan Zwanenburg. {\em A Constructive Proof of the
Fundamental Theorem of Algebra without Using the Rationals.} In Types for Proofs and
Programs, volume 2277 of Lecture Notes in Computer Science. Springer, 2002.}
\end{itemize}

\subsection{Аутоматско доказивање теорема}

Један од првих аутоматских доказивача теорема био је аутоматски
доказивач за геометрију. Интересовање за аутоматско доказивање у
геометрији постоји још одавно, од времена Тарског. Он је развио
алгебарску методу за Еуклидску геометрију, али је та метода била
неупотребљива за комликованије теореме. Ипак, највећи напредак је
направљен тек средином XX-ог века када је Ву предложио своју
алгебарску методу за доказивање теорема у Еуклидској
геометрији. Његова метода је могла да докаже и веома комплексне
теореме. Још једна алгебарска метода која се развила у исто време је
метода Гребнерових база. Ови методи имају аналитички приступ у
доказивању и заснивају се на репрезентацији тачака коришћењем
координата. Мо\-де\-рни доказивачи теорема који се заснивају на овим
методима су коришћени да се докажу стотине нетривијалних
теорема. Ипак, велика мана ових система је што производе доказе који
нису читљиви. Деведесетих година XX-ог века постојало је више покушаја
да се овај проблем реши и развијене су нове методе засноване на
аксиоматизацији синтетичке геометрије -- метода површина, метода пуног
угла. Ипак, њихова главна мана је што су далеко мање ефикасни у односу
на алгебарске методе.

Примена система за аутоматско доказивање теорема у геометрији је
велика. Пре свега ови системи се могу користити у образовању. Поред
тога користе се у начним областима као што су роботика, биологија,
препознавање слика и друге.

Значајни радови из ове области су:
\begin{itemize}
\item {\lat Wen-Ts\"un Wu.  {\em On the decision problem and the  mechanization  of  theorem  proving  in  elementary  geometry.}
Scientia  Sinica ,  21:157--179, 1978}

\item {\lat Deepak Kapur. {\em Using Gr\"obner bases to reason about geometry problems.} Journal of Symbolic Computation, 2(4):399--408, 1986}

\item {\lat  Shang-Ching Chou, Xiao-Shan Gao, and Jing-Zhong  Zhang. {\em Machine  Proofs  in  Geometry.} World Scientific, Singapore, 1994}
\end{itemize}

\section{Циљеви тезе}

\subsection{Формализација аналитичке геометрије Декартове равни}

Синтетичка геометрија се обично изучава ригорозно, као пример
ригорозног аксиоматског извођења. Са друге стране, аналитичка
геометрија се углавном изучава неформално. Често се ова два приступа
представљају независно и веза између њих се ретко показује.

Циљ тезе је да повеже ове празнине. Први циљ је да се формализује
аналитичка геометрија тј. Декартова раван у оквиру система за
интерактивно доказивање теорема. Циљ је пре\-дста\-ви\-ти добро изграђену
формализацију Декартове геометрије равни у оквиру система
Isabelle/HOL. Потом даћемо неколико различитих дефиниција Декартове
координатне равни и показати да су све дефиниције
еквивалентне. Дефиниције ће бити преузете из стандарних уџбеника.
Међутим, да би их исказали у формалном окружењу асистента за
доказивање теорема, потребно је подићи ниво ригорозности. На пример,
када дефинишемо праве преко једначина, неки уџбеници помињу да
различите једначине репрезентују исту праву ако су њихови коефицијенти
``про\-по\-рцио\-на\-лни'', док неки други уџбеници често ово важно тврђење и
не наведу. У текстовима се обично не помињу конструкције као што су
релација еквиваленције и класа еквиваленције које ће морати да буду
уведене у формалним дефиницијама.

Потребно је формално показати да Декартова коорднатна раван задовољава
све аксиоме Тарског и већину аксиома Хилберта (укључујући и аксиому
непрекидности). Даљи циљ је анализирати доказе и упоредити који од два
система аксиома је лакши за формализацију.

Чињеница да је аналитичка геометрија модел синететичке геометрије се
често подразумева као једна једноставна чињеница. Наш циљ је да
формално покажемо ову чињеницу и ана\-ли\-зи\-ра\-мо колико су докази заиста
једноставни. Такође, циљ је извести неке закључке из доказа, на
пример, да ли постоји нека техника која је најпогоднија за коришћење у
доказима јер омогућава упрошћавање доказа или је боље све доказе
изводити директним приступом. Важно питање које се поставља је и
коришћење реалних бројева у доказима. Наиме, питање је да ли је могуће
за доказе користити било које нумеричко поље или је заиста неопходно
заиста користити поље реалних бројева и својства која у њему важе.

\subsection{Формализација геометрије комплексне равни}

Постоји јако пуно радова и књига које описују геометрију комплексне
равни. Циљ тезе је да представи потпуно развијену теорију проширене
комплексне равни, њених објеката (правих и кругова) и њених
трансформација (Мебијусове трансформације). Иако је већина појмова већ
описана у различитим књигама, циљ тезе је да споји бројне приступе у
један униформни приступ у коме ће бити коришћен јединствен и прецизан
језик за описивање појмова.  Наиме, чак и у оквиру исте књиге дешава
се да аутори користе исти назив за различите појмове.  Поред тога, циљ
је анализирати и формално показати све случајеве који често остану
недовољно истражени јер их више раличитих аутора сматра
тривијалним. 

Прво ће бити анализирана и формално представљена проширена комплексна
раван (ко\-мпле\-ксна раван која садржи тачку бесконачно), као и
хомогене координате комплексне тачке. У оквиру комплексне равни биће
дефинисани различити појмови и анализирана њихова сво\-јства, као што
су аритметичке операције, размера и дворазмера, метрика, група
Ме\-би\-ју\-со\-вих тра\-нсфор\-ма\-ци\-ја и њихово деловање на
проширену комплексну раван, потом неке важне подгрупе Мебијусових
тра\-нсфо\-рма\-ци\-ја (као што су аутоморфизми диска, Еуклидске
сли\-чно\-сти), уопштена права и њена веза са правама и круговима,
Риманова сфера, дејство Мебијусофих трансформација на уопштене праве,
јединственост уопштене праве, типови уо\-пште\-не праве и кардиналост
скупа уопштене праве, оријентисане уопштене праве, угао и очување угла
након дејства Мебијусових трансформација или инверзије итд.

Природно се намећу два приступа формализацији: геометријски и
алгебарски, као и питање да ли избор приступа утиче на ефикасност
формалног доказивања. У раду ће бити детаљно дискутовани односи између
два наведена приступа у формализацији као и њихове предности и
мане. Биће анализиране технике које се користе у доказима, као и
могућност коришћења аутоматизације. Такође, посматраћемо да ли је
доказе лакше извести у моделу Риманове сфере или у моделу хомогених
координата.

\subsection{Формализација Поинкареовог диск модела}

Циљ тезе је да формализује Поинкарев диск модел. Иако постоји много
радова који упућују да би формализација Поинкареовог диск модела могла
бити тривијална, до сада нисмо пронашли ни један рад, нити књигу који
садрже доказ да Поенкареов модел заиста представља модел синтетичке
геомерије представљен у довољно прецизном облику да би се могао
формализовати у оквиру интерактивног доказивача. Једна од потешкоћа
је, на пример, како погодно дефинисати релацију ,,између'' за три
тачке Поинкареове диск равни. Иако једна од основних релација, њену
дефиницију нисмо пронашли до сада, барем не у истакнутој литератури из
ове области. Потом, циљ је анализирати како Мебијусове трансформације
утичу на релацију ,,између'' и доказати да Поинкареов диск модел је
модел геометрије Лобачевског.

\subsection{Формална анализа алгебарских метода и разматрање њихове
  примене на проблеме у стереометрији}

Већина система са аналитичким приступом за доказивање теорема се
користи као софтвер којем се верује иако нису повезани са модерним
интерактивним доказивачима теорема. Да би се повећала њихова
поузданост потребно их је повезати са модерним интерактивним
до\-ка\-зи\-ва\-чи\-ма теорема и то је могуће учинити на два начина --
њиховом имплементацијом у оквиру интерактивног доказивача теорема и
показивањем њихове исправности или коришћењем ин\-тер\-ак\-тив\-них
доказивача да провере њихова тврђења. Неколико корака у овом правцу је
већ направљено.

Циљ тезе је да допуни ова истраживања и да понуди формално верификован
систем за аутоматско доказивање у геометрији који користи метод
Гребнерових база или Вуову методу.

Поред овога циљ је направити систем за доказивање тврђења у
стереометрији. Први корак је представити стереометријске објекте и
тврђења у одговарајућем облику коришћењем полинома. Потом је циљ
направити софтвер који би омогућио запис геометрисјких објеката и
тврђења у једноставном облику који је разумљив човеку, а потом
превођење тог записа у систем полинома на који би била примењена Вуова
метода. Даљи циљ је примена система на решавање различитих задатака из
уџбеника за средње школе и факултет и задатака са математичких
такмичења, као и анализа ефикасности оваквог приступа.

\section{Прелиминарни садржај тезе}

Теза ће се састојати из 4 велике целине подељене на мање делове при
чему се неки од делова могу мењати у зависности од тока истраживања:

\begin{enumerate}
\item \emph{Увод.} У овом поглављу биће описани основни појмови и
  главни циљеви тезе.

\item \emph{Интерактивни доказивачи теорема.} Биће описани
  софтверски системи за интерактивно формално доказивање теорема са
  нагласком на систему {\lat Isabelle/HOL} који ће бити ко\-ри\-шћен у
  раду.
      
\item \emph{Различити приступи и тренутни резултати у формализацији геометрије.}
  \begin{enumerate}[label*=\arabic*.]
      \item {\em Аутоматско доказивање у геометрији.} Ово поглавље ће
        представити различите приступе у овој области и биће наведени
        најзначајнији резултати.
      \item {\em Интерактивно доказивање у геометрији.}  Биће
        представљени различити приступи у интерактивном доказивању у
        геометрији и биће истакнути сви значајни резултати из стално
        растућег скупа нових радова и истраживања.
      \end{enumerate}

\item \emph{Формализација аналитичке геометрије.}
      \begin{enumerate}[label*=\arabic*.]
      \item {\em Модел аксиоматског система Тарског.} Биће описана формализација аналитичке
        геометрије у аксиоматском систему Тарског. Биће представљене дефиниције
        појмова и докази аксиома Тарског у аналитичког геометрији.
      \item {\em Модел аксиоматског система Хилберта.} Биће представљене дефиниције појмова
        и докази аксиома Хилберта у аналитичкој геометрији.
      \end{enumerate}
      
\item \emph{Формализација хиперболичке геометрије.}
      \begin{enumerate}[label*=\arabic*.]
      \item {\em Формализација геометрије комплексне равни.} Биће
        приказана формализација мно\-гих појмова комплексне геометрије:
        Мебијусове трансформације, круг, права, угао итд. У оквиру
        овог поглавља биће анализирани различити приступи у
        формали\-за\-ци\-ји.
      \item {\em Формализација Поинкареовог диск модела.} У овом
        поглављу биће анализиране аксиоме Тарског у Поинкареовом диск
        моделу.
      \end{enumerate}

\item \emph{Формална анализа алгебарских метода и разматрање њихове примене на проблеме у сте\-ре\-о\-мет\-ри\-ји}
      \begin{enumerate}[label*=\arabic*.]
      \item {\em Формална анализа алгебарских метода у систему {\lat
            Isabelle/HOL}.} Ово поглавље је посвећено анализи полинома
        којима се представљају геометријске конструкције и тврђења у
        оквиру система {\lat Isabelle/HOL}.
      \item {\em Примена алгебарских метода на проблеме у
          стереометрији.} Биће анализирана примена алгебарских метода
        на задатке из стереометрије.
      \end{enumerate}
\end{enumerate}


\section{Досадашњи резулатати кандидата}

Формализација аналитичке геометрије је заршена. Показано је да је она модел
за две аксиоматски засноване геометрије, геометрије Тарског и геометрије 
Хилберта. Ови резултати су приказани у раду:
\begin{itemize}
\item Petrović, Danijela, and Filip Marić. {\em Formalizing Analytic
    Geometries.} Paper presented at ADG 2012: The 9th
      International Workshop on Automated Deduction in Geometry, held
      on September 17--19, 2012 at the University of Edinburgh.
\end{itemize}

Поред овог рада, урађена је и формализација геометрије комплексне
равни. Дефинисани су основни појмови и показане су многе особине
Мебијусових трансформација. Такође, истражено је како Мебијусове
трансформације утичу на објекте комплексне равни, као што су права,
круг, угао, оријентација. Резултати овог истраживања приказани су у
раду:
\begin{itemize}
\item Marić, Filip, and Danijela Petrović. {\em Formalizing complex
    plane geometry.} Annals of Mathematics and Artificial
      Intelligence: 1-38.
\end{itemize}
\end{document} 



