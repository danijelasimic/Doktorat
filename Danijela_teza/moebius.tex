

\section{Conclusions and Further Work}
\label{sec:concl}
In this paper we have described some elements of our formalization of
the geometry of the complex plane $\extC$ both as complex projective
line and the Riemann sphere, arithmetic operations in $\extC$, ratio
and cross-ratio, chordal metric in $\extC$, the group of M\"obius
transformations and their action on $\extC$, some of its special
subgroups (Euclidean similarities, sphere rotations, disk
automorphisms), circlines and their connection with circles and lines,
the chordal metric, and the Riemann sphere, M\"obius action of
circlines, circline uniqueness, circline types and set cardinality,
oriented circlines, relations between M\"obius transformations and the
orientation, angle preservation properties of M\"obius
transformations, etc. Our current development counts around 12,000
lines of Isabelle/HOL code (all proofs are structured and written in
the proof language Isabelle/Isar, and our early attempts that are
subsumed by shorter algebraic proofs are not included), around 125
definitions and around 800 lemmas.

The crucial step in our formalization was our decision to use the
algebraic representation of all relevant objects (vectors of
homogeneous coordinates, matrices for M\"obius transformations,
Hermitean matrices for circlines, etc.). Although this is not a new
approach (for example, Schwerdtfeger's classic book
\cite{schwerdtfeger} follows this approach quite consistently), it is
not so common in the literature (and in the course material available
online). Instead, other, more geometrically oriented approaches
prevail. We have tried to follow that kind of geometric reasoning in
our early work on this subject, but we have encountered many
difficulties and did not have so much success. Based on this
experience, we conclude that introducing the powerful techniques of
linear algebra, makes the work on formalization an order of magnitude
simpler then when using just plain geometric reasoning.

It can be argued that sometimes geometrical arguments give better
explanations of some theorems, but when only justification is
concerned, the algebraic approach is clearly superior. However, to
keep the connection with the standard, geometric intuition, several
definitions must be introduced (more geometric, and more algebraic
ones) and they must be proved equivalent. For example, when the
definition of angles is given only through algebraic operations on
matrices and their determinants, the angle preservation property is
very easy to prove, but for educational purposes, this becomes
relevant only when that definition is connected with the standard
definition of angle between curves (i.e., their tangent vectors) ---
otherwise, the formalization becomes a game with meaningless symbols.

Another important conclusion that we make is that in formal documents,
case analysis should be avoided and extensions that help avoiding it
should be pursued whenever possible (e.g., it was much better to use
the homogeneous coordinates instead of a single distinguished infinity
point, it was much simpler to work with circlines then to distinguish
between circles and lines, etc.). Keeping different models of the same
concept (for example, in our case, homogeneous coordinates and the
Riemann sphere) also helps, as some proofs are easier in one, and some
proofs are easier in other models.

In principle, our proofs are not long (15-20 lines in
average). However, some tedious reasoning was sometimes required,
especially when switching between real and complex numbers (by the
conversion functions {\tt Re} and {\tt cor}). These conversion are
usually not present in informal texts and some better automation of
reasoning about these functions would be welcome. Isabelle's
automation was quite powerful in equational reasoning about ordinary
complex numbers using {\tt (simp add: field\_simps)} (with some minor
exceptions), but in the presence of inequalities, the automation was
not so good and we had to prove many things manually, that would be
considered trivial in informal texts.

In our further work we plan to use these results for formalizing
non-Euclidean geometries and their models (especially, spherical model
of the elliptic geometry and the Poincar\'e disc and upper half-plane
models of hyperbolic geometry).


\acknowledgement{The authors are grateful to Pascal Schreck, Pierre
  Boutry, and Julien Narboux for many valuable suggestions and
  advice.}

\bibliography{moebius}{}
\bibliographystyle{plain}

\end{document}
