\chapter{Формализација хиперболичке геометрије}
\label{chapter::hiperbolicka}

% ------------------------------------------------------------------------------
\section{Увод}
% ------------------------------------------------------------------------------

Постоjи много радова и књига коjе описуjу геометриjу комплексне равни,
а у овом поглављу ми ћемо представити резултате наше
формализације. Постоји више циљева које смо желели да остваримо.

\begin{enumerate}
\item Формализовати теорију проширене комплексне равни (комплексна
  раван коjа садржи тачку бесконачно) и њених објеката (правих и
  кругова) и њених трансформациjа (на пример, инверзијa и Мебиjусових
  трансформациjа).
\item Споjити броjне приступе које можемо срести у препорученој
  литератури (\cite{needham}, \cite{schwerdtfeger},
  \cite{mateljevic2006kompleksne}) у jедан униформни приступ у коме ће
  бити коришћен jединствен и прецизан jезик за описивање поjмова.
\item Aнализирати и формално доказати све случаjеве коjи често остану
  недовољно истражени jер их више различитих аутора сматра тривиjалним.
\item У раду ће бити детаљно дискутовани односи између два приступа у
  формализациjи као и њихове предности и мане. Природно се намећу два
  приступа формализациjи: геометриjски (рецимо приступ који предлаже
  Нидам \cite{needham}) и алгебарски (приступ који можемо видети у
  раду Швердфегерa \cite{schwerdtfeger}), као и питање да ли избор
  приступа утиче на ефикасност формалног доказивања.
\item Анализирати технике коjе се користе у доказима, као и могућност
  коришћења аутоматизациjе.
\item Посматрати да ли jе доказе лакше извести у моделу Риманове сфере
  или у моделу хомогених координата.
\end{enumerate}

У овој тези, ради сажетости, ми ћемо представити само основне
резултате наше формализације --- најважније дефиниције и тврђења. Ова
теза садржи само кратку рекапитулацију оригиналног формалног извођења и
многа својства која су формално доказана неће бити презентована у овом
раду. Додатно, ни један доказ неће бити приказан или описан, а све је
доступно у оквиру званичне \emph{Isabelle/HOL}
документације \footnote{\emph{Isabelle} документи у којима су теорије и
  докази доступни се налазе на адреси
  \url{http://www.matf.bg.ac.rs/~danijela/Moebius.zip}}. У тези ће
бити описане технике које смо користили, као и леме и помоћна тврђења
која су била потребна у доказима, али докази неће бити комплетно
приказани јер су превише технички, често веома велики и читаоцу могу
бити неинтересантни.

\paragraph{Организација поглавља.} 
У поглављу \ref{complex-osnove} дате су дефиниције основних појмова
који се користе -- комплексни бројеви, дводимензиони вектори и
аритметичке операције са њима, матрице димензије $2 \times 2$ и
основне аритметичке операције са матрицама, као и дефиниције
адјунговане, хермитске и унитарне матрице. У поглављу
\ref{complex--homogene} дата је дефиниција хомогених координата којима
се представљају комплексни бројеви проширене комплексне равни, потом
су дефинисане аритметичке операције над хомогеним координатама и дате
су леме о њиховим својствима, а потом су дефинисане размера и
дворазмера и доказана су основна својства ових операција. У поглављу
\ref{complex--riman} успостављена је веза између Риманове сфере и
проширене комплексне равни коришћењем стереографске пројекције. Ова
веза је важна јер се често показало лакшим доказати тврђење на
Римановој сфери него у проширеној комплексној равни. Такође, уведена
је и метрика коришћењем тетивног растојања. У поглављу
\ref{subsec:mobius} дефинисане су Мебијусове трансформације и дата је
њихова карактеризација коришћењем матрица димензије $2 \times
2$. Доказано је да Мебијусове трансформације формирају групу над
композицијом. Потом је дефинисано дејство Мебијусових трансформација
на тачке проширене комплексне равни. Као значајна група Мебијусових
трансформација издвајају се еуклидске сличности које такође формирају
групу и доказано је да се свака еуклидска сличност може добити
композицијом ротације, транслације и хомотетије. Једно од
најзначајнијих тврђења је да је дворазмера такође Мебијусова
трансформација. Ово својство је коришћено да се докаже да се
Мебијусовом трансформацијом било које три тачке могу сликати у било
које три тачке. Доказивањем те чињенице, многи докази су олакшани јер
се тврђење могло свести на доказивање да оно важи за неке специјално
изабране тачке, односно коришћењем механизма "без губитка на
општости". У поглављу \ref{subsec:circlines} уведен је појам уопштеног
круга (кругоправа) којом се у проширеној комплексној равни могу
представити и еуклидски кругови, али и еуклидске праве. Успостављена
је веза између равни у простору и кругоправих коришћењем стереографске
пројекције. Доказано је да се Мебијусовим трансформацијама кругоправа
слика у кругоправу. Потом је доказано да се карактеризација
кругоправих може поделити на три врсте: имагинарне кругоправе (које
немају тачака), тачка кругоправе (имају једну тачку) и реалне
кругоправе (имају барем три тачке). Уведена је оријентација
кругоправих и тиме је омогућено дефинисање диска, односно унутрашњости
кругоправе. Доказана су и тврђења која се односе на дејство
Мебијусових трансформација на оријентисане кругоправе. Једно од
најзначајнијих тврђења је да Мебијусове трансформације чувају угао
између кругоправих. У поглављу \ref{subsec:classification} дата је
класификација Мебијусових трансформација. Најзначајнији су
аутоморфизми диска који сликају унутрашњост диска у унутрашњост
диска. Такође, доказано је да се у односу на фиксне тачке, Мебијусове
трансформације могу класификовати у неколико група у којима све
трансформације имају заједничка својства. У поглављу \ref{sec:discuss}
описан је геометријски приступ у доказивању да Мебијусова
трансформација чува угао и дискутована је разлика између алгебарског и
геометријског приступа у доказу. У поглављу \ref{complex--poincare}
посматра се јединични диск и потребно је доказати да он представља модел
геометрије Лобачевског. Потребно је доказати да у њему важе све
аксиоме Тарског, осим Еуклидове аксиоме. Дефинисана је релација између
и доказана су њена основна својства, на пример, како Мебијусова
трансформација утиче на релацију између. Потом је доказано да неке
аксиоме Тарског важе на овом диску. На крају су сумирани сви закључци
ове главе.

\section{Основни појмови геометрије комплексне равни}
\label{complex-osnove}

\paragraph{Комплексни бројеви.}
Иако у систему \emph{Isabelle/HOL} постоји основна подршка за
комплексне бројеве (са {\tt complex} је означен тип комплексних
бројева у систему \emph{Isabelle/HOL}), то није било довољно за наше
потребе, па смо морали да направимо додатни напор и да ову теорију
проширимо. Многе леме које смо доказали су углавном веома техничке и
нису интересантне за виши ниво формализације коју описујемо и зато их
нећемо спомињати у овом тексту (нпр. \selectlanguage{english}{\tt
  \textbf{lemma} "arg $i$ = $pi/2$"} или {\tt \textbf{lemma} "$|z|^2$
  = Re ($z$ * cnj $z$)"})\selectlanguage{serbian}. Једна од
најзначајнијих дефиниција је дефиниција функције за канонизацију угла
$\downharpoonright \_ \downharpoonleft$, која узима у обзир $2\pi$
периодичност функција $\sin$ и $\cos$ и пресликава сваки угао у његову
каноничну вредносту која припада интервалу $(-\pi, \pi]$. Са овом
  функцијом, на пример, мултипликативна својства функције {\tt arg}
  могу се лако изразити и доказати.  \selectlanguage{english} {\tt
  \begin{tabbing}
    \hspace{5mm}\=\hspace{5mm}\=\hspace{5mm}\=\hspace{5mm}\=\hspace{5mm}\=\kill
    \textbf{lemma} "$z_1 * z_2$ $\neq$ 0 $\Longrightarrow$ arg($z_1 * z_2$) = $\downharpoonright$arg $z_1$ $+$ arg $z_2$$\downharpoonleft$"
  \end{tabbing}
}
\selectlanguage{serbian}

\noindent Како се комплексни бројеви често третирају и као вектори,
увођење скаларног производа два комплексна броја (што је дефинисано
као $\langle z_1, z_2\rangle = (z_1*\mathtt{cnj}\ z_2 +
z_2*\mathtt{cnj}\ z_1) / 2$) се показало веома корисним за сажето
приказивање неких услова.

\paragraph{Линеарна алгебра.}
Следећа важна теорија за даљу формализацију је теорија линеарне
алгебре $\mathbb{C}^2$.  Представљање вектора и матрица различитих
димензија у логици вишег реда представља изазов, због недостатка
зависних типова \cite{harrison05}, али у нашој формализацији треба
само да разматрамо просторе коначне димензије $\mathbb{C}^2$ и у неким
ситуацијама $\mathbb{R}^3$, тако да је наш задатак био
једноставнији. Комплексни вектори се дефинишу са {\tt \textbf{
    type\_synonym} C2\_vec = complex $\times$ complex}. Слично,
комплексне матрице ({\tt C2\_mat}) се дефинишу као четворка
комплексних бројева (матрица $\left(\begin{array}{cc}A & B\\C &
  D\end{array}\right)$ репрезентована је са $(A, B, C, D)$). Скаларно
  множење вектора означавамо са $*_{sv}$, а скаларно множење са
  матрицом означавамо са $*_{sm}$. Скаларни производ два вектора
  означен је са $*_{vv}$, производ вектора и матрице је означено са
  $*_{vm}$, производ матрице и вектора је означено са $*_{mv}$, а
  производ две матрице са $*_{mm}$. Нула матрица је означена са {\tt
    mat\_zero}, јединична матрица је означена са {\tt eye}, нула
  вектор је означен са {\tt vec\_zero}, детерминанта матрице је
  означена са \mbox{{\tt mat\_det}}, њен траг (сума елемената на
  главној дијагонали) са {\tt mat\_trace}, инверзна матрица са {\tt
    mat\_inv}, транспонована матрица са {\tt mat\_transpose},
  коњугација сваког елемента вектора са {\tt vec\_cnj}, коњугација
  сваког елемента матрице са {\tt mat\_cnj}, итд. Уведени су многи
  стандардни појмови линеарне алгебре. На пример, сопствене вредности
  су дефинисане и окарактерисане на следећи начин:
  \selectlanguage{english} {\tt
  \begin{tabbing}
    \hspace{5mm}\=\hspace{5mm}\=\hspace{5mm}\=\hspace{5mm}\=\hspace{5mm}\=\kill
\textbf{definition} eigenval :: "C2\_mat $\Rightarrow$ complex $\Rightarrow$ bool" \textbf{where}\\
\>"eigenval $k$ $A$ $\longleftrightarrow$ ($\exists v.$ $v$ $\neq$ vec\_zero $\wedge$ $A *_{mv} v = k *_{sv} v$)"\\
\textbf{lemma} "eigenval k H $\longleftrightarrow$ $k^2 - \mathtt{mat\_trace}\ H * k + \mathtt{mat\_det}\ H = 0$"
  \end{tabbing}
}
\selectlanguage{serbian}
\emph{Адјунгована} матрица је транспонована коњугована
матрица. \emph{Хермитске} матрице су оне које су једнаке својој
адјунгованој матрици, док су \emph{унитарне} матрице оне чији инверз је
једнак њиховој адјунгованој матрици.
\selectlanguage{english}
{\tt
\begin{tabbing}
\hspace{5mm}\=\hspace{5mm}\=\hspace{5mm}\=\hspace{5mm}\=\hspace{5mm}\=\kill
\textbf{definition} mat\_adj \textbf{where} "mat\_adj $H$ $=$ mat\_cnj (mat\_transpose $H$)"\\
\textbf{definition} hermitean \textbf{where} "hermitean $H$ $\longleftrightarrow$ mat\_adj $H$ = $H$"\\
\textbf{definition} unitary \textbf{where} "unitary $M$ $\longleftrightarrow$ mat\_adj $M$ $*_{mm}$  $M$ = eye"
\end{tabbing}
}
\selectlanguage{serbian}

Други основни појмови који су потребни у овом раду ће бити уведени у
даљем тексту, а читалац може пронаћи више информација у нашој
формализацији \footnote{\url{http://www.matf.bg.ac.rs/~danijela/Moebius.zip}}.

\label{sec:main}

\subsection{Проширена комплексна раван}
\label{subsec:extc}
Веома важан корак у развоју геометрије комплексне равни је проширена
комплексна раван која има један додатни елемент у односу на комплексну
раван $\C$ (који се третира као тачка бесконачно). Проширену
комплексну раван ћемо означити са $\extC$. Постоји више различитих
приступа \cite{needham,schwerdtfeger} за дефинисање
$\extC$. Најпривлачнији начин са становишта израчунавања је приступ
који се базира на хомогеним координатама, а најпривлачнији приступ
визуелно је заснован на стереографској пројекцији Риманове сфере.


\section{Хомогене координате}
\label{complex--homogene}
Проширена комплексна раван $\extC$ се идентификује са комплексном
пројективном правом (једнодимензиони пројективни простор над
комплексним пољем, понекад означаван са $\C P^1$). Свака тачка $\extC$
је репрезентована паром комплексних хомогених координата (од којих
нису оба једнака нули), а два пара хомогених координата представљају
исту тачку у $\extC$ акко су они пропорционални са неким ненула
комплексним фактором. Формализација овог својства у систему
\emph{Isabelle/HOL} се ослања на \emph{lifting/transfer} пакет за колинички
тип \cite{lifting-transfer} и састоји се из три фазе \footnote{Једна
  од фаза може бити прескочена ако би се користио \emph{lifting/transfer}
  пакет за парцијални количнички тип. Ову могућност ми нисмо користили
  у нашој формализацији због неких проблема који су постојали у
  ранијим верзијама количничког пакета. У међувремену, сви проблеми су
  исправљени, али наша формализације је у том тренутку већ увелико
  била развијена.}. Прво се уводи тип за пар комплексних бројева који
је различит од нуле (и који се истовремено посматра и као ненула
комплексни вектор).
\selectlanguage{english}
{\tt
\begin{tabbing}
\hspace{5mm}\=\hspace{5mm}\=\hspace{5mm}\=\hspace{5mm}\=\hspace{5mm}\=\kill
\textbf{typedef} C2\_vec$_{\neq 0}$ = "\{v::C2\_vec. v $\neq$ vec\_zero\}"
\end{tabbing}
}
\selectlanguage{serbian}
\noindent Одавде добијамо функцију за репрезентацију {\tt
  Rep\_C2\_vec$_{\neq 0}$} (коју ћемо означити са $\Repnzv{\_}$) која
враћа (ненула) пар комплексних бројева за сваки дати елемент помоћног
типа {\tt C2\_vec$_{\neq 0}$} и враћа функцију за апстракцију {\tt
  Abs\_C2\_vec$_{\neq 0}$} (коју ћемо ми означити са $\Absnzv{\_}$)
која враћа {\tt C2\_vec$_{\neq 0}$} елемент за сваки дати не-нула пар
комплексних бројева. Друго, кажемо да су два елемента типа {\tt
  C2\_vec$_{\neq 0}$} еквивалентна акко су њихове репрезентације
пропорционалне.

\selectlanguage{english}
{\tt
\begin{tabbing}
\hspace{5mm}\=\hspace{5mm}\=\hspace{5mm}\=\hspace{5mm}\=\hspace{5mm}\=\kill
\textbf{definition} $\approxhc$ :: "C2\_vec$_{\neq 0}$ $\Rightarrow$ C2\_vec$_{\neq 0}$ $\Rightarrow$ bool" \textbf{where}\\
\> "$z_1$ $\approxhc$ $z_2$ $\longleftrightarrow$ ($\exists$ (k::complex). k $\neq$ 0 $\wedge$ $\Repnzv{z_2}$ = k *$_{sv}$ $\Repnzv{z_1}$)"
\end{tabbing}
}
\selectlanguage{serbian}

\noindent Било је могуће увести и нешто другачију дефиницију, односно
да се уместо постојања параметра $k$ испитује вредност детерминанте,
односно да ли је $|\Repnzv{z_1}, \Repnzv{z_2}| = 0$.

\noindent Прилично је лако доказати да је $\approxhc$ релација
еквиваленције. Коначно, тип комплексних бројева проширене комплексне
равни дат хомогеним координатама се дефинише као класа еквиваленције
релације $\approxhc$ и уводи се преко наредног количничког типа.

\selectlanguage{english}
{\tt
\begin{tabbing}
\hspace{5mm}\=\hspace{5mm}\=\hspace{5mm}\=\hspace{5mm}\=\hspace{5mm}\=\kill
\textbf{quotient\_type} complex$_{hc}$ = C2\_vec$_{\neq 0}$ / $\approxhc$
\end{tabbing}
}
\selectlanguage{serbian}

Да сумирамо, на најнижем нивоу репрезентације постоји тип комплексних
бројева, на следећем нивоу је тип ненула комплексних $2\times 2$
вектора (који се представљају претходним типом), а на највишем нивоу
је количнички тип који има класу еквиваленије --- рад са овим
количничким типом (његовом репрезентацијом и апстракцијом) се ради у
позадини, коришћењем пакета \emph{lifting/transfer} \cite{lifting-transfer}.
Ова три нивоа апстракције могу на математичаре деловати збуњујуће, али
они су неопходни у формалном окружењу где сваки објекат мора имати
јединствени тип (на пример, често се узима да је $(1, i)$ истовремено
пар комплексних бројева и ненула комплексни вектор, али у нашој
формализацији $(1, i)$ је пар комплексних бројева, док је $\Absnzv{(1,
  i)}$ ненула комплексни вектор). 

\paragraph{Обични и бесконачни бројеви.}
Сваки обични комплексни број може бити конвертован у проширени
комплексни број.

\selectlanguage{english}
{\tt
\begin{tabbing}
\hspace{5mm}\=\hspace{5mm}\=\hspace{5mm}\=\hspace{5mm}\=\hspace{5mm}\=\kill
\textbf{definition} of\_complex\_rep :: "complex $\Rightarrow$ C2\_vec$_{\neq 0}$" \textbf{where}\\
\>of\_complex\_rep z = \Absnzv{(z, 1)}\\
\textbf{lift\_definition} of\_complex :: "complex $\Rightarrow$ complex$_{hc}$" \textbf{is}\\
\>  of\_complex\_rep
\end{tabbing}
}
\selectlanguage{serbian}

\noindent Тачка бесконачности се дефинише на следећи начин:
\selectlanguage{english}
{\tt
\begin{tabbing}
\hspace{5mm}\=\hspace{5mm}\=\hspace{5mm}\=\hspace{5mm}\=\hspace{5mm}\=\kill
\textbf{definition} inf\_hc\_rep :: C2\_vec$_{\neq 0}$ \textbf{where} inf\_hc\_rep = \Absnzv{(1, 0)}\\
\textbf{lift\_definition} $\infty_{hc}$ :: "complex$_{hc}$" is inf\_hc\_rep
\end{tabbing}
}
\selectlanguage{serbian}

Лако се доказује да су сви проширени комплексни бројеви или
$\infty_{hc}$ (акко је њихова друга координата једнака нули) или се
могу добити конвертовањем обичних комплексних бројева (акко њихова
друга координата није нула).
\selectlanguage{english}
{\tt
\begin{tabbing}
\hspace{5mm}\=\hspace{5mm}\=\hspace{5mm}\=\hspace{5mm}\=\hspace{5mm}\=\kill
\textbf{lemma}  "z = $\infty_{hc}$ $\vee$ ($\exists$\ x.\ z = of\_complex x)"
\end{tabbing}
}
\selectlanguage{serbian}

Нотација $0_{hc}$, $1_{hc}$ и $i_{hc}$ се користи да означи комплексне
бројеве $0$, $1$, и $i$ у проширеној комплексној равни (у хомогеним
координатама).

\paragraph{Аритметичке операције.} Аритметичке операције обичних
комплексних бројева могу бити проширене тако да се могу применити у
проширеној комплексној равни.

На најнижем, репрезентативном нивоу, сабирање $(z_1, z_2)$ и
$(w_1, w_2)$ се дефинише као $(z_1*w_2 + w_1*z_2, z_2*w_2)$, тј. 

\selectlanguage{english}
{\tt
\begin{tabbing}
\textbf{definition} plus\_hc\_rep :: "C2\_vec$_{\neq 0}$ $\Rightarrow$ C2\_vec$_{\neq 0}$ $\Rightarrow$ C2\_vec$_{\neq 0}$"\\
\hspace{5mm}\=\textbf{where} "}plus\_hc\_rep $z$ $w$ = (l\=et \= ($z_1$, $z_2$) \== $\Repnzv{z}$; ($w_1$, $w_2$) = $\Repnzv{w}$ \\
  \> \> in $\Absnzv{(z_1*w_2 + w_1*z_2, z_2*w_2)})$"}
\end{tabbing}
}
\selectlanguage{serbian}
\noindent Овим се добија ненула пар хомогених координата осим ако су
и $z_2$ и $w_2$ нула (у супротном, добија се лоше дефинисани елемент
{\tt $\Absnzv{(0, 0)}$})\footnote{Све функције (укључујући и
  апстрактну функцију $\Absnzv{\_}$) у HOL су тоталне. Ипак, све леме
  о тој функцији које су доказане, садрже један додатни услов, а то је
  да њихов аргумент није $(0, 0)$. Зато, не постоји разлог да
  резонујемо о вредности $\Absnzv{(0, 0)}$ и може се сматрати као лоше
  дефинисана вредност.}. Дефиниција је подигнута на ниво количничког
типа:

\selectlanguage{english}
{\tt
\begin{tabbing}
  \textbf{lift\_definition} $+_{hc}$ :: "complex$_{hc}$ $\Rightarrow$ complex$_{hc}$ $\Rightarrow$ complex$_{hc}$" is \\
  \hspace{5mm}\=plus\_hc\_rep
\end{tabbing}
}
\selectlanguage{serbian}
\noindent Ова дефиниција генерише следећи обавезан услов који треба
доказати $\lbrakk z \approxhc z'; w \approxhc w'\rbrakk
\Longrightarrow z +_{hc} w \approxhc z' +_{hc} w'$, а он се лако
доказује анализом случајева да ли су $z_2$ и $w_2$ оба једнака
нули. Приметимо да због HOL захтева да све функције буду тоталне, ми
не можемо дефинисати функцију само за добро дефинисане случајеве, и у
доказу морамо да посматрамо и лоше дефинисане случајеве.

Даље, доказује се да ова операција проширује уобичајено сабирање
комплексних бројева (операцију $+$ у $\mathbb{C}$).

\selectlanguage{english}
{\tt
\begin{tabbing}
\textbf{lemma} "of\_complex z $+_{hc}$ of\_complex w = of\_complex (z + w)"
\end{tabbing}
}
\selectlanguage{serbian}

\noindent Сума обичних комплексних бројева и $\infty_{hc}$ је
$\infty_{hc}$ (ипак, $\infty_{hc} +_{hc} \infty_{hc}$ је лоше
дефинисана).

\selectlanguage{english}
{\tt
\begin{tabbing}
\textbf{lemma} "of\_complex z $+_{hc}$ $\infty_{hc}$ = $\infty_{hc}$"\\
\textbf{lemma} "$\infty_{hc}$ $+_{hc}$ of\_complex z = $\infty_{hc}$"
\end{tabbing}
}
\selectlanguage{serbian}

Операција $+_{hc}$ је асоцијативна и комутативна, али $\infty_{hc}$
нема инверзни елемент, што прекида лепа алгебарска својства операције
$+$ на $\mathbb{C}$.

Друге аритметичке операције су такође проширене. На најнижем,
репрезентативном нивоу, унарни минус $(z_1, z_2)$ је $(-z_1, z_2)$,
производ $(z_1, z_2)$ и $(w_1, w_2)$ је $(z_1*z_2, w_1*w_2)$, а
реципрочна вредност $(z_1, z_2)$ је $(z_2, z_1)$ -- ове операције су
онда подигнуте на апстрактни количнички тип што производи операције
означене са {\tt uminus}$_{hc}$, $*_{hc}$, и {\tt
  recip}$_{hc}$. Одузимање (означено са $-_{hc}$) је дефинисано
коришћењем $+_{hc}$ и {\tt uminus}$_{hc}$, а дељење (означено са
$:_{hc}$) се дефинише коришћењем $*_{hc}$ и {\tt recip}$_{hc}$. Као и
у случају сабирања, доказано је да све ове операције одговарају
обичним операцијама коначне комплексне равни
(нпр. \selectlanguage{english}{\tt \textbf{lemma} "uminus$_{hc}$
  (of\_complex $z$) = of\_complex
  $(-z)$"})\selectlanguage{serbian}. Следеће леме показују понашање
ових операција када се у њима појављује и тачка бесконачно (приметимо
да изрази $0_{hc} *_{hc} \infty_{hc}$, $\infty_{hc} *_{hc} 0_{hc}$,
$0_{hc} :_{hc} 0_{hc}$, и $\infty_{hc} :_{hc} \infty_{hc}$ су лоше
дефинисани).

\selectlanguage{english}
{\tt
\begin{tabbing}
  \textbf{lemma} "uminus$_{hc}$\ $\infty_{hc}$ = $\infty_{hc}$"\\
  \textbf{lemma} "recip$_{hc}$\ $\infty_{hc}$ = $0_{hc}$" "recip$_{hc}$\ $0_{hc}$ = $\infty_{hc}$"\\
  \textbf{lemma} "$z \neq 0_{hc}$ $\Longrightarrow$ $z$ $*_{hc}$ $\infty_{hc}$ = $\infty_{hc}$ $\ \wedge\ $  $\infty_{hc}$ $*_{hc}$ $z$ = $\infty_{hc}$"\\
  \textbf{lemma} "$z \neq 0_{hc}$ $\Longrightarrow$ $z$ $:_{hc}$ $\infty_{hc}$ = $0_{hc}$"\\
  \textbf{lemma} "$z \neq \infty_{hc}$ $\Longrightarrow$ $\infty_{hc}$ $:_{hc}$ $z$ = $\infty_{hc}$"
\end{tabbing}
}
\selectlanguage{serbian}

Такође, проширен је и комплексни коњугат (на репрезентативном типу
$(z_1, z_2)$ је мапирано на $(\cnj{z_1}, \cnj{z_2})$), што даје
операцију {\tt cnj}$_{hc}$. Веома важна операција у комплексној
геометрији је \emph{инверзија у односу на јединични круг}:

\selectlanguage{english}
{\tt
\begin{tabbing}
\hspace{5mm}\=\hspace{5mm}\=\hspace{5mm}\=\hspace{5mm}\=\hspace{5mm}\=\kill
\textbf{definition} inversion$_{hc}$ :: "complex$_{hc}$ $\Rightarrow$ complex$_{hc}$" \textbf{where}\\
\> "inversion$_{hc}$ = cnj$_{hc}$ $\circ$ recip$_{hc}$"
\end{tabbing}
}
\selectlanguage{serbian}

Основне особине инверзије се лако доказују.
\selectlanguage{english}
{\tt
\begin{tabbing}
\hspace{5mm}\=\hspace{5mm}\=\hspace{5mm}\=\hspace{5mm}\=\hspace{5mm}\=\kill
\textbf{lemma} "inversion$_{hc}$ $\circ$ inversion$_{hc}$ = $id$"\\
\textbf{lemma} "inversion$_{hc}$ $0_{hc}$ = $\infty_{hc}$" "inversion$_{hc}$ $\infty_{hc}$ = $0_{hc}$"
\end{tabbing}
}
\selectlanguage{serbian}

\subsubsection{Размера и дворазмера.}

Размера и дворазмера су веома важни појмови у пројективној геометрији
и у проширеној комплексној равни (дворазмера се карактерише као
инваријанта Мебијусових трансформација -- основних трансформација у
$\extC$, и могуће је дефинисати праве коришћењем размера и круга
коришћењем дворазмере).

Размера тачака $z$, $v$ и $w$ се обично дефинише као
$\frac{z-v}{z-w}$. Наша дефиниција уводи хомогене координате.
\selectlanguage{english}
{\tt
\begin{tabbing}
\hspace{5mm}\=\hspace{5mm}\=\hspace{5mm}\=\hspace{5mm}\=\hspace{5mm}\=\kill
\textbf{definition} ratio\_rep \textbf{where} "}ratio\_rep z v w =  \\
\>(\textbf{l}\=\textbf{et} ($z_1$, $z_2$) = $\Repnzv{z}$; ($v_1$, $v_2$) = $\Repnzv{v}$; ($w_1$, $w_2$) = $\Repnzv{w}$ \\
\>\>\textbf{in} $\Absnzv{((z_1*v_2 - v_1*z_2)*w_2, (z_1*w_2 - w_1*z_2)*v_2)}$)"}\\
\textbf{lift\_definition} ratio :: \\
\>"complex$_{hc}$ $\Rightarrow$ complex$_{hc}$ $\Rightarrow$ complex$_{hc}$ $\Rightarrow$ complex$_{hc}$" \textbf{is} ratio\_rep
\end{tabbing}
}
\selectlanguage{serbian}
\noindent Приметимо да је ово добро дефинисано у свим случајевима осим
када важи $z=w=v$ или $z=v=\infty_{hc}$ или $z=w=\infty_{hc}$ или
$v=w=\infty_{hc}$ (ипак, у доказима код подизања на количнички тип ови
лоше дефинисани случајеви такође морају бити анализирани). Додатно,
оригинална размера разлика је дефинисана у свим случајевима осим када
$z=w=v$ или $z=\infty_{hc}$ или $v=w=\infty_{hc}$, тако да наша
дефиниција у хомогеним координатама природно проширује оригиналну
дефиницију. Следеће леме показују понашање размере у свим добро
дефинисаним случајевима (одговара оригиналној размери разлика кад год
је она дефинисана).
\selectlanguage{english}
{\tt
\begin{tabbing}
\hspace{5mm}\=\hspace{5mm}\=\hspace{5mm}\=\hspace{5mm}\=\hspace{5mm}\=\kill
\textbf{lemma} "}$\lbrakk$$z \neq v \vee z \neq w;\ z \neq \infty_{hc};\  v \neq \infty_{hc} \vee w \neq \infty_{hc}$$\rbrakk$ $\Longrightarrow$ \\
\>ratio $z$ $v$ $w$ = ($z$ $-_{hc}$ $v$) $:_{hc}$ ($z$ $-_{hc}$ $w$)"}\\
\textbf{lemma} "$\lbrakk$$v \neq \infty_{hc}$;\ $w \neq \infty_{hc}$$\rbrakk$ $\Longrightarrow$ ratio $\infty_{hc}$ $v$ $w$ = $1_{hc}$"\\
\textbf{lemma} "$\lbrakk$$z \neq \infty_{hc}$;\ $w \neq \infty_{hc}$$\rbrakk$ $\Longrightarrow$ ratio $z$ $\infty_{hc}$ $w$ = $\infty_{hc}$"\\
\textbf{lemma} "$\lbrakk$$z \neq \infty_{hc}$;\ $v \neq \infty_{hc}$$\rbrakk$ $\Longrightarrow$ ratio $z$ $v$ $\infty_{hc}$ = $0_{hc}$"
\end{tabbing}
}
\selectlanguage{serbian}

\noindent Последње две леме су последице прве леме. Такође, приметимо
да размера не може бити дефинисана на природан начин у случају када су
барем две тачке бесконачно (тако да функција размере остане непрекидна
по свим својим параметрима).

Дворазмера је дефинисана над 4 тачке $(z, u, v, w)$, обично као
$\frac{(z-u)(v-w)}{(z-w)(v-u)}$. Поново, ми је дефинишемо користећи
хомогене координате.

\selectlanguage{english}
{\tt
\begin{tabbing}
\hspace{3mm}\=\hspace{3mm}\=\hspace{3mm}\=\hspace{3mm}\=\hspace{3mm}\=\kill
\textbf{definition} cross\_ratio\_rep \textbf{where} "}cross\_ratio\_rep z u v w = \\
\>(\textbf{l}\=\textbf{et} \=($z_1$, $z_2$) = \Repnzv{$z$}; ($u_1$, $u_2$) = \Repnzv{$u$};\\
\>\>\> ($v_1$, $v_2$) = \Repnzv{$v$}; ($w_1$, $w_2$) = \Repnzv{$w$} \textbf{in}\\
\>$\Absnzv{(z_1*u_2 - u_1*z_2)*(v_1*w_2 - w_1*v_2), (z_1*w_2 - w_1*z_2)*(v_1*u_2 - u_1*v_2))}$"}\\
\textbf{lift\_definition} cross\_ratio :: "}complex$_{hc}$ $\Rightarrow$ complex$_{hc}$ $\Rightarrow$ \\
\>complex$_{hc}$ $\Rightarrow$ complex$_{hc}$ $\Rightarrow$ complex$_{hc}$"} \textbf{is} cross\_ratio\_rep
\end{tabbing}
}
\selectlanguage{serbian}

Ово је добро дефинисано у свим случајевима осим када $z=u=w$ или
$z=v=w$ или $z=u=v$ или $u=v=w$ (приметимо да су бесконачне вредности
за $z$, $u$, $v$ или $w$ дозвољене, што није случај у оригиналној
формулацији разломка). Нека основна својства дворазмере су дата
следећим лемама.

\selectlanguage{english}
{\tt
\begin{tabbing}
\hspace{5mm}\=\hspace{5mm}\=\hspace{5mm}\=\hspace{5mm}\=\hspace{5mm}\=\kill
\textbf{lemma} "}$\lbrakk$$(z \neq u \wedge v \neq w) \vee (z \neq w \wedge u \neq v)$;$z \neq \infty_{hc}$;$u \neq \infty_{hc}$;$v \neq \infty_{hc}$ $w \neq \infty_{hc}$$\rbrakk$\\ 
\> $\Longrightarrow$ cross\_ratio $z$ $u$ $v$ $w$ = $((z-_{hc}u) *_{hc} (v-_{hc}) :_{hc} ((z-_{hc}w) *_{hc} (v-_{hc}u))$"}\\
\textbf{lemma} "cross\_ratio $z$ $0_{hc}$ $1_{hc}$ $\infty_{hc}$ = $z$"\\
\textbf{lemma} "$\lbrakk$ $z_1 \neq z_2$;$z_1 \neq z_3$ $\rbrakk$ $\Longrightarrow$ cross\_ratio $z_1$ $z_1$ $z_2$ $z_3$ = $0_{hc}$"\\
\textbf{lemma} "$\lbrakk$ $z_2 \neq z_1$;$z_2 \neq z_3$ $\rbrakk$ $\Longrightarrow$ cross\_ratio $z_2$ $z_1$ $z_2$ $z_3$ = $1_{hc}$"\\
\textbf{lemma} "$\lbrakk$ $z_3 \neq z_1$;$z_3 \neq z_2$ $\rbrakk$ $\Longrightarrow$ cross\_ratio $z_3$ $z_1$ $z_2$ $z_3$ = $\infty_{hc}$"
\end{tabbing}
}
\selectlanguage{serbian}

\section{Риманова сфера и стереографска пројекција}
\label{complex--riman}

Проширена комплексна раван се може идентификовати са Римановом
(јединичном) сфером $\Sigma$ коришћењем стереографске пројекције
\cite{needham,schwerdtfeger}. Сфера се пројектује из свог сeвeрног
пола $N$ на $xOy$ раван (коjу означавамо са $\mathbb{C}$). Oва
проjeкциjа успоставља биjeктивно прeсликавањe $\stpr{}$ измeђу $\Sigma
\setminus N$ и коначнe комплeкснe равни $\mathbb{C}$. Тачка бeсконачно
je дeфинисана као слика од $N$.

У \emph{Isabelle/HOL} систему, сфeра $\Sigma$ je дeфинисана као нови тип.

\selectlanguage{english}
{\tt
  \begin{tabbing}
    \hspace{5mm}\=\hspace{5mm}\=\hspace{5mm}\=\hspace{5mm}\=\hspace{5mm}\=\kill
\textbf{typedef} riemann\_sphere = "\{$(x, y, z)$::R3\_vec. $x^2+y^2+z^2 = 1$\}"
  \end{tabbing}
}
\selectlanguage{serbian}

Као и раниje, ово дeфинишe функциjу {\tt Rep\_riemann\_sphere} (коjа
je означeна са $\Reprs{\_}$) и функциjу {\tt Abs\_riemann\_sphere}
(коjа je означeна са $\Absrs{\_}$) коjа повeзуje тачкe апстрактног
типа ({\tt riemann\_sphere}) и тачкe рeпрeзантативног типа (троjкe
рeалних броjeва). Стeрeографска проjeкциjа сe уводи на слeдeћи начин:

\selectlanguage{english}
{\tt
\begin{tabbing}
\hspace{4mm}\=\hspace{4mm}\=\hspace{4mm}\=\hspace{4mm}\=\hspace{4mm}\=\kill
\textbf{definition} stereographic\_rep :: "riemann\_sphere $\Rightarrow$ C2\_vec$_{\neq 0}$" where \\
 \> "}stereographic\_rep $M$ = \\
\>\> (l\=et ($x$, $y$, $z$) = $\Reprs{M}$ \\
\>\>\>  in if ($x$, \= $y$, $z$) $\neq$ (0, 0, 1) then $\Absnzv{(x + i * y,\, 1 - z)}$ \\
\>\>\>  else $\Absnzv{(1,\,0)}$)"}\\
\textbf{lift\_definition} stereographic :: "riemann\_sphere $\Rightarrow$ complex$_{hc}$" \textbf{is}\\
\>stereographic\_rep
\end{tabbing}
}
\selectlanguage{serbian}

За свe тачкe, ово je добро дeфинисано (вeктор $(x + i * y, 1 - z)$ je
нeнула jeр $(x, y, z) \neq (0, 0, 1)$, и $(1, 0)$ je очито нeнула).

Инвeрзна стeрeографска проjeкциjа сe дeфинишe на слeдeћи начин. 

\selectlanguage{english}
{\tt
\begin{tabbing}
  \textbf{def}\=\textbf{inition} inv\_stereographic\_rep :: "C2\_vec$_{\neq 0}$ $\Rightarrow$ riemann\_sphere" \\
  \textbf{where} \\
  \> "}inv\=\_stereographic\_rep $z$ = \\
  \>      \> (\textbf{l}\=\textbf{et} ($z_1$, $z_2$) = $\Repnzv{z}$  \\
  \>      \>   \>\textbf{in} \= \textbf{if} \= $z_2$ = 0 \textbf{then} $\Absrs{(0, 0, 1)}$ \\
  \>      \>   \>    \>\textbf{else} \textbf{l}\=\textbf{et} \= $z$ = $z_1$/$z_2$; $XY$ = (2*$z$)/cor (1+$|z|^2$); \\
  \>      \>   \>    \>$Z$ = ($|z|^2$-1)/(1+$|z|^2$) \\
  \>      \>   \>    \>\> \textbf{in} $\Absrs{(Re\ XY,\ Im\ XY,\ Z)}$)"}\\
\textbf{lift\_definition} inv\_stereographic :: "complex$_{hc}$ $\Rightarrow$ riemann\_sphere" \textbf{is} \\
\>inv\_stereographic\_rep
\end{tabbing}
}
\selectlanguage{serbian}

\noindent За свe тачкe, ово je добро дeфинисано (сума квадрата три
координатe je 1 у оба случаjа, па сe можe примeнити функциjа {\tt
  Abs\_riemann\_sphere}).

Вeза измeђу двe функциje je дата слeдeћим лeмама.

\selectlanguage{english}
{\tt
\begin{tabbing}
\hspace{5mm}\=\hspace{5mm}\=\hspace{5mm}\=\hspace{5mm}\=\hspace{5mm}\=\kill
\textbf{lemma} "stereographic $\circ$ inv\_stereographic = id"\\
\textbf{lemma} "inv\_stereographic $\circ$ stereographic = id"\\
\textbf{lemma} "bij stereographic" "bij inv\_stereographic"\\
\end{tabbing}
}
\selectlanguage{serbian}
\noindent Докази нису тeшки, али захтeваjу формализациjу врло
нeзгодних израчунавања.

\paragraph{Тетивно растојање.} %da li koristiti Ramanovo rastojanje??
Риманова сфера може бити метрички простор. Најчешћи начин да се уведе
метрички простор је коришћењем \emph{тетивне, Риманове метрике} --
растојање између две тачке на сфери је дужина тетиве која их спаја.
\selectlanguage{english} {\tt
\begin{tabbing}
\hspace{5mm}\=\hspace{5mm}\=\hspace{5mm}\=\hspace{5mm}\=\hspace{5mm}\=\kill
\textbf{definition} dist$_{rs}$ :: "riemann\_sphere $\Rightarrow$ riemann\_sphere $\Rightarrow$ real" \textbf{where}\\
\>  "}dist$_{rs}$ $M_1$ $M_2$ = (l\=et ($x_1$, $y_1$, $z_1$) = $\Reprs{M_1}$; ($x_1$, $y_1$, $z_1$) = $\Reprs{M_2}$\\
\>\>       in norm ($x_1$ - $x_2$, $y_1$ - $y_2$, $z_1$ - $z_2$))"}
\end{tabbing}
}
\selectlanguage{serbian}

Други приступ је да се узме \emph{лучно растојање}, односно узети
дужину одговарајућег кружног лука, односно одговарајући централни угао
у радијанима.

Функција {\tt norm} је уграђена функција и у овом случају она рачуна
eуклидску векторску норму. Коришћењем (сада већ познате) чињенице да
је $\mathbb{R}^3$ метрички простор (са функциjом растојања
$\lambda\ x\ y.\ {\tt norm}(x - y)$), није било тешко доказати да је
тип {\tt riemann\_sphere} опремљен са {\tt dist$_{rs}$} метрички
простор, тј. да је он инстанца локала {\tt metric\_space}. Иако је
дефинисана на сфери, тетивна метрика има своју репрезентацију и у
равни.

\selectlanguage{english}
{\tt
\begin{tabbing}
\hspace{2mm}\=\hspace{5mm}\=\hspace{5mm}\=\hspace{5mm}\=\hspace{5mm}\=\kill
\textbf{lemma} \textbf{assumes}\\
\>"stereographic $M_1$ = of\_complex $m_1$"\\
\>"stereographic $M_2$ = of\_complex $m_2$"\\
\>\textbf{shows} "dist$_{rs}$ $M_1$ $M_2$ = 2*|$m_1$-$m_2$| / (sqrt (1+|$m_1$|$^2$)*sqrt (1+|$m_2$|$^2$))"\\
\textbf{lemma} \textbf{assumes} \\
\>"stereographic $M_1$ = $\infty_{hs}$" \\
\>"stereographic $M_2$ = of\_complex $m$"\\
\>\textbf{shows} "dist$_{rs}$ $M_1$ $M_2$ = 2 / sqrt (1+|$m$|$^2$)"\\
\textbf{lemma} \textbf{assumes} \\
\>"stereographic $M_1$ = of\_complex $m$" \\
\>"stereographic $M_2$ = $\infty_{hs}$"\\
\>\textbf{shows} "dist$_{rs}$ $M_1$ $M_2$ = 2 / sqrt (1+|$m$|$^2$)"\\
\textbf{lemma} \textbf{assumes} "stereographic $M_1$ = $\infty_{hs}$" "stereographic $M_2$ = $\infty_{hs}$"\\
\>\textbf{shows} "dist$_{rs}$ $M_1$ $M_2$ = 0"
\end{tabbing}
}
\selectlanguage{serbian}

Ове леме праве разлику између коначних и бесконачних тачака, али се
ова анализа случаја може избећи коришћењем хомогених координата.

\selectlanguage{english}
{\tt
\begin{tabbing}
\hspace{5mm}\=\hspace{5mm}\=\hspace{5mm}\=\hspace{5mm}\=\hspace{5mm}\=\kill
\textbf{definition} "$\llangle z, w \rrangle$ = (vec\_cnj $\Repnzv{z}$) $*_{vv}$ ($\Repnzv{w}$)"\\
\textbf{definition} "$\llangle z \rrangle$ = sqrt (Re $\llangle z, z\rrangle$)"\\
\textbf{definition} "dist\_hc\_rep = 2*sqrt(1 - |$\llangle z,w \rrangle$|$^2$ / ($\llangle z \rrangle^2$ $*$ $\llangle w \rrangle^2$))"\\
\textbf{lift\_definition} dist$_{hc}$ :: "complex$_{hc}$ $\Rightarrow$ complex$_{hc}$ $\Rightarrow$ real" \textbf{is} \\
\>dist\_hc\_rep\\
\textbf{lemma} "dist$_{rs}$ $M_1$ $M_2$ = dist$_{hc}$ (stereographic $M_1$) (stereographic $M_2$)"
\end{tabbing}
}
\selectlanguage{serbian}
\noindent Понекад се ова форма зове Фубини--Стади
(енг.~\emph{Fubini--Study}) метрика.

Тип {\tt complex$_{hc}$} опремљен са {\tt dist$_{hc}$} метриком је
такође инстанца локала {\tt metric\_space}.  Ово тривијално следи из
последње леме која је повезује са метричким простором на Римановој
сфери. Постоје и директни докази ове чињенице (нпр. Хил
(енг.~\emph{Hille}) \cite{hille} даје директан доказ захваљујући
Какутани (енг.~\emph{Kakutani}), али доказ је некомплетан јер
занемарује могућност да једна тачка буде бесконачно), а ми смо и те
директне доказе формализовали\footnote{Наша формализација је започета
  без анализирања Риманове сфере, тако да смо у почетку једино и могли
  користити директне доказе, али у неком тренутку увели смо појам
  Риманове сфере и то је помогло да се многи докази упросте,
  укључујући и овај.}. Испоставило се да је нека својства лакше
доказати на Римановој сфери коришћењем функције {\tt dist$_{rs}$}
(нпр. неједнакост троугла), али нека својства је било лакше доказати у
пројекцији коришћењем функције {\tt dist$_{hc}$} (нпр. да је метрички
простор савршен, тј. да нема изолованих тачака), што показује значај
постојања различитих модела за исти концепт.

Коришћењем тетивне метрике у проширеној комплексној равни, и eуклидске
метрике на сфери у $\mathbb{R}^3$, доказано је да су стереографска
пројекција и инверзна стереографска пројекција непрекидне.

\selectlanguage{english}
{\tt
  \begin{tabbing}
    \hspace{5mm}\=\hspace{5mm}\=\hspace{5mm}\=\hspace{5mm}\=\hspace{5mm}\=\kill
\textbf{lemma} \="continuous\_on UNIV stereographic" \\
\> "continuous\_on UNIV inv\_stereographic"
  \end{tabbing}
}
\selectlanguage{serbian}

Приметимо да је у претходној леми, метрика имплицитна (у систему
\emph{Isabelle/HOL} претпоставља се да коришћена метрика је управо она
метрика која је коришћена да се докаже да је дати тип инстанца локала
{\tt metric\_space}).

\section{Мебијусове трансформације}
\label{subsec:mobius}
Мебијусове трансформације (које се још називају и холоморфна
пресликавања, линеарна фракциона трансформација или билинеарна
пресликавања) су основне трансформације проширене комплексне равни. У
нашој формализацији оне су уведене алгебарски. Свака трансформација је
представљена регуларном (несингуларном, недегенерисаном) $2\times 2$
матрицом која линеарно делује на хомогене координате. Како
пропорционалне хомогене координате представљају исту тачку у $\extC$,
тако и пропорционалне матрице представљају исту Мебијусову
трансформацију. Поново, формализација се састоји из три корака
коришћењем \emph{lifting/transfer} пакета. Прво, уводи се тип
регуларних матрица.

\selectlanguage{english}
{\tt
\begin{tabbing}
\textbf{typedef} C2\_mat\_reg = "\{$M$ :: C2\_mat. mat\_det $M$ $\neq$ 0\}"
\end{tabbing}
}
\selectlanguage{serbian}

\noindent Функција репрезентације {\tt Rep\_C2\_mat\_reg} ће бити
означена са $\Reprm{\_}$, а апстрактна функција {\tt
  Abs\_C2\_mat\_reg} ће бити означена са $\Absrm{\_}$. Регуларне
матрице формирају групу у односу на множење и она се често назива
\emph{генерална линеарна група} и означава се са $GL(2, \C)$. У неким
случајевима се разматра само њена подгрупа, \emph{специјална линеарна
  група}, означена са $SL(2, \C)$, која садржи само оне матрице чија
је детерминанта једнака 1.

\paragraph{Мебијусова група.}

Кажемо да су две регуларне матрице еквивалентне акко су њихове
репрезентације пропорционалне.  
\selectlanguage{english}
{\tt
  \begin{tabbing}
    \hspace{5mm}\=\hspace{5mm}\=\hspace{5mm}\=\hspace{5mm}\=\hspace{5mm}\=\kill
\textbf{definition} $\approxrm$ :: "C2\_mat\_reg $\Rightarrow$ C2\_mat\_reg $\Rightarrow$ bool" \textbf{where}  \\
\> "$M_1$ $\approxrm$ $M_2$ $\longleftrightarrow$ ($\exists$ (k::complex). k $\neq$ 0 $\wedge$ $\Reprm{M_2}$ = k *$_{sm}$ $\Reprm{M_1}$)"
  \end{tabbing}
}
\selectlanguage{serbian}

\noindent Лако се доказује да је ова релација заправо релација
еквиваленције. Елементи Мебијусове групе се уводе као класа
еквиваленције над овом релацијом.

\selectlanguage{english}
{\tt
\begin{tabbing}
\textbf{quotient\_type} mobius = C2\_mat\_reg / $\approxrm$
\end{tabbing}
}
\selectlanguage{serbian}

\noindent Понекад ћемо користити помоћни конструктор {\tt mk\_mobius}
који враћа елемент Мебијусове групе (класу еквиваленције) за дата 4
комплексна параметра (што има смисла само када је одговарајућа матрица
регуларна).

Мебијусови елементи формирају групу над композицијом. Ова група се
назива \emph{пројективна генерална линеарна група} и означена је са
$PGL(2, \mathbb{C})$. Поново, могу се разматрати само они елементи
\emph{специјалне пројективне групе} $SGL(2, \mathbb{C})$ чија је
детерминанта једнака $1$. Композиција Мебијусових елемената се постиже
множењем матрица које их репрезентују.

\selectlanguage{english}
{\tt
\begin{tabbing}
\textbf{def}\=\textbf{inition} mobius\_comp\_rep :: "C2\_mat\_reg $\Rightarrow$ C2\_mat\_reg $\Rightarrow$ C2\_mat\_reg" \\
\> \textbf{where} "}moe\=bius\_comp\_rep $M_1$ $M_2$ = $\Absrm{\Reprm{M_1}\ *_{mm}\ \Reprm{M_2}}$"}\\
\textbf{lift\_definition} mobius\_comp :: "mobius $\Rightarrow$ mobius $\Rightarrow$ mobius" \textbf{is}\\
\>mobius\_comp\_rep
\end{tabbing}
}
\selectlanguage{serbian}

\noindent Слично, инверзна Мебијусова трансформација се добија
инверзијом матрице која је представља.

\selectlanguage{english}
{\tt
\begin{tabbing}
\textbf{def}\=\textbf{inition} mobius\_inv\_rep :: "C2\_mat\_reg $\Rightarrow$ C2\_mat\_reg" \textbf{where} \\
         \>"mobius\_inv\_rep $M$ = $\Absrm{\mathtt{mat\_inv}\ \Reprm{M}}$"\\
\textbf{lift\_definition} mobius\_inv :: "mobius $\Rightarrow$ mobius" is "mobius\_inv\_rep"
\end{tabbing}
}
\selectlanguage{serbian}
\noindent Коначно, Мебијусова трансформација која је идентитет је
представљена јединичном матрицом.  
\selectlanguage{english}
{\tt
\begin{tabbing}
\textbf{def}\=\textbf{inition} mobius\_id\_rep :: "C2\_mat\_reg" \textbf{where} \\
\> "mobius\_id\_rep = $\Absrm{\mathtt{eye}}$"\\
\textbf{lift\_definition} mobius\_id :: "mobius" is mobius\_id\_rep
\end{tabbing}
}
\selectlanguage{serbian}

Све ове дефиниције увек уводе добро дефинисане објекте (јер је
производ регуларних матрица регуларна матрица, а инверз регуларне
матрице је такође регуларна матрица). Обавезни услови да би се
дефиниција могла подићи (нпр. {\tt $M_1$ $\approxrm$ $M_2$
  $\Longrightarrow$ mobius\_inv\_rep $M_1$ $\approxrm$
  mobius\_inv\_rep $M_2$}) се лако доказују. Онда, доказује се да је
тип {\tt mobius} заједно са овим операцијама инстанца локала {\tt
  group\_add} који је већ уграђен у систем \emph{Isabelle/HOL}. Зато,
ми ћемо понекад означавати {\tt mobius\_comp} са $+$, {\tt
  mobius\_inv} са унарним $-$, и {\tt mobius\_id} са $0$.

%nasla sam da ima ovo "dejstvo" kod nas da se kaze; da li postoji bolji izraz
\paragraph{Дејство Мебијусове групе.}

Мебијусове трансформације су дефинисане као дејство Мебијусове групе
на тачке проширене комплексне равни (које су дате у хомогеним
координатама).

\selectlanguage{english}
{\tt
\begin{tabbing}
\hspace{5mm}\=\hspace{5mm}\=\hspace{5mm}\=\hspace{5mm}\=\hspace{5mm}\=\kill
\textbf{definition} mobius\_pt\_rep :: "C2\_mat\_reg $\Rightarrow$ C2\_vec$_{\neq 0}$ $\Rightarrow$ C2\_vec$_{\neq 0}$"  \\
\> \textbf{where} "}moe\=bius\_pt\_rep $M$ $z$ = $\Absnzv{\Reprm{M}\ *_{mv}\ \Repnzv{z}}$"}\\
\textbf{lift\_definition} mobius\_pt :: "mobius $\Rightarrow$ complex$_{hc}$ $\Rightarrow$ complex$_{hc}$" \textbf{is}\\
\> mobius\_pt\_rep
\end{tabbing}
}
\selectlanguage{serbian}
\noindent Како производ регуларне матрице и ненула вектора је увек
ненула вектор, резултат је увек добро дефинисан. Подизање дефиниција
генерише обавезан услов {\tt $\lbrakk$$M$ $\approxrm$ $M'$; $z$
  $\approxhc$ $z'$$\rbrakk$ $\Longrightarrow$ mobius\_pt\_rep $M$ $z$
  $\approxhc$ mobius\_pt\_rep $M'$ $z'$} који се прилично лако
доказује.

Када се узима у обзир дејство групе на проширену комплексну раван,
онда се може видети да операције групе заиста одговарају композицији
пресликавања, инверзном пресликавању и идентичном пресликавању.  
\selectlanguage{english}
{\tt
\begin{tabbing}
\textbf{lemma} "}mobius\_pt \= (mobius\_comp $M_1$ $M_2$) = \\
\>(mobius\_pt $M_1$) $\circ$ (mobius\_pt $M_2$)"}\\
\textbf{lemma} "mobius\_pt (mobius\_inv $M$) = inv (mobius\_pt $M$)"\\
\textbf{lemma} "mobius\_pt (mobius\_id) = id"
\end{tabbing}
}
\selectlanguage{serbian}
Дејство је транзитивно (јер је увек бијективно пресликавање).
\selectlanguage{english} 
{\tt
\begin{tabbing}
\textbf{lemma} "bij (mobius\_pt M)"
\end{tabbing}
}
\selectlanguage{serbian}

У класичној литератури Мебијусове трансформације се обично
представљају у форми $\frac{az+b}{cz+d}$, и наредна лема заиста
оправдава и овакав запис (али у њој разликујемо специјалан случај када
је $z$ тачка бесконачно).
\selectlanguage{english}
{\tt 
\begin{tabbing}
\textbf{lemma} \=\textbf{assumes} "mat\_det ($a$, $b$, $c$, $d$) $\neq$ 0" \\
  \>\textbf{shows} "}moeb\=ius\_pt (mk\_mobius $a$ $b$ $c$ $d$) $z$ =  \\
  \>                    \>(\textbf{if} $z$ \= $\neq$ $\infty_{hc}$ \textbf{then}  \\
  \>                    \>        \>((of\_complex $a$) $*_{hc}$ $z$ $+_{hc}$ (of\_complex $b$)) $:_{hc}$  \\
  \>                    \>        \>((of\_complex $c$) $*_{hc}$ $z$ $+_{hc}$ (of\_complex $d$)) \\
  \>                    \>\textbf{else} (of\_complex $a$) $:_{hc}$ (of\_complex c))"}
\end{tabbing}
}
\selectlanguage{serbian}

Произвољна трансформација у $\extC$ ће бити звана Мебијусовом
трансформацијом акко је она дејство неког елемента Мебијусове групе.
\selectlanguage{english}
{\tt
\begin{tabbing}
\hspace{5mm}\=\hspace{5mm}\=\hspace{5mm}\=\hspace{5mm}\=\hspace{5mm}\=\kill
\textbf{definition} is\_mobius :: "(complex$_{hc}$ $\Rightarrow$ complex$_{hc}$) $\Rightarrow$ bool" \textbf{where}\\
\>"is\_mobius $f$ $\longleftrightarrow$ ($\exists$ $M$. $f$ = mobius\_pt $M$)"
\end{tabbing}
}
\selectlanguage{serbian}

Приметимо да већина до сада изнетих резултата зависи од чињенице да је
матрица репрезентације Мебијусове трансформације регуларна -- у
супротном, дејство би било дегенерисано и целу раван $\extC$ би
сликало у једну тачку.

\subsubsection{Неке специјалне Мебијусове трансформације.}

Многе трансформације са којима се сусрећемо у геометрији су заправо
специјална врста Мебијусових трансформација. Веома важна подгрупа је
група \emph{eуклидских сличности} (које се још називају и
\emph{интегралне трансформације}). Оне су одређене са два комплексна
параметра (и представљају Мебијусову трансформацију када први од та
два параметра није нула). 
\selectlanguage{english}
{\tt
\begin{tabbing}
\hspace{5mm}\=\hspace{5mm}\=\hspace{5mm}\=\hspace{5mm}\=\hspace{5mm}\=\kill
\textbf{definition} similarity :: "complex $\Rightarrow$ complex $\Rightarrow$ mobius" \textbf{where} \\
\>"similarity $a$ $b$ = mk\_mobius $a$ $b$ 0 1"
\end{tabbing}
}
\selectlanguage{serbian}
\noindent Сличности формирају групу (која се понекад назива и
\emph{параболичка група}).  
\selectlanguage{english}
{\tt
\begin{tabbing}
\hspace{5mm}\=\hspace{5mm}\=\hspace{5mm}\=\hspace{5mm}\=\hspace{5mm}\=\kill
\textbf{lemma} "}$\lbrakk$$a\neq 0$; $c \neq 0$$\rbrakk$ $\Longrightarrow$ \\
\>mobius\_comp (similarity $a$ $b$) (similarity $c$ $d$) = \\
\>similarity $(a*c)$ $(a*d+b)$"}\\
\textbf{lemma} "}$a \neq 0$ $\Longrightarrow$ \\
\>mobius\_inv (similarity $a$ $b$) = similarity $(1/a)$ $(-b/a)$"}\\
\textbf{lemma} "id\_mobius = similarity $1$ $0$"\\
\end{tabbing}
}
\selectlanguage{serbian}

Њихово дејство је линеарна трансформација у $\C$, а свака линеарна
трансформација у $\C$ која није константна је дејство елемента групе
eуклидских сличности.  
\selectlanguage{english}
{\tt
\begin{tabbing}
\hspace{5mm}\=\hspace{5mm}\=\hspace{5mm}\=\hspace{5mm}\=\hspace{5mm}\=\kill
\textbf{lemma} "}$a \neq 0$ $\Longrightarrow$ \=mobius\_pt (similarity $a$ $b$) = \\
\>($\lambda$ $z$. (of\_complex $a$) $*_{hc}$ $z$ $+_{hc}$ (of\_complex $b$))"}
\end{tabbing}
}
\selectlanguage{serbian}

Еуклидске сличности су једини елементи Мебијусове групе такви да је
тачка $\infty_{hc}$ фиксна тачка.  
\selectlanguage{english}
{\tt
\begin{tabbing}
\hspace{5mm}\=\hspace{5mm}\=\hspace{5mm}\=\hspace{5mm}\=\hspace{5mm}\=\kill
\textbf{lemma} "}mobius\_pt $M$ $\infty_{hc}$ = $\infty_{hc}$ $\longleftrightarrow$ \\
\>\>($\exists$ $a$ $b$. $a \neq 0$ $\wedge$ $M$ = similarity $a$ $b$)"}
\end{tabbing}
}
\selectlanguage{serbian}

Ако су и тачка $\infty_{hc}$ и тачка $0_{hc}$ фиксне, онда је то
сличност са коефицијентима $a \neq 0$ и $b=0$, а дејство је облика {\tt
  $\lambda$ $z$. (of\_complex $a$) $*_{hc}$ $z$}.

\selectlanguage{english}
{\tt
\begin{tabbing}
  \hspace{5mm}\=\hspace{5mm}\=\hspace{5mm}\=\hspace{5mm}\=\hspace{5mm}\=\kill
  \textbf{lemma} "}mobius\_pt $M$ $\infty_{hc}$ = $\infty_{hc}$ $\wedge$ mobius\_pt $M$ $0_{hc}$ = $0_{hc}$ $\longleftrightarrow$ \\
\>($\exists$ $a$. $a \neq 0$ $\wedge$ $M$ = similarity $a$ $0$)"}
\end{tabbing}
}
\selectlanguage{serbian}

Еуклидске сличности укључују транслацију, ротацију и хомотетију и
свака еуклидска сличност се може добити као композиција ове три врсте
пресликавања.  
\selectlanguage{english} {\tt
\begin{tabbing}
\hspace{5mm}\=\hspace{5mm}\=\hspace{5mm}\=\hspace{5mm}\=\hspace{5mm}\=\kill
\textbf{definition}\ "}translation $v$ \== similarity 1 $v$"}\\
\textbf{definition}\ "}rotation $\phi$ \>= similarity (cis $\phi$) 0"}\\
\textbf{definition}\ "}dilatation $k$ \>= similarity (cor $k$) 0"}\\
\textbf{lem}\=\textbf{ma} "}$a \neq 0$ $\Longrightarrow$ similarity $a$ $b$ = \\
\>(translation $b$) + (rotation (arg $a$)) + (dilatation $|a|$)"}
\end{tabbing}
}
\selectlanguage{serbian}

Реципрочна вредност ($1_{hc}:_{hc}z$) је такође Мебијусова трансформација.
\selectlanguage{english} 
{\tt
\begin{tabbing}
\hspace{5mm}\=\hspace{5mm}\=\hspace{5mm}\=\hspace{5mm}\=\hspace{5mm}\=\kill
\textbf{definition} "reciprocation = mk\_mobius (1, 0, 0, 1)"\\
\textbf{lemma} "recip$_{hc}$ = mobius\_pt reciprocation"
\end{tabbing}
}
\selectlanguage{serbian}
\noindent Са друге стране, инверзија није Мебијусова трансформација
(то је основни пример такозваних анти-Мебијусових трансформација, или
антихоломорфне функције). 

Веома важна чињеница је да се свака Мебијусова трансформација може
добити композицијом eуклидских сличности и реципрочне функције. Један
од начина како се ово може постићи дат је следећом лемом (када је
$c=0$ је случај eуклидских сличности и ово је раније већ анализирано).

\selectlanguage{english}
{\tt
\begin{tabbing}
\hspace{5mm}\=\hspace{5mm}\=\hspace{5mm}\=\hspace{5mm}\=\hspace{5mm}\=\kill
\textbf{lemma} \= \textbf{assumes} "$c\neq 0$" and "$a*d - b*c \neq 0$"\\
\textbf{shows} "}mk\_mobius a b c d = \\
\>\>\> translation (a/c) + rotation\_dilatation ((b*c - a*d)/(c*c)) + \\
\>\>\> reciprocal + translation (d/c)"}
\end{tabbing}
}
\selectlanguage{serbian}

\noindent Декомпозиција је коришћена у многим доказима. Наиме, да
бисмо доказали да свака Мебијусова трансформација има неко својство,
довољно је доказати да реципрочна функција и eуклидске сличности
задовољавају то својство и да композиција чува то својство (обично,
најтеже је доказати у случају реципрочне функције, а остала два корака
буду углавном много једноставнија).

\selectlanguage{english}
{\tt
\begin{tabbing}
\hspace{5mm}\=\hspace{5mm}\=\hspace{5mm}\=\hspace{5mm}\=\hspace{5mm}\=\kill
\textbf{lemma} \textbf{assumes} "$\bigwedge$ $v$. $P$ (translation $v$)" "$\bigwedge$ $\alpha$. $P$ (rotation $\alpha$)"\\
\> "$\bigwedge$ $k$. $P$ (dilatation $k$)" "$P$ (reciprocation)"\\
\> "$\bigwedge$ $M_1$ $M_2$. $\lbrakk$ $P$ $M_1$; $P$ $M_2$ $\rbrakk$ $\Longrightarrow$ $P$ $(M_1 + M_2)$"\\
\>\textbf{shows} "$P$ $M$"
\end{tabbing}
}
\selectlanguage{serbian}

\subsubsection{Дворазмера као Мебијусова трансформација}

За било које три фиксне тачке $z_1$, $z_2$ и $z_3$, {\tt cross\_ratio
  $z$ $z_1$ $z_2$ $z_3$} се може посматрати као функција једне
променљиве $z$. Следећа лема гарантује да је ова функција Мебијусова
трансформација и да према особина дворазмере она слика $z_1$ у
$0_{hc}$, $z_2$ у $1_{hc}$ и $z_3$ у $\infty_{hc}$.

\selectlanguage{english}
{\tt
\begin{tabbing}
\hspace{5mm}\=\hspace{5mm}\=\hspace{5mm}\=\hspace{5mm}\=\hspace{5mm}\=\kill
\textbf{lemma} "}$\lbrakk$ $z_1 \neq z_2; z_1 \neq z_3; z_2 \neq z_3$ $\rbrakk$ $\Longrightarrow$ \\
\>is\_mobius ($\lambda$ $z$. cross\_ratio $z$ $z_1$ $z_2$ $z_3$)"}
\end{tabbing}
}
\selectlanguage{serbian}

Доказавши ово тврђење, дворазмера се може користити да се докаже да
постоји Мебијусова трансформација која слика било које три различите
тачке редом у $0_{hc}$, $1_{hc}$ и $\infty_{hc}$. Како Мебијусове
трансформације чине групу, једноставна последица овога је да постоји
Мебијусова трансформација која слика било које три различите тачке у
било које три различите тачке.

\selectlanguage{english}
{\tt
\begin{tabbing}
\hspace{5mm}\=\hspace{5mm}\=\hspace{5mm}\=\hspace{5mm}\=\hspace{5mm}\=\kill
\textbf{lemma} "}$\lbrakk$ $z_1 \neq z_2$; $z_1 \neq z_3$; $z_2 \neq z_3$ $\rbrakk$ $\Longrightarrow$ $(\exists$ $M$. mobius\_pt $M$ $z_1$ = $0_{hc}$ $\wedge$ \\
\> mobius\_pt $M$ $z_2$ = $1_{hc}$ $\wedge$ mobius\_pt $M$ $z_3$ = $\infty_{hc}$)"}
\end{tabbing}
}
\selectlanguage{serbian}

Следећа лема има веома важну примену у даљем развоју теорије јер
омогућава закључивање "без губитка на општости (бгно)"
\cite{wlog}. Наиме, ако Мебијусова трансформација чува неко својство,
онда уместо три произвољне тачке може се посматрати само случај
специјалних тачака $0_{hc}$, $1_{hc}$, и $\infty_{hc}$.

\selectlanguage{english}
{\tt
\begin{tabbing}
\hspace{5mm}\=\hspace{5mm}\=\hspace{5mm}\=\hspace{5mm}\=\hspace{5mm}\=\kill
\textbf{lemma} \textbf{assumes} "$P$ $0_{hc}$ $1_{hc}$ $\infty_{hc}$" "$z_1 \neq z_2$" "$z_1 \neq z_3$" "$z_2 \neq z_3$"\\
\>"}$\bigwedge$ $M$ $u$ $v$ \= $w$.\ $P$ $u$ $v$ $w$ $\Longrightarrow$ \\
\>\> $P$ (mobius\_pt $M$ $u$) (mobius\_pt $M$ $b$) (mobius\_pt $M$ $c$)"}\\
\>\textbf{shows} "$P$ $z_1$ $z_2$ $z_3$"
\end{tabbing}
}
\selectlanguage{serbian}

Једна од првих примена "бгно" резоновања за Мебијусове трансформације
је у анализи фиксних тачака Мебијусових трансформација. Лако се
доказује да једино идентично пресликавање има фиксне тачке $0_{hc}$,
$1_{hc}$, и $\infty_{hc}$. Такође важи да ако Мебијусова
трансформација $M$ има три различите фиксне тачке, онда је она
идентитет, али директан доказ овога се заснива на чињеници да $2\times
2$ матрица има највише два независна сопствена вектора, а овакво
закључивање се лако може избећи коришћењем "бгно" резоновања (како
било које три тачке можемо сликати редом у $0_{hc}$, $1_{hc}$, и
$\infty_{hc}$ неким пресликавањем $M'$, а онда пресликавање $M'+M-M'$
има ове три тачке фиксне па мора бити једнако $0$).

\selectlanguage{english}
{\tt
\begin{tabbing}
\hspace{5mm}\=\hspace{5mm}\=\hspace{5mm}\=\hspace{5mm}\=\hspace{5mm}\=\kill
\textbf{lemma} "}$\lbrakk$ mobius\_pt $M$ $0_{hs}$ = $0_{hs}$; mobius\_pt $M$ $1_{hs}$ = $1_{hs}$; \\
\>mobius\_pt $M$ $\infty_{hs}$ = $\infty_{hs}$ $\rbrakk$ $\Longrightarrow$ M = id\_mobius"}\\
\textbf{lemma} "}$\lbrakk$ mobius\_pt $M$ $z_1$ = $z_1$; mobius\_pt $M$ $z_2$ = $z_2$; \\
\>mobius\_pt $M$ $z_3$ = $z_3$; $z_1 \neq z_2$; $z_1 \neq z_3$; $z_2 \neq z_3$ $\rbrakk$ $\Longrightarrow$ M = id\_mobius"}
\end{tabbing}
}
\selectlanguage{serbian}

Последица овога је да постоји јединствена Мебијусова трансформација
која слика три различите тачке у друге три различите тачке (већ је
доказано да такво пресликавање постоји, а ако би постојала два таква
пресликавања онда би њихова разлика морала имати три фиксне тачке, што
значи да би била идентитет).

\selectlanguage{english}
{\tt
\begin{tabbing}
\hspace{3mm}\=\hspace{5mm}\=\hspace{5mm}\=\hspace{5mm}\=\hspace{5mm}\=\kill
\textbf{lemma} "}$\lbrakk z_1 \neq z_2$; $z_1 \neq z_3$; $z_2 \neq z_3$; $w_1 \neq w_2$; $w_1 \neq w_3$; $w_2 \neq w_3 \rbrakk$ $\Longrightarrow$ $\exists$! $M$.\\ 
\>mobius\_pt $M$ $z_1$ = $w_1$ $\wedge$ mobius\_pt $M$ $z_2$ = $w_2$ $\wedge$ mobius\_pt $M$ $z_3$ = $w_3$"}
\end{tabbing}
}
\selectlanguage{serbian}

Мебијусове трансформације чувају дворазмеру. Поново, директан доказ би
био компликован, па је  формализован елегантан индиректни доказ (у
основи, разлика {\tt $\lambda z$. cross\_ratio $z$ $z_1$ $z_2$ $z_3$}
и $M$ слика ($M$ $z_1$) у $0_{hc}$, ($M$ $z_2$) у $1_{hc}$, и ($M$
$z_3$) у $\infty_{hc}$, па зато мора бити једнака {\tt $\lambda
  z$. cross\_ratio $z$ ($M$ $z_1$) ($M$ $z_2$) ($M$ $z_3$)}, и тврђење
следи замењујући ($M$ $z$) са $z$).

\selectlanguage{english}
{\tt
\begin{tabbing}
\hspace{3mm}\=\hspace{5mm}\=\hspace{5mm}\=\hspace{5mm}\=\hspace{5mm}\=\kill
\textbf{lemma} "}$\lbrakk z_1 \neq z_2$; $z_1 \neq z_3$; $z_2 \neq z_3 \rbrakk$ $\Longrightarrow$\\
\>\>cross\_ratio $z$ $z_1$ $z_2$ $z_3$ = \\
\>\>cross\_ratio \=(mobius\_pt $M$ $z$) (mobius\_pt $M$ $z_1$)\\
\>\>\>(mobius\_pt $M$ $z_2$) (mobius\_pt $M$ $z_3$)"}
\end{tabbing}
}
\selectlanguage{serbian}

\section{Кругоправа}
\label{subsec:circlines}
Веома важно својство проширене комплексне равни је могућност да праве
и кругове посматрамо униформно. Основни објекат је \emph{уопштен круг}
или скраћено \emph{кругоправа}. У нашој формализацији ми смо пратили
приступ који је описао Швердфегер \cite{schwerdtfeger} и представили
смо кругоправе хермитским, ненула $2\times 2$ матрицама. У оригиналној
формулацији, матрица $\left(\begin{array}{cc}A & B\\C &
  D\end{array}\right)$ одговара једначини $A*z*\mathtt{cnj}\,z +
  B*\mathtt{cnj}\,z + C*z + D = 0$, где је $C = \mathtt{cnj}\,B$ и $A$
  и $D$ су реални (јер је матрица хермитска).  Кључно је да ова
  једначина представља праву када је $A=0$, а иначе круг.

Поново, наша формализација се састоји из три корака. Прво, уведен је
тип хермитских, ненула матрица.

\selectlanguage{english}
{\tt
\begin{tabbing}
\hspace{5mm}\=\hspace{5mm}\=\hspace{5mm}\=\hspace{5mm}\=\hspace{5mm}\=\kill
\textbf{definition} is\_C2\_mat\_herm :: "C2\_mat $\Rightarrow$ bool" \textbf{where}\\
\> "is\_C2\_mat\_herm H $\longleftrightarrow$ hermitean $H$ $\wedge$ $H$ $\neq$ mat\_zero"\\
\textbf{typedef} C2\_mat\_herm = "\{$H$ :: C2\_mat. is\_C2\_mat\_herm $H$\}"
\end{tabbing}
}
\selectlanguage{serbian}

Функција репрезентације {\tt Rep\_C2\_mat\_herm} ће бити означена са
$\Repcm{\_}$, а апстрактна функција {\tt Abs\_C2\_mat\_herm} ће бити
означена са $\Abscm{\_}$.  Имајући на уму интерпретацију у форми
једначине, јасно је да би поново пропорционалне матрице требало
сматрати еквивалентним. Овог пута, фактор пропорционалности матрица је
реалан ненула број.

\selectlanguage{english}
{\tt
\begin{tabbing}
\hspace{5mm}\=\hspace{5mm}\=\hspace{5mm}\=\hspace{5mm}\=\hspace{5mm}\=\kill
\textbf{definition} $\approxcm$ :: "C2\_mat\_herm $\Rightarrow$ C2\_mat\_herm $\Rightarrow$ bool" \textbf{where}\\
\>"$H_1$ $\approxcm$ $H_2$ $\longleftrightarrow$ ($\exists$ ($k$::real). $k$ $\neq$ $0$ $\wedge$ $\Repcm{H_2}$ = cor $k$ *$_{sm}$ $\Repcm{H_1}$)"
\end{tabbing}
}
\selectlanguage{serbian}

Лако се доказује да је ово релација еквиваленције, а кругоправе се
дефинишу коришћењем количничке конструкције као класа еквиваленције.

\selectlanguage{english}
{\tt
\begin{tabbing}
\hspace{5mm}\=\hspace{5mm}\=\hspace{5mm}\=\hspace{5mm}\=\hspace{5mm}\=\kill
\textbf{quotient\_type} circline = C2\_mat\_herm / $\approxcm$
\end{tabbing}
}
\selectlanguage{serbian}

Помоћни конструктор {\tt mk\_circline} даје кругоправу (класу
еквиваленције) за дата четири комплексна броја $A$, $B$, $C$ и $D$
(под претпоставком да они формирају хермитску, ненула матрицу).

Свака кругоправа одређује одговарајући скуп тачака. Поново, опис који
је дат у хомогеним координатама је нешто бољи него оригинални опис
који је дат за обичне комплексне бројеве. Тачка са хомогеним
координатама $(z_1, z_2)$ ће припадати скупу тачака кругоправе акко
$A*z_1*\mathtt{cnj}\,z_1 + B*\mathtt{cnj}\,z_1*z_2 +
C*z_1*\mathtt{cnj}\,z_2 + D*z_2*\mathtt{cnj}\,z_2 = 0$. Приметимо да
је ово квадратна форма која је одређена вектором хомогених координата
и хермитском матрицом. Зато, скуп тачака на датој кругоправoj се
формализује на следећи начин (поред дефиниције кругоправе на овом
месту дајемо и дефиниције билинеарне и квадратне форме које су уведене
у нашој основној теорији линеарне алгебре).

\selectlanguage{english}
{\tt
\begin{tabbing}
\hspace{5mm}\=\hspace{5mm}\=\hspace{5mm}\=\hspace{5mm}\=\hspace{5mm}\=\kill
\textbf{definition} "bilinear\_form $H$ $z_1$ $z_2$ = (vec\_cnj $z_1$) $*_{vm}$ H $*_{vv}$ $z_2$"\\
\textbf{definition} "quad\_form $H$ $z$ = bilinear\_form $H$ $z$ $z$"\\
\textbf{definition} on\_circline\_rep :: "C2\_mat\_herm $\Rightarrow$ C2\_vec$_{\neq 0}$ $\Rightarrow$ bool" \textbf{where}\\
\>"on\_circline\_rep $H$ $z$ $\longleftrightarrow$ quad\_form $\Repcm{H}$ $\Repnzv{z}$ = 0"\\
\textbf{lift\_definition} on\_circline :: "circline $\Rightarrow$ complex$_{hc}$ $\Rightarrow$ bool" \textbf{is}\\
\> on\_circline\_rep\\
\textbf{definition} circline\_set :: "complex$_{hc}$ set" \textbf{where} \\
\>"circline\_set $H$ = \{$z$. on\_circline $H$ $z$\}"
\end{tabbing}
}
\selectlanguage{serbian}

\noindent Подизање дефиниције {\tt on\_circline} ствара услове
$\lbrakk H_1 \approxcm H_2$; $z_1 \approxhc z_2 \rbrakk$
$\Longrightarrow$ $\mathtt{on\_circline\_rep}\ H_1\ z_1
\longleftrightarrow \mathtt{on\_circline\_rep}\ H_2\ z_2$ који се лако
доказују.


\paragraph{Неке специјалне кругоправе.}
Међу свим кругоправама најзначајније су јединични круг, $x$-оса и
имагинарни јединични круг.
\selectlanguage{english}
{\tt
\begin{tabbing}
\hspace{5mm}\=\hspace{5mm}\=\hspace{5mm}\=\hspace{5mm}\=\hspace{5mm}\=\kill
\textbf{definition} "unit\_circle\_rep = \Abscm{(1, 0, 0, -1)}"\\
\textbf{lift\_definition} unit\_circle :: "circline" \textbf{is} unit\_circle\_rep\\
\textbf{definition} "x\_axis\_rep = \Abscm{(0, i, -i, 0)}"\\
\textbf{lift\_definition} x\_axis :: "circline" \textbf{is} x\_axis\_rep\\
\textbf{definition} "imag\_unit\_circle\_rep = \Abscm{(1, 0, 0, 1)}"\\
\textbf{lift}\=\textbf{\_definition} imag\_unit\_circle :: "circline" \textbf{is}\\
\> imag\_unit\_circle\_rep
\end{tabbing}
}
\selectlanguage{serbian}

Лако се доказују нека основна својства ових кругоправих. На пример:
\selectlanguage{english}
{\tt
\begin{tabbing}
\hspace{5mm}\=\hspace{5mm}\=\hspace{5mm}\=\hspace{5mm}\=\hspace{5mm}\=\kill
\textbf{lemma} \="$0_{hc} \in $ circline\_set x\_axis" "$1_{hc} \in $ circline\_set x\_axis" \\
\>"$\infty_{hc} \in $ circline\_set x\_axis"
\end{tabbing}
}
\selectlanguage{serbian}

\subsubsection{Повезаност са правама и круговима у обичној eуклидској равни.}

У проширеној комплексној равни не постоји разлика између појма праве и
појма круга. Ипак, праве могу бити дефинисане као оне кругоправе код
којих матрице имају коефицијент $A = 0$, или, еквивалентно као оне
кругоправе које садрже тачку $\infty_{hc}$.

\selectlanguage{english}
{\tt
\begin{tabbing}
\hspace{5mm}\=\hspace{5mm}\=\hspace{5mm}\=\hspace{5mm}\=\hspace{5mm}\=\kill
\textbf{definition} is\_line\_rep  \textbf{where} \\
\>"is\_line\_rep $H$ $\longleftrightarrow$ (let ($A$, $B$, $C$, $D$) = $\Repcm{H}$ in $A = 0$)"\\
\textbf{lift\_definition} is\_line :: "circline $\Rightarrow$ bool" \textbf{is} is\_line\_rep\\
\textbf{definition} is\_circle\_rep  \textbf{where} \\
\>"is\_circle\_rep $H$ $\longleftrightarrow$ (let ($A$, $B$, $C$, $D$) = $\Repcm{H}$ in $A \neq 0$)"\\
\textbf{lift\_definition} is\_circle :: "circline $\Rightarrow$ bool" \textbf{is} is\_circle\_rep\\
\textbf{lemma}\ "is\_line $H$ $\longleftrightarrow$ $\neg$ is\_circle $H$"   "is\_line $H$ $\vee$ is\_circle $H$"\\
\textbf{lemma} \="is\_line $H$ $\longleftrightarrow$ $\infty_{hc} \in$ circline\_set $H$"\\
\>"is\_circle $H$ $\longleftrightarrow$ $\infty_{hc} \notin$ circline\_set $H$"
\end{tabbing}
} \selectlanguage{serbian}

Сваки eуклидски круг и eуклидкса права (у обичној комплексној равни,
коришћењем стандардне, eуклидске метрике) може бити представљена
коришћењем кругоправе.  
\selectlanguage{english}
{\tt
  \begin{tabbing}
    \hspace{5mm}\=\hspace{5mm}\=\hspace{5mm}\=\hspace{5mm}\=\hspace{5mm}\=\kill
\textbf{definition} mk\_circle\_rep $\mu$ $r$ = $\Abscm{(1,\ -\mu,\ -\mathtt{cnj}\ \mu,\ |\mu|^2-(\mathtt{cor}\ r)^2)}$\\
\textbf{lift\_definition} mk\_circle :: "complex $\Rightarrow$ real $\Rightarrow$ circline" \textbf{is} \\
\> mk\_circle\_rep\\
\textbf{lemma} "}$r \ge 0$ $\Longrightarrow$ circ\=line\_set (mk\_circle $\mu$ $r$) = \\
\> of\_complex ` $\{z.\ |z - \mu| = r\}$"}\\
\textbf{de}\=\textbf{finition} mk\_line\_rep \textbf{where} "}mk\_line\_rep $z_1$ $z_2$ = \\
\>  (let $B = i*(z_2-z_1)$ in $\Abscm{(0,\ B,\ \mathtt{cnj}\ B,\ -(B*\mathtt{cnj}\ z_1+\mathtt{cnj}\ B*z_1)}$)"}\\
\textbf{li}\=\textbf{ft\_definition} mk\_line :: "complex $\Rightarrow$ complex $\Rightarrow$ circline" \textbf{is} \\
\> mk\_line\_rep\\
\textbf{lemma} "}$z_1 \neq z_2$ $\Longrightarrow$ circ\=line\_set (mk\_line $z_1$ $z_2$) - $\{\infty_{hc}\}$ = \\
\> of\_complex ` \{$z$. collinear $z_1$ $z_2$ $z$\}"}
  \end{tabbing}
}
\selectlanguage{serbian}

Супротно такође важи, скуп тачака који су одређени кругоправом је увек
или eуклидски круг или eуклидска права. Следећа функција одређује
параметре круга или параметре праве (центар и полупречник у случају
круга или две различите тачке у случају праве) за дату кругоправу.
\selectlanguage{english}
{\tt
  \begin{tabbing}
    \hspace{5mm}\=\hspace{5mm}\=\hspace{5mm}\=\hspace{5mm}\=\hspace{5mm}\=\kill
\textbf{definition} euclidean\_circle\_rep \textbf{where} "}euclidean\_circle\_rep $H$ = \\
\> $($\textbf{l}\=\textbf{et} $(A, B, C, D)$ = $\Repcm{H}$\\
\>\> \textbf{in} ($-B/A$, sqrt$($Re $((B*C - A*D)/(A*A))))$$)$"}\\
\textbf{li}\=\textbf{ft\_definition} euclidean\_circle :: "circline $\Rightarrow$ complex $\times$ real" \textbf{is}\\
\> euclidean\_circle\_rep\\
\textbf{definition} euclidean\_line\_rep \textbf{where} "}euclidean\_line\_rep $H$ = \\
\>(\textbf{l}\=\textbf{et} \=$(A, B, C, D)$ = $\Repcm{H}$; \\
\>\>\>$z_1$ = $-(D*B)/(2*B*C)$;\\
\>\>\>$z_2$ = $z_1$ + $i$ * sgn (\textbf{if} arg $B$ > 0 \textbf{then} $-B$ \textbf{else} $B$)\\
\>\>\textbf{in} ($z_1$, $z_2$))"}\\
\textbf{lift\_definition} euclidean\_line :: "circline $\Rightarrow$ complex $\times$ complex" \textbf{is}\\
\> euclidean\_line\_rep
  \end{tabbing}
}
\selectlanguage{serbian}
\noindent Приметимо да је нормални вектор праве вектор који је
нормалан на вектор који је одређен координатним почетком и комплексним
бројем $B$ (коефицијент матрице), односно $z_2=z_1+i*B$. Да бисмо могли
да подигнемо дефиницију (тако да су добијене тачке исте за сваку
матрицу која репрезентује исту кругоправу) у дефиницији друге тачке
вектор $B$ је нормализован. Ово даје нешто већи израз од израза
$z_2=z_1+i*B$.

Додатно, кардиналност скупа тачака кругоправе зависи од знака израза
$\mathtt{Re} ((B*C - A*D)/(A*A))$. Зато, кругоправе могу бити
класификоване у три категорије у зависности од знака детерминанте
(која је увек реалан број, јер је матрица хермитска).

\selectlanguage{english}
{\tt
\begin{tabbing}
\hspace{5mm}\=\hspace{5mm}\=\hspace{5mm}\=\hspace{5mm}\=\hspace{5mm}\=\kill
\textbf{definition} circline\_type\_rep \textbf{where}\\
\>"circline\_type\_rep $H$ = sgn (Re (mat\_det ($\Repcm{H}$)))"\\
\textbf{lift\_definition} circline\_type :: "circline $\Rightarrow$ real" \textbf{is}\\
\> circline\_type\_rep
\end{tabbing}
}
\selectlanguage{serbian}
\noindent Обавезан услов $H \approxcm H' \Longrightarrow$ {\tt
  circline\_type\_rep $H$ = circline\_type\_rep $H'$} се лако
доказује, јер {\tt Re (mat\_det ($k$ $*_{sm}$ $H$)) = $($Re $k)^2$ *
  Re (mat\_det $H$)} важи за све хермитске матрице $H$ и за све $k$ са
имагинарним делом $0$.

Сада постаје јасно да је скуп тачака на датој кругоправој празан акко
је тип кругоправе позитиван (ове кругоправе се зову \emph{ имагинарне
  кругоправе}), да садржи само једну тачку акко је тип кругoправе
једнак нули (\emph{тачка кругоправе}) и да је бесконачан акко је тип
негативан (\emph{реалне кругоправе}).  Оно што је било изненађујуће је
да се испоставило да је веома тешко формално доказати ово тврђење и
било га је могуће доказати само када је формализовано дејство
Мебијусових трансформација на кругоправе, што је омогућило да се
користи "бгно" резоновање. Приметимо да не постоје имагинарне праве
јер кад је $A = 0$, онда {\tt mat\_det $H$ $\ge 0$}.

Коначно, веза између реалних кругоправих и eуклидских правих и кругова
се може успоставити.
\selectlanguage{english}
{\tt
\begin{tabbing}
\hspace{5mm}\=\hspace{5mm}\=\hspace{5mm}\=\hspace{5mm}\=\hspace{5mm}\=\kill
\textbf{lemma}\\
\> \textbf{assumes} "is\_circle H" "($\mu$, $r$) = euclidean\_circle H"\\
\>  \textbf{shows} "circline\_set H = of\_complex ` \{$z.$ $|z-\mu|$ = $r$\}"\\
\textbf{lemma}\\
\>  \textbf{assumes} \="is\_line $H$" "($z_1$, $z_2$) = euclidean\_line H"\\
\>\> "circline\_type $H$ $< 0$"\\
\>  \textbf{shows} \\
\> "circline\_set $H$ - $\{\infty_{hc}\}$ = of\_complex ` \{$z$. collinear $z_1$ $z_2$ $z$\}"
\end{tabbing}
}
\selectlanguage{serbian}
\noindent Приметимо да прва лема такође важи за имагинарни и тачка
круг јер су оба скупа празна. Ипак, друга лема једино важи за реалне
праве јер у случају тачка праве важи да $z_1=z_2$, па је леви скуп
празан, а десни је универзални скуп.

\subsubsection{Кругоправе на Римановој сфери.}

Кругоправе у равни одговарају круговима на Римановој сфери, и ми смо
формално доказали ову везу. Сваки круг у тродимензионом простору се
може добити као пресек сфере и равни. Успоставили смо један-на-један
пресликавање између кругова на Римановој сфери и равни у
простору. Приметимо и да није неопходно да раван сече сферу и тада
ћемо рећи да она дефинише јединствен имагинаран круг. Веза између
равни у простору и кругоправих у проширеној комплексној равни је
описао Швердфегер \cite{schwerdtfeger}. Ипак, аутор није приметио
да за једну специјалну кругоправу (ону чија је репрезентативна матрица
јединична матрица) не постоји раван у $\mathbb{R}^3$ која јој одговара
--- и да бисмо могли да имамо такву раван, потребно је да уместо
посматрања равни у $\mathbb{R}^3$, узмемо у обзир тродимензионални
пројективни простор и коначну хиперраван. Зато, ми дефинишемо раван на
следећи начин (опет у три корака).

\selectlanguage{english}
{\tt
\begin{tabbing}
\hspace{5mm}\=\hspace{5mm}\=\hspace{5mm}\=\hspace{5mm}\=\hspace{5mm}\=\kill
\textbf{typedef} R4\_vec$_{\neq 0}$ = "\{$(a, b, c, d)$ :: R4\_vec. $(a, b, c, d) \neq$ vec\_zero\}"
\end{tabbing}
}
\selectlanguage{serbian}

Приметимо да ће у $\mathbb{R}^3$, један од бројева $a$, $b$, или $c$
бити различит од $0$. Ипак, наша дефиниција дозвољава постојање равни
$(0, 0, 0, d)$ која лежи у бесконачности. Функција репрезентације ће
бити означена са $\Reppl{\_}$, а апстрактна функција ће бити означена
са $\Abspl{\_}$. Поново, две равни су еквивалентне акко су
пропорционалне (овог пута за неки ненула реални фактор).  \selectlanguage{english} {\tt
\begin{tabbing}
\hspace{5mm}\=\hspace{5mm}\=\hspace{5mm}\=\hspace{5mm}\=\hspace{5mm}\=\kill
\textbf{definition} $\approxp$ :: "R4\_vec$_{\neq 0}$ $\Rightarrow$ R4\_vec$_{\neq 0}$ $\Rightarrow$ bool" \textbf{where}\\
\>"$\alpha_1 \approxp \alpha_2 \longleftrightarrow (\exists k.\ k \neq 0 \wedge \Reppl{\alpha_2} = k*\Reppl{\alpha_1})$"
\end{tabbing}
}
\selectlanguage{serbian}

Коначно, равни (кругови који су у њима су добијени пресеком са
Римановом сфером) се дефинишу као класа еквиваленције ове релације.

\selectlanguage{english}
{\tt
\begin{tabbing}
\hspace{5mm}\=\hspace{5mm}\=\hspace{5mm}\=\hspace{5mm}\=\hspace{5mm}\=\kill
\textbf{quotient\_type} plane = R4\_vec$_{\neq 0}$ / $\approxp$
\end{tabbing}
}
\selectlanguage{serbian}

Коефицијенти равни дају линеарну једначину а тачка на Римановој сфери
лежи на кругу одређеном са равни акко њена репрезентација задовољава
линеарну једначину.

\selectlanguage{english}
{\tt
\begin{tabbing}
\hspace{5mm}\=\hspace{5mm}\=\hspace{5mm}\=\hspace{5mm}\=\hspace{5mm}\=\kill
\textbf{definition} on\_sphere\_circle\_rep \textbf{where}\\
\>"}on\_sphere\_circle\_rep $\alpha$ $M$ $\longleftrightarrow$ \\
\>\>(\textbf{l}\=\textbf{et} ($a$, $b$, $c$, $d$) = $\Reppl{\alpha}$; ($X$, $Y$, $Z$) = $\Reprs{M}$\\
\>\>\>  \textbf{in} $a*X + b*Y + c*Z + d = 0$)"}\\
\textbf{lift\_definition} on\_sphere\_circle :: \\
\> "plane $\Rightarrow$ riemann\_sphere $\Rightarrow$ bool" \textbf{is} on\_sphere\_circle\_rep\\
\textbf{definition} sphere\_circle\_set :: "riemann\_sphere set" \textbf{where}\\
\>"sphere\_circle\_set $\alpha$ = $\{A.$ on\_sphere\_circle $\alpha$ $A\}$"
\end{tabbing}
}
\selectlanguage{serbian}
\noindent Приметимо да нисмо морали да уведемо тачке у тродимензионом
пројективном простору (и њихове хомогене координате) јер смо ми једино
заинтересовани за тачке на Римановој сфери које нису бесконачне.

Следеће, ми уводимо стереографску и инверзну стереографску пројекцију
између кругова на Римановој сфери и кругова у проширеној комплексној
равни.

\selectlanguage{english}
{\tt
\begin{tabbing}
\hspace{5mm}\=\hspace{5mm}\=\hspace{5mm}\=\hspace{5mm}\=\hspace{5mm}\=\kill
\textbf{definition} stereographic\_circline\_rep \textbf{where} \\
\>"}stereographic\_circline\_rep $\alpha$  =\\
\>\>(\textbf{l}\=\textbf{et} \=$(a, b, c, d)$ = $\Reppl{\alpha}$; $A=\mathtt{cor}\,((c+d)/2)$; $B=(\mathtt{cor}\,a + i* \mathtt{cor}\,b)/2)$;\\
\>\>\>\>$C=(\mathtt{cor}\,a - i*\mathtt{cor}\,b)/2$; $D=\mathtt{cor}\,((d-c)/2))$\\
\>\>\>\textbf{in} $\Abscm{(A, B, C, D)}$"}\\
\textbf{lift\_definition} stereographic\_circline :: "plane $\Rightarrow$ circline" \textbf{is}\\
\>stereographic\_circline\_rep\\
\textbf{definition} inv\_stereographic\_circline\_rep \textbf{where} \\
\>"}inv\_stereographic\_circline\_rep $H$  =\\
\>\>(\textbf{l}\=\textbf{et} $(A, B, C, D)$ = $\Repcm{H}$\\
\>\>\>\textbf{in} $\Abspl{(\mathtt{Re}(B+C), \mathtt{Re}(i*(C-B)), \mathtt{Re}(A-D), \mathtt{Re}(D+A))}$"}\\
\textbf{lift\_definition} inv\_stereographic\_circline :: "circline $\Rightarrow$ plane" \textbf{is}\\
\>inv\_stereographic\_circline\_rep
\end{tabbing}
}
\selectlanguage{serbian}

Ова два пресликавања су бијективна и међусобно инверзна. Пројекција
скупа тачака круга на Римановој сфери је управо скуп тачака на
кругоправoj која се добија управо уведеном стереографском пројекцијом
круга.
\selectlanguage{english} {\tt
\begin{tabbing}
\hspace{5mm}\=\hspace{5mm}\=\hspace{5mm}\=\hspace{5mm}\=\hspace{5mm}\=\kill
\textbf{lemma} "stereographic\_circline $\circ$ inv\_stereographic\_circline = id"\\
\textbf{lemma} "inv\_stereographic\_circline $\circ$ stereographic\_circline = id"\\
\textbf{lemma} "bij stereographic\_circline" "bij inv\_stereographic\_circline"\\
\textbf{lemma} "}\=stereographic ` sphere\_circle\_set $\alpha$ = \\
\>circline\_set (stereographic\_circline $\alpha$)"}
\end{tabbing}
}
\selectlanguage{serbian}


\subsubsection{Риманове кругоправе.}

Још једна интересантна чињеница је да су реалне кругоправе ништа друго
до скупови тачака које су на једнаком одстојању од неких датих тачака
(заправо увек постоје тачно две такве тачке), али посматрајући
одстојање у тетивној метрици. На Римановој сфери ове две тачке
(зваћемо их тетивни центри) се добијају пресеком сфере и праве која
пролази кроз центар круга и нормална је на раван која садржи тај круг.

Тетивна кругоправа са датом тачком $a$ и полупречником $r$ је одређена
на следећи начин.

\selectlanguage{english}
{\tt
  \begin{tabbing}
    \hspace{5mm}\=\hspace{5mm}\=\hspace{5mm}\=\hspace{5mm}\=\hspace{5mm}\=\kill
\textbf{definition} chordal\_circle\_rep \textbf{where} "}chordal\_circle\_rep $\mu_c$ $r_c$ = \\
\>  (l\=et \=($\mu_1$, $\mu_2$) = $\Repnzv{\mu_c}$;\\
\>\>\>$A$ = 4*$|\mu_2|^2$ - (cor $r_c$)$^2$*($|\mu_1|^2 + |\mu_2|^2$); $B$ = $-4$*$\mu_1$*cnj $\mu_2$;\\
\>\>\>$C$ = -4*cnj $\mu_1$*$\mu_2$; $D$ = 4*$|\mu_1|^2$ - (cor $r_c$)$^2$*($|\mu_1|^2 + |\mu_2|^2$)\\
\>\>in mk\_circline\_rep $A$ $B$ $C$ $D$)"}\\
\textbf{lift\_definition} chordal\_circle :: "complex$_{hc}$ $\Rightarrow$ real $\Rightarrow$ circline" \textbf{is}\\
\> chordal\_circle\_rep\\
\textbf{lemma} "}\=$z$ $\in$ circline\_set (chordal\_circle $\mu_c$ $r_c$) $\longleftrightarrow$\\
\>$r_c \ge 0$ $\wedge$ dist$_{hc}$ $z$ $\mu_c$ = $r_c$"}
  \end{tabbing}
}
\selectlanguage{serbian}

\noindent За дату кругоправу, њен центар и радијус се могу одредити
ослањајући се на следеће леме (у зависности да ли су коефицијенти $B$ и
$C$ у репрезентативној матрици једнаки нули).  
\selectlanguage{english}
{\tt
  \begin{tabbing}
    \hspace{2mm}\=\hspace{5mm}\=\hspace{5mm}\=\hspace{5mm}\=\hspace{5mm}\=\kill
\textbf{lemma}\\
\>\textbf{assumes} \="is\_C2\_mat\_herm $(A, B, C, D)$" "Re ($A*D$) $<$ $0$" "$B = 0$"\\
\>\textbf{sh}\=\textbf{ows}\\
\>\>"}mk\_circli\=ne $A$ $B$ $C$ $D$ = \\
\>\>\>chordal\_circle $\infty_{hc}$ sqrt$($Re $((4*A)/(A-D))$$)$"}\\
\>\>"}mk\_circline $A$ $B$ $C$ $D$ = \\
\>\>\>chordal\_circle $0_{hc}$ sqrt$($Re $((4*D)/(D-A))$$)$"}\\
\textbf{lemma} \\
\> \textbf{assumes} \= "Re (mat\_det $(A, B, C, D)$) $<$ $0$" "$B \neq 0$"\\
\> \> "is\_C2\_mat\_herm $(A, B, C, D)$" "$C * \mu_c^2  + (D - A) * \mu_c - B = 0$" \\
\> \> "$r_c$ = sqrt$((4 + \mathtt{Re}((4 * \mu_c/B) * A)) / (1 + \mathtt{Re} (|\mu_c|^2)))$"\\
\>\textbf{shows} "mk\_circline $A$ $B$ $C$ $D$ = chordal\_circle (of\_complex $\mu_c$) $r_c$"
  \end{tabbing}
}
\selectlanguage{serbian}

Као и у претходним случајевима, може се увести функција која враћа
тетивне параметре (потребно је направити разлику међу случајевима
$B=0$ и $B \neq 0$ и у другом случају је потребно решити квадратну
једначину која описује тетивни центар).

\paragraph{Симетрија.}
Још од античке Грчке, инверзија кругa је посматрана као аналогија
осној рефлексији. У проширеној комплексној равни не постоји суштинска
разлика између кругова и правих, тако да ћемо ми посматрати само једну
врсту релације и за две тачке ћемо рећи да су \emph{симетричне у
  односу на круг} ако се оне сликају једна у другу коришћењем било
рефлексије или инверзије у односу на произвољну праву или круг. Када
смо тражили алгебраску репрезентацију ове релације изненадили смо се
колико је била једноставна и елегантна -- тачке су симетричне акко је
билинеарна форма њиховог репрезентативног вектора и репрезентативне
матрице кругоправе једнака нули.
\selectlanguage{english}
{\tt
\begin{tabbing}
\hspace{5mm}\=\hspace{5mm}\=\hspace{5mm}\=\hspace{5mm}\=\hspace{5mm}\=\kill
\textbf{definition} circline\_symmetric\_rep \textbf{where}\\
\>"}circline\_sym\=metric\_rep $z_1$ $z_2$ $H$ $\longleftrightarrow$ \\
\>\>bilinear\_form $\Repnzv{z_1}$ $\Repnzv{z_2}$ $\Repcm{H}$ $= 0$"}\\
\textbf{lift\_definition} circline\_symmetric :: "}complex$_{hc}$ $\Rightarrow$ complex$_{hc}$ $\Rightarrow$ \\
\>circline $\Rightarrow$ bool"} \textbf{is} circline\_symmetric\_rep
\end{tabbing}
}
\selectlanguage{serbian}

Посматрајући скуп тачака на кругоправoj и поредећи наше две дефиниције,
постаје јасно да тачке на кругоправoj су управо оне које су
инваријантне у односу на симетрију у односу на ту кругоправу.
\selectlanguage{english}
{\tt
\begin{tabbing}
\hspace{5mm}\=\hspace{5mm}\=\hspace{5mm}\=\hspace{5mm}\=\hspace{5mm}\=\kill
\textbf{lemma} "on\_circline $H$ $z$ $\longleftrightarrow$ circline\_symmetric $H$ $z$ $z$"
\end{tabbing}
}
\selectlanguage{serbian}

\subsubsection{Дејство Мебијусових трансформација на кругоправе.}

Већ смо видели како Мебијусове трансформације делују на тачке $\extC$.
Оне такође делују и на кругоправе (и дефиниција је изабрана тако да су
два дејства компатибилна). Додатно, дајемо и дефиницију сличности
две матрице (која је дефинисана у нашој помоћној теорији линеарне
алгебре).

\selectlanguage{english}
{\tt
\begin{tabbing}
\hspace{5mm}\=\hspace{5mm}\=\hspace{5mm}\=\hspace{5mm}\=\hspace{5mm}\=\kill
\textbf{definition} "congruence $M$ $H$ = mat\_adj $M$ $*_{mm}$ $H$ $*_{mm}$ $M$"\\
\textbf{definition} mobius\_circline\_rep ::\\
\>"C2\_mat\_reg $\Rightarrow$ C2\_mat\_herm $\Rightarrow$ C2\_mat\_herm" \textbf{where}\\
\>"mobius\_circline\_rep $M$ $H$ = $\Abscm{\mathtt{congruence}\ (\mathtt{mat\_inv}\ \Reprm{M})\ \Repcm{H}}$"\\
\textbf{lift\_definition} mobius\_circline :: "mobius $\Rightarrow$ circline $\Rightarrow$ circline"\\
\> \textbf{is} mobius\_circline\_rep
\end{tabbing}
}
\selectlanguage{serbian}

\noindent Својства која има дејство Мебијусових трансформација на
кругоправе је врло слично као и код дејства Мебијусових трансформација
на тачке. На пример,

\selectlanguage{english}
{\tt
\begin{tabbing}
\hspace{5mm}\=\hspace{5mm}\=\hspace{5mm}\=\hspace{5mm}\=\hspace{5mm}\=\kill
\textbf{lemma} "}\=mobius\_circline (mobius\_comp $M_1$ $M_2$) = \\
\>mobius\_circline $M_1$ $\circ$ mobius\_circline $M_2$"}\\
\textbf{lemma} "mobius\_circline (mobius\_inv $M$) = inv (mobius\_circline $M$)"\\
\textbf{lemma} "mobius\_circline (mobius\_id) = id"\\
\textbf{lemma} "inj mobius\_circline"
\end{tabbing}
}
\selectlanguage{serbian}

Централна лема у овом одељку успоставља везу између дејства Мебијусових
трансформација на тачкама и на кругоправама (и што је основно,
доказује се да Мебијусове трансформације сликају кругоправе на
кругоправе).

\selectlanguage{english}
{\tt
\begin{tabbing}
\hspace{5mm}\=\hspace{5mm}\=\hspace{5mm}\=\hspace{5mm}\=\hspace{5mm}\=\kill
\textbf{lemma} "}\=mobius\_pt $M$ ` circline\_set $H$ = \\
\>circline\_set (mobius\_circline $M$ $H$)"}
\end{tabbing}
}
\selectlanguage{serbian}

\noindent Поред овога чува се и тип кругоправе (што повлачи, на
пример, да се реалне кругоправе сликају на реалне кругоправе).

\selectlanguage{english}
{\tt
\begin{tabbing}
\hspace{5mm}\=\hspace{5mm}\=\hspace{5mm}\=\hspace{5mm}\=\hspace{5mm}\=\kill
\textbf{lemma} "circline\_type (mobius\_circline $M$ $H$) = circline\_type $H$"
\end{tabbing}
}
\selectlanguage{serbian}

Још једно важно својство (које је нешто општије него претходно
наведено) је да је симетрија тачака очувана након дејства Мебијусових
трансформација (што се још назива и \emph{прицип симетрије}).

\selectlanguage{english}
{\tt
\begin{tabbing}
\hspace{5mm}\=\hspace{5mm}\=\hspace{5mm}\=\hspace{5mm}\=\hspace{5mm}\=\kill
\textbf{lemma} \=\textbf{assumes} "circline\_symmetric $z_1$ $z_2$ $H$"\\
\>\textbf{shows} "}circline\_symmetric \=(mobius\_pt $M$ $z_1$) (mobius\_pt $M$ $z_2$)\\
\>\>(mobius\_circline $M$ $H$)"}
\end{tabbing}
}
\selectlanguage{serbian}

Последње две леме су веома важни геометријски резултати, и захваљујући
веома погодној алгебарској репрезентацији, њих је било прилично лако
доказати у нашој формализацији. Оба доказа се заснивају на следећој
једноставној чињеници из линеарне алгебре.  
\selectlanguage{english}
{\tt
\begin{tabbing}
\hspace{5mm}\=\hspace{5mm}\=\hspace{5mm}\=\hspace{5mm}\=\hspace{5mm}\=\kill
\textbf{lem}\=\textbf{ma} "}mat\_det $M$ $\neq$ $0$ $\Longrightarrow$ bilinear\_form $z_1$ $z_2$ $H$ = \\
\>bilinear\_form ($M *_{mv} z_1$) ($M *_{mv} z_2$) (congruence (mat\_inv $M$) $H$)"}
\end{tabbing}
}
\selectlanguage{serbian}


\subsubsection{Јединственост кругоправе.}

У eуклидској геометрији добро је позната чињеница да постоји
јединствена права кроз две различите тачке и јединствени круг кроз три
неколинеарне различите тачке. Слични резултати важе и у $\extC$.
Ипак, да би се дошло до закључака потребно је извршити анализу
случајева према типу кругоправе.  Кругоправе озитивног типа не садрже
тачке па код њих не постоји јединственост. Кругоправе нула типа садрже
једну тачку и за сваку тачку постоји јединствена кругоправа нула типа
која је садржи. Постоји јединствена кругоправа кроз било које три
различите тачке (и она мора бити негативног типа).
 
\selectlanguage{english}
{\tt
\begin{tabbing}
\hspace{5mm}\=\hspace{5mm}\=\hspace{5mm}\=\hspace{5mm}\=\hspace{5mm}\=\kill
\textbf{lemma} "$\exists !$ $H$. circline\_type $H$ = 0 $\wedge$ $z$ $\in$ circline\_set $H$"\\
\textbf{lemma} "}$\lbrakk z_1 \neq z_2;\ z_1 \neq z_3;\ z_2 \neq z_3\rbrakk$ $\Longrightarrow$ \\
\>$\exists !$ $H$. \= $z_1 \in $ circline\_set $H$ $\wedge$ $z_2 \in$ circline\_set $H$ $\wedge$ \\
\>\>$z_3 \in$ circline\_set $H$"}
\end{tabbing}
}
\selectlanguage{serbian}

\noindent Веома изненађујуће, директно доказивање ових лема је било
веома тешко. Ипак, након примене "бгно" резоновања и након
пресликавања тачака у канонску позицију ($0_{hc}$, $1_{hc}$ и
$\infty_{hc}$) добили смо веома кратак и елегантан доказ (јер је
могуће доказати, коришћењем израчунавања, да је $x$-оса једина
кругоправа кроз ове три канонске тачке). Како су праве карактеризоване
као управо оне кругоправе које садрже $\infty_{hc}$, постаје јасно да
постоји јединствена права кроз било које две различите коначне тачке.

\paragraph{Скуп кардиналности кругоправе.}
Још једна од ствари која се узима "здраво за готово" је кардиналност
кругоправи различитог типа. Већ смо рекли да ови докази захтевају
"бгно" резоновање, али овог пута користили смо другачију врсту "бгно"
резоновања. Испоставља се да је у многим случајевима лакше резоновати
о круговима уколико је њихов центар у координатном почетку --- у том
случају, њихова матрица је дијагонална.  Ми смо формализовали
специјалан случај чувеног резултата из линеарне алгебре да је
$2\times2$ хермитска матрица слична са реалном дијагоналном матрицом
(штавише, елементи на дијагонали су реалне сопствене вредности
матрице, а подударност је успостављена коришћењем унитарних матрица
--- подударност се такође може успоставити коришћењем једноставније
матрице (матрице транслације), али онда она не би имала многа лепа
својства).

\selectlanguage{english}
{\tt
\begin{tabbing}
\hspace{5mm}\=\hspace{5mm}\=\hspace{5mm}\=\hspace{5mm}\=\hspace{5mm}\=\kill
\textbf{lemma} \textbf{assumes} "hermitean $H$"\\
\>\textbf{shows} "}$\exists\,k_1\,k_2\,M$. \=mat\_det $M$ $\neq 0$ $\wedge$ unitary $M$ $\wedge$\\
\>\>congruence $M$ $H$ = (cor $k_1$, 0, 0, cor $k_2$)"}
\end{tabbing}
}
\selectlanguage{serbian}

Последица је да за сваку кругоправу постоји унитарна Мебијусова
трансформација која слика кругоправу тако да је њен центар у
координатном почетку (заправо, постоје две такве трансформације ако су
сопствене вредности различите). Видећемо да унитарне трансформације
одговарају ротацијама Риманове сфере, тако да последња чињеница има
једноставно геометријско објашњење. Кругоправе се могу
дијагонализовати коришћењем само транслација, али унитарне
трансформације често имају лепша својства.

\selectlanguage{english}
{\tt
\begin{tabbing}
\hspace{5mm}\=\hspace{5mm}\=\hspace{5mm}\=\hspace{5mm}\=\hspace{5mm}\=\kill
\textbf{lemma} "}$\exists$ $M$ $H'$. \= unitary\_mobius $M$ $\wedge$ \\
\>mobius\_circline $M$ $H$ = $H'$ $\wedge$ circline\_diag $H'$"}\\
\textbf{lemma} \=\textbf{assumes} \="$\bigwedge$ $H'$. circline\_diag $H'$ $\Longrightarrow$ $P\ H$"\\
\>\> "$\bigwedge$ $M$ $H$. $P\ H$ $\Longrightarrow$ $P$ (mobius\_circline $M$ $H$)"\\
\>\textbf{shows}\>"$P$ $H$"
\end{tabbing}
} 
\selectlanguage{serbian}

\noindent Приметимо да је {\tt unitary\_mobius} предикат који подиже
{\tt unitary} својство са $\mathbb{C}^2$ матрица на тип {\tt
mobius}. Слично, {\tt circline\_diag} подиже услов
дијагоналне матрице на тип {\tt circline}.

Коришћењем овакве врсте "бгно" резоновања постаје прилично јасно како
доказати следећу карактеризацију за кардиналност скупа кругоправе.

\selectlanguage{english}
{\tt
\begin{tabbing}
\hspace{5mm}\=\hspace{5mm}\=\hspace{5mm}\=\hspace{5mm}\=\hspace{5mm}\=\kill
\textbf{lemma} "circline\_type $H$ > $0$ $\longleftrightarrow$ circline\_set $H$ = $\{\}$"\\
\textbf{lemma} "circline\_type $H$ = $0$ $\longleftrightarrow$ $\exists z$. circline\_set $H$ = $\{z\}$"\\
\textbf{lem}\=\textbf{ma} "}circline\_type $H$ < $0$ $\longleftrightarrow$\\
\> $\exists\,z_1\,z_2\,z_3$. $z_1 \neq z_2$ $\wedge$ $z_1 \neq z_3$ $\wedge$ $z_2 \neq z_3$ $\wedge$ circline\_set $H$ $\supseteq$ $\{z_1, z_2, z_3\}$"}\\
\end{tabbing}
}
\selectlanguage{serbian}

Важна, нетривијална, последица јединствености кругоправе и
кардиналности скупа кругоправе је да је функција {\tt circline\_set}
инјективна, тј. за сваки непразан скуп тачака кругоправе, постоји
јединствена класа пропорционалних матрица која их све одређује ({\tt
  circline\_set} је празан за све имагинарне кругоправе, што значи да
ово својство не важи када је скуп тачака кругоправе празан).
\selectlanguage{english}
{\tt
\begin{tabbing}
\hspace{5mm}\=\hspace{5mm}\=\hspace{5mm}\=\hspace{5mm}\=\hspace{5mm}\=\kill
\textbf{lemma} "}$\lbrakk$ circline\_set $H_1$ = circline\_set $H_2$; circline\_set $H_1$ $\neq \{\}$ $\rbrakk$ \\
\> $\Longrightarrow$ $H_1 = H_2$"}
\end{tabbing}
}
\selectlanguage{serbian}

\subsection{Оријентисане кругоправе}
\label{subsec:orientation}
У овом одељку ми ћемо описати како је могуће увести оријентацију за
кругоправе. Многи важни појмови зависе од оријентације.  Један од
најважнијих појмова је појам \emph{диска} --- унутрашњост
кругоправе. Слично као што је то био случај код скупа тачака
кругоправе, скуп тачака диска се уводи коришћењем квадратне форме у
чијем изразу се налази матрица кругоправе --- скуп тачака диска
кругоправе је скуп тачака за које важи $A*z*\mathtt{cnj}\,z +
B*\mathtt{cnj}\,z + C*z + D < 0$, при чему је
$\left(\begin{array}{cc}A & B\\C & D\end{array}\right)$ матрица
  која репрезентује кругоправу. Како скуп тачака диска мора бити
  инваријантан у односу на избор представника, јасно је да су матрице
  оријентисане кругоправе еквивалентне само ако су оне
  пропорционалне у односу на неки реални фактор (подсетимо се да код
  неоријентисаних кругоправих фактор може бити произвољан реалан
  ненула број).

\selectlanguage{english}
{\tt
\begin{tabbing}
\hspace{5mm}\=\hspace{5mm}\=\hspace{5mm}\=\hspace{5mm}\=\hspace{5mm}\=\kill
\textbf{definition} $\approxocm$ :: "C2\_mat\_herm $\Rightarrow$ C2\_mat\_herm $\Rightarrow$ bool" \textbf{where}\\
\>"$H_1$ $\approxocm$ $H_2$ $\longleftrightarrow$ ($\exists$ ($k$::real). $k$ $>$ 0 $\wedge$ $\Repcm{H_2}$ = cor $k$ *$_{sm}$ $\Repcm{H_1}$)"
\end{tabbing}
}
\selectlanguage{serbian}

Лако се доказује да је овa дефинисана релација релација еквиваленције,
тако да су кругоправе дефинисане преко количничке конструкције као
класе еквиваленције.

\selectlanguage{english}
{\tt
\begin{tabbing}
\hspace{5mm}\=\hspace{5mm}\=\hspace{5mm}\=\hspace{5mm}\=\hspace{5mm}\=\kill
\textbf{quotient\_type} o\_circline = C2\_mat\_herm / $\approxocm$
\end{tabbing}
}
\selectlanguage{serbian}

Сада можемо користити квадратну форму да дефинишемо унутрашњост,
спољашњост и границу оријентисане кругоправе.

\selectlanguage{english}
{\tt
\begin{tabbing}
\hspace{5mm}\=\hspace{5mm}\=\hspace{5mm}\=\hspace{5mm}\=\hspace{5mm}\=\kill
\textbf{definition} on\_o\_circline\_rep :: "C2\_mat\_herm $\Rightarrow$ C2\_vec$_{\neq 0}$ $\Rightarrow$ bool"\\
\>\textbf{where} "on\_o\_circline\_rep $H$ $z$ $\longleftrightarrow$ quad\_form $\Repcm{H}$ $\Repnzv{z}$ = 0"\\
\textbf{definition} in\_o\_circline\_rep :: "C2\_mat\_herm $\Rightarrow$ C2\_vec$_{\neq 0}$ $\Rightarrow$ bool"\\
\>\textbf{where} "in\_o\_circline\_rep $H$ $z$ $\longleftrightarrow$ quad\_form $\Repcm{H}$ $\Repnzv{z}$ < 0"\\
\textbf{definition} out\_o\_circline\_rep :: "C2\_mat\_herm $\Rightarrow$ C2\_vec$_{\neq 0}$ $\Rightarrow$ bool"\\
\>\textbf{where} "out\_o\_circline\_rep $H$ $z$ $\longleftrightarrow$ quad\_form $\Repcm{H}$ $\Repnzv{z}$ > 0"
\end{tabbing}
}
\selectlanguage{serbian}

\noindent Ове дефиниције се подижу на {\tt on\_o\_circline}, {\tt
  in\_o\_circline}, и {\tt out\_o\_circline} (при томе доказујемо
неопходне услове), и, коначно, уводе се следеће три дефиниције.

\selectlanguage{english}
{\tt
\begin{tabbing}
\hspace{5mm}\=\hspace{5mm}\=\hspace{5mm}\=\hspace{5mm}\=\hspace{5mm}\=\kill
\textbf{definition} o\_circline\_set :: "complex$_{hc}$ set" \textbf{where} \\
\>"o\_circline\_set $H$ = \{$z$. on\_o\_circline $H$ $z$\}"\\
\textbf{definition} disc :: "complex$_{hc}$ set" \textbf{where} \\
\>"disc $H$ = \{$z$. in\_o\_circline $H$ $z$\}"\\
\textbf{definition} disc\_compl :: "complex$_{hc}$ set" \textbf{where} \\
\>"disc\_compl $H$ = \{$z$. out\_o\_circline $H$ $z$\}"
\end{tabbing}
}
\selectlanguage{serbian}

Ова три скупа су међусобно дисјунктна и заједно испуњавају целу раван.
\selectlanguage{english}
{\tt
\begin{tabbing}
\hspace{5mm}\=\hspace{5mm}\=\hspace{5mm}\=\hspace{5mm}\=\hspace{5mm}\=\kill
\textbf{lemma} \="disc $H$ $\cap$ disc\_compl $H$ = $\{\}$" \\
\>"disc $H$ $\cap$ o\_circline\_set $H$ = $\{\}$"\\
\>"disc\_compl $H$ $\cap$ o\_circline\_set $H$ = $\{\}$"\\
\>"disc $H$ $\cup$ disc\_compl $H$ $\cup$ o\_circline\_set $H$ = UNIV"
\end{tabbing}
}
\selectlanguage{serbian}

За дату оријентисану кругоправу, може се тривијално одредити њен
неоријентисани део, а ове две кругоправе имају исти скуп тачака.
\selectlanguage{english}
{\tt
  \begin{tabbing}
    \hspace{5mm}\=\hspace{5mm}\=\hspace{5mm}\=\hspace{5mm}\=\hspace{5mm}\=\kill
\textbf{lift\_definition} of\_o\_circline ($\ofocircline{\_}$) :: "o\_circline $\Rightarrow$ circline" \textbf{is} id \\
\textbf{lemma} "circline\_set ($\ofocircline{H}$) = o\_circline\_set $H$"
  \end{tabbing}
}
\selectlanguage{serbian}

У \textbf{lift\_definition} увели смо краћи запис функције \mbox{{\tt
  of\_o\_circline}}, тако да, на пример, $\ofocircline{H}$ у леми је
скраћеница за {\tt of\_o\_circline\ $H$}.

За сваку кругоправу, постоји тачно једна супротно оријентисана
кругоправа.
\selectlanguage{english}
{\tt
  \begin{tabbing}
    \hspace{5mm}\=\hspace{5mm}\=\hspace{5mm}\=\hspace{5mm}\=\hspace{5mm}\=\kill
\textbf{definition} "opp\_o\_circline\_rep $H$ = $\Abscm{-1 *_{sm} \Repcm{H}}$"\\
\textbf{lift\_definition} opp\_o\_circline ($\oppocircline{\_}$) :: "o\_circline $\Rightarrow$ o\_circline" \textbf{is} \\
\>opp\_o\_circline\_rep
  \end{tabbing}
}
\selectlanguage{serbian}
\noindent Одређивање супротне кругоправе је идемпотентно јер супротне
кругоправе имају исти скуп тачака, али размењују диск и његов
комплемент.  
\selectlanguage{english}
{\tt
  \begin{tabbing}
    \hspace{5mm}\=\hspace{5mm}\=\hspace{5mm}\=\hspace{5mm}\=\hspace{5mm}\=\kill
\textbf{lemma} "$\oppocircline{(\oppocircline{H})} = H$"\\
\textbf{lemma} \="o\_circline\_set ($\oppocircline{H}$) = o\_circline\_set $H$"\\
\>"disc ($\oppocircline{H}$) = disc\_compl $H$" "disc\_compl ($\oppocircline{H}$) = disc $H$"
  \end{tabbing}
}
\selectlanguage{serbian}

Функције $\ofocircline{\_}$ и {\tt o\_circline\_set} су у одређеном
смислу инјективне.  
\selectlanguage{english}
{\tt
  \begin{tabbing}
    \hspace{5mm}\=\hspace{5mm}\=\hspace{5mm}\=\hspace{5mm}\=\hspace{5mm}\=\kill
\textbf{lem}\=\textbf{ma} "$\ofocircline{H_1} = \ofocircline{H_2}$ $\Longrightarrow$ $H_1$ = $H_2$ $\vee$ $H_1$ = $\oppocircline{H_2}$"\\
\textbf{lemma} \\
\> "}$\lbrakk$o\_circline\_set $H_1$ = o\_circline\_set $H_2$; o\_circline\_set $H_1$ $\neq$ $\{\}$$\rbrakk$ \\
\>\> $\Longrightarrow$ $H_1 = H_2$ $\vee$ $H_1 = \oppocircline{H_2}$"}
  \end{tabbing}
}
\selectlanguage{serbian}

Дата хермитска матрица кругоправе представља тачно једну од две могуће
оријентисане кругоправе. Избор шта ћемо звати позитивно оријентисана
кругоправа је произвољан. Ми смо одлучили да пратимо приступ који је
предложио Швердфегер \cite{schwerdtfeger}, где се користи водећи
коефицијент $A$ као први критеријум, који каже да се кругоправe са
матрицом у којој важи $A > 0$ зову позитивно оријентисане, а ако у
матрици важи $A < 0$ онда се зову негативно оријентисане.  Ипак,
Швердфегер није дискутовао још један могући случај када је $A = 0$
(случај правих), тако да смо ми морали да проширимо његову дефиницију
да би имали потпуну карактеризацију.

\selectlanguage{english}
{\tt
  \begin{tabbing}
    \hspace{5mm}\=\hspace{5mm}\=\hspace{5mm}\=\hspace{5mm}\=\hspace{5mm}\=\kill
\textbf{definition} pos\_o\_circline\_rep \textbf{where} "}pos\_o\_circline\_rep $H$ $\longleftrightarrow$\\
\>(\textbf{l}\=\textbf{et} ($A$, $B$, $C$, $D$) = $\Repcm{H}$\\ 
\>\>  \textbf{in} \=Re $A > 0$ $\vee$ \\
\>\>\>(Re $A = 0$ $\wedge$ (($B \neq 0$ $\wedge$ arg $B > 0$) $\vee$ ($B = 0$ $\wedge$ Re $D > 0$))))"}\\
\textbf{lift\_definition} pos\_o\_circline :: "o\_circline $\Rightarrow$ bool" \\
\> \textbf{is} pos\_o\_circline\_rep
  \end{tabbing}
}
\selectlanguage{serbian}

\noindent Сада, тачно једна од две супротно оријентисане кругоправе је
позитивно оријентисана.

\selectlanguage{english}
{\tt
  \begin{tabbing}
    \hspace{5mm}\=\hspace{5mm}\=\hspace{5mm}\=\hspace{5mm}\=\hspace{5mm}\=\kill
\textbf{lemma} \="pos\_o\_circline $H$ $\vee$ pos\_o\_circline ($\oppocircline{H}$)"\\
\>  "pos\_o\_circline ($\oppocircline{H}$) $\longleftrightarrow$ $\neg$ pos\_o\_circline $H$"
  \end{tabbing}
}
\selectlanguage{serbian}

Оријентација кругова је и алгебарски једноставна (посматра се знак
коефицијента $A$) и геометријски природна захваљујући следећој
једноставној карактеризацији.

\selectlanguage{english}
{\tt
  \begin{tabbing}
    \hspace{5mm}\=\hspace{5mm}\=\hspace{5mm}\=\hspace{5mm}\=\hspace{5mm}\=\kill
\textbf{lemma} "}\=$\infty_h$  $\notin$ o\_circline\_set $H$ $\Longrightarrow$ \\
\> pos\_o\_circline $H$ $\longleftrightarrow$ $\infty_h$ $\notin$ disc $H$"}
  \end{tabbing}
}
\selectlanguage{serbian}

\noindent Још једна лепа геометријска карактеризација за позитивну
оријентацију је да је eуклидски центар позитивно оријентисаних
eуклидових кругова садржан у њиховом диску.

\selectlanguage{english}
{\tt
  \begin{tabbing}
    \hspace{5mm}\=\hspace{5mm}\=\hspace{5mm}\=\hspace{5mm}\=\hspace{5mm}\=\kill
\textbf{lemma} \=\textbf{assumes} \="is\_circle ($\ofocircline{H}$)" "circline\_type ($\ofocircline{H}$) < 0"\\
\>\> "($a$, $r$) = euclidean\_circle ($\ofocircline{H}$)"\\
\>\textbf{shows} "pos\_oriented H $\longleftrightarrow$ of\_complex a $\in$ disc H" 
  \end{tabbing}
}
\selectlanguage{serbian}

\noindent Приметимо да је оријентација правих и тачака кругова вештачки
уведена (само да бисмо имали тотално дефинисану позитивну
оријентацију), и она нема природну геометријску интерпретацију. Због
овога, непрекидност оријентације је прекинута и ми мислимо да није
могуће увести оријентацију правих тако да функција оријентације буде
свуда непрекидна. Зато, када у неким наредним лемама будемо говорили о
оријентацији ми ћемо експлицитно искључити случај правих.

Tоталнa карактеризацијa за позитивну оријентацију нам омогућава да
створимо пресликавање из неоријентисаних у оријентисане кругоправе
(добијамо увек позитивно оријентисане кругоправе).

\selectlanguage{english}
{\tt
  \begin{tabbing}
    \hspace{5mm}\=\hspace{5mm}\=\hspace{5mm}\=\hspace{5mm}\=\hspace{5mm}\=\kill
\textbf{definition} of\_circline\_rep :: "C2\_mat\_herm $\Rightarrow$ C2\_mat\_herm" \textbf{where}\\
\> "}of\_circline\_rep $H$ = (\=\textbf{if} pos\_o\_circline\_rep $H$ \textbf{then} $H$\\
\>\>\textbf{else} opp\_o\_circline\_rep $H$)"}\\
\textbf{lift\_definition} of\_circline ($\ofcircline{\_}$) :: "circline $\Rightarrow$ o\_circline" \textbf{is} \\
\>of\_circline\_rep
  \end{tabbing}
}
\selectlanguage{serbian}
\noindent Доказана су бројна својства функције {\tt of\_circline}, а
овде ћемо навести само најзначајнија.

\selectlanguage{english}
{\tt
  \begin{tabbing}
    \hspace{5mm}\=\hspace{5mm}\=\hspace{5mm}\=\hspace{5mm}\=\hspace{5mm}\=\kill
\textbf{lemma} \="o\_circline\_set ($\ofcircline{H}$) = circline\_set $H$"\\
\textbf{lemma} \>"pos\_o\_circline ($\ofcircline{H}$)"\\
\textbf{lemma} \>"$\ofocircline{(\ofcircline{H})}$ = $H$" "pos\_o\_circline $H$ $\Longrightarrow$ $\ofcircline{(\ofocircline{H})}$ = $H$"\\
\textbf{lemma} "$\ofcircline{H_1}$ = $\ofcircline{H_2}$ $\Longrightarrow$ $H_1 = H_2$"
  \end{tabbing}
}
\selectlanguage{serbian}


\paragraph{Дејство Мебијусових трансформација на оријентисане кругоправе.} 
На репрезентативном нивоу дејство Мебијусових трансформација на
оријентисане кругоправе је исто као и дејство на неоријентисане
кругоправе.  
\selectlanguage{english}
{\tt
  \begin{tabbing}
    \hspace{5mm}\=\hspace{5mm}\=\hspace{5mm}\=\hspace{5mm}\=\hspace{5mm}\=\kill
\textbf{lift\_definition} mobius\_o\_circline :: \\
\>"mobius $\Rightarrow$ o\_circline $\Rightarrow$ o\_circline" \textbf{is} mobius\_circline\_rep
  \end{tabbing}
}
\selectlanguage{serbian}

\noindent Дејство Мебијуса на (неоријентисане) кругоправе се може
дефинисати коришћењем дефиниције за дејство Мебијуса на оријентисане
кругоправе, али обрнуто не би могло.  
\selectlanguage{english}
{\tt
  \begin{tabbing}
    \hspace{5mm}\=\hspace{5mm}\=\hspace{5mm}\=\hspace{5mm}\=\hspace{5mm}\=\kill
\textbf{lemma} "}\=mobius\_circline $M$ $H$ = $\ofocircline{(\mathtt{mobius\_o\_circline}\ M\ (\ofcircline{H}))}$"}\\
\textbf{lemma} "}\textbf{l}\=\textbf{et} \=$H_1$ = mobius\_o\_circline $M$ $H$; \\
\>\> $H_2$ = $\ofcircline{(\mathtt{mobius\_circline}\ M\ (\ofocircline{H}))}$ \\
\>\textbf{in}  $H_1$ = $H_2$ $\vee$ $H_1$ = $\oppocircline{H_2}$"}
  \end{tabbing}
}
\selectlanguage{serbian}

\noindent Дејство Мебијусових трансформација на оријентисане
кругоправе има слична својства као и дејство Мебијусових
трансформација на неоријентисане кругоправе. На пример, оне се слажу у
погледу инверза (\selectlanguage{english}{\tt \textbf{lemma}
  "mobius\_o\_circline (mobius\_inv $M$) = inv (mobius\_o\_circline
  $M$)"}\selectlanguage{serbian}), композиције, идентитета, обе су
инјективне ({\tt inj mobius\_circline}), и тако даље. Централне леме у
овом одељку повезују дејства Мебијусових трансформација на тачкама,
оријентисаним кругоправама и дисковима.

\selectlanguage{english}
{\tt
  \begin{tabbing}
    \hspace{5mm}\=\hspace{5mm}\=\hspace{5mm}\=\hspace{5mm}\=\hspace{5mm}\=\kill
\textbf{lemma} "}\=mobius\_pt $M$ ` o\_circline\_set H = \\
\>o\_circline\_set (mobius\_o\_circline $M$ $H$)"}\\
\textbf{lemma} "}\=mobius\_pt $M$ ` disc H = disc (mobius\_o\_circline $M$ $H$)"}\\
\textbf{lemma} "}\=mobius\_pt $M$ ` disc\_compl H = \\
\> disc\_compl (mobius\_o\_circline $M$ $H$)"}
  \end{tabbing}
}
\selectlanguage{serbian}

Све eуклидске сличности чувају оријентацију кругоправе.  
\selectlanguage{english}
{\tt
  \begin{tabbing}
    \hspace{5mm}\=\hspace{5mm}\=\hspace{5mm}\=\hspace{5mm}\=\hspace{5mm}\=\kill
\textbf{le}\=\textbf{mma}
  \textbf{assumes} \= "$a$ $\neq$ $0$" "$M$ = similarity $a$ $b$" \\
\>\> "$\infty_{hc}$ $\notin$ o\_circline\_set $H$"\\
\>  \textbf{sh}\=\textbf{ows} \\
\>\>"pos\_o\_circline $$H$$ $\longleftrightarrow$ pos\_o\_circline (mobius\_o\_circline $M$ $H$)"
  \end{tabbing}
}
\selectlanguage{serbian}

\noindent Оријентација слике дате оријентисане кругоправе $H$ након
дате Мебијусове трансформације $M$ зависи од тога да ли пол $M$ (тачка
коју трансформација $M$ слика у $\infty_{hc}$) лежи на диску или у
диску који је комплементаран $H$ (ако је у скупу $H$, онда се слика у
праву, а у том случају не дискутујемо оријентацију).

\selectlanguage{english}
{\tt
  \begin{tabbing}
  \hspace{3mm}\=\hspace{5mm}\=\hspace{5mm}\=\hspace{5mm}\=\hspace{5mm}\=\kill
\textbf{lemma}\\
\>"}$0_{hc}$ $\in$ disc\_compl $H$ $\Longrightarrow$ \\
\>\>pos\_o\_circline (mobius\_o\_circline reciprocation $H$)"}\\
\>"}$0_{hc}$ $\in$ disc $H$ $\Longrightarrow$ \\
\>\>$\neg$ pos\_o\_circline (mobius\_o\_circline reciprocation $H$)"}\\
\textbf{lemma}\\
\>\textbf{assumes} "$M$ = mk\_mobius a b c d" "c $\neq$ 0" "a*d - b*c $\neq$ 0"\\
\>\textbf{shows} \="}\=pole $M$ $\in$ disc $H$ $\longrightarrow$ \\
\>\>\>$\neg$ pos\_o\_circline (mobius\_o\_circline $M$ $H$)"}\\
\>\>"}pole $M$ $\in$ disc\_compl $H$ $\longrightarrow$\\
\>\>\>pos\_o\_circline (mobius\_o\_circline $M$ $H$)"}
  \end{tabbing}
}
\selectlanguage{serbian}

\noindent Приметимо да је ово другачије него што тврди Швердфегер
\cite{schwerdtfeger}: "Реципроцитет чува оријентацију круга који не
садржи 0, али инвертује оријентацију било ког круга који садржи 0 као
унутрашњу тачку. Свака Мебијусова трансформација чува оријентацију
било ког круга који не садржи свој пол. Ако круг садржи свој пол, онда
круг који се слика има супротну оријентацију". Наша формализација
доказује да оријентација резултујућег круга не зависи од оријентације
полазног круга (на пример, у случају реципроцитета, оријентација
полазног круга показује релативну позицију круга и тачке бесконачно
што је одређено знаком коефицијента $A$ у репрезентативној матрици и
то је сасвим независно од релативне позиције круга и нула тачке које
су одређене знаком коефицијента $D$ --- ова два коефицијента се
размењују приликом примене трансформације реципроцитета).


\subsubsection{Очување угла}

Мебијусове трансформације су конформно пресликавање, што значи да оне
чувају оријентисане углове међу оријентисаним кругоправама. Ако се
угао дефинише коришћењем чисто алгебарског приступа (пратећи
\cite{schwerdtfeger}), онда је врло лако доказати ово својство. Поред
дефиниције угла, навешћемо и дефиницију мешовите детерминанте коју смо
дефинисали раније у нашој основној теорији.  
\selectlanguage{english}
{\tt
  \begin{tabbing}
    \hspace{3mm}\=\hspace{5mm}\=\hspace{5mm}\=\hspace{5mm}\=\hspace{5mm}\=\kill
\textbf{fun} mat\_det\_mix :: "C2\_mat $\Rightarrow$ C2\_mat $\Rightarrow$ complex" \textbf{where}\\
\> "}mat\_det\_mix $(A_1, B_1, C_1, D_1)$ $(A_2, B_2, C_2, D_2)$ =\\
\>\> $A_1*D_2 - B_1*C_2 + A_2*D_1 - B_2*C_1$"}\\
\textbf{definition} cos\_angle\_rep \textbf{where}\\
\>  "}cos\_ang\=le\_rep $H_1$ $H_2$ = \\
\> \> - Re (mat\_det\_mix $\Repcm{H_1}$ $\Repcm{H_2}$) / \\
\> \> 2 * (sqrt (Re (mat\_det $\Repcm{H_1}$ * mat\_det $\Repcm{H_2}$))))"}\\
\textbf{li}\=\textbf{ft\_definition} cos\_angle :: "o\_circline $\Rightarrow$ o\_circline $\Rightarrow$ complex"\\
\>  \textbf{is} cos\_angle\_rep\\
\textbf{lemma} "}cos\_angle $H_1$ $H_2$ = \\
\> cos\_angle (moebius\_o\_circline $M$ $H_1$) (moebius\_o\_circline $M$ $H\_2$)"}
  \end{tabbing}
}
\selectlanguage{serbian}

Ипак, ова дефиниција није интуитивна, и из педагошких разлога желели
смо да је повежемо са нешто уобичајенијом дефиницијом. Прво,
дефинисали смо угао између два комплексна вектора ($\downharpoonright
\_ \downharpoonleft$ означава функцију за нормализацију угла која је
описана раније).  
\selectlanguage{english}
{\tt
  \begin{tabbing}
    \hspace{5mm}\=\hspace{5mm}\=\hspace{5mm}\=\hspace{5mm}\=\hspace{5mm}\=\kill
\textbf{definition} ang\_vec ("$\measuredangle$") \textbf{where} "$\measuredangle$ $z_1$ $z_2$ = $\downharpoonright$arg $z_2$ - arg $z_1$$\downharpoonleft$"    
  \end{tabbing}
}
\selectlanguage{serbian}

За дати центар $\mu$ обичног eуклидског круга и тачку $z$ на њему,
дефинишемо тангентни вектор у $z$ као радијус вектор
$\overrightarrow{\mu z}$, ротиран за $\pi/2$, у смеру казаљке на сату
или у супротном смеру у зависности од оријентације.  
\selectlanguage{english}
{\tt
  \begin{tabbing}
    \hspace{5mm}\=\hspace{5mm}\=\hspace{5mm}\=\hspace{5mm}\=\hspace{5mm}\=\kill
\textbf{definition} tang\_vec :: "complex $\Rightarrow$ complex $\Rightarrow$ bool $\Rightarrow$ complex" \textbf{where}\\
\>"tang\_vec $\mu$ $z$ $p$ = sgn\_bool $p$ * $i$ * ($z$ - $\mu$)"
  \end{tabbing}
}
\selectlanguage{serbian}
\noindent У логичкој променљивој $p$ енкодира се оријентација круга, а
функција \mbox{{\tt sgn\_bool $p$}} враћа $1$ када је $p$ тачно, а $-1$ када
је $p$ нетачно. Коначно, угао између два оријентисана круга у њиховој
заједничкој тачки $z$ се дефинише као угао између тангентних вектора у
$z$.

\selectlanguage{english}
{\tt
  \begin{tabbing}
    \hspace{5mm}\=\hspace{5mm}\=\hspace{5mm}\=\hspace{5mm}\=\hspace{5mm}\=\kill
\textbf{definition} ang\_circ \textbf{where}\\
\> "ang\_circ $z$ $\mu_1$ $\mu_2$ $p_1$ $p_2$ = $\measuredangle$ (tang\_vec $\mu_1$ $z$ $p_1$) (tang\_vec $\mu_2$ $z$ $p_2$)"
  \end{tabbing}
}
\selectlanguage{serbian}

\noindent Коначно, веза између алгебарске и геометријске дефиниције
косинуса угла дата је следећом лемом.

\selectlanguage{english}
{\tt
  \begin{tabbing}
    \hspace{5mm}\=\hspace{5mm}\=\hspace{5mm}\=\hspace{5mm}\=\hspace{5mm}\=\kill
\textbf{lemma} \textbf{assumes} "is\_circle ($\ofocircline{H_1}$)" "is\_circle ($\ofocircline{H_2}$)"\\
\>\>  "circline\_type ($\ofocircline{H_1}$) $< 0$" "circline\_type ($\ofocircline{H_2}$) $< 0$"\\
\>\>  "($\mu_1$, $r_1$) = euclidean\_circle ($\ofocircline{H_1}$)"\\
\>\>  "($\mu_2$, $r_2$) = euclidean\_circle ($\ofocircline{H_2}$)"\\
\>\>  "of\_complex $z$ $\in$ o\_circline\_set H1 $\cap$ o\_circline\_set H2"\\
\>\textbf{shows} "}cos\_angle $H_1$ $H_2$ = \\
\>\>cos (ang\_circ $z$ $\mu_1$ $\mu_2$ (pos\_o\_circline $H_1$) (pos\_o\_circline $H_2$))"}
  \end{tabbing}
}
\selectlanguage{serbian}
\noindent Да бисмо доказали ову лему било је неопходно доказати закон
косинуса у систему \emph{Isabelle/HOL}, али се ово показало као веома
једноставан задатак.


\section{Неке важне подгрупе Мебијусових трансформација}
\label{subsec:classification}

Већ смо описали параболичку групу (групу eуклидских сличности), кључну
за eуклидску геометрију равни. Сада ћемо описати карактеристике две
веома важне подгрупе Мебијусове групе --- групу сферних ротација,
важну за елиптичку планарну геометрију, и групу аутоморфизама диска
која је важна за хиперболичку планарну геометрију.

\paragraph{Ротације сфере.}
Генерална унитарна група, коју означавамо са $GU_2(\mathbb{C})$ је
група која садржи све Мебијусове трансформације које су репрезентоване
уопштеним унитарним матрицама.  
\selectlanguage{english}
{\tt
  \begin{tabbing}
    \hspace{3mm}\=\hspace{5mm}\=\hspace{5mm}\=\hspace{5mm}\=\hspace{5mm}\=\kill
\textbf{definition} unitary\_gen \textbf{where}\\
\>"}\=unitary\_gen $M$ $\longleftrightarrow$\\
\>\>$(\exists$ $k$::complex. $k \neq 0$ $\wedge$ mat\_adj $M *_{mm} M$ = $k$ $*_{sm}$ eye$)$"}
  \end{tabbing}
}
\selectlanguage{serbian}

\noindent Иако је у дефиницији дозвољено да $k$ буде комплексан
фактор, испоставља се да је једино могуће да $k$ буде
реалан. Генерализоване унитарне матрице могу бити растављене на обичне
унитарне матрице и јединичне матрице које су помножене неким
позитивним фактором.

\selectlanguage{english}
{\tt
  \begin{tabbing}
    \hspace{5mm}\=\hspace{5mm}\=\hspace{5mm}\=\hspace{5mm}\=\hspace{5mm}\=\kill
\textbf{definition} unitary \textbf{where} "unitary $M$ $\longleftrightarrow$ mat\_adj $M *_{mm} M$ = eye"\\
\textbf{lemma} "}unitary\_gen $M$ $\longleftrightarrow$ \\
\> $($$\exists\ k\ M'$. $k > 0$ $\wedge$ unitary $M'$ $\wedge$ $M$ = (cor $k$ $*_{sm}$ eye) $*_{mm}$ $M'$)"}
  \end{tabbing}
}
\selectlanguage{serbian}

Група унитарних матрица је веома важна јер описује све ротације
Риманове сфере (изоморфна је реалној специјалној ортогоналној групи
$SO_3(\mathbb{R})$). Једна од могућих карактеризација
$GU_2(\mathbb{C})$ у $\extC$ је да је то група трансформација таквих
да је имагинарни јединични круг фиксан (ово је круг чија је матрица
репрезентације јединична и налази се у равни у бесконачности).

\selectlanguage{english}
{\tt
  \begin{tabbing}
    \hspace{5mm}\=\hspace{5mm}\=\hspace{5mm}\=\hspace{5mm}\=\hspace{5mm}\=\kill
\textbf{lemma} "}mat\_det $(A, B, C, D)$ $\neq 0$ $\Longrightarrow$ unitary\_gen ($A$, $B$, $C$, $D$)  $\longleftrightarrow$\\
\>moebius\_circline (mk\_moebius $A$ $B$ $C$ $D$) imag\_unit\_circle = \\
\>imag\_unit\_circle"}
  \end{tabbing}
}
\selectlanguage{serbian}


Карактеризација генерализованих унитарних матрица у координатама је
дата следећом лемом. 
\selectlanguage{english}
{\tt
  \begin{tabbing}
    \hspace{5mm}\=\hspace{5mm}\=\hspace{5mm}\=\hspace{5mm}\=\hspace{5mm}\=\kill
\textbf{lemma} "}unitary\_gen M $\longleftrightarrow$ $($$\exists$ $a$ $b$ $k$.\ \textbf{let} $M' = (a,\,b,\,-\mathtt{cnj}\ b,\,\mathtt{cnj}\ a)$ \textbf{in} \\
\>$k \neq 0$ $\wedge$ mat\_det $M' \neq 0$ $\wedge$ $M = k *_{sm} M'$$)$"}
  \end{tabbing}
}
\selectlanguage{serbian}

Додатно, дефинисали смо специјалну унитарну групу $SU_2(\mathbb{C})$,
која садржи генерализоване унитарне матрице са детерминантом једнаком
један (оне се препознају по форми
$(a,\,b,\,-\mathtt{cnj}\ b,\,\mathtt{cnj}\ a)$), без множитеља $k$, и
ову специјалну групу користимо да бисмо извели координатну форму
генерализованих унитарних матрица.


\paragraph{Аутоморфизми диска.}
Дуална група претходној групи трансформација је група генерализованих
унитарних матрица чија сигнатура је $1-1$ ($GU_{1,1}(\mathbb{C})$).

\selectlanguage{english}
{\tt
  \begin{tabbing}
    \hspace{5mm}\=\hspace{5mm}\=\hspace{5mm}\=\hspace{5mm}\=\hspace{5mm}\=\kill
\textbf{definition} unitary11 \textbf{where}\\
\>"unitary11 $M$ $\longleftrightarrow$ mat\_adj $M *_{mm} (1, 0, 0, -1) *_{mm} M = (1, 0, 0, -1)$"\\
\textbf{definition} unitary11\_gen \textbf{where}\\
\>"}unitary11\_gen $M$ $\longleftrightarrow$ $(\exists$ $k$::complex. $k \neq 0$ $\wedge$\\
\>\>mat\_adj $M$ $*_{mm} (1, 0, 0, -1) *_{mm}$ $M$ = $k$ $*_{sm}$ $(1, 0, 0, -1)$$)$"}
  \end{tabbing}
}
\selectlanguage{serbian}
\noindent Поново, дефиниција дозвољава комплексан фактор $k$, 
али се показује да једино реални фактори имају смисла.

Карактеризација $GU_{1,1}(\mathbb{C})$ је да она садржи све Мебијусове
трансформације које фиксирају јединични круг.

\selectlanguage{english}
{\tt
  \begin{tabbing}
    \hspace{5mm}\=\hspace{5mm}\=\hspace{5mm}\=\hspace{5mm}\=\hspace{5mm}\=\kill
\textbf{lemma} "}mat\_det $(A, B, C, D)$ $\neq 0$ $\Longrightarrow$ unitary11\_gen ($A$, $B$, $C$, $D$)  $\longleftrightarrow$\\
\>moebius\_circline (mk\_moebius $A$ $B$ $C$ $D$) unit\_circle = unit\_circle"}
  \end{tabbing}
}
\selectlanguage{serbian}

Карактеризација генерализоване унитарне 1-1 матрице у координатама је
дата следећим лемама.  
\selectlanguage{english}
{\tt
  \begin{tabbing}
    \hspace{5mm}\=\hspace{5mm}\=\hspace{5mm}\=\hspace{5mm}\=\hspace{5mm}\=\kill
\textbf{lemma} "}unitary11\_gen $M$ $\longleftrightarrow$ $($$\exists$ $a$ $b$ $k$. \textbf{let} $M' = (a,\,b,\,\mathtt{cnj}\ b,\,\mathtt{cnj}\ a)$ \textbf{in} \\
\> $k \neq 0$ $\wedge$ mat\_det $M' \neq 0$ $\wedge$ \\
\> $(M = k *_{sm} M'$ $\vee$ $M = k *_{sm} (\mathtt{cis}\ pi,\,0,\,0,\,1) *_{sm} M'$$))$"}\\
\textbf{lemma} "}unitary11\_gen M $\longleftrightarrow$ $($$\exists$ $a$ $b$ $k$. \textbf{let} $M' = (a,\,b,\,\mathtt{cnj}\ b,\,\mathtt{cnj}\ a)$ \textbf{in} \\
\> $k \neq 0$ $\wedge$ mat\_det $M' \neq 0$ $\wedge$ $M = k *_{sm} M'$ $)$"}
  \end{tabbing}
}
\selectlanguage{serbian}
\noindent Приметимо да је прва лема садржана у другој леми. Ипак, било
је лакше доказати прву лему јер добијамо матрице следећег облика $k
*_{sm} (a,\,b,\,-{\tt cnj}\ b,\, -{\tt cnj}\ a)$ --- геометријски,
друга група трансформација комбинује прву групу са додатном централном
симетријом.

Још једна важна карактеризација ових трансформација је коришћењем
такозваног Блашке фактора. Свака трансформација је композиција Блашке
фактора (рефлексије која неку тачку која је на јединичној кружници
слика у нула) и ротације.

\selectlanguage{english}
{\tt
  \begin{tabbing}
    \hspace{5mm}\=\hspace{5mm}\=\hspace{5mm}\=\hspace{5mm}\=\hspace{5mm}\=\kill
\textbf{lemma} \=\textbf{assumes} \="$k \neq 0$" "$M' = (a,\,b,\,\mathtt{cnj}\ b,\,\mathtt{cnj}\ a)$"\\
\>\>"$M = k *_{sm} M'$" "mat\_det $M' \neq 0$" "$a \neq 0$"\\
\hspace{5mm}\=\kill
\>\textbf{shows} "}\=$\exists$ $k'$ $\phi$ $a'$. $k' \neq 0$ $\wedge$ $a' * \mathtt{cnj}\ a' \neq 1$ $\wedge$\\
\>\> $M = k' *_{sm} (\mathtt{cis}\ \phi,\,0,\,0,\,1) *_{mm} (1,\,-a',\,-\mathtt{cnj}\ a',\,1)$"}
  \end{tabbing}
}
\selectlanguage{serbian}
\noindent Изузетак је у случају када је $a=0$ и онда се уместо Блашке
фактора, користи реципроцитет (бесконачно замењује $a'$ у претходној
леми).  
\selectlanguage{english}
{\tt
  \begin{tabbing}
    \hspace{5mm}\=\hspace{5mm}\=\hspace{5mm}\=\hspace{5mm}\=\hspace{5mm}\=\kill
\textbf{lemma} \=\textbf{assumes} \="$k \neq 0$" "$M' = (0,\,b,\,\mathtt{cnj}\ b,\,0)$" "$b \neq 0$" "$M = k *_{sm} M'$" \\
\>\textbf{shows} "$\exists$ $k'$ $\phi$. $k' \neq 0$ $\wedge$ $M = k' *_{sm} (\mathtt{cis}\ \phi,\,0,\,0,\,1) *_{mm} (0,\,1,\,1,\,0)$"
  \end{tabbing}
}
\selectlanguage{serbian}

Матрице $GU_{1,1}(\mathbb{C})$ се природно деле у две подгрупе.  Све
трансформације фиксирају јединични круг, али прва подгрупа се састоји
од трансформација које мапирају јединични диск у самог себе (такозвани
\emph{аутоморфизми диска}), док се друга подгрупа састоји из
трансформација које размењују јединични диск и његов комплемент. За
дату матрицу, њена подгрупа се једино може одредити посматрајући знак
детерминанте $M' = (a,\,b,\,\mathtt{cnj}\ b,\,\mathtt{cnj}\ a)$. Ако
је само $M = (a_1, b_1, c_1, d_1)$ дато, а нису дати $M'$, а ни $k$,
онда је критеријум за утврђивање подгрупе вредност $\mathtt{sgn}
(\mathtt{Re}\ ((a_1*d_1)/(b_1*c_1)) - 1)$.

Приметимо да су све важне подгрупе овде описане једино у терминима
алгебре. Формализовали смо и неке геометријске доказе који дају
еквивалентну карактеризацију тврђењима које смо већ описали. Додатно,
важи да су сви аналитички аутоморфизми диска једнаки композицији
Блашке фактора и ротација (ипак, доказ се заснива на математичкој
анализи, принципу максималног модула и Шварцовој леми -- техникама
које ми нисмо узимали у обзир). Чак и слабије тврђење да су сви
Мебијусови аутоморфизми диска ове форме није још формално доказано
(кључни корак је доказати да аутоморфизми диска фиксирају јединични
круг, а то је нешто што нисмо могли доказати без детаљног испитивања
топологије на чему тренутно радимо).

\subsection{Сличне Мебијусове трансформације и класификација Мебијусових трансформација}

Да бисмо могли да класификујемо Мебијусове трансформације прво је било
потребно увести пар нових појмова и анализирати њихова својства. Пре
свега, анализирали смо фиксне тачке Мебијусових трансформација. Раније
смо спомињали да су еуклидске сличности једине Мебијусове
трансформације којима је $\infty_{hc}$ фиксна тачка. Ипак, за доказе
потребне у овом одељку морали смо да нешто више анализирамо фиксне
тачке. Увели смо дефиницију фиксне тачке и дефиницију фиксне тачке
која је коначна.

\selectlanguage{english}
{\tt 
  \begin{tabbing}
    \hspace{5mm}\=\hspace{5mm}\=\hspace{5mm}\=\hspace{5mm}\=\hspace{5mm}\=\kill
\textbf{definition} moebius\_fixed\_points \textbf{where} \\
  \> "moebius\_fixed\_points $M$ $\gamma$ $\longleftrightarrow$ moebius\_pt $M$ $\gamma$ = $\gamma$" \\
\textbf{definition} moebius\_fixed\_points\_finite\_rep \textbf{where} \\
\>"}\=moebius\_fixed\_points\_finite\_rep $M$ $\gamma$ $\longleftrightarrow$\\
\>\>(\textbf{l}\=\textbf{et} $(a, b, c, d) = \Reprm{M}$ \\
\>\>\>\textbf{in} $c*\gamma*\gamma - (a - d)*\gamma - b = 0$)"} \\
\textbf{lift\_definition} moebius\_fixed\_points\_finite :: \\
\> "}moe\=bius $\Rightarrow$ complex $\Rightarrow$ bool"} \textbf{is}\\
\>\> moebius\_fixed\_points\_finite\_rep
\end{tabbing}
}
\selectlanguage{serbian}

За Мебијусову трансформацију могу постојати највише две фиксне тачке
које могу бити и једнаке. Оне су обе коначне ако за коефицијент
репрезентативне матрице важи $c \neq 0$; једна од њих је коначна, а
једна бесконачна ако за коефицијенте важи $c= 0$ и $a \neq d$ и оне су
обе једнаке бесконачно ако за коефицијенте важи $c = 0$ и $a = d$.
Ово тврђење смо доказали у наредним лемама. 


\selectlanguage{english}
{\tt 
  \begin{tabbing}
    \hspace{5mm}\=\hspace{5mm}\=\hspace{5mm}\=\hspace{5mm}\=\hspace{5mm}\=\kill
\textbf{lemma} \\
\>\textbf{assumes} "mat\_det $(a, b, c, d) \neq 0$" "$c \neq 0$" \\
\>\textbf{shows} "}$\exists \gamma_1 \gamma_2.$ \= moebius\_fixed\_points (mk\_moebius $a$ $b$ $c$ $d$) $\gamma_1$ $\land$ \\
\>                                             \>  moebius\_fixed\_points (mk\_moebius $a$ $b$ $c$ $d$) $\gamma_2$ $\land$ \\ 
\>                                             \> $\gamma_1 \neq \infty_{hc}$ \ \ $\land$ \ \ $\gamma_2 \neq \infty_{hc}$"} \\
\textbf{lemma} \\
\> \textbf{assumes} "mat\_det $(a, b, c, d) \neq 0$" "$c = 0$" "$a \neq d$" \\
\> \textbf{shows} \="moebius\_fixed\_points (mk\_moebius $a$ $b$ $c$ $d$) $\infty_{hc}$" \\
\>                \>"$\exists \gamma.$ moebius\_fixed\_points (mk\_moebius $a$ $b$ $c$ $d$) $\gamma$ $\land$ $\gamma \neq \infty_{hc}$" \\
\textbf{lemma} \\
\>\textbf{assumes} "mat\_det $(a, b, c, d) \neq 0$" "$c = 0$" "$a = d$" \\
\>\textbf{shows} "moebius\_fixed\_points (mk\_moebius $a$ $b$ $c$ $d$) $\infty_{hc}$"
\end{tabbing}
}
\selectlanguage{serbian}

\noindent Ова тврђења није било тешко доказати, али су имала бројне
кораке и случајеве које је требало размотрити и захтевала су
доказивање доста ситних алгебарских корака.

Потом дефинишемо како се за две дате Мебијусове трансформације може
одредити слична Мебијусова трансформација. Овде уједно дајемо и
дефиницију сличних матрица коју смо увели и чија својства смо доказали
у нашој основној теорији линеарне алгебре.

\selectlanguage{english}
{\tt 
  \begin{tabbing}
    \hspace{5mm}\=\hspace{5mm}\=\hspace{5mm}\=\hspace{5mm}\=\hspace{5mm}\=\kill
\textbf{definition} similarity\_matrices \textbf{where} \\
\> "similarity\_matrices $I$ $M$ = $I *_{mm} M  *_{mm}$ mat\_inv $I$" \\
\textbf{definition} moebius\_mb\_rep \textbf{where } \\
\> "moebius\_mb\_rep $I$ $M$ =  $\lceil$ similarity\_matrices $\Reprm{I}$ $\Reprm{M}$ $\rceil^{M}$" \\
\textbf{lift\_definition} moebius\_mb :: "moebius $\Rightarrow$ moebius $\Rightarrow$ moebius" \textbf{is}\\
\>  moebius\_mb\_rep
\end{tabbing}
}
\selectlanguage{serbian}

Сада је могуће дефинисати релацију сличности између две Мебијусове
трансформације. Додатно, доказали смо и да је ово и релација
еквиваленције, тј. доказано је да за релацију важи својство
рефлексивности, симетричности и транзитивности и докази су били
прилично директни и кратки.

\selectlanguage{english}
{\tt 
  \begin{tabbing}
    \hspace{5mm}\=\hspace{5mm}\=\hspace{5mm}\=\hspace{5mm}\=\hspace{5mm}\=\kill
\textbf{definition} similar \textbf{where} \\
\> "similar $M_1$ $M_2$ $\longleftrightarrow$ ($\exists I.$ moebius\_mb $I$ $M_1$ = $M_2$)"\\
\textbf{lemma} "similar $M$ $M$" \\
\textbf{lemma} \textbf{assumes} "similar $M_1$ $M_2$" \\
\>  \textbf{shows} "similar $M_2$ $M_1$" \\
\textbf{lemma} \textbf{assumes} "similar $M_1$ $M_2$" "similar $M_2$ $M_3$" \\
\>  \textbf{shows} "similar $M_1$ $M_3$"
\end{tabbing}
}
\selectlanguage{serbian}

Врло важно тврђење је да је свака Мебијусова трансформација слична
некој интегралној трансформацији (eуклидској сличности). Доказивање
овог тврђења се свело на одређивање параметара eуклидксе сличности,
$a$ и $b$, за произвољну Мебијусову трансформацију. Да би одредили ове
параметре било је потребно одредити фиксне тачке Мебијусове
трансформације и користити својства за фиксне тачке које смо нешто
раније доказали.

\selectlanguage{english}
{\tt 
  \begin{tabbing}
    \hspace{5mm}\=\hspace{5mm}\=\hspace{5mm}\=\hspace{5mm}\=\hspace{5mm}\=\kill
\textbf{lemma} \\
\>  "$\exists$ $k$ $t.$ $k \neq 0$ $\land$ similar $M$ (similarity $a$ $b$)"
\end{tabbing}
}
\selectlanguage{serbian}

Веома важан параметар за Мебијусове трансформације јесте
\emph{инваријанта Мебијусових трансформација}. Она се дефинише
коришћењем репрезентативне матрице за дату Мебијусову трансформацију,
као однос између трага и детерминанте матрице. Потом, дефиницију са
репрезентативног нивоа подижемо на ниво количничког типа.

\selectlanguage{english}
{\tt 
  \begin{tabbing}
    \hspace{5mm}\=\hspace{5mm}\=\hspace{5mm}\=\hspace{5mm}\=\hspace{5mm}\=\kill
\textbf{definition} similarity\_invar\_rep \textbf{where} \\
\>  "}\=similarity\_invar\_rep $M$ = \\
\>\>   (\textbf{l}\=\textbf{et} $M$ = $\Reprm{M}$  \textbf{in} $\frac{(\mathtt{mat\_trace\ } M)^2}{\mathtt{mat\_det\ } M} - 4)$"} \\
\textbf{lift\_definition} similarity\_invar :: "moebius $\Rightarrow$ complex" \textbf{is}\\
\>  similarity\_invar\_rep
\end{tabbing}
}
\selectlanguage{serbian}

Број $4$ који се добија у дефиницији произилази из чињенице да суS
матрице идентичне трансформације сличне, тј. $E = I^{-1}EI$, где је $E
= \Reprm{\mathtt{mobius\_id}}$, a $I = \Reprm{I}$. Како је
$\mathtt{mat\_trace\ }E = 2$, a $\mathtt{mat\_det\ } E = 1$, њихов
количник је $4$.

Важно својство овог параметра је да су Мебијусове трансформације (које
нису идентитет) сличне акко имају једнаке инваријанте.

\selectlanguage{english}
{\tt 
  \begin{tabbing}
    \hspace{5mm}\=\hspace{5mm}\=\hspace{5mm}\=\hspace{5mm}\=\hspace{5mm}\=\kill
\textbf{lemma} \textbf{assumes} "$M_1$ $\neq$ id\_moebius" "$M_2$ $\neq$ id\_moebius" \\
\>  \textbf{s}\=\textbf{hows}  \\
\> "similarity\_invar $M_1$ = similarity\_invar $M_2$ $\longleftrightarrow$ similar $M_1$ $M_2$"
\end{tabbing}
}
\selectlanguage{serbian}

\noindent Доказивање у једном смеру ("ако су сличне имају једнаке
инваријанте") било је једноставно и кратко. Међутим, супротан смер
("ако имају једнаке инваријанте, онда су сличне") је представљао
изазов, било је потребно раздвојити случајеве када је инваријанта
једнака $0$ и када је различита од $0$, а потом у доказу су коришћене
алгебарске трансформације, као и бројна својства релације сличности
матрица. Оно што је интересантно је да је доказ у Швердфегеру значајно
краћи (само пар редова), мада се мора рећи да је аутор нотирао све
важне тачке доказа, али је у формалном доказу било неопходно ући у
дубљу анализу и сваку од ових тачака детаљније испитати.

Коначно, стижемо до класификације Мебијусових трансформација која се
управо карактерише коришћењем инваријанте. За Мебијусову
трансформацију кажемо да је то пресликавање које је:
\begin{tabbing}
\emph{параболичко,}~~~~~~~~~~~~~~~~~~~~~~ \= {\tt similarity\_invar} $= 0$, \\
\> има само једну фиксну тачку \\
\emph{елиптичко,}  \> инваријанта је реална и \\
\> $-4 \le $ {\tt similarity\_invar} $< 0$ \\
\emph{правилно хиперболичко,}  \> инваријанта је реална и \\
\> {\tt similarity\_invar} $ > 0$ \\
\emph{неправилно хиперболичко,}  \> инваријанта је реална и \\
\> {\tt similarity\_invar} $\le -4$ \\
\emph{локсодромичко,} \> инваријанта није реална
\end{tabbing}

\section{Дискусија}
\label{sec:discuss}
Визуелно, геометријски аргументи се често користе у доказима у
уџбеницима. Као пример, ми ћемо демонстрирати доказ о очувању својства
угла након примене Мебијусових трансформација на који се често може
наићи у различитим књигама о овој теми (у овом поглављу ми ћемо
пратити Нидамов приступ \cite{needham} који има за циљ да представи
област без формалних детаља, па самим тим књига није стриктно формално
писана али, ипак, овакав начин резоновања присутан је и код многих
других аутора).

Прво важно питање је појам угла. Углови могу бити дефинисани између
оријентисаних или неоријентисаних кривих, а и сами углови могу бити
оријентисани или неоријентисани. Нидам дефинише угао између две криве
на следећи начин: "Нека су $S_1$ и $S_2$ криве које се секу у тачки
$z$. Као што је илустровано, ми можемо повући њихове тангенте $T_1$ и
$T_2$ у тачки $z$. Угао између кривих $S_1$ и $S_2$ у њиховој
заједничкој тачки $z$ је оштар угао $\alpha$ од $T_1$ до $T_2$. Значи
овај угао $\alpha$ има знак који му је додељен: угао између $S_2$ и
$S_1$ је минус илустровани угао између $S_1$ и $S_2$."  То значи да је
угао дефинисан само између неоријентисаних кривих (и то је различито у
односу на нашу дефиницију), али сам угао је оријентисан (а то је исто
као и у нашој финалној дефиницији).  У раној фази наше формализације
ми смо дефинисали и користили неоријентисани конвексан и оштар угао
између два вектора.

\selectlanguage{english}
{\tt
  \begin{tabbing}
    \hspace{5mm}\=\hspace{5mm}\=\hspace{5mm}\=\hspace{5mm}\=\hspace{5mm}\=\kill
\textbf{definition} "$\measuredangle_c$" \textbf{where} "$\measuredangle_c$ $z_1$ $z_2$ $\equiv$ abs ($\measuredangle$ $z_1$ $z_2$)"\\
\textbf{definition} acutize \textbf{where} "acutize $\alpha$ = (\textbf{if} $\alpha$ $>$ $\frac{\pi}{2}$ \textbf{then} $\pi$ - $\alpha$ \textbf{else} $\alpha$)"\\
\textbf{definition} "$\measuredangle_a$" \textbf{where} "$\measuredangle_a$ $z_1$ $z_2$ $\equiv$ acutize ($\measuredangle_c$ $z_1$ $z_2$)"
  \end{tabbing}
}
\selectlanguage{serbian}

Како су наше кругоправе оријентисане од старта, ми смо доказали да на
оштар угао између два круга не утиче оријентација и да се он може
изразити у терминима три тачке (тачке пресека и тачака које
представљају центре кругова).

\selectlanguage{english}
{\tt
  \begin{tabbing}
    \hspace{5mm}\=\hspace{5mm}\=\hspace{5mm}\=\hspace{5mm}\=\hspace{5mm}\=\kill
\textbf{lemma} "}\=$\lbrakk z \neq \mu_1$;$z \neq \mu_2\rbrakk$ $\Longrightarrow$\\
\> ang\_circ\_a $z$ $\mu_1$ $\mu_2$ $p_1$ $p_2$ = $\measuredangle_a\ (z - \mu_1)\ (z - \mu_2)$"}
\end{tabbing}
}
\selectlanguage{serbian}
\noindent Функција {\tt ang\_circ\_a} је дефинисана као оштар угао
између два тангентна вектора (слично функцији {\tt ang\_circ} у нашој
коначној формализацији).

Доказ да Мебијусова трансформација чува угао који стоји у уџбенику
\cite{needham} се ослања на чињеницу да се свака Мебијусова
трансформација може раставити на транслацију, ротацију, хомотетију и
инверзију. Чињеница да транслације, ротације и дилетације чувају угао
је узета као подразумевана и није доказивана (и да будемо искрени
формализација ове чињенице није била тешка када смо успели да све
појмове формално дефинишемо на одговарајући начин). Централни изазов
је доказати да инверзија чува углове, тј. доказати тврђење "Инверзија
је антикомфорно пресликавање".  Доказ се заснива на "чињеници да за
било коју дату тачку $z$ која није на кругу инверзије $K$, постоји
тачно један круг који је ортогоналан на $K$ и пролази кроз $z$ у било
ком правцу". Даље, доказ се наставља са "Претпоставимо да се две криве
$S_1$ и $S_2$ секу у $z$, и да су њихове тангенте $T_1$ и $T_2$, а
угао између њих је $\alpha$. Да бисмо сазнали шта се дешава са углом
након инверзије у односу на $K$, заменимо $S_1$ и $S_2$ јединственим
круговима $R_1$ и $R_2$ ортогоналним на $K$ који пролазе кроз $z$ у
истом смеру као што је и смер $S_1$ и $S_2$, тј., круговима чије
тангенте у $z$ су $T_1$ и $T_2$. Како инверзија у односну на $K$ слика
сваки од ових кругова на саме себе нови угао у $\tilde{z}$ је
$-\alpha$. Крај."

У нашем ранијем покушају ми смо формализовали овај "доказ", али је ово
захтевало веома велику количину уложеног труда у поређењу са углађеним
алгебарским доказом у нашој финалној формализацији.  Прво, уџбеник је
често врло непрецизан у томе да ли се користи "комплексна инверзија"
или "геометријска инверзија" (тј. према нашим терминима које смо
раније увели -- да ли се користи реципроцитет или инверзија). У доказу
из уџбеника аутор користи инверзију у односу на произвољан круг $K$,
али је довољно посматрати само реципроцитет (који је увек дат у односу
на јединични круг). Формализација резоновања које је дато у уџбенику
је већ дала прилично велике формуле, и било би још компликованије и
монотоније (ако је уопште и могуће) завршити доказ коришћењем
инверзије у односу на произвољни круг. На пример, једноставан
реципроцитет круга са центром $\mu$ и радијусом $r$ даје круг са
центром $\tilde{\mu} = \mu / \mathtt{cor}\ (|\mu|^2 - r^2)$, и
радијусом $\tilde{r} = r / ||\mu|^2 - r^2|$, и ова веза би била још
комплекснија за произвољну Мебијусову трансформацију, ако би била
записана у координатама, без коришћења појма матрица као што смо ми
радили у нашој главној формализацији.

Формални запис тврђења о очувању угла је следећи.

\selectlanguage{english}
{\tt
\begin{tabbing}
\hspace{5mm}\=\hspace{5mm}\=\hspace{5mm}\=\hspace{5mm}\=\hspace{5mm}\=\kill
\textbf{lemma} \\
 \> \textbf{assumes} \= "$z$ $\in$ circle $\mu_1$ $r_1$" "$z$ $\in$ circle $\mu_2$ $r_2$"\\
  \>               \> "inv ` circle $\mu_1$ $r_1$ = circle $\tilde{\mu_1}$ $\tilde{r_1}$"   \\
  \>               \> "inv ` circle $\mu_2$ $r_2$ = circle $\tilde{\mu_2}$ $\tilde{r_2}$"\\
  \> \textbf{shows} "ang\_circ\_a $z$ $\mu_1$ $\mu_2$ = ang\_circ\_a $\tilde{z}$ $\tilde{\mu_1}$ $\tilde{\mu_2}$"
\end{tabbing}
}
\selectlanguage{serbian}

Поред тога што недостаје дискусија за бројне специјалне случајеве, у
неформалном доказу недостаје и један значајан део.  Наиме, лако је
доказати да је $\tilde{z}$ пресек $R_1$ и $R_2$ (то је пресек
$\tilde{S_1}$ и $\tilde{S_2}$, које су слике $S_1$ и $S_2$ након
инверзије), али доказати да $R_1$ и $\tilde{S_1}$ и да $R_2$ и
$\tilde{S_2}$ имају исту тангенту у $\tilde{z}$ је захтевало не тако
тривијална израчунавања (тај доказ се заснива на чињеници да су центар
$\mu_i'$ круга $R_i$, центар $\tilde{\mu_i}$ круга $\tilde{S_i}$, и
$\tilde{z}$ колинеарни).

Једноставан аргумент симетрије који каже да су углови између два круга
у њиховим двема различитим тачкама пресека једнаки поново није било
једноставно формализовати.
\selectlanguage{english}
{\tt
  \begin{tabbing}
    \hspace{5mm}\=\hspace{5mm}\=\hspace{5mm}\=\hspace{5mm}\=\hspace{5mm}\=\kill
\textbf{lemma} \textbf{assumes} \="$\mu_1$ $\neq$ $\mu_2$" "$r_1$ $>$ 0" "$r_2$ $>$ 0"\\
\>"$\{z_1, z_2\}$ $\subseteq$ circle $\mu_1$ $r_1$ $\cap$ circle $\mu_2$ $r_2$" "$z_1$ $\neq$ $z_2$" \\
    \hspace{5mm}\=\hspace{5mm}\=\hspace{5mm}\=\hspace{5mm}\=\hspace{5mm}\=\kill
  \> \textbf{shows} "ang\_circ\_a $z_1$ $\mu_1$ $\mu_2$ = ang\_circ\_a $z_2$ $\mu_1$ $\mu_2$"
  \end{tabbing}
}
\selectlanguage{serbian}
\noindent Ми смо доказали ову лему тек након примене "бгно" резоновања
и померањем слике тако да центри два круга који се посматрају буду на
$x$-оси.

У доказу смо идентификовали бројне дегенерисане случајеве који су
морали да се анализирају одвојено. Прво смо морали да докажемо да
кругови који се секу могу имати исти центар (тј. да
$\mu_1$$=$$\mu_2$) само ако су једнаки и тада је оштар угао између
њих једнак $0$. Са друге стране, ако су оба центра колинеарна са
пресечном тачком $z$ (тј. ако важи {\tt collinear $\mu_1$ $\mu_2$
  $z$}), два круга се додирују (било споља или изнутра), и опет је
оштар угао једнак $0$.

Постојање круга $R_i$ који је ортогоналан на јединичну кружницу и који
има исту тангенту у датој тачки $z$ као и дати круг са центром $\mu_i$
је дато следећом лемом (заправо у леми се даје центар $\mu_i'$ тог
новог круга).  
\selectlanguage{english}
{\tt
  \begin{tabbing}
    \hspace{5mm}\=\hspace{5mm}\=\hspace{5mm}\=\hspace{5mm}\=\hspace{5mm}\=\kill
\textbf{lemma} \\
\> \textbf{assumes} \="$\langle$$\mu_i$ - $z$, $z$$\rangle$ $\neq 0$"\\
\>\> "$\mu_i'$ = $z$ + (1 - $z$*$\mathtt{cnj}\ z$) * ($\mu_i$ - $z$) / $(2 * \langle$$\mu_i$ - $z$, $z$$\rangle$)"\\
  \> \textbf{shows} "collinear $z$ $\mu_i$ $\mu_i'$" "$z$ $\in$ ortho\_unit\_circ $\mu'_i$" \\
  \end{tabbing}
}
\selectlanguage{serbian}
\noindent Аналитички израз је открио још неке дегенерисане
случајеве. Бројилац може бити нула једино ако се кругови секу на
јединичној кружници (тј. када је $z*\mathtt{cnj}\ z = 1$). У том
случају, доказ из уџбеника се не може применити јер је $\mu_1' =
\mu_2' = z$, и кругови $R_1$ и $R_2$ се не могу конструисати (они су
празни кругови). Случај када је именилац једнак нули (било за $\mu_1'$
или $\mu_2'$) је такође дегенерисан. Ово се дешава када су вектори
$\mu_i - z$ и $z$ ортогонални. Геометријски, у том случају се круг
$R_i$ дегенерише у праву (што и није проблем у проширеној комплексној
равни, али јесте проблем у поставци која важи у оригиналном доказу
која се налази у обичној комлексној равни). Зато, овај специјалан
случај мора да се анализира одвојено. Тако је наша формална анализа
брзо показала да једноставно тврђење у Нидамовом уџбенику
\cite{needham} да "за дату било коју тачку $z$ која није на кругу
инверзије $K$, постоји тачно један круг који је ортогоналан на $K$ и
пролази кроз $z$ у било ком задатом правцу" није тачно у многим
случајевима.


\section{Закључци и даљи рад у формализацији геометрије комплексне равни}
\label{sec:concl}
У овом раду смо показали неке елементе наше формализације геометрије
проширене комплексне равни $\extC$ коришћењем комплексне пројективне
равни, али и Риманове сфере. Формализовали смо аритметичке операције у
$\extC$, размеру и дворазмеру, тетивну метрику у $\extC$, групу
Мебијусових трансформација и њихово дејство на $\extC$, неке њене
специјалне подгрупе (eуклидске сличности, ротације сфере, аутоморфизме
диска), кругоправе и њихову везу са круговима и правама, тетивном
метриком, Римановом сфером, јединственост кругоправи, дејство
Мебијусових трансформација на кругоправе, типове и кардиналност скупа
кругоправе, оријентисане кругоправе, однос између Мебијусових
трансформација и оријентације, својство очувања угла након дејства
Мебијусових трансформација итд. Наша тренутна теорија има око 12,000
линија \emph{Isabelle/HOL} кода (сви докази су структурни и записани
су у језику за доказе \emph{Isabelle/Isar} и наши ранији покушаји су
замењени краћим алгебарским доказима и нису укључени у финалну
формализацију), око 125 дефиниција и око 800 лема.

Кључан корак у нашој формализацији је била одлука да се користи
алгебарска репрезентација свих важних објеката (вектора хомогених
координата, матрица за Мебијусове трансформације, хермитске матрице за
кругоправе итд.). Иако ово није нов приступ (на пример, Швердфегерова
класична књига \cite{schwerdtfeger} прати овај приступ прилично
конзистентно), он ипак није тако уобичајен у литератури (и у
материјалима курсева који се могу наћи на интернету). Уместо њега
преовладао је геометријски приступ.  Ми смо покушали да пратимо такву
врсту геометријског резоновања у раној фази нашег рада на овој теми,
али смо наишли на бројне потешкоће и нисмо имали много успеха. На
основу овог искуства, закључујемо да увођење моћне технике линеарне
алгебре омогућава значајно лакши рад на формализацији него што је то
случај када се користи геометријско резоновање.

Може се дискутовати да ли у неким случајевима геометријски аргументи
дају боље објашњење неких теорема, али када се посматра само
доказивање тврђења, алгебраски приступ је јасно супериорнији. Ипак, да
би имали везу са стандардним приступом у коме се користи геометријска
интуиција увели смо неколико додатних дефиниција (које су више
геометријске или више алгебарске) и морали смо доказати да су ове
дефиниције еквивалентне. На пример, када је дефиниција угла дата само
коришћењем алгебарских операција на матрицама и њиховим
детерминантама, својство очувања угла је било веома лако доказати, али
због образовне сврхе ово постаје значајно једино када се та дефиниција
споји са стандардном дефиницијом угла између кривих (тј. њихових
тангентних вектора) --- у супротном, формализација постаје игра са
симболима који немају никакво значење.

Још један важан закључак до ког смо дошли је да у формалним
документима треба што чешће избегавати анализу случајева и екстензије
које омогућавају резоновање без анализе случајева треба што чешће
користити (нпр. било је много боље користити хомогене координате
уместо једне одвојене тачке бесконачно коју би морали засебно да
анализирамо у сваком тврђењу или дефиницији; слично, било је много
лакше радити са кругоправама него разликовати случај прави и кругова,
итд.). Увођење два модела истог концепта (на пример, у нашем случају,
хомогених координата и Риманове сфере) такође помаже, јер су неки
докази лакши у једном моделу, а неки у другом.

У принципу наши докази нису дугачки (15-20 линија у просеку). Ипак,
понекад је било потребно изводити веома досадне закључке, поготову
када се пребацивало са реалних на комплексне бројеве и обратно
(коришћењем функција за конверзију {\tt Re} и {\tt cor}). Ове
конверзије се углавном и не појављују у неформалном тексту и добро би
дошла нека аутоматизација оваквог закључивања. Аутоматизација система
\emph{Isabelle} је прилично моћна у резоновању једнакости у којима су
обични комплексни бројеви и ту смо често користили метод {\tt (simp
  add: field\_simps)} (са неким мањим изузецима), али када су у питању
неједнакости, аутоматизација није била добра и много тога смо морали
да доказујемо ручно, корак по корак, а оваква тврђења се често
сматрају веома тривијалним у неформалном тексту.

У нашем даљем раду планирамо да користимо ове резултате у
формализацији неeуклидских геометрија и њихових модула (посебно,
сферични модел елиптичке геометрије, Поeнкареов диск модел и модел
горње полуравни хиперболичке геометрије).



\section{Формализација Поeнкареовог диск модела}
\label{complex--poincare}

Циљ је доказати да Поeнкареов диск модел представља модел свих аксиома
Тарског са изузетком Еуклидове аксиоме која у овом моделу није
тачнa. Потребно је дефинисати основне појмове, тј. релацију
\emph{између} и растојање и показати да дефинисани појмови
задовољавају нека својства. Нажалост, због разлога које ћемо касније
изложити, нисмо успели да формално докажемо све аксиоме. Ипак,
изложићемо нека интересантна својства и закључке до којих смо дошли.

Прво, дефинишемо тип података којим се представљају тачке Поeнкареовог
диск модела, односно тачке које припадају унутрашњости јединичног
диска.

\selectlanguage{english}
{\tt
\textbf{typedef} unit\_disc = "\{$z$::complex\_homo. in\_ocircline ounit\_circle $z$\}"
}
\selectlanguage{serbian}

Специјална тачка која припада унутрашњости је и $0$. Иако већ постоји
дефинисана $0_h$ било је потребно дефинисани и $0_u$, односно нулу
која припада јединичном диску. Иако тривијално важи да $0_h$ припада
јединичном диску, приликом дефинисања појмова или задавања лема ово
није познато и систем може пријављивати грешку јер тип није
одговарајући. Управо зато, потребно је дати још једну дефиницију за
$0$.

\selectlanguage{english}
{\tt
\textbf{lift\_definition} zero\_homo\_unit :: unit\_disc ("$0_u$") \textbf{is} zero\_homo
}
\selectlanguage{serbian}


\subsubsection{Растојање}

Растојање над тачкама јединичног диска се дефинише исто као и
растојање над тачкама проширене комплексне равни.

\selectlanguage{english}
{\tt
\begin{tabbing}
  \textbf{lif}\=\textbf{t\_definition} dist\_poincare :: "unit\_disc $\Rightarrow$ unit\_disc $\Rightarrow$ real" \textbf{is} \\
  \> dist\_homo
\end{tabbing}
}
\selectlanguage{serbian}


\subsubsection{Релација \emph{између}}

Дефиниција релације \emph{између} се ослања на већ дефинисани појам
\mbox{{\tt cross\_ratio}} за који су већ доказана бројна
својства. Релација \emph{између} се прво дефинише над проширеном
комплексном равни, а потом се подиже на тип \mbox{{\tt unit\_disc}}.

\selectlanguage{english}
{\tt
\begin{tabbing}
\textbf{def}\=\textbf{inition} between \textbf{where} \\
\> "}bet\=ween $z_1$ $z_2$ $z_3$ $\longleftrightarrow$ (($z_1$ = $z_2$ $\land$ $z_2$ = $z_3$) $\lor$ \\
\> \> (\= \textbf{let} CR = to\_complex(cross\_ratio $z_1$ $z_2$ $z_3$ (inversion\_homo $z_2$)) \\
       \>\>\> \textbf{in} \\
       \>\>\> is\_real CR $\land$ Re CR $\le$ $0$))"} \\

\textbf{lift\_definition} between\_poincare :: \\
       \> "unit\_disc $\Rightarrow$ unit\_disc $\Rightarrow$ unit\_disc $\Rightarrow$ bool" \textbf{is} between
\end{tabbing}
}
\selectlanguage{serbian}

Као што се може видети у дефиницији разликујемо два случаја. Код
Тарског, за тачке које су једнаке такође важи релација \emph{између},
односно у моделу Тарског је допуштено да тачке буду једнаке. Како {\tt
  cross\_ratio} није дефинисан када су три тачке једнаке, то случај
једнаких тачака одвајамо посебно.

Aко све четири тачке припадају једној кругоправoj, онда ће двострука
размера за те четири тачке бити реална. Неформално, тачке за које важи
релација \emph{између} припадају једној кругоправoj и то не било
каквој кругоправoj, већ оној која је нормална на јединичну
кружницу. Кругоправе нормалне на јединичну кружницу могу бити кругови
нормални на јединичну кружницу или праве које пролазе кроз координатни
почетак. Додатно, инверзија у односу на јединични круг било које од
ове три тачке ће такође припадати кругоправoj нормалној на јединичну
кружницу. Зато, у двоструку размеру уврстимо три тачке и инверзију
једне од њих и за ове четири тачке двострука размера мора бити реална.

Додатно, ако је дворазмера негативна онда је друга тачка између прве и
треће, а у супротном није. Ако је дворазмера нула онда су две од три
тачке једнаке и тада исто важи релација \emph{између}.

Наравно, да би потврдили да је овако дефинисана релација заиста
релација \emph{између} потребно је доказати аксиоме Тарског за овај
модел.


\subsubsection{Мебијусове трансформације и релација \emph{између}}

Не посматрају се све Мебијусове трансформације већ само оне које сликају
унутрашњост диска у унутрашњост диска. Раније смо видели да је
$GU_{1,1}(\mathbb{C})$ група оних Мебијусових трансформација које
фиксирају јединични круг, али да и ту постоје две групе
трансформација, тј. оне трансформације које сликају унутрашњост круга
у унутрашњост (и које су у овом контексту значајне) и друга група
трансформација које размењују унутрашњост и спољашњост диска.

Дефинисаћемо својство којим се описују оне Мебијусове трансформације
које сликају унутрашњост диска у унутрашњост диска. И ова дефиниција
се одвија у два корака, прво се дефинише над матрицама, а потом се
подигне на тип {\tt moebius}.

\selectlanguage{english}
{\tt
\begin{tabbing}
\textbf{def}\=\textbf{inition} Unitary11\_gen\_direct\_rep \textbf{where} \\ 
  \> "}Unitary11\_gen\_direct\_rep $M$ $\longleftrightarrow$ \\
  \>(\textbf{let} ($A$, $B$, $C$, $D$) = $\Reprm{M}$ \\
  \> \textbf{in} unitary11\_gen ($A$, $B$, $C$, $D$) $\land$ ($B = 0$ $\lor$ Re (($A$*$D$)/($B$*$C$)) $>$ 1))"} \\ \\
\textbf{lift\_definition} Unitary11\_gen\_direct :: "moebius $\Rightarrow$ bool" \textbf{is} \\
\> Unitary11\_gen\_direct\_rep
\end{tabbing}
}
\selectlanguage{serbian}

Може се доказати следеће тврђење

\selectlanguage{english}
{\tt
\begin{tabbing}
\textbf{lem}\=\textbf{ma} \\
\> "}moebius\_ocircline $M$ ounit\_circle = ounit\_circle $\longleftrightarrow$ \\
\> Unitary11\_gen\_direct $M$"}
\end{tabbing}
}

\noindent односно, унутрашњост диска је очувана ако и само ако
Мебијусова тра\-нсфо\-рма\-ци\-ја задовољава својство {\tt
  Unitary11\_gen\_direct}.

Сада се може дефинисати тип, тј. нова група Мебијусових трансформација
које чувају унутрашњост диска

\selectlanguage{english}
{\tt
\textbf{typedef}
 moebius\_unitary = "\{$M$::moebius. Unitary11\_gen\_direct $M$\}" 
}
\selectlanguage{serbian}

Слично као и раније, Мебијусове трансформације ове групе су дате као
дејство над тачкама јединичног диска. Ипак, за ову дефиницију се може
искористити дефиниција Мебијусових трансформација над тачкама
проширене комплексне равни, тј. потребно је подићи ту дефиницију за
тип \mbox{{\tt mobius\_unitary}}  и {\tt unit\_disc}.

\selectlanguage{english}
{\tt
\begin{tabbing}
  \textbf{lif}\=\textbf{t\_definition} moebius\_pt\_poincare :: \\
  \> "moebius\_unitary $\Rightarrow$ unit\_disc $\Rightarrow$ unit\_disc" \textbf{is} moebius\_pt
\end{tabbing}
}
\selectlanguage{serbian}

Ова дефиниција ствара обавезу да се докаже

\selectlanguage{english}
{\tt
\begin{tabbing}
 $\forall$ $M$ \= $z$. \\
       \> $\lbrakk$ Unitary11\_gen\_direct $M$; in\_ocircline ounit\_circle $z$ $\rbrakk$ \\
       \> $\Longrightarrow$ in\_ocircline ounit\_circle (moebius\_pt $M$ $z$)
\end{tabbing}
}
\selectlanguage{serbian}

\noindent  што се лако доказује у неколико корака коришћењем горе дате леме.

Поред ових дефиниција интересантно је дефинисати и инверзну трансформацију.

\selectlanguage{english}
{\tt
\begin{tabbing}
  \textbf{lif}\=\textbf{t\_definition} moebius\_inv\_poincare:: \\
  \> "moebius\_unitary $\Rightarrow$ moebius\_unitary" \textbf{is} moebius\_inv
\end{tabbing}
}
\selectlanguage{serbian}

И ова дефиниција ствара обавезу да се докаже да је инверзна
трансформација такође {\tt Unitary11\_gen\_direct} што се доказује
коришћењем једноставних алгебарских трансформација над матрицама.


Веома важно тврђење које смо доказали је да Мебијусове трансформације
које чувају унутрашњост диска, такође чувају и релацију
\emph{између}. Ово тврђење се веома лако доказује јер смо раније већ
доказали да Мебијусове трансформације чувају дворазмеру. Ипак, било је
потребно доказати и да је очувана инверзија у односу на јединични
круг. У општем случају инверзија није очувана, али за Мебијусове
трансформације које чувају унутрашњост јединичног диска, јесте.

\selectlanguage{english}
{\tt
\begin{tabbing}
\textbf{lem}\=\textbf{ma} \\
   \> \textbf{assumes} "Unitary11\_gen\_direct $M$" \\
   \> \textbf{shows} "}\=moebius\_pt $M$ (inversion\_homo $z$) = \\
   \> \> inversion\_homo (moebius\_pt $M$ $z_2$)"}
\end{tabbing}
}
\selectlanguage{serbian}

\selectlanguage{english}
{\tt
\begin{tabbing}
\textbf{lem}\=\textbf{ma} \\
  \> \textbf{assumes} \= "$z_1'$ = moebius\_pt\_poincare $M$ $z_1$" \\
  \> \> "$z_2'$ = moebius\_pt\_poincare $M$ $z_2$" \\
  \> \> "$z_3'$ = moebius\_pt\_poincare $M$ $z_3$" \\
  \> \> "between\_poincare $z_1$ $z_2$ $z_3$" \\
  \> \textbf{shows} "between\_poincare $z_1'$ $z_2'$ $z_3'$"
\end{tabbing}
}
\selectlanguage{serbian}


Поред ових доказано је још пуно помоћних тврђења, а издвојићемо
неколико интересантнијих. Једна од често коришћених чињеница у
доказима је да инверзна слика тачке јединичног диска не припада
јединичном диску.

\selectlanguage{english}
{\tt
\begin{tabbing}
\textbf{lem}\=\textbf{ma} \\
  \> \textbf{assumes} "$x$ $\in$ unit\_disc" \\
  \> \textbf{shows} "inversion\_homo $x$ $\notin$ unit\_disc"
\end{tabbing}
}
\selectlanguage{serbian}

Такође, тачке које се налазе у јединичном диску су по модулу мање од
$1$.

\selectlanguage{english}
{\tt
\begin{tabbing}
\textbf{lem}\=\textbf{ma} \\
  \> \textbf{assumes} "in\_ocircline ounit\_circle $z$" \\
  \> \textbf{shows} "cmod (to\_complex $z$) $<$ $1$"
\end{tabbing}
}
\selectlanguage{serbian}


Важно тврђење је да свака Мебијусова трансформација која је
композиција ротације и Блашке фактора чији параметар је по модулу мањи
од $1$ је трансформација која слика унутрашњост диска у унутрашњост
диска.

\selectlanguage{english}
{\tt
\begin{tabbing}
\textbf{lem}\=\textbf{ma} \\
  \> \textbf{assumes} \= "cmod a < 1" "a * cnj a $\neq$ $1$" "$a$ $\neq$ $0$" \\
          \> \> "$M$ = rotation\_moebius $\phi$ + blaschke $a$" \\
  \> \textbf{shows} "Unitary11\_gen\_direct $M$"
\end{tabbing}
}
\selectlanguage{serbian}


Прво се докаже да се било које две тачке могу сликати у $0_u$ и у неку
тачку на реалној оси, што је тврђење дато у следећој леми.
\label{blaske_cuvajedinicni}

\selectlanguage{english}
{\tt
\begin{tabbing}
\textbf{lem}\=\textbf{ma} \\
  \> "}$\exists$ $a$ $M$. \= $0_u$ = moebius\_pt\_poincare $M$ $z_1$ $\land$ \\ 
                \> \> is\_real (to\_complex (Rep\_unit\_disc $a$)) $\land$ \\ 
                \> \> $a$ = moebius\_pt\_poincare $M$ $z_2$"}
\end{tabbing}
}
\selectlanguage{serbian}

Доказ овог тврђења се састоји из два корака. Прво се прва тачка слика
у $0_u$ трансформацијом {\tt $M'$ = blaschke (to\_complex $z_1$)}, а
потом се врши ротација за угао који одговара тачки која се добила
пресликавањем тачке $z_2$ трансформацијом $M'$. Како је ротација око
координатног почетка, то се $0_u$ слика у $0_u$, а друга тачка се
ротацијом слика на реалну осу и тиме се управо и добијају жељене слике
тачака. У доказу се користи раније доказана чињеница да је композиција
ротације и Блашке фактора чији је параметар имао модуо мањи од $1$
Мебијусова трансформација која чува унутрашњост диска. Иако је идеја
доказа једноставна, постоји неколико случајева које треба размотрити
(ако су тачке једнаке или ако је већ нека тачка једнака $0_h$), а и у
доказу постоји много ситних корака које је требало доказати. Зато је
доказ готово 300 линија дугачак.

Потом се може доказати да ако за три тачке важи релација
\emph{између}, онда се оне могу сликати на реалну осу (а једна од њих
у $0$). У овом доказу се користи претходно тврђење и чињеница да ако
је дворазмера реална, ако су њена три параметра реална, онда и четврти
параметар мора бити реалан.

\selectlanguage{english}
{\tt
\begin{tabbing}
\textbf{lem}\=\textbf{ma} \\
  \> \textbf{assumes} "between\_poincare $z_1$ $z_2$ $z_3$" \\ 
  \> \textbf{shows} \= "}$\exists$ $a$ $b$ $M$. $0_u$ = moebius\_pt\_poincare $M$ $z_1$ $\land$ \\ 
                \> \> is\_real (to\_complex (Rep\_unit\_disc $a$)) $\land$ \\
                \> \> $a$ = moebius\_pt\_poincare $M$ $z_2$ $\land$ \\ 
                \> \> is\_real (to\_complex (Rep\_unit\_disc $b$)) $\land$ \\ 
                \> \> $b$ = moebius\_pt\_poincare $M$ $z_3$"}

\end{tabbing}
}
\selectlanguage{serbian}


Коришћењем свих претходних тврђења може се користити резоновање "без
губитка на општости". Наиме, тврђење се може доказати на реалној оси
на којој је лакше доказати да неко тврђење важи, а онда се може
уопштити да важи за било које три тачке за које важи релација
\emph{између} у Поeнкареовом диск моделу.

\subsubsection{Аксиоме подударности}

Аксиоме подударности се тривијално доказују јер су својства растојања
већ раније доказана и само је потребно позвати се на та својства.

\selectlanguage{english}
{\tt
\begin{tabbing}
\textbf{lem}\=\textbf{ma} ax$_1$: \\ 
  \> "dist\_poincare $z_1$ $z_2$ = dist\_poincare $z_2$ $z_1$" \\ 

\textbf{lemma} ax$_2$: \\
  \> \textbf{assumes} "dist\_poincare $x$ $y$ = dist\_poincare $z$ $z$" \\
  \> \textbf{shows} "$x$ = $y$" \\

\textbf{lemma} ax$_3$: \\
  \> \textbf{assumes} \= "dist\_poincare $x$ $y$ = dist\_poincare $z$ $u$" \\
                   \> \> "dist\_poincare $x$ $y$ = dist\_poincare $v$ $w$" \\
  \> \textbf{shows} "dist\_poincare $z$ $u$ = dist\_poincare $v$ $w$"
\end{tabbing}
}
\selectlanguage{serbian}


\subsubsection{Аксиоме релације \emph{између}}

Две аксиоме ове групе се тривијално доказују. Аксиома идентитета се
доказује контрадикцијом. Аксиома горње димензије тврди да постоје три
тачке за које не важи релација \emph{између}. Доказује се тако што се
одаберу такве три тачке, на пример, $0_u$, $1/2$ и {\tt ii}$/2$,
докаже се да све три тачке заиста припадају Поeнкареовом диск моделу,
али да дворазмера није реална, па самим тим и релација \emph{између}
не важи.

\selectlanguage{english}
{\tt
\begin{tabbing}
\textbf{lem}\=\textbf{ma} ax$_4$: \\
  \> \textbf{assumes} "between\_poincare $x$ $y$ $x$" \\
  \> \textbf{shows} "$x$ = $y$" \\

\textbf{lemma} ax$_6$: \\
\> "}$\exists$ $a$ $b$ $c$. $\neg$ between\_poincare $a$ $b$ $c$ $\land$ $\neg$ between\_poincare $b$ $c$ $a$ \\
\> $\land$ $\neg$ between\_poincare $c$ $a$ $b$"}
\end{tabbing}
}
\selectlanguage{serbian}

Овој групи аксиома припада аксиома непрекидности.

\selectlanguage{english}
{\tt
\begin{tabbing}
\textbf{lem}\=\textbf{ma} ax$_7$: \\ 
  \> \textbf{assumes} "$\exists$ $a$. $\forall$ $x$. $\forall$ $y$. $\phi$ $x$ $\land$ $\psi$ $y$ $\longrightarrow$ between\_poincare a $x$ $y$" \\
  \> \textbf{shows} "$\exists$ $b$. $\forall$ $x$. $\forall$ $y$. $\phi$ $x$ $\land$ $\psi$ $y$ $\longrightarrow$ between\_poincare $x$ $b$ $y$"
\end{tabbing}
}
\selectlanguage{serbian}

Да би доказали ово тврђење било је потребно доказати пуно помоћних
лема. Доказ се започиње једноставним испитивањем случајева у којима
тврђење тривијално важи. Први случај је да не постоје ни $x$, ни $y$
такви да важи $\phi$ $x$ и $\psi$ $y$. Следећи случај је да не постоји
$x$ такво да важи $\phi$ $x$. Трећи случај је да не постоји $y$ такво
да важи $\psi$ $y$. И последњи тривијалан случај је да постоји $b$
такво да важи $\phi$ $b$ и $\psi$ $b$.

Доказивање општег случаја се састоји из неколико корака. Прво се све
тачке сликају на реалну осу на којој је тврђење лакше
доказати. Оправданост овог пресликавања смо видели раније. Потом је
потребно доказати једно тврђење за релацију \emph{између} које важи на
реалној оси. То тврђење тврди да тачке које су у релацији
\emph{између} задовољавају неки поредак.

\selectlanguage{english}
{\tt
\begin{tabbing}
\textbf{lem}\=\textbf{ma} \\ 
  \> \textbf{assumes} \= "is\_real (to\_complex (Rep\_unit\_disc $z$))" \\
          \> \> "is\_real (to\_complex (Rep\_unit\_disc $u$))" \\
          \> \> "is\_real (to\_complex (Rep\_unit\_disc $v$))" \\
          \> \> "$rz$ = Re (to\_complex (Rep\_unit\_disc $z$))" \\
          \> \> "$ru$ = Re (to\_complex (Rep\_unit\_disc $u$))" \\
          \> \> "$rv$ = Re (to\_complex (Rep\_unit\_disc $v$))" \\
  \> \textbf{shows} "}between\_poincare $z$ $u$ $v$ $\longleftrightarrow$ \\
          \> \> ($rz$ $\le$ $ru$ $\land$ $ru$ $\le$ $rv$) $\lor$ \\
          \> \> ($rz$ $\ge$ $ru$ $\land$ $ru$ $\ge$ $rv$)"}
\end{tabbing}
}
\selectlanguage{serbian}

Потом доказујемо аксиому непрекидности за реалне бројеве:

\selectlanguage{english}
{\tt
\begin{tabbing}
\textbf{lem}\=\textbf{ma} \\
  \> \textbf{assumes} \= "$\forall$ $x$::real. $\forall$ $y$::real. $\phi$ $x$ $\land$ $\psi$ $y$ $\longrightarrow$ $x$ $<$ $y$" \\
          \> \> "$\exists$ $x$. $\phi$ $x$" "$\exists$ $y$. $\psi$ $y$" \\
  \> \textbf{shows} "$\exists$ $b$. $\forall$ $x$. $\forall$ $y$. $\phi$ $x$ $\land$ $\psi$ $y$ $\longrightarrow$ ($x$ $\le$ $b$ $\land$ $b$ $\le$ $y$)"
\end{tabbing}
}
\selectlanguage{serbian}

Oво тврђење се једноставно доказује коришћењем својства
супремума. Наиме, посматра се скуп \selectlanguage{english}{\tt $P$ =
  "\{$x$::real. $\phi$ $x$\}"}\selectlanguage{serbian}. Докаже се да
супремум овог скупа управо испуњава тражено тврђење.

Комбинујући последње две леме, доказује се и тврђење аксиоме. 

Преостале аксиоме Тарског нисмо успели да докажемо. Узмимо Пашову
аксиому у разматрање.

\selectlanguage{english}
{\tt
\begin{tabbing}
\textbf{lem}\=\textbf{ma} ax$_5$: \\
    \> \textbf{assumes} \= "between\_poincare $x$ $u$ $z$" \\
    \> \>        "between\_poincare $y$ $v$ $z$" \\
    \> \textbf{shows} "$\exists$ $a$. between\_poincare $u$ $a$ $y$ $\land$ between\_poincare $v$ $a$ $x$"
\end{tabbing}
}
\selectlanguage{serbian}


\paragraph{Проблем пресека кругоправих}
%napisati ceo racun -- kako sa Mebijusom slikamo u (0, 0)

Да би доказали ово тврђење потребно је одредити $x$ које испуњава
тражена својства. То значи да треба одредити $x$ као пресек две
кругоправе нормалне на јединичну кружницу. Прва кругоправа садржи
тачке $u$ и $y$, а друга кругоправа садржи тачке $v$ и $x$. То значи
да треба одредити пресек два круга. 

Коришћењем раније показаних својства лако се може одредити хермитска
матрица $H_1$ кругоправе која садржи $u$ и $v$. То је матрица која је
нормална на јединични круг, па је облика $\left(\begin{array}{cc}A &
  B\\\bar{B} & A\end{array}\right)$. Поред овога, још важи {\tt
    on\_circline\_rep} $H_1$ $u$ (односно {\tt quad\_form}
  $\Repcm{H_1}$ $\Repnzv{u}$ = $0$, тј. $\bar{u'}\cdot H_1\cdot u' =
  0$, при чему $u'$ представља хомогене координате тачке $u$), и
  слично и за другу тачку, односно {\tt on\_circline\_rep} $H_1$ $v$
  (тј. {\tt quad\_form} $\Repcm{H_1}$ $\Repnzv{v}$ = $0$). Развијањем
  се добије систем једначина по коефицијентима кругоправе $H_1$ који
  се може лако решити. Слично је и у другом случају, односно приликом
  одређивања хермитске матрице кругоправе $H_2$ којој припадају тачке
  $v$ и $x$. Када добијемо ове две матрице, потребно је одредити тачку
  која припада обема, односно тачку која задовољава услове {\tt
    quad\_form} $\Repcm{H_1}$ $\Repnzv{q}$ и {\tt quad\_form}
  $\Repcm{H_2}$ $\Repnzv{q}$, односно
$$\bar{q'}\cdot H_1\cdot q' = 0$$
$$\bar{q'}\cdot H_2\cdot q' = 0$$

\noindent при чему $q'$ представља хомогене координате пресечне тачке
и то је непозната променљива. Идеално решење би било када би овај
систем једначина могао да се реши коришћењем матричних трансформација,
али ми нисмо успели да нађемо такво решење. Расписивањем овог система
по координатама матрице и хомогених вектора добија се квадратна
једначина и резултат пресека ће бити корен неког великог израза.

Претходни систем је могуће за нијансу упростити тако што би се
упростила матрица која представља кругоправу. Наиме, посматрајмо прву
кругоправу која пролази кроз тачке $v$ и $x$ и чија хермитска матрица
је $H_1$. Коришћењем Мебијусових трансформација које чувају јединични
диск (било је речи раније \ref{blaske_cuvajedinicni}) могуће је
сликати тачку $v$ у координатни почетак, а тачку $x$ на
$x$--осу. Наиме, посматрајмо матрицу {\tt $M$ = rotation\_moebius
  $\phi$ + blaschke $v$} при чему је {\tt $\phi$ = arg ((blaschke $v$)
  $x$)}. Ова матрица представља Мебијусову трансформацију која слика
кругоправу $H_1$ на $x$--осу, а тачку $v$ на $(0, 0)$. Ову матрицу
можемо применити на обе кругоправе, тј. {\tt $H_1' =$ mat\_adj $M
  \cdot H_1 \cdot M$} и {\tt $H_2' =$ mat\_adj $M \cdot H_2 \cdot
  M$}. Тада се систем једначина своди на
$$\bar{q'}\cdot H_1'\cdot q' = 0$$
$$\bar{q'}\cdot H_2'\cdot q' = 0$$

\noindent при чему знамо да је $H_1'$ једнака $x$--оси, односно да је
облика $(0, i, -i, 0)$. Ипак, друга кругоправа може бити било шта
(круг или права), што значи да тражимо пресек праве и круга, што је и
даље квадратна једначина и решење ће поново бити комплексан израз.

Алгебарским трансформацијама може се одредити корени израз који
представља пресек кругова (и како су кругови, у општем случају биће
два пресека). Ипак, са добијеним комплексним изразом је тешко
резоновати. Први изазов је одредити који од два пресека заиста припада
диску и јесте тражени пресек. Потом је потребно комплексан израз који
садржи квадратни корен уврстити у једначину дворазмере, и проверити да
ли је добијени израз реалан и негативан. Ово се показало као веома
тежак задатак. Наиме, није могуће лако се ослободити корена, а изрази
који се добијају су веома комплексни и тешко је радити са
неједнакостима и доћи до жељених закључака. Зато, нажалост, овај доказ
остаје незавршен.

Исти су проблеми и у другим доказима. Такође је потребно пронаћи
пресеке кругоправих, а онда уврстити то у вектор или матрицу и
резоновати са тим комплексним изразима.

У многим уџбеницима \cite{needham, schwerdtfeger} смо наишли на
тврђење да се тривијално доказује да је Поeнкареов диск модел модел
аксиома Тарског изузимајући Еуклидову аксиому. Ипак, ни у једном
уџбенику, за сада, нисмо пронашли доказ овог тврђења. Нама није успело
да самостално доказ и довршимо, а сматрамо да сам доказ није
тривијалан.

\subsection{Закључци и даљи рад}

Иако је тврђење да је Поeнкареов диск модел модел геометрије
Лобачевског део математичког фолклора, показало се да је ово тврђење
јако тешко формализовати. Један од првих изазова је била дефиниција
релације \emph{између}. Релација се може дефинисати на два начина,
пратећи геометријски приступ или пратећи алгебарски приступ. Као и
раније, алгебарски приступ се показао далеко супериорнији и зато смо
одабрали да ову релацију дефинишемо коришћењем дворазмере. У оквиру
формализације комплексне равни доказана су бројна својства дворазмере,
па је доказивање многих својстава релације \emph{између} било веома
једноставно. Показано је да шест аксиома Тарског важе у Поeнкареовом
диск моделу и да Еуклидова аксиома паралелности не важи. Ипак,
доказивање да остале аксиоме важе је било проблематично јер за доказ
тих аксиома је потребно одредити пресек кругоправих што рачун и доказе
чини значајно комплекснијим. То је разлог зашто ова формализација није
завршена до краја. И поред овог проблема, бројна својства релације
\emph{између} и релације \emph{подударно} у оквиру Поeнкареовог диск
модела су показана, а показано је и да једна група Мебијусових
трансформација чува ове две релације. То се може користити у даљим
формализацијама.

