











Returning to the proof that our Cartesian plane satisfy the full
Pasch's axiom, first several degenerate cases need to be
considered. First group of degenerate cases arise when some points in
the construction are equal. For example, $\bett{A}{P}{C}$ allows that
$A=P=C$, that $A=P\neq C$, that $A\neq P=C$ and that $A \neq P \neq
C$. A direct approach would be to analyze all these cases
separately. However, a better approach is to carefully analyze the
conjecture and identify which cases are substantially different. It
turns out that only two different cases are relevant. If $P=C$, then
$Q$ is the point sought. If $Q=C$, then $P$ is the point sought. Next
group of degenerate cases arise when all points are collinear. In this
case, either $\bett{A}{B}{C}$ or $\bett{B}{A}{C}$ or $\bett{B}{C}{A}$
holds. In the first case $B$ is the point sought, in the second case
it is the point $A$, and in the third case it is the point
$P$.\footnote{ Note that all degenerate cases that arise in the
  Pasch's axioms were proved directly by using these elementary
  properties and that coordinate computations did not need to be used
  in those cases. This suggests that degenerate cases of Pasch's axiom
  are equivalent to the conjunction of the given properties. Further,
  this suggests that if Tarski's axiomatics was changed so that it
  included these elementary properties, then the Pasch's axiom could
  be weakened so that it includes only the central case of
  non-collinear, distinct points.}

Finally, the central case remains. In that case, algebraic
calculations are used to calculate the coordinates of the point $X$
and prove the conjecture. To simplify the proof, isometries are used,
as described in Section \ref{sec:iso}. The configuration is
transformed so that $A$ becomes the origin $(0, 0)$, and so that $P =
(0, y_P)$ and $C = (0, y_C)$ lie on the positive part of the $y$-axis.
Let $B=(x_B, y_B)$, $Q=(x_Q, y_Q)$ and $X = (x_X, y_X)$. Since
$\bett{A}{P}{C}$ holds, there is a real number $k_3$, $0 \le k_3 \le
1$, such that $y_P = k_3\cdot y_C$.  Similarly, since $\bett{B}{Q}{C}$
holds, there is a real number $k_4$, $0 \le k_4 \le 1$, such that
$(x_B - x_A) = k_2 \cdot (x_Y - x_A)$, and $x_Q - x_B = -k_4*x_B$ and
$y_Q - y_B = k_4*(y_C - y_B)$. Then, we can define a real number $k_1
= \frac{k_3\cdot (1 - k_4)}{k_4 + k_3 - k_3\cdot k_4}.$ $A \neq P \neq
C$ and points are not collinear, then, using straightforward algebraic
calculations, it can be shown that $0 \le k1 \le 1$, and that $ x_X =
k_1 \cdot x_B$, and $y_X - y_P = k_1\cdot (y_B - y_P)$, and therefore
$\bett{P}{X}{B}$ holds. Similarly, we can define a real number $k_2 =
\frac{k_4\cdot (1 - k_3)}{k_4 + k_3 - k_3\cdot k_4}$ and show that $0
\le k_2 \le 1$ and that following holds: $x_X - x_Q = -k_2\cdot x_Q$
and $y_X - y_Q = - k_2\cdot y_Q$ and thus $\bett{Q}{X}{A}$ holds.
>From these two conclusion we have determined point X.




\subsection{Axioms of Congruence and Betweenness.}
\paragraph{Upper dimension axiom.}
Three points equidistant from two distinct points form a line. Hence
any model of these axioms must have dimension less than 3.

{\tt
\begin{tabbing}
\hspace{5mm}\=\kill
$\congrt{A}{P}{A}{Q} \,\wedge\, \congrt{B}{P}{B}{Q} \,\wedge\, \congrt{C}{P}{C}{Q} \,\wedge\,  P \neq Q \ \longrightarrow \colint{A}{B}{C}$
\end{tabbing}
}

\begin{center}
\input{ax_t_10.tkz}
\end{center}



\paragraph{Segment construction axiom.}
{\tt
\begin{tabbing}
\hspace{5mm}\=\kill
$\exists E.\ \bett{A}{B}{E}\ \wedge\ \congrt{B}{E}{C}{D}$
\end{tabbing}
}

The proof that our Cartesian plane models this axiom is simple and
starts by transforming the points so that $A$ becomes the origin and
that $B$ lies on the positive part of the $y$-axis. Then $A = (0, 0)$
and $B = (0, b)$, $b \ge 0$. Let $d = \sqrt{\agsqdist{C}{D}}$. Then $E
= (0, b + d)$.

\paragraph{Five segment axiom.}
{\tt
\begin{tabbing}
\hspace{5mm}\=assumes\ \=\kill
$\congrt{A}{B}{A'}{B'} \,\wedge\, \congrt{B}{C}{B'}{C'}  \,\wedge\,  \congrt{A}{D}{A'}{D'}  \,\wedge\,  \congrt{B}{D}{B'}{D'} \,\wedge\,$\\
\>$\bett{A}{B}{C} \,\wedge\, \bett{A'}{B'}{C'} \,\wedge\, A \neq B \longrightarrow  \congrt{C}{D}{C'}{D'}$
\end{tabbing}
}

Proving that our model satisfies this axiom was rather
straightforward, but it required complex calculations. To simplify the
proofs, points $A$, $B$ and $C$ were transformed to the positive part
of the $y$-axis. Since calculations involved square roots, we did not
manage to use much automatisation and many small steps needed to be
spelled out manually.

\paragraph{The axiom of Euclid.}
{\tt
\begin{tabbing}
\hspace{5mm}\=assumes\ \=\kill
$\bett{A}{D}{T} \,\wedge\, \bett{B}{D}{C} \,\wedge\, A \neq D \ \longrightarrow\ $\\
\>$(\exists X Y.\ (\bett{A}{B}{X}\ \wedge\ \bett{A}{C}{Y}\ \wedge\ \bett{X}{T}{Y}))$
\end{tabbing}
}

The corresponding picture when all points are distinct is:

\begin{center}
\input{ax_t_6.tkz}
\end{center}


_____________________________________*******************Hilbert axioms (naci ono sto je prevedeno na fax-u(!!!!!!!!!!!!!!!!!!!!!!!!!!!!!!!!!!!!!!!!!!!!!!!!!!)

\paragraph{Axiom of Pasch.}

{\tt
\begin{tabbing}
\hspace{5mm}\=assumes\ \=\kill
$A \neq B\,\wedge\,B \neq C\,\wedge\,C \neq A\,\wedge\,\beth{A}{P}{B}\,\wedge$\\
$\,\inh{P}{l}\,\wedge\,\neg \inh{C}{l}\,\wedge\,\neg \inh{A}{l}\,\wedge\,\neg \inh{B}{l}h\ \longrightarrow$\\
\>$\exists Q.\ (\beth{A}{Q}{C}\  \wedge\ \inh{Q}{l})\ \vee\
               (\beth{B}{Q}{C}\  \wedge\  \inh{Q}{l})$
\end{tabbing}
}

\begin{center}
\input{ax_h_Pasch.tkz}
\end{center}

In the original Pasch axiom there is one more assumption -- points $A$,
$B$ and $C$ are not collinear, so the axiom is formulated only for the
central, non-degenerate case. However, in our model the statement
holds trivially if they are, so we have shown that our model satisfies
both the central and the degenerate case of collinear points. Note
that, due to the properties of the Hilbert's between relation, the
assumptions about the distinctness of points cannot be omitted.

The proof uses the standard technique. First, isometric
transformations are used to translate points to the $y$-axis, so that
$A = (0, 0)$, $B = (x_B, 0)$ and $P = (x_P, 0)$. Let $C = (x_C, y_C)$
and $\mathit{Rep\_line\_coeffs}\ l = (l_A, l_B, l_C)$. We distinguish two
major cases, depending in which of the given segments requested point
lies. Using the property $\beth{A}{P}{B}$ it is shown that $l_A\cdot
y_B \neq 0$ and then, two coefficient $k_1 = \frac{-l_C}{l_A\cdot
  y_B}$ and $k_2 = \frac{l_A\cdot y_B + l_C}{l_A\cdot y_B}$ are
defined. Next, it is shown that it holds $0 < k_1 < 1$ or $0 < k_2 <
1$. Using $0 < k_1 < 1$, the point $Q = (x_Q, y_Q)$ is determined by
$x_Q = k_1\cdot x_C$ and $y_Q = k_1\cdot y_C$, thus $\beth{A}{Q}{C}$
holds. In the other case, when the second property holds, the point
$Q=(x_q, y_q)$ is determined by $x_Q = k_2\cdot (x_C - x_B) + x_B$ and
$y_Q = k_2\cdot y_C$, thus $\bett{B}{Q}{C}$ holds.

\subsection{Axioms of Congruence}
The first axiom gives the possibility of constructing congruent
segments on given lines. In Hilbert's Grundlagen \cite{hilbert} it is
formulated as follows: \emph{If $A$, $B$ are two points on a line $a$, and
  $A'$ is a point on the same or another line $a'$ then it is always
  possible to find a point $B'$ on a given side of the line $a'$ through
  $A'$ such that the segment $AB$ is congruent to the segment $A'B'$.}
However, in our formalization part \emph{on a given side} is changed
and two points are obtained (however, that is implicitly stated in the
original axiom).

{\tt
\begin{tabbing}
\hspace{5mm}\=assumes\ \=\kill
$A \neq B\,\wedge\,\inh{A}{l}\,\wedge\,\inh{B}{l}\,\wedge\,\inh{A'}{l'}\ \longrightarrow$\\
\> $\exists B'\, C'.\ \inh{B'}{l'}\,\wedge\,\inh{C'}{l'}\,\wedge\,\beth{C'}{A'}{B'}\,\wedge\,\congrh{A}{B}{A'}{B'}\,\wedge\,\congrh{A}{B}{A'}{C'}$
\end{tabbing}
}

The proof that this axiom holds in our Cartesian model, starts with
isometric transformation so that $A'$ becomes $(0, 0)$ and $l'$
becomes the x-axes. Then, it is rather simple to find the two points
on the x-axis by determining their coordinates using condition that
$\agsqdist{}{}$ between them and the point $A'$ is same as the
$\agsqdist{A}{B}$.


The following two axioms are proved straightforward by unfolding the
corresponding definitions, and automatically performing algebraic
calculations and the Gr\"obner bases method.

{\tt
\begin{tabbing}
\hspace{5mm}\=assumes\ \=\kill
$\congrh{A}{B}{A'}{B'}\,\wedge\,\congrh{A}{B}{A''}{B''}\ \longrightarrow\ \congrh{A'}{B'}{A''}{B''}$\\
$\beth{A}{B}{C}\,\wedge\,\beth{A'}{B'}{C'}\,\wedge\,\congrh{A}{B}{A'}{B'}\,\wedge\,\congrh{B}{C}{B'}{C'} \ \longrightarrow\ \congrh{A}{C}{A'}{C'}$
\end{tabbing}
}

Next three axioms in the Hilbert's axiomatization are concerning the
notion of angles, and we have not yet considered angles in our
formalization.


\subsection{Axiom of Parallels}

{\tt
\begin{tabbing}
\hspace{5mm}\=assumes\ \=\kill
$\neg \inh{P}{l} \ \longrightarrow\ \exists!\,l'.\ \inh{P}{l'}\ \wedge\ \neg (\exists\ P_1.\ \inh{P_1}{l}\ \wedge\  \inh{P_1}{l'})$
\end{tabbing}
}

The proof of this axiom consists of two parts. First, it is shown that
such line exists and second, that it is unique.  Showing the existence
is done by finding coefficients of the line sought.  Let $P = (x_P,
y_P)$ and $\mathit{Rep\_line\_coeffs} l = (l_A, l_B, l_C)$.  Then
coefficients of the requested line are $(l_A, l_B, -l_A\cdot x_P -
l_B\cdot y_P)$. In the second part, the proof starts from the
assumption that there exist two lines that satisfy the condition
$\inh{P}{l'}\ \wedge\ \neg (\exists\ P_1.\ \inh{P_1}{l}\ \wedge\  \inh{P_1}{l'})$. In
the proof it is shown that their coefficients are proportional and
thus the lines are equal.

\subsection{Axioms of Continuity}

\paragraph{Axiom of Archimedes.}
Let $A_1$ be any point upon a straight line between the arbitrarily
chosen points A and B.  Choose the points $A_2, A_3, A_4, \ldots$ so
that $A_1$ lies between A and $A_2$, $A_2$ between $A_1$ and $A_3$,
$A_3$ between $A_2$ and $A_4$ etc. Moreover, let the segments $AA_1,
A_1A_2, A_2A_3, A_3A_4, \ldots$ be equal to one another. Then, among
this series of points, there always exists a point $A_n$ such that B
lies between A and $A_n$.

It is rather difficult to represent series of points in a manner as
stated in the axiom and our solution was to use list. First, we
define a list such that each four consecutive points are congruent
and betweenness relation holds for each three consecutive points.

{\tt
\begin{tabbing}
\hspace{5mm}\=assumes\ \=\kill
definition \\
\> congruentl $l \longrightarrow length\ l \ge 3\ \wedge$\\
\>\>  $\forall i.\ 0 \le i\ \wedge\ i+2 < length\ l \longrightarrow$ \\
\>\>  $\congrh{(l\ !\ i)}{(l\ !\ (i+1))}{(l\ !\ (i+1))}{(l\ !\ (i+2))}\ \wedge $\\
\>\>  $\beth{(l\ !\ i)}{(l\ !\ (i+1))}{(l\ !\ (i+2))}$
\end{tabbing}
}

Having this, the axiom can be bit transformed, but still with the same
meaning, and it states that there exists a list of points with
properties mentioned above such that for at least one point A' of the
list, $\bett{A}{B}{A'}$ holds. In Isabelle/HOL this is formalized as:

{\tt
\begin{tabbing}
\hspace{5mm}\=\kill
$\beth{A}{A_1}{B}\ \longrightarrow$\\
\> $(\exists l.\ congruentl (A\ \#\ A1\ \#\ l)\ \wedge\ (\exists i.\ \beth{A}{B}{(l\ !\ i)}))$
\end{tabbing}
}

The main idea of the proof is in the statements $\agsqdist{A}{A'} >
\agsqdist{A}{B}$ and $\agsqdist{A}{A'} = t\cdot
\agsqdist{A}{A_1}$. So, in the first part of the proof we find such
$t$ that $t\cdot \agsqdist{A}{A_1} > \agsqdist{A}{B}$ holds. This is
achieved by applying Archimedes' rule for real numbers. Next, it is
proved that there exists a list $l$ such that \verb|congruentl| $l$
holds, that it is longer then $t$, and that it's first two elements
are $A$ and $A_1$. This is done by induction on the parameter $t$. The
basis of induction, when $t = 0$ trivially holds. In the induction
step, the list is extended by one point such that it is congruent with
the last three elements of the list and that between relation holds
for the last two elements and added point. Using these conditions,
coordinates of the new point are easily determined by algebraic
calculations. Once constructed, the list satisfies the conditions of
the axiom, what is easily showed in the final steps of the proof. The
proof uses some additional lemmas which are mostly used to describe
properties of the list that satisfies condition \verb|congruentl| $l$.

% ------------------------------------------------------------------------------
\section{Related work}
\label{sec:related}
% ------------------------------------------------------------------------------

There are a number of formalizations of fragments of various
geometries within proof assistants.

Formalizing Tarski geometry using Coq proof assistant was done by
Narboux \cite{narboux}.  Many geometric properties are derived,
different versions of Pasch axiom, betweenness and congruence
properties.The paper is concluded with the proof of existence of
midpoint of a segment.

Another formalization using Coq was done for projective plane geometry
by Magaud, Narboux and Schreck \cite{projective-coq1,projective-coq2}.
Some basic properties are derived, and the principle of duality
for projective geometry.  Finely the consistency of the axioms are
proved in three models, both finite and infinite. In the end authors
discuss the degenerate cases and choose ranks and flats to deal with
them.

First attempt to formalize first groups of Hilbert's axioms and their
consequences within a proof assistant was done by Dehliger, Dufourd
and Schreck in intuitionistic manner in Coq \cite{hilbert-coq}. The
next attempt in Isabelle/Isar was done by Meikle and Fleuriot
\cite{hilbert-isabelle}. The authors argue the common believed
assumption that Hilbert's proofs are less intuitive and more
rigorous. Important conclusion is that Hilbert uses many assumptions
that in formalization checked by a computer could not be made and
therefore had to be formally justified.

Guilhot connects Dynamic Geometry Software (DGS) and formal theorem
proving using Coq proof assistant in order to ease studying the
Euclidean geometry for high school students \cite{guilhot}. Pham,
Bertot and Narboux suggest a few improvements \cite{coqly}. The first
is to eliminate redundant axioms using a vector approach. They
introduced four axioms to describe vectors and tree more to define
Euclidean plane, and they introduced definitions to describe geometric
concepts. Using this, geometric properties are easily proved. The
second improvement is use of area method for automated theorem
proving. In order to formally justify usage of the area method,
Cartesian plane is constructed using geometric properties previously
proved.

Avigad describes the axiomatization of Euclidean geometry
\cite{avigad}. Authors start from the claim that Euclidean geometry
describes more naturally geometry statements than some axiomatizations
of geometry done recently and it's diagrammatic approach is not so
full of weaknesses as some might think. In order to prove this, the
system E is introduced in which basic objects such as point, line,
circle are described as literals and axioms are used to describe
diagram properties from which conclusions could be made. The authors
also illustrate the logical framework in which proofs can be
constructed. In the work are presented some proofs of geometric
properties as well as equivalence between Tarski's system for
ruler-and-compass and E. The degenerate cases are avoided by making
assumptions and thus only proving general case.

In \cite{william} is proposed the minimal set of Hilbert axioms and
set theory is used to model it. The main properties and theorems are
carried out within this model.

In many of these formalizations discussion about degenerate cases is
omitted. Although, usually the general case expresses important
geometry property, observing degenerate cases usually leads to
conclusion about some basic properties such as transitivity or
symmetry, and thus makes them equally important.

Beside formalization of geometries many authors tried to formalize
automated proving in geometry.

Gr\'egoire, Pottier and Th\'ery combine a modified version of
Buchberger’s algorithm and some reflexive techniques to get an
effective procedure that automatically produces formal proofs of
theorems in geometry \cite{grobnercoq}.

G\'enevaux, Narboux and Schreck formalize Wu's simple method in Coq
\cite{wucoq}. Their approach is based on verification of certificates
generated by an implementation in Ocaml of a simple version of Wu's
method.

Fuchs and Th{\'e}ry formalize Grassmann-Cayley algebra in Coq proof
assistant \cite{grassman}. The second part, more interesting in the
context of this paper, presents application of the algebra on the
geometry of incidence. Points, lines and there relationships are
defined in form of algebra operations. Using this, theorems of Pappus
and Desargues are interactively proved in Coq. Finally the authors
describe the automatisation in Coq of theorem proving in geometry
using this algebra. The drawback of this work is that only those
statements where goal is to prove that some points are collinear can
automatically be proved and that only non-degenerate cases are
considered.

Pottier presents programs for calculating Grobner basis, F4, GB and
gbcoq and compares them \cite{grobner_coq}.  A solution with
certificates is proposed and this shortens the time for computation
such that gbcoq, although made in Coq, becomes competitive with two
others. Application of Gr\"obner basis on algebra, geometry and
arithmetic are represented through three examples.



% ------------------------------------------------------------------------------
\section{Conclusions and Further Work}
\label{sec:concl}
% ------------------------------------------------------------------------------

In this paper, we have developed a formalization of Cartesian plane
geometry within Isabelle/HOL. Several different definitions of the
Cartesian plane were given, but it was shown that they are all
equivalent. The definitions were taken from the standard
textbooks. However, to express them in a formal setting of a proof
assistant, much more rigour was necessary. For example, when
expressing lines by equations, some textbooks mention that equations
represent the line if their coefficients are ``proportional'', while
some other fail even to mention this. The texts usually do not mention
constructions like equivalence relations and equivalence classes that
underlie our formal definitions.

We have formally shown that the Cartesian plane satisfies all Tarski's
axioms and most of the Hilbert's axioms (including the continuity
axiom). Proving that our Cartesian plane model satisfies all the
axioms of the Hilbert's system is left for further work (as we found
the formulation of the completeness axiom and the axioms involving the
derived notion of angles problematic).

Our experience shows that proving that our model satisfies simple
Hilbert's axioms was easier than showing that it satisfies Tarski's
axioms. This is mostly due to the definition of the betweenness
relation. Namely, Tarski's axiom allows points connected by the
betweenness relation to be equal. This gives rise to many degenerate
cases that need to be considered separately, what complicates
reasoning and proofs. However, Hilbert's axioms are formulated using
derived notions (e.g., angles) what posed problems for our
formalization.

The fact that analytic geometry models geometric axioms is usually
taken for granted, as a rather simple fact. However, our experience
shows that, although conceptually simple, the proof of this fact
requires complex computations and is very demanding for
formalization. It turned out that the most significant technique used
to simplify the proof was ``without loss of generality reasoning'' and
using isometry transformations. For example, we have tried to prove
the central case of the Pasch's axiom, without applying isometry
transformations first. Although it should be possible do a proof like
that, the arising calculations were so difficult that we did not
manage to finish such a proof. After applying isometry
transformations, calculations remained non-trivial, but still, we
managed to finish this proof (however, many manual interventions had
to be used because even powerful tactics relying on the Gr\"obner
bases did not manage to do all the algebraic simplifications
automatically). From this experiment on Pasch's axiom, we learned the
significance of isometry transformations and we did not even try to
prove other lemmas directly.

Our formalization of the analytic geometry relies on the axioms of
real numbers and properties of reals are used throughout our
proofs. Many properties would hold for any numeric field (and
Gr\"obner bases tactics used in our proofs would also work in that
case). However, for showing the continuity axioms, we used the
supremum property, not holding in an arbitrary field. In our further
work, we would like to build analytic geometries without using the
axioms of real numbers, i.e., define analytic geometries within
Tarski's or Hilbert's axiomatic system. Together with the current
work, this would help analyzing some model theoretic properties of
geometries. For example, we want to show the categoricity of both
Tarski's and Hilbert's axiomatic system (and prove that all models are
isomorphic and equivalent to the Cartesian plane).

Our present and further work also includes formalizing analytic models
of non-Euclidean geometries. For example, we have given formal
definitions of the Poincar\'e disk (were points are points in the unit
disk and lines are circle segments perpendicular to the unit circle)
using the Complex numbers available in Isabelle/HOL and currently we
are showing that these definitions satisfy all axioms except the
parallel axiom.

Finally, we want to connect our formal developments to the
implementation of algebraic methods for automated deduction in
geometry, making formally verified yet efficient theorem provers for
geometry.

% ------------------------------------------------------------------------------
% Bibliography
% ------------------------------------------------------------------------------
%\bibliography{article}{}
%\bibliographystyle{plain}

\begin{thebibliography}{10}

\bibitem{avigad}
Jeremy Avigad, Edward Dean, and John Mumma.
\newblock A formal system for {E}uclid's \emph{{E}lements}.
\newblock {\em Review of Symbolic Logic}, 2(4):700--768, 2009.

\bibitem{locales}
Clemens Ballarin.
\newblock Interpretation of locales in isabelle: Theories and proof contexts.
\newblock In {\em MKM}, LNCS 4108, pp.~31--43. Springer, 2006.

\bibitem{buchberger}
Bruno Buchberger.
\newblock Bruno Buchberger's PHD thesis 1965: An algorithm for finding the
  basis elements of the residue class ring of a zero dimensional polynomial
  ideal.
\newblock {\em J. Symb. Comput.}, 41(3-4):475--511, 2006.


\bibitem{hilbert-coq}
Christophe Dehlinger, Jean-François Dufourd and Pascal Schreck.
\newblock Higher-Order Intuitionistic Formalization and Proofs in Hilbert’s Elementary Geometry.
\newblock In {\em Automated Deduction in Geometry}, ADG'01, LNCS 2061, Springer. 2001.



\bibitem{grassman}
Laurent Fuchs and Laurent Th{\'e}ry.
\newblock A formalization of Grassmann-Cayley algebra in COQ and its
  application to theorem proving in projective geometry.
\newblock In {\em Automated Deduction in Geometry}, ADG'10, pp.~51--67. Springer. 2011.

\bibitem{wucoq}
Jean-David G{\'e}nevaux, Julien Narboux, and Pascal Schreck.
\newblock Formalization of Wu's simple method in COQ.
\newblock In {\em CPP}, LNCS 7086, pp.~71--86. Springer, 2011.

\bibitem{grobnercoq}
Benjamin Gr{\'e}goire, Lo\"{\i}c Pottier, and Laurent Th{\'e}ry.
\newblock Proof certificates for algebra and their application to automatic
  geometry theorem proving.
\newblock In {\em Automated Deduction in Geometry}, LNCS 6301, pp.~42--59. Springer, 2011.

\bibitem{guilhot}
F.~Guilhot.
\newblock Formalisation en COQ et visualisation d’un cours de g\'eom\'etrie
  pour le lyc\'e.
\newblock {\em TSI 24, 1113–1138}, 2005.


\bibitem{wlog}
John Harrison.
\newblock Without loss of generality.
\newblock In {\em TPHOLs}, LNCS 5674, pp.~43--59. Springer, 2009.

\bibitem{hilbert}
David Hilbert.
\newblock {\em Grundlagen der Geometrie}.
\newblock Leipzig, B.G. Teubner, 1903.

\bibitem{isabelle-quotient}
Cezary Kaliszyk and Christian Urban.
\newblock Quotients revisited for Isabelle/HOL.
\newblock In {\em SAC}, pp.~1639--1644. ACM, 2011.

\bibitem{kapur}
Deepak Kapur.
\newblock Using Gr\"obner bases to reason about geometry problems.
\newblock {\em Journal of Symbolic Computation}, 2(4), 1986.

\bibitem{projective-coq1}
Nicolas Magaud, Julien Narboux, and Pascal Schreck.
\newblock Formalizing projective plane geometry in coq.
\newblock In {\em Automated Deduction in Geometry} 2008., pages 141--162.

\bibitem{projective-coq2}
Nicolas Magaud, Julien Narboux, and Pascal Schreck.
\newblock A case study in formalizing projective geometry in COQ: Desargues
  theorem.
\newblock {\em Comput. Geom.}, 45(8):406--424, 2012.

\bibitem{thedu}
Filip Mari\'c, Ivan Petrovi\'c, Danijela Petrovi\'c, and Predrag Jani\v{c}i\'c.
\newblock Formalization and implementation of algebraic methods in geometry.
\newblock In {\em THEDU'11}, {\em Electronic Proceedings in Theoretical Computer Science} 79, pp.~63--81. 2012.

\bibitem{hilbert-isabelle}
Laura~I. Meikle and Jacques~D. Fleuriot.
\newblock Formalizing Hilbert's Grundlagen in Isabelle/Isar.
\newblock In {\em TPHOLs}, LNCS 2758, pp.~319--334. Springer, 2003.

\bibitem{narboux}
Julien Narboux.
\newblock Mechanical theorem proving in Tarski's geometry.
\newblock In {\em Automated Deduction in Geometry}, LNCS 4869, pp.~139--156. Springer, 2006.

\bibitem{Isabelle}
Tobias Nipkow, Lawrence~C. Paulson, and Markus Wenzel.
\newblock {\em Isabelle/HOL - A Proof Assistant for Higher-Order Logic}, LNCS 2283.
\newblock Springer, 2002.

\bibitem{coqly}
Tuan-Minh Pham, Yves Bertot, and Julien Narboux.
\newblock A COQ-based library for interactive and automated theorem proving in
  plane geometry.
\newblock In {\em ICCSA (4)}, LNCS 6785, pp.~368--383. Springer, 2011.

\bibitem{grobner_coq}
Lo\"{\i}c Pottier.
\newblock Connecting Gr{\"o}bner bases programs with COQ to do proofs in
  algebra, geometry and arithmetics.
\newblock {\em CoRR}, abs/1007.3615, 2010.

\bibitem{william}
William Richter.
\newblock A minimal version of Hilbert's axioms for plane geometry.
\newblock \url{www.math.northwestern.edu/~richter/hilbert.pdf}

\bibitem{tarski}
W.~Schwabhäuser, W.~Szmielew, and A.~Tarski.
\newblock {\em Metamathematische Methoden in der Geometrie}.
\newblock Springer-Verlag, 1983.

\bibitem{wu}
Wen-Ts\"un Wu.
\newblock On the decision problem and the mechanization of theorem proving in
  elementary geometry.
\newblock {\em Scientia Sinica}, 21, 1978.
\end{thebibliography}


\end{document}
