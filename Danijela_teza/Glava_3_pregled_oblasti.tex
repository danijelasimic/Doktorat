\chapter[Доказивање у геометрији]{Преглед формализације и аутоматског доказивања у геометрији}
\label{chapter::pregled_oblasti}

У последњих десет година направљени су значајни резултати у
формализацији математике ко\-ри\-шће\-њем интерактивних доказивача. У
овом поглављу навешћемо неке најзначајније радове (како класичне, тако
и најновије) у формализацији геометрије. У сваком поглављу радови су
наведени хронолошки према годинама када су објављени.

\section{Аутоматско доказивање у геометрији}

У раду "Аутоматско резоновање у геометрији" \cite{hbook} постоји веома
детаљан историјски опис развоја аутоматских доказивача до почетка овог
века, као и детаљан опис различитих приступа у аутоматском доказивању
у геометрији и нешто од тих идеја је приказано у овом поглављу.

\subsection{Aлгебарски доказивачи}

\paragraph{Вуов метод и метод Гребнерових база.} 
Највећи напредак у аутоматском доказивању теорема у геометрији
направио је Ву (енг.~\emph{Wu}). Он је ограничио скуп проблема које је
разматрао на проблеме са једнакостима. На такве проблеме је могао да
примени моћан метод који је могао да докаже и компликована
тврђења. Овај метод је представио у оквиру свог рада \cite{wu}
1978.~године. Како је овим методом показано 130 тврђења (међу којима
је Фојербахова (нем.~\emph{Feuerbach}) теорема, Карнотова
(фр.~\emph{Carnot}) теорема, тангента--секанс теорема, Птоломејева
теорема, Ојлерова теорема, Стјуартова теорема и многе друге)
\cite{chou1984}, он постаје све популарнији. Бројни аутори су овај
метод имплементирали и унапређивали разним хеуристикама \cite{gao1990,
  sen1987mechanical, kapur1990refutational}. Убрзо, постаје јасно да
се Вуов приступ могао извести из Ритовог (енг.~\emph{Ritt}) рада
\cite{ritt1950differential}, па се често овај метод још назива и
\emph{Ву--Рит метод}.

Успех овог метода утицао је на развој нових метода. Један од успешних
је метод Гребнерових база, који се заснива на \emph{Бухбергеровом
  алгоритаму} (нем.~\emph{Buchberger}) \cite{buchberger} и може се
применити на исту класу проблема као и Вуов метод. Бухбергеров
алгоритам су унапређивали бројни аутори и данас постоје разне
хеуристике које служе да повећају ефикасност алгоритма. Коришћењем
овог метода показана су многа геометријска тврђења
\cite{buchberger1998grobner}, као што су Гусова теорема, Папусова
теорема, Дезаргова теорема, теорема о Ојлеровој прави троугла и многе
друге. Постоје бројне имплементације, а неке и у комерцијалним
програмима (нпр.~\emph{Matlab} и \emph{Mathematica}).

Главна мана ова два метода је што се са њима \emph{не могу доказивати
  неједнакости} и самим тим се не могу разматрати теореме које говоре
о распореду тачака. За решавање проблема са неједнакостима, Ву је
предложио метод који се заснива на проналажењу минималне или
максималне вредности полиномијалне функције под одређеним условима
\cite{Wu1992}. Поред доказивања у елементарној геометрији, Ву је
представио и \emph{метод за доказивање у диференцијалној геометрији}
\cite{wen1991mechanical}. Постоје и проширења која омогућавају да се
\emph{метод користи и за хиперболичку геометрију}
\cite{yang1998automated}.

\subsection{Полусинтетички доказивачи}

\paragraph{Метод површина и његова проширења.}
Сви набројани методи преводе геометријско тврђење у једначине
коришћењем координата тачака које се посматрају, а потом се примењују
алгебарске технике на ове једначине. Ови доказивачи дају одговор "да"
или "не", али не дају никакву информацију о извођењу која би била
разумљива човеку и слична доказима у школским уџбеницима. Постоје
бројни покушаји да се направе доказивачи који би поред доказивања
уједно производили \emph{читљиве доказе}. Један од најзначајнијих је
\emph{метод површина} \cite{chou1993automated}. Овај метод користи
геометријске величине као што су површина, размера, Питагорина разлика
и слично. Главна предност овог доказивача је што сваки корак у
доказивању има јасно геометријско значење. Додатно, експерименти су
показали да су докази коришћењем метода површина краћи. Метод је
могуће проширити тако да је могуће радити и са неједнакостима. Главна
идеја метода је изразити хипотезе теореме коришћењем скупа
конструктивних тврђења, од којих свако тврђење уводи нову тачку или
нову праву и представља закључак као једнакост израза у којем се налазе
геометријске величине, а без коришћења Декартових координата. Доказ се
заснива на елиминацији (у обрнутом редоследу) тачака и линија
коришћењем одговарајућих лема. Када се елиминишу сви уведени елементи,
тренутни циљ постаје једнакост између два израза са геометријским
величинама датих само слободним тачкама. Ако је ова једнакост
тривијално тачна, онда је оригинално тврђење доказано; ако је
тривијално нетачна, онда је доказано да је почетна претпоставка
нетачна; иначе, тврђење није ни доказано, ни оповргнуто.

Чу (енг.~\emph{Chou}) и сарадници су такође представили и \emph{метод
  пуног угла} \cite{chou1996automated}. Метод пуног угла као величину
користи новоуведени појам \emph{пун угао} преко ког изражава односе
између тачака и дужи. \emph{Пун угао} је дат као уређен пар правих, а
веома је битан однос између два пуна угла, тј. угао одређен правама
$m$ и $n$ је једнак пуном углу одређеном правама $u$ и $v$ ако постоји
ротација $\rho$ таква да $\rho(m) || u$ и $\rho(n) || v$. У
експериментима је примећено да се метод површина веома добро понаша за
конструктивне теореме у афиној геометрији. Са друге стране, метод
пуног угла је погодан за проблеме у којима има много кругова и углова
и за овакве проблеме чешће производи краће доказе него што је то
случај са методом површина. Овај метод се користи као допуна метода
површина.

Исти аутори су у раду \cite{chou1995volume} представили могућност
\emph{проширења методе површина на проблеме у стереометрији}. Хипотезе
се задају конструктивно, а закључци су полиномијалне једначине
неколико геометријских величина, као што су запремина, размера дужи,
размера површина и Питагорине разлике. Главна идеја овог метода је да
елиминише тачке из закључка геометријског тврђења коришћењем основних
својстава запремине.

У раду из 2016. године, Шао (енг.~\emph{Shao}), Ли (енг.~\emph{Li}) и
Хуанг (енг.~\emph{Huang}) \cite{shao2016challenging} посматрају
задатке из стереометрије са Олимпијских такмичења из математике. Они у
раду представљају три различита проблема и дају полиноме које су
извели на папиру и којима се у полиномијалном облику задају посматрана
геометријска тврђења. Коришћењем ова три примера они показују да се
алгебарски методи могу користити за доказивање у стереометрији. За
сваки пример користили су три различита метода: метод карактеристичног
скупа \cite{wu2007mathematics, wang1998decomposing,
  gao1991computations, chen2002projection}, метод Гребнерових база
\cite{cox1992ideals, kutzler1986application, stifter1993geometry,
  chou1987characteristic} и метод вектора
\cite{lord1985method}. Методи се пореде и закључак је да метод вектора
даје бољи геометријски доказ, али формуле могу бити дуге и незгодне за
манипулацију и израчунавање. Ипак, они не нуде неки систематичан начин
како се геометријска тврђења могу аутоматски алгебризовати, односно
представити полиномима.

Поред ова два поменута рада, није нам познато да постоји још радова
који се баве аутоматским доказивањем у стереометрији.

Иако је метод површина описан још деведесетих година прошлог века, до
скора нису детаљно описана имплементациона питања, али ни испитана
оправданост коришћења самог метода. У раду \cite{janivcic2012area}
аутори управо скрећу пажњу на ове проблеме. Они \emph{веома детаљно
  описују метод и формално доказују у систему Coq важне дефиниције и
  леме које омогућавају коришћење метода}. Детаљно описују и нека
важна имплементациона питања, јер метод површина, иако је једноставан
за разумевање, је тежак за имплементацију јер постоји много детаља на
које би требло обратити пажњу.

\subsection{Синтетички доказивачи}

Стојановић са сарадницима демонстрира коришћење \emph{синтетичког
  доказивача за доказивање теорема у геометрији Тарског}
\cite{dhurdjevic2015automated}. Пре свега је интересантно то што је
повезано интерактивно и аутоматско доказивање, што значи да су сви
аутоматски генерисани докази уједно и формално верификовани. Систем
поред доказивања теорема генерише и машински проверене, читљиве доказе
који су веома слични доказима из уџбеника. Аутори користе кохерентну
логику, део логике првог реда као основну логику система. Примењују
резолуцијски доказивач, доказивач теорема у кохерентној логици, и
\emph{XML} алатке за кохерентну логику који им омогућавају да доказе
трансформишу у машински проверљиве доказе и у доказе разумљиве
човеку. Систем примењују на доказивање тврђења из првог дела књиге
"Mатематички методи у геометрији" \cite{tarski} и успешно, потпуно
аутоматски доказују 37\% теорема.

Сличан приступ је описан у раду \cite{beesonfinding} у којем  се описује
\emph{полуаутоматски приступ за доказивање у геометрији
  Тарског}. Аутори су посматрали више група доказа, од којих су неки
краћи од 40 корака, потом доказе који су између 40 и 100 корака и који
се сматрају тежим за човека и коначно доказе дуже од 100 корака који
најчешће представљају теме докторских радова. За доказивање користе
доказивач \emph{OTTER}, који је и раније коришћен за доказивање теорема
\cite{quaife1992automated}. Један од циљева је била и анализа утицаја
хардверског напретка, али и нових техника за аутоматско доказивање у
геометрији на успешност аутоматских доказивача. Потпуно механички је
изведена већина кратких доказа. Испрва за дугачке доказе није било
могуће добити механичке доказе, и аутори су коришћењем доказа из књиге
конструисали формалне доказе. Потом су применили нову технику, где су
доказивачу прослеђивали све потребне аксиоме и претходно доказана
тврђења, као и неке кораке доказа и систем је успевао да нађе
механичке доказе за тврђења која су се доказивала у више од 100
корака.

\subsection{Остали доказивачи и примене}

Један од првих радова у области аутоматског доказивања у геометрији је
Гелертнеров (енг.~\emph{Gelernter}) рад \cite{gelernter1959}, који је
користио методе вештачке интелигенције, и његов приступ се заснивао на
прављењу доказа сличних онима које пише човек. 

Новији рад који такође користи методе вештачке интелигенције је систем
\emph{Geometrix} \cite{gressier2013}.

Вос (енг.~\emph{Wos}) и његови сарадници \cite{mccharen2000problems}
су користили резолуцијски доказивач за доказивање у геометрији
Тарског. Програм је тестиран на проблемима из геометрије Тарског, али
поред тога и на алгебарским проблемима, теорији категоричности и на
проблемима верификације програма.

Додатно, постоји и рад који представља приступ за доказивање
геометријских тврђења коришћењем елиминације квантора у линеарним и
квадратним формулама над реалним бројевима
\cite{dolzmann1998new}. Приступ се може користити за доказивање у
реалној равни или вишим димензијама. За разлику од других приступа,
овом методом се могу показати тврђења која иначе не могу да се покажу
у пољу комплексних бројева. Додатно, формулација проблема може
садржати и неједнакости (јер не морају теореме да буду универзално
квантификоване) и алтернације квантификатора. Услови недегенерисаности 
се генеришу аутоматски, а чак је могуће аутоматски утврдити за тврђења 
која не важе у уопштеном случају који услови недегенерисаности и које
претпоставке су потребне да би тврђење важило.

Интензивно поље истраживања није само аутоматско доказивање него и
коришћење аутоматских доказивача за решавање проблема у геометрији. У
раду Весне Маринковић са сарадницима \cite{marinkovic2014computer}
аутори се баве \emph{проблемом конструкције помоћу лењира и
  шестара}. У овом раду, фокус је на проблему конструкције троугла где
су дате тачке и услови који за њих морају да важе. Приликом
проналажења конструкције пролази се кроз све четири фазе (анализа,
конструкција, доказ и дискусија), аутоматски се проналазе кораци
конструкције, као и доказ исправности који се даје у облику који је
читљив човеку. Додатно, користе се алгебарски методи за аутоматско
доказивање да ли је могуће извршити конструкцију или није могуће
извршити конструкцију, што је посебно интересантно јер постоји много
проблема који су неконструктивни.


\subsection{Различити системи за задавање конструкција и аутоматско доказивање у геометрији и њихова примена у образовању}

Ванг (енг.~\emph{Wang}) је у раду \cite{wang2002geother} описао систем
\emph{GEOTHER} који може послужити за аутоматско доказивање теорема у
геометрији. За развој система користи \emph{Maple}. \emph{Maple}
омогућава симболичко и нумеричко израчунавње који садржи динамички
типизиран императивни програмски језик који највише личи на
\emph{Pascal}. Поред тога има и могућност визуелизације, анализе
података и рада са матрицама. Коришћењем програма \emph{Maple}, Ванг
геометријска тврђења репрезентује коришћењем предикатске
спецификације, а те спецификације је могуће аутоматски превести на
тврђења записана на енглеском или кинеском, или на алгебарске
једнакости или на логичке формуле. На основу спецификација могуће је
конструисати и дијаграме које је потом могуће мењати коришћењем
миша. Оно што је посебно интересантно је што је у систему
имплементирано више аутоматских доказивача теорема. Имплеметиран је
доказивач заснован на Вуовој методи. Потом доказивач заснован на
Кацлер---Стифтер (нем.~\emph{Kutzler--Stifter}) методи и доказивач
заснован на Капуровој (енг.~\emph{Kapur}) методи (оба метода су
заснована на идејама методе Гребнерових база). Имплеметирани су још и
методи засновани на нула декомпозицији и обичној диференцијалној нула
декомпозицији \cite{wang1995elimination}. Доказивачи су упоређивани
над више различитих теорема, а Вуов метод се за већину тврђења показао
као најефикаснији.

Значајно је поменути и систем \emph{Geometry Expert}
\cite{chou1996introduction} који има имплементиран Вуов метод, метод
Гребнерових база, метод вектора, метод пуног угла и метод
површина. Посебно је интересантно његово проширење, систем \emph{Java
  Geometry Expert} \cite{ye2008introduction}. Ова алатка је занимљива
јер поред аутоматског доказивања теорема нуди и визуелну, динамичку
репрезентацију доказа. Производи серију визуелних ефеката за
презентацију доказа и у својој бази садржи преко шест стотина примера.

Систем \emph{Geometry Explorer} \cite{wilson2005combining} производи
читљиве доказе о својствима конструисаних објеката коришћењем метода
пуног угла.

Важно је поменути и систем \emph{Discover} \cite{botana2002dynamic} за
аутоматско откривање у Еуклидској геометрији, који користи алгебарски
софтвер \emph{CoCoA} \cite{abbottcocoa} и \emph{Mathematica}
\cite{wolfram2008mathematica}.

Значајан је и систем \emph{GCLC} \cite{janivcic2010geometry} који
омогућава запис конструкције и тврђења и превођење истих у различите
формате (рецимо, у формат .tkz који је значајан јер је погодан за
уметање слика у \emph{TeX} документ). Овај систем је посебно значајан
јер има интегрисана три доказивача теорема: Вуов метод, метод
Гребнерових база и метод површина.

Нешто другачији приступ у односу на поменуте системе има систем
\emph{Cinderella} \cite{kortenkamp2004using} који омогућава
интерактивно задавање конструкције, а онда проверава да ли је добијена
конструкција заиста тражено решење. Обављају се насумичне провере
теорема да би се анализирале акције корисника, али се не даје доказ за
дату конструкцију у било којој форми. Ово је систем који није
симболички, ни динамички већ користи пробабилистичке методе да провери
да ли је дата претпоставка највероватније теорема. Метод је заснован
на леми Шварц--Зифел (нем.~\emph{Schwartz--Zippel}) која одређује број
нула мултиваријантног полинома за дати максимални укупни степен. Више
о свему овоме се може пронаћи у \cite{richter1999foundations}. Оно што
је важно је да програм може брзо да одлучи да ли два елемента у
конструкцији су иста јер их теорема на то приморава или не. Програм
враћа "није теорема" као одговор ако није успео да докаже теорему иако
је могуће теорему доказати другим методама. Са друге стране, програм
никада не враћа "јесте теорема" када то није случај. Систем
\emph{Cinderella} омогућава наивну подршку за нееуклидске
геометрије. Софтвер је писан у програмском језику \emph{Java} па се
може користити на више платформи.

Веома широко распрострањени у образовању су \emph{динамички
  геометријски алати}. Коришћењем тог софтвера корисник лако може
креирати и мењати геометријске конструкције. Конструкција обично
започиње задавањем тачака или неких објеката (праве, кругови), а онда
се креирају зависне тачке и објекти. Потом, почетна конфигурација се
може мењати и може се посматрати како мењање почетних положаја утиче
на крајњи резултат. На тај начин могуће је тестирати претпоставке,
рецимо, да ли су три тачке увек колинеарне без обзира на
конфигурацију. Како је могуће само тестирање, али не и доказивање,
нови правци у развоју оваквог софтвера су управо у додавању
аутоматских доказивача у оквиру динамичке геометријске
алатке. Најпознатији динамичко геометријски софтвер је
\emph{GeoGebra} \footnote{\url{https://www.geogebra.org/}} и користи
се у многим земљама, укључујући и Србију, као помоћно наставно
средство.

Алгоритам који је проширење метода Гребнерових база и који се заснива
на анализи система са параметрима се користи у раду
\cite{botana2014automatic} ради \emph{проналажења хипотеза које морају
  да важе (поред претпоставки које су већ дате) да би дато тврђење
  било могуће извести}. Систем је имплементиран тако да се може
користити у оквиру система \emph{GeoGebra}. Значајно је што систем
симболички одређује геометријско место тачака и потом показује
валидност геометријског тврђења. Посебно је значајно и то што се у
процесу одређивања геометријског места тачака, нотирају и ирелевантне
тачке (најчешће су то у питању дегенерисани случајеви) и избацују из
разматрања.

Поред поменутог, значајан \emph{додатак систему GeoGebra je аутоматски
  доказивач заснован на Вуовој методи}
\cite{botana2015automated}. Поред могућности доказивања, систем може
да идентификује "интересантна" својства дате конструкције, односно да
аутоматски одреди неке релације између конструисаних објеката.


\section{Интерактивно доказивање у геометрији}

\subsection{Интерактивно доказивање у Хилберовој геометрији }

Постоји велики број формализација фрагмената различитих геометрија у
оквиру инте\-ра\-кти\-вних доказивача теорема.  Први покушај да се
формализује \emph{прва група Хилбертових аксиома и њихове последице}
је био у оквиру асистената за доказивање теорема \emph{Coq}, у
интуиционистичком окружењу \cite{hilbert-coq}. Следећи покушај је био
у систему \emph{Isabelle/Isar} и ову формализацију су радили Mеикле
(фр.~\emph{Meikle}) и Флорио (фр.~\emph{Fleuriot})
\cite{hilbert-isabelle}. Аутори оповргавају уобичајено мишљење да су
Хилбертови докази мање интуитивни, а више ригорозни. Важан закључак је
да је Хилберт користио бројне претпоставке које у формализацији са
рачунаром нису могле да буду направљене и стога су морале да буду
формално верификоване и оправдане. Наставак ове формализације, пратећи
Хилбертову књигу "Основи геометрије" \cite{hilbert}, урадио је Скот у
оквиру своје мастер тезе \cite{scott2008mechanising}.

У раду \cite{william} је \emph{предложен минималан скуп Хилбертових
  аксиома} и за модел је коришћена теорија скупова. Изведена су и
формално показана основна својства и тврђења у овом моделу.

\subsection{Интерактивно доказивање у геометрији Тарског}

\emph{Велике делове геометрије Тарског} \cite{tarski} је формализовао
Нарбу (фра.~\emph{Narboux}) у систему \emph{Coq}
\cite{narboux-tarski}. Бројна геометријска својства су изведена,
доказано је више облика Пашове (нем.~\emph{Pacsh}) аксиоме, показана
су бројна својства релације \emph{подударно} и релације
\emph{између}. Рад се завршава доказом о постојању средишта тачке
дужи.

У оквиру своје мастер тезе Макариос (енг.~\emph{Makarios}) је показао
\emph{независност аксиоме паралелности} \cite{makarios2012}. За
доказивање је изабрао аксиоматски систем Тарског зато што је тај
систем категоричан. Да би могао да покаже независност, прво је
формализовао аксиоме Тарског у оквиру система \emph{Isabelle}.  Онда
је формализовао и Клајн Белтрами модел (енг.~\emph{Klein--Beltrami
  model}) нееуклидске геометрије Тарског и показао је да је ово модел
за све аксиоме Тарског осим за аксиому паралелности, односно Еуклидову
аксиому. На тај начин је показао да је ова аксиома независна од
осталих аксиома Тарског за планарну геометрију. За неке аксиоме
Тарског су у литератури недостајали докази да Клајн--Белтрами модел
задовољава аксиому или су докази били некомплетни, па је рад попунио
ове празнине. Као део рада, дефинисана је реална пројективна раван у
систему \emph{Isabelle/HOL} и показане су неке њене карактеристике.

Формализацију еквиваленције између различитих верзија \emph{Еуклидовог
  петог постулата} дали су Бутри (фра.~\emph{Boutry}), Нарбу и Шрек
(фра.~\emph{Schreck}) \cite{boutry2015parallel}. Овај постулат је
посебно значајан јер је било много покушаја да се он докаже. Наиме,
иако је пети постулат Еуклид записао као аксиому, веома рано је
настала идеја да би он могао да се изведе из прва четири
постулата. Бројни покушаји да се постулат докаже су били погрешни јер
су се у доказима често користиле претпоставке које нису биле
доказане. Аутори су у раду показали да је десет различитих тврђења
еквивалентно Еуклидовом петом постулату. Такође, они су у раду
разматрали како избор различитих верзија Еуклидовог постулата утиче на
проблем одлучивања у геометријским доказима.

У раду \cite{braun2012tarski} аутори су формализовали \emph{првих
  дванаест поглавља књиге "Математички методи у геометрији"}
\cite{tarski} и на основу доказаних својстава су механички успели да
докажу да се аксиоме Хилберта могу извести из аксиома Тарског. Браун
(фра.~\emph{Braun}) и Нарбу су формализовали синтетички доказ Папусове
теореме у геометрији Тарског \cite{braun2015synthetic}. Ова теорема је
веома важна за \emph{конструкцију координатне равни и представља један
  од важних корака у успостављању везе између аналитичке и синтетичке
  геометрије}. Ова веза је посебно важна јер омогућава коришћење
алгебарског приступа у аутоматском закључивању у геометрији. Поред
појмова који су дати у књизи, аутори су дали и формализацију вектора,
четвороуглова, паралелограма, пројекције, оријентације праве и
другог. Наставак овог рада и коначни производ вишегодишњег пројекта
(први рад је објављен 2012. године) приказан је у раду
\cite{boutry2016tarski} из 2016. године. Аутори су завршили
формализацију књиге "Математички методи у геометрији" и у овом раду су
показали како су формализвали последња три поглавља. Ову формализацију
су искористили да \emph{геометријски дефинишу аритметичке операције и
  доказали су да ове операције чине уређено поље. Потом су увели
  Декартову координатну раван и показали су својства основних
  геометријских релација. Ови резултати су веома важни јер оправдавају
  коришћење алгебарских метода за доказивање у геометрији.} Аутори то
и демонстрирају у раду тако што користе метод Гребнерових база да
докажу теорему о девет тачака на кругу.

\subsection{Интерактивно доказивање у неколико различитих геометрија}

Маго (фр.~\emph{Magaud}), Нарбу и Шрек су урадили још једну
формализацију коришћењем система \emph{Coq} и то за \emph{геометрију
  пројективне равни} \cite{projective-coq1,projective-coq2}. Показана
су нека основна својства и доказан је принцип дуалности за пројективну
геометрију. Коначно, доказана је конзистенција аксиома у три модела,
од којих су неки коначни, а неки бесконачни. На крају аутори дискутују
о дегенерисаним случајевима и да би се са њима изборили користе
рангове и монотоност.

Кан (енг.~\emph{Kahn}) je формализовао \emph{вон Платову
  (фин.~\emph{von Plato}) конструктивну геометрију} такође у систему
\emph{Coq} \cite{vonPlato,von-plato-formalization}.

Потом, Гиљо (фра.~\emph{Guilhot}) користећи \emph{Coq} \emph{повезује
  Софтвер за интерактивну геометрију (СИГ) и формално доказивање} у
намери да олакша учење еуклидске геометрије у средњој школи
\cite{guilhot}. Фам (фра.~\emph{Pham}), Берто (фра.~\emph{Bertot}) и
Нарбу су предложили и неколико унапређења \cite{coqly}. Прво је да се
елиминишу сувишне аксиоме коришћењем вектора. Они су додали четири
аксиоме да опишу векторе и још три аксиоме да дефинишу eуклидску раван
и увели су додатне дефиниције да би описали геометријске
концепте. Коришћењем ових аксиома и дефиниција, геометријска својства
су лако доказана. Друго унапређење је коришћење методе површина за
аутоматско доказивање теорема. Да би се формално оправдало коришћење
методе површина, морала је да се конструише Декартова координатна
раван коришћењем геометријских својстава која су раније доказана.

Дупрат (фр.~\emph{Duprat}) формализује \emph{геометрију лењира и
  шестара} \cite{duprat2008}.  Авигад (енг.~\emph{Avigad}) нуди још
једну аксиоматизацију eуклидске геометрије \cite{avigad}. Он полази од
чињенице да eуклидска геометрија описује природније геометријска
тврђења него новије аксиоматизације геометрије. Он сматра да
посматрање слике, односно дијаграма, није пуно мана као што многи
мисле. У намери да ово докаже, уводи систем \emph{Е} у коме су основни
објекти тачке и праве. Аксиоме се користе да опишу својства дијаграма
на основу којих се може закључивати. Аутори такође илуструју логички
оквир у коме се могу изводити докази. У раду су презентовани неки
докази геометријских својстава, као и доказ еквивалентности између
система Тарског за геометрију лењира и шестара и система
\emph{Е}. Дегенерисани случајеви су избегнути коришћењем претпоставки
и стога се доказује само централни случај.

Као део пројекта \emph{Flyspeck}, Харисон је развио веома богату
теорију (која укључује алгебру, топологију и анализу) \emph{Еуклидског
  $n$-димензионог простора $\mathbb{R}^n$} у доказивачу теорема
\emph{HOL Light} \cite{harrison05,harrison2013hol}.

Показани су и различити резултати из \emph{комплексне анализе} у
оквиру доказивача теорема. Милевски (пол.~\emph{Milewski}) је доказао
основну теорему алгебре у систему Мизар \cite{milewski2001fundamental},
Хуверс (фр.~\emph{Geuvers}) са сарадницима је показао исту теорему у
систему \emph{Coq} \cite{geuvers2000constructive}, а Харисон је
имплементирао комплексну елиминацију квантификтора за логику вишег
реда и то је користио у разним формализацијама, укључујући и
формализације геометрије.


\section{Везе између интерактивних и аутоматских доказивача}

Поред формализације геометријских тврђења многи истраживачи су
покушали да формализују аутоматско доказивање у геометрији.

Грегоар (фр.~\emph{Gr\'egoire}), Потје (фр.~\emph{Pottier}) и Тери
(фр.~\emph{Th\'ery}) комбинују модификовану верзију
\emph{Бухбергеровог алгоритма} и неке технике рефлексије да би добили
ефективну процедуру која аутоматски производи формални доказ теорема у
геометрији \cite{grobnercoq}.

Женево (фр.~\emph{G\'enevaux}), Нарбу и Шрек су формализовали
\emph{упрошћен Вуов метод} у систему \emph{Coq} \cite{wucoq}. Њихов
приступ се базира на верификацији сертификата које генерише програм за
упрошћен Вуов метод писан у \emph{Ocaml}.

Фуч (фр.~\emph{Fuchs}) и Тери су формализовали Грасман---Кајл
(енг.~\emph{Grassmann---Cayle}) алгебру y систему \emph{Coq}
\cite{grassman}. Други део рада, који је интересантнији са аспекта
наше формализације, представља \emph{примену алгебре на геометрију
  инциденције}. Тачке, праве и њихови односи су дефинисани у форми
алгебарских операција. Коришћењем ових дефиниција, Папусова теорема и
Дезаргова теорема су интерактивно доказане у систему
\emph{Coq}. Коначно, аутори описују аутоматизацију у систему
\emph{Coq} за доказивање теорема у геометрији коришћењем ове
алгебре. Мане овог приступа су у томе што је могуће показати само она
тврђења где се доказује колинеарност међу тачкама и што се разматрају
само недегенерисани случајеви.

Програме за \emph{одређивање Гребнерове базе}, \emph{F4} и \emph{GB},
презентује Потје \cite{pottier2010connecting} и упоређује их са
\emph{gbcoq} \cite{grobner_coq}. Он предлаже решење са сертификатима и
ово скраћује време које је потребно за израчунавања, тако да
\emph{gbcoq}, иако направљен у систему \emph{Coq}, постаје упоредив са
друга два програма. Примена Гребнерових база на алгебру, геометрију и
аритметику је приказана кроз три примера.

Веома интересантан је рад који су представили Браун и Нарбу који се
бави \emph{проналажењем специјалних тачака троугла и доказивањем
  својстава које те тачке задовољавају}
\cite{narboux2015towards}. Наиме, под руководством Кимберлинга
(енг.~\emph{Kimberling}) направљена је електронска енциклопедија
важних тачака троугла у којој се тренутно налазе дефиниције о више од
7000 тачака, као и својства које те тачке задовољавају. Ипак, ова
својства често немају доказ или је доказ задат неформално. Како се у
енциклопедији налази веома велики број тачака и њихових својстава,
било би прилично напорно ручно преписивати и доказивати сва та
својства. Браун и Нарбу су у оквиру система \emph{Coq} дефинисали
аутоматске методе за дефинисање тачака и аутоматске методе за
доказивање њихових својстава. Веома важну улогу имају геометријске
трансформације које помажу и у налажењу тачака и у доказивању
одговарајућих лема.

Ботри (фр.~\emph{Boutry}), Нарбу и Шрек су у систему \emph{Coq}
\emph{формализовали и имлементирали рефлексивну тактику за аутоматско
  генерисање доказа о инциденцији} \cite{boutry2015reflexive}. Тврђења
о инциденцији се често јављају у формалним доказима разних
геометријских тврђења, али су у доказима који су записани на папиру
често изостављена јер често не доприносе разумевању доказа. Ипак,
приликом формалне верификације у оквиру асистента за доказивање
теорема, леме и докази о инциденцији морају бити записани. Аутори су
представили генеричку тактику која је примењива на било коју теорију
чији је циљ да аутоматски докаже та ситна тврђења. Уједно, ово је
један од низа корака да се формални доказ приближи доказима из
уџбеника у којима се често изостављају "очигледна" тврђења.




























