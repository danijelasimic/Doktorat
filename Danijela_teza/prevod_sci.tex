\chapter{Prevod SCI}

\section{Abstract}
  Тесна веза између комплексних бројева и геометрије је добро позната
  и пажљиво је изучавана вековима уназад. Основни објекти који се
  изучавају су комплексна раван (обично проширена једном бесконачном
  тачком), њени објекти (тачке, праве и кругови) и група
  трансформација која делује на ове објекте (нпр. инверзије и
  Мебијусове трансформације).  У овом раду ми приступамо геометрији
  комплексних бројева формално и представљамо потпуно механички
  верификовану теорију у оквиру доказивача теорема Isabelle/HOL. Поред
  примена у формализацији математике и у математичком образовању, овај
  рад представља основу за формално изучавање различитих не-Еуклидских
  геометрија и за дубљу анализу њихових међусобних веза. Дискутујемо
  различите приступе у формализацији и анализирамо главне предности
  приступа који је више алгебарски оријентисан.

\section{Увод}
Веза између комплексних бројева и геометрије су тесне и дубоке. Иако
су комплексни бројев познати више од 450 година уназад, њихова
геометријска репрезентација је дата тек крајем 18-ог века у радовима
Wessel, Argand и Гауса \cite{needham}. Њихове најзначајније примене у
геометрији су додатно развијали Коши, Риман, Мебијус, Beltrami,
Поинкаре и многи други током 18-ог века \cite{needham}. Комплексни
бројеви представљају веома згодан алат за анализу својстава објеката у
многим различитим геометријама. Аналитичко изучавање геометрије датира
од Декарта и Декартова координантна раван ($\mathbb{R}^2$) је често
узимана као домен за моделе различитих геометрија (посебно је то
случај са Еуклидским геометријама). Међутим, заменом Декартове
координатне равни комплексном равни омогућава да формуле које описују
геометријске објекте буду једноставније и компактније, поједностављује
се израчунавање и баца ново светло на посматрану тематику. Зато,
комплексна раван или неки њени делови (нпр. јединични диск или горња
полураван) се често узимају као домен у којима су модели различитих
геометрија (као еуклидских, тако и нееуклидских) су формализовани.
Комплексна раван је такође важан домен за изучавање модерне физике
(погледати, на пример, Penrose and Rindler \cite{penrose}). Због своје
важности, геометрија комплексних бројева је добро описана у
литератури. Постоје многи уџбеници који описују ову област са много
детаља (током нашег рада, ми смо интезивно користили уџбенике које су
писали Needham \cite{needham} и Schwerdtfeger \cite{schwerdtfeger}).
Такође, постоји велики избор материјала за ову област (слајдова,
белешки, приручника) који су доступни на вебу. Ипак, ми нисмо упознати
да постоји формализација ове области, и у овом раду, ми представљамо
наше поптпуно формално, механички проверено представљање геометрије
комплексне равни које је, према нашем сазнању, прво такве врсте.

Потреба за ригорозним представљањем геометријских аргумената постоји
више од две хиљаде година --- Еуклидови ``Елементи'' су једни од првих
случаја математичке дедукције и представљају један од најлепших и
најутицајних радова науке у историји човечанства. У последњем
веку. радови Хилберта \cite{hilbert} и Тарског \cite{tarski} су нас
обогатили са много прецизнијим развојем синтетичке геометрије. У
последњих неколико деценија, са напретком доказивача теорема и
интерактивник асистената за доказивање теорема, ниво формалности и
ригорозности у геометријском резоновању је подигнут на највиши ниво. У
оквиру заједнице која се бави формалним доказивањем теорема често се
истиче да, поред посматрања формалижације као ларпурлатизам, ипак
постоје бројне користи формализације (пре свега у математичком
образовању). Надамо се да ће све више и више математичара прихватити
овакво гледиште. Кроз историју математике, ниво ригорозности се стално
повећавао и ми сматрамо да ће механички проверавачи теорема помоћи да
се достигне ултимативни идеал потпуно ригорозних доказа. Формална,
механички проверена анализа неке теорије обично попуњава многе
празнине које су често присутне у класичним уџбеницима, и чини да
аутор размишља много дубље и детаљније о теми коју истражује. Наше
искуство у овом раду показује, као што је и обично случај у
формализацији математике, да не постоји много погрешних тврђења у
неформалним текстовима и уџбеницима. Ипак, ми смо у уџбеницима које
смо анализирали наишли на нека нетривијална тврђења која су била
погрешна и која није било могуће показати. Још већи проблем су докази
који су непрецизни, који немају покривене све случајеве или чија важна
нетривијална оправданост није показана.

Коначни резултат нашег рада је добро развијена теорија проширене
комплексне равни (дата као комплексан пројективни простор, али и као
Риманова сфера), њени објекти (кругови и праве) и њене трансформације
(Мебијусове трансформације). Може да служи као веома важан блок за
изградњу будућих формалних модела различитих геометрија (нпр, наша
мотивација за овај рад је била управо у покушају да се формализује
Поинкареов диск модел хиперболичке геометрије). Већина концепата које
смо формализовали већ је описана у литератури (иако је постојало пуно
детаља које смо морали да измислимо јер их нисмо нашли у литератури
који смо разматрали). Ипак, наш рад је захтевао обједињавање
различитих извора у једану јединствену, формалну репрезентацију и
пребацивање у један јединствен језик имајући на уму да је првобитно
било описано на много различитих начина. На пример, чак и у оквиру
истог уџбеника, без икаквог формалног оправдања, аутори често лако
прелазе из једне поставке у другу (рецимо, из обичне комплексне равни
у проширену комплексну раван), прелазе између геометријског и
алгебарског представљања, често користе многе недоказане, нетривијалне
чињенице (посматрајући их као део математичког ``фолклора'') и
др. Један од наших најзначајнијих доприноса је управо расветљавање
ових непрецизности и креирање униформног, јасног и самосталног
материјала.

Додано, ми сматрамо да једнако (или чак више) од коначног резултата је
важно искуство које смо стекли приликом различитих покушаја да
достигнемо коначни циљ. Наиме, постоји много различитих начина како је
област изложена у литератури. Поредећи, на пример, Needham
\cite{needham} и Schwerdtfeger \cite{schwerdtfeger}, представљају два
врло различита начина приказивања исте приче --- један приступ је више
геометријски оријентисан, док је други више алгебарски
оријентисан. Наше искуство показује да избор правог приступа је важан
корак у остваривању циља да формализација буде изводљива у оквиру
асистенат за доказивање теорема --- и показало да што је више приступ
алгебарски оријентисан, то је формализација једноставнија, лепша,
флексибилнија и више робусна.

У овој тези, ради сажетости, ми ћемо представити само основне
резултате наше формализације --- најважније дефиниције и тврђења. Овај
рад садржи само кратку рекапитулацију оргиналног формалног извођења и
многа својства која су формално доказана неће бити презентована у овом
раду. Додано, ни један доказ неће бити показан или описан, а све је
доступно у оквиру званичне Isabelle/HOL
документације \footnote{Isabell документи у којима су теорије и докази
  су доступни на \url{http://argo.matf.bg.ac.rs/formalizations/}}. У
представљању рада најчеће ћемо користити нотацију система
Isabelle/HOL, коју ћемо мало упростити на неким местима да би била
приступачнија широј популацији.


\subsection{Повезани радови}
\label{sec:related}

Током последњих десет година било је много резултата у формализацији
геометрије у оквиру асистената за доказивање теорема. Делови првобитне
Хилбертове књиге "Основи геометрије" \cite{hilbert} су формализовани и
у систему Coq, али и у оквиру система Isabelle/Isar. Формализација
прве две групе аксиома у систему Coq, у интуиционистичком окружењу је
урађена од стране Dehlinger et al. \cite{hilbert-coq}. Прву
формализацију у систему Isabelle/HOL су урадили Флориот и Meikele
\cite{hilbert-isabelle}, а наставак ове формализације је урадио Скот у
оквиру своје мастер тезе \cite{hilbert-scott}. Велики делови
геометрије Тарског \cite{tarski} је формализовао Нарбу at al. у
систему Coq \cite{narboux-tarski}. Kahn je формализовао вон Плато
конструктивну геометрију такође у систему Coq
\cite{vonPlato,von-plato-formalization}, потом, Guilhot је
формализовао геометрију за средње школе у Француској \cite{guilhot},
Duprat геометрију шестра и лењира \cite{ruler-compass}, Magaud et
al. пројективну геометрију ~\cite{narboux-projective}, итд.

У нашем претходном раду \cite{algmethods,analytic}, ми смо
формализовали Декартову координатну раван као модел Еуклидске
геометрије. Тимоти Макариос је показао независност Еуклидске аксиоме
Тарског тако што је формалитовао моделе еуклидске геометрије Тарског и
нееуклидске геоемтрије Тарског (Klein-Beltrami модел)
\cite{makarios}. Као део тог рада, реална пројективна раван је
формализована у систему Isabelle/HOL.

Као део пројекта Flyspeck, Харисон је ражвио веома богату теорију
(која укључује алгебру, топологију и анализу) Еуклидског
$n$-димензионог простора $\mathbb{R}^n$ у доказивачу теорема HOL Light
\cite{harrison05,harrison13}.

Неки аутоматски докзивачи теорема у геометрији су такође интегрисани у
оквиру асистената за доказивање теорема. На пример, Јаничић at
al. описује детаљну формализацију (укључујући и имплементационе
детаље) методе површи \cite{areamethod}. Повезивање алгебарских метода
(Гребнерове базе и Вуова метода) са системом Coq су урадили
Gr{\'e}goire et al.~\cite{certificates} и G\'eneveaux et
al.~\cite{wu-certificates}.

Показани су и различити резултати из комплексне анализе у оквиру
доказивача теорема. Милевски је доказао основу теорему алгебре у
систему Мизар \cite{ftamizar}, Geuvers et al. је показао исту теорему
у систему Coq \cite{ftacoq}, а Харисон је имплементирао комплексну
елиминацију квантификтора за логику вишег реда и то је користио у
разним формализацијама, укључујући и формализације геометрије, и др.

\section{Основни појмови}

\label{sec:background}
У овом одељку ћемо представити доказивач теорема Isabelle/HOL који смо
користили у нашој формализацији, његове основне појмове и
нотацију. Такође, кратко ћемо описати неке основне резултате који су
део наше формализације, али који су доста општи и генерални (неке леме
о комплексним бројевима, и теорија линеарне алгебре у простору
$\mathbb{C}^2$).

\subsection{Isabelle/HOL}
Isabelle \cite{isabelle} је генерички асистент за доказивање теорема,
али његова најразвијенија примена је за логику вишег реда
(Isabelle/HOL). Формализација математичких теорија се састоји из
дефиниције нових појмова (типови, константе, функције и др.), и
доказивања тврђења која за њих важе (леме, теореме и др.). Обично се
користи декларативни језик за доказе Isabelle/Isar \cite{isar}. Isar
је веома богат језик, и ми ћемо овде описати само оне конструкције
које су коришћене у овом раду. Дефиниције се задају коришћењем
синтаксе {\tt {\bf definition} $x$ {\bf where} "$x$ = ..."}, где је
$x$ константа која се дефинише. Леме се задају коришћењем синтаксе
{\tt {\bf lemma} {\bf assumes} $assms$ {\bf shows} $concl$} при чему
је $assms$ претпоставка, а $concl$ је закључак тврђења. Ако нема
претпоставки, кључна реч {\tt {\bf shows}} може бити
изостављена. Такође, користићемо синтаксу {\tt {\bf lemma} "$\bigwedge
  x_1, \ldots x_k.\ \lbrakk$$asm_1$; \ldots; $asm_n$$\rbrakk$
  $\Longrightarrow$ $concl$"} при чему су $asm_1$, \ldots, $asm_n$
претпоставке, $concl$ је закључак, а $x_1$, \ldots, $x_k$ су
универзално квантификоване променљиве.

Логичке формуле су записане у логици вишег реда коришћењем стандардне
нотације (нпр. везници су $\wedge$, $\vee$, $\longrightarrow$, $\neg$,
а квантификатори су $\forall$ и $\exists$). У термовима могу да се
користе let-конструкције (нпр. {\tt let $x=3$ in $3*x$}), а
if-then-else изрази (нпр., {\tt if $x > 0$ then $x$ else $-x$}), у
оквиру стандарне семантике.

HOL je типизирана логика. Да би изразили да је неко $x$ типа $\tau$
пишемо $x :: \tau$. Предефинисан тип {\tt bool} означава Booleans,
{\tt nat} означава природне бројеве, {\tt int} означава целе бројеве,
{\tt real} означава реалне бројеве, док тип {\tt complex} означава
комплексне бројеве. Имагинарна јединица се обележава са $ii$. Ови
типови подржавају основне аритметичке операције (нпр., $+$, $-$, $*$,
$/$). Конверзија из реалног у комплексан број се означава са {\tt
  cor}, реални и имагинарни део комплексног броја са {\tt Re} и {\tt
  Im}, комплексни коњугат са {\tt cnj}, модуо комплексног броја са
$|\_|$, и аргумент комплексног броја са {\tt arg} (у систему
Isabelle/HOL он је увек у интервалу $(-\pi, \pi]$). Комплексна
функција за знак {\tt sgn} одређује комплексан број на јединичној
крућници који има исти аргумент као и дати не-нула комплексан број
(нпр., {\tt sgn} $z$ = $z / |z|$). Ова функција преоптерећена и исто
се примењује и за реалне бројеве (преоптерећење је математички
оправдано у овом случају јер важи {\tt sgn}\ (x + $ii$*0) = {\tt
sgn}\ x).  Функција ${\tt cis}$ примењена на $\alpha$ израчунава
{\tt cos}\ $\alpha$\ $+$ $ii*${\tt sin} $\alpha$.


Типизирани скуп се означава са $\tau\ \mathtt{set}$. Isabelle/HOL
скуповна теорија је веома слична скуповној теорији у стандарној
математици, са неколико мањих изузетака. Разлика скупова се означава
са $X - Y$, а слика функције $f$ над скупом $X$ се записује са $f `
X$. Тип производа је означен са $\tau_1 \times \tau_2$. Тип функције
се значава са као $\tau_1 \Rightarrow \tau_2$. Функције су углавном
curried, а примена функције се углавном записује у префиксној форми,
што је често случај код функционалног програмирања, као {\tt f $x$}
(уместо $f(x)$, што је ближе стандардној математичкој
нотацији). Предикат {\tt inj} означава да је функција ињективна, {\tt
  bij} да је бијекција, а {\tt continuous\_on} да је непрекидна на
датом скупу (претпостављајући да је одговарајући тип метрички простор
за неку дату функцију растојања).

Нов тип се може увести на неколико различитих начина. Најједноставнији
начин је да се користи команда {\tt type\_synonym} која уводи ново име
за већ постојеће типове. Нови типови се могу увести и коришћењем
дефиниција. Сваки нови тип је цпецификован да буде изоморфан са неким
непразним подскупом постојећег типа. На пример, тип се може увести као
{\tt {\bf typedef} three = "\{0::nat, 1, 2\}"}, а то генерише обавезу
да се докаже да је тип непразан. Бијекција између новог абстрактног
типа и његове репрезентације дата је са две функције: {\tt Rep\_three
  :: three $\Rightarrow$ nat}, и {\tt Abs\_three :: nat $\Rightarrow$
  three}, које задовољавају услове {\tt Rep\_three $x$ $\in$ $\{0, 1,
  2\}$}, {\tt Rep\_three (Abs\_three $x$) = $x$}, and {\tt $y \in \{0,
  1, 2\}$ $\Longrightarrow$ Abs\_three (Rep\_three $y$) = $y$}.

Други начин да се уведу нови типови, често коришћен и у математици,
јесу количнички типови. У Isabell/HOL постоји више пакета који
омугућавају рад са количничким типом, и у нашој формализацији ми смо
користили lifting/transfer пакет \cite{lifting-transfer}. Први корак у
дефинисању количничког типа је дефиниција релације еквиваленције
$\approx$ над неким постојећим (репрезентативним) типом
$\tau$. Количнички тип $\kappa$ се онда дефинише са {\tt
  quotient\_type $\kappa$ = $\tau$ / $\approx$}. Функције над
количничким типом се дефинишу у два корака. Прво, функција {\tt
  $f_{\tau}$ :: \ldots $\tau$ \ldots} се дефинише над репрезентативним
типом $\tau$. Онда, функција се подиже на количнички тип коришћењем
{\tt {\bf lift\_definition} $f_{\kappa}$ :: \ldots $\kappa$ \ldots
  {\bf is} $f_{\tau}$}. Ово ствара обавезу да се покаже да дефиниција
не зависи од избора представника. Више детаља се може пронаћи у
литератури \cite{isabelle-quotient,lifting-transfer}.

\subsection{Неке основне теорије}

\paragraph{Комплексни бројеви.}
Иако у систему Isabelle/HOL постоји нека основна подршка за комплексне
бројеве, то није било довољно за наше потребе, па смо морали да
направимо додати напор и да ову теорију проширимо. Многе леме које смо
показали су углавном веома техничке и нису интересантне за виши ниво
формализације коју описујемо и зато их нећемо спомињати у овом тексту
(нпр., {\tt {\bf lemma} "arg $i$ = $pi/2$"} или {\tt {\bf lemma}
  "$|z|^2$ = Re ($z$ * cnj $z$)"}). Једна од најзначајнијих дефиниција
је дефиниција функције за канонизацију угла $\downharpoonright \_
\downharpoonleft$, која узима у обзир $2\pi$ периодичнотс синуса и
косинуса и мапира сваки угао у његову каноничну вредносту која лежи у
оквиру интервала $(-\pi, \pi]$. Са овом функцијом, на пример,
мултипликативна својства функције {\tt arg} могу се лако изразити и
доказати.
{\tt
  \begin{tabbing}
    \hspace{5mm}\=\hspace{5mm}\=\hspace{5mm}\=\hspace{5mm}\=\hspace{5mm}\=\kill
    {\bf lemma} "$z_1 * z_2$ $\neq$ 0 $\Longrightarrow$ arg($z_1 * z_2$) = $\downharpoonright$arg $z_1$ $+$ arg $z_2$$\downharpoonleft$"
  \end{tabbing}
} 

\noindent Како се комплексни бројеви често третирају и као вектори,
увођење скаларног производа између два комплексна броја (што је
дефинисано као $\langle z_1, z_2\rangle = (z_1*\mathtt{cnj}\ z_2 +
z_2*\mathtt{cnj}\ z_1) / 2$) се показало веома корисно за сажето
приказивање неких услова.


\paragraph{Линеарна алгебра.}
Следећа важна теорија за даљу формализацију је теорија линеарне
алгебре $\mathbb{C}^2$.  Представљање вектора и матрица различитих
дизија у логици вишег реда представља изазов, због недостатка зависних
типова \cite{harrison05}, али у нашој формализацији треба само да
разматрамо просторе коначне димензије $\mathbb{C}^2$ и у неким
ситуацијама $\mathbb{R}^3$, тако да је наш задатак био
једноставнији. Комплексни вектори се дефинишу са {\tt {\bf
    type\_synonym} C2\_vec = complex $\times$ complex}. Слично,
комплексне матрице ({\tt C2\_mat}) се дефинишу као четворка
комплексних бројева (матрица $\left(\begin{array}{cc}A & B\\C &
  D\end{array}\right)$ репрезенотана је са $(A, B, C, D)$). Скаларно
  множење вектора означавамо са $*_{sv}$, а скаларно множење са
  матрицом означавамо са $*_{sm}$. Скаларни производ два вектора
  означен је са $*_{vv}$, производ вектора и матрице је означено са
  $*_{vm}$, производ матрице и вектора је означено са $*_{mv}$, а
  производ две матрице са $*_{mm}$. Нула матрица је означена са {\tt
    mat\_zero}, јединична матрица је означена са {\tt eye}, нула
  вектор је означен са {\tt vec\_zero}, детерминанта матрице је
  означена са {\tt mat\_det}, њен траг (сума елемената на главној
  дијагонали) са {\tt mat\_trace}, инверзна матрица са {\tt mat\_inv},
  транспонована матрица са {\tt mat\_transpose}, коњугација сваког
  елемента вектора са {\tt vec\_cnj}, коњугација сваког елемента
  матрице са {\tt mat\_cnj}, итд. Уведени су многи стандарни појмови
  линеарне алгебре. На пример, сопствене вредности су дефинисане и
  карактериоване на следећи начин
{\tt
  \begin{tabbing}
    \hspace{5mm}\=\hspace{5mm}\=\hspace{5mm}\=\hspace{5mm}\=\hspace{5mm}\=\kill
{\bf definition} eigenval :: "C2\_mat $\Rightarrow$ complex $\Rightarrow$ bool" {\bf where}\\
\>"eigenval $k$ $A$ $\longleftrightarrow$ ($\exists v.$ $v$ $\neq$ vec\_zero $\wedge$ $A *_{mv} v = k *_{sv} v$)"\\
{\bf lemma} "eigenval k H $\longleftrightarrow$ $k^2 - \mathtt{mat\_trace}\ H * k + \mathtt{mat\_det}\ H = 0$"
  \end{tabbing}
} 
\emph{Адјунгована} матрица је транспонована коњугована
матрица. \emph{Хермитеове} матрице су оне које су једнаке својој
адјунгованој матрици,док су \emph{унитарне} матрице оне чији инверз је
једнак њиховој адјунгованој матрици.
{\tt
\begin{tabbing}
\hspace{5mm}\=\hspace{5mm}\=\hspace{5mm}\=\hspace{5mm}\=\hspace{5mm}\=\kill
{\bf definition} mat\_adj {\bf where} "mat\_adj $H$ $=$ mat\_cnj (mat\_transpose $H$)"\\
{\bf definition} hermitean {\bf where} "hermitean $H$ $\longleftrightarrow$ mat\_adj $H$ = $H$"\\
{\bf definition} unitary {\bf where} "unitary $M$ $\longleftrightarrow$ mat\_adj $M$ $*_{mm}$  $M$ = eye"
\end{tabbing}
}

Други основни појмови који су потребни у овом раду ће бити уведени у
даљем тексту, а читалац може пронаћи више информација у нашем
оргиналном документу.


\section{Главни резултати}
\label{sec:main}
\subsection{Проширена комплексна раван}
\label{subsec:extc}
Веома важан корак у развоју гомерије комплексне равни је проширена
комплексна раван која има један додатни елемент у односу на комплексну
раван $\C$ (који се третира као тачка бесконачно). Проширену
комплексну раван ћемо означити са $\extC$. Постоји више различитих
приступа \cite{needham,schwerdtfeger} за дефинисање
$\extC$. Најпривлачнији начин са становишта израчунавања је приступ
који се базира на хомогеним координатама, а најпривлачнији приступ
визуелно је заснован на стереографској пројекцији Риманове сфере.


\subsubsection{Хомогене координате}
Проширена комплексна раван $\extC$ се идентификује са комплексном
пројективном линијом (једнодимензиони пројективни простор над
комплексним пољем, понекад означаван са $CP^1$). Свака тачка $\extC$
је репрезентована паром комплексних хомогених координата (од којих
нису оба једнака нули), а два пара хомогених координата представљају
исту тачку у $\extC$ акко су они пропорционални са неким не-нула
комплексним фактором. Формализација овог својства у систему
Isabelle/HOL се ослања на lifting/transfer пакет за колинички тип
\cite{lifting-transfer} и састоји се из три фазе \footnote{Једна од
  фаза може бити прескочена ако би се користио lifting/transfer пакет
  за парцијални колинички тип. Ово својство ми нисмо користили у нашој
  формализацији због неких проблема који су постојали у ранијим
  верзијама количничког пакета. У међувремену, сви проблеми су
  исправљени, али наша формализације је у том тренутку већ увелико
  била развијена и било би заиста мнотоно све мењати из
  почетка.}. Прво се уводи тип за различит од нуле пар комплексних
бројева (који се истовремено посматра и као не-нула комплексни
вектор).

{\tt
\begin{tabbing}
\hspace{5mm}\=\hspace{5mm}\=\hspace{5mm}\=\hspace{5mm}\=\hspace{5mm}\=\kill
{\bf typedef} C2\_vec$_{\neq 0}$ = "\{v::C2\_vec. v $\neq$ vec\_zero\}"
\end{tabbing}
}
\noindent Одавде добијамо функцију за репрезентацију {\tt
  Rep\_C2\_vec$_{\neq 0}$} (коју ћемо означити са $\Repnzv{\_}$) која
враћа (не-нула) пар комплексних бројева за сваки дати елемент помоћног
типа {\tt C2\_vec$_{\neq 0}$} и функцију за абстракцију {\tt
  Abs\_C2\_vec$_{\neq 0}$} (коју ћемо ми означити са $\Absnzv{\_}$)
која враћа {\tt C2\_vec$_{\neq 0}$} елемент за сваки дати не-нула пар
комплексних бројева. Друго, кажемо да су два елемента типа {\tt
  C2\_vec$_{\neq 0}$} еквивалентна акко су њихове репрезентације
пропорционалне.

{\tt
\begin{tabbing}
\hspace{5mm}\=\hspace{5mm}\=\hspace{5mm}\=\hspace{5mm}\=\hspace{5mm}\=\kill
{\bf definition} $\approxhc$ :: "C2\_vec$_{\neq 0}$ $\Rightarrow$ C2\_vec$_{\neq 0}$ $\Rightarrow$ bool" {\bf where}\\
\> "$z_1$ $\approxhc$ $z_2$ $\longleftrightarrow$ ($\exists$ (k::complex). k $\neq$ 0 $\wedge$ $\Repnzv{z_2}$ = k *$_{sv}$ $\Repnzv{z_1}$)"
\end{tabbing}
}

\noindent Прилично је лако показати да $\approxhc$ је релација
еквиваленције. Коначно, тип комплексних бројева проширене комплексне
равни дат хомогеним координатама се дефинише као класа еквиваленције
релације $\approxhc$ и уводи се преко наредног количничког типа.

{\tt
\begin{tabbing}
\hspace{5mm}\=\hspace{5mm}\=\hspace{5mm}\=\hspace{5mm}\=\hspace{5mm}\=\kill
{\bf quotient\_type} complex$_{hc}$ = C2\_vec$_{\neq 0}$ / $\approxhc$
\end{tabbing}
}

Да сумирамо, на најнижем нивоу репрезентације постоји тип комплексних
бројева, на следећем нивоу је тип не+нула комплексних $2\times 2$
вектора (који се представљају претходним типом), а на највишем нивоу
је количнички тип који има класу еквивалеције --- рад са овим
количничким типом (његовом репрезетацијом и апстракцијом) се ради у
позадини, коришћењем пакета lifting/transfer \cite{lifting-transfer}.
Ова три нивоа апстракције могу на математичаре деловати збуњујуће, али
они су неопходни у формалном окружењу где сваки објекат мора имати
јединстевни тип (на пример, често се узима да је $(1, i)$ истовремено
пар комплексних бројева и не-нула комплексни вектор, али у нашој
формализацији $(1, i)$ је пар комплексних бројева, док је $\Absnzv{(1,
  i)}$ не-нула комплексни вектор). У овом тексту користићемо
неагресивну нотацију ($\lfloor\_\rfloor$ и $\lceil\_\rceil$) за
функцију репрезенатције и за функцију апстракције и игноришући ове
ознаке текст може бити разумљивији и сличнији уобичајеним математичким
текстовима.

\paragraph{Обични и бесконачни бројеви.}
Сваки обични комплексни број може бити конверован у проширени
комплексни број.

{\tt
\begin{tabbing}
\hspace{5mm}\=\hspace{5mm}\=\hspace{5mm}\=\hspace{5mm}\=\hspace{5mm}\=\kill
{\bf definition} of\_complex\_rep :: "complex $\Rightarrow$ C2\_vec$_{\neq 0}$" \textbf{where}\\
\>of\_complex\_rep z = \Absnzv{(z, 1)}\\
{\bf lift\_definition} of\_complex :: "complex $\Rightarrow$ complex$_{hc}$" \textbf{is} of\_complex\_rep
\end{tabbing}
}

\noindent Тачка и бесконачности се дефинише на следећи начин
{\tt
\begin{tabbing}
\hspace{5mm}\=\hspace{5mm}\=\hspace{5mm}\=\hspace{5mm}\=\hspace{5mm}\=\kill
{\bf definition} inf\_hc\_rep :: C2\_vec$_{\neq 0}$ {\bf where} inf\_hc\_rep = \Absnzv{(1, 0)}\\
{\bf lift\_definition} $\infty_{hc}$ :: "complex$_{hc}$" is inf\_hc\_rep
\end{tabbing}
}

Лако се показује да су сви проширени комплексни бројеви или
$\infty_{hc}$ (акко је њихова друга координата једнака нули) или се
могу добити конвертовањем обичних комплексних бројева (акко њихова
друга координата није нула).
{\tt
\begin{tabbing}
\hspace{5mm}\=\hspace{5mm}\=\hspace{5mm}\=\hspace{5mm}\=\hspace{5mm}\=\kill
{\bf lemma}  "z = $\infty_{hc}$ $\vee$ ($\exists$\ x.\ z = of\_complex x)"
\end{tabbing}
}

Нотација $0_{hc}$, $1_{hc}$ и $i_{hc}$ се користи да означи проширене
комплексне делове бројева $0$, $1$, и $i$.

\paragraph{Аритметичке операције.} Аритметичке операције обичних
комплексних бројева могу бити проширене на проширену комплексну
раван.

На најнижем, репрезентативном нивоу, сабирање $(z_1, z_2)$ и
$(w_1, w_2)$ се дефинише као $(z_1*w_2 + w_1*z_2, z_2*w_2)$, тј, 

{\tt
\begin{tabbing}
{\bf definition} plus\_hc\_rep :: "C2\_vec$_{\neq 0}$ $\Rightarrow$ C2\_vec$_{\neq 0}$ $\Rightarrow$ C2\_vec$_{\neq 0}$"\\
\hspace{5mm}\={\bf where} "plus\_hc\_rep $z$ $w$ = (l\=et \= ($z_1$, $z_2$) \== $\Repnzv{z}$; ($w_1$, $w_2$) = $\Repnzv{w}$ \\
  \> \> in $\Absnzv{(z_1*w_2 + w_1*z_2, z_2*w_2)})$"
\end{tabbing}
}
\noindent Овим се добија не-нула пар хомогених координата осима ако су
и $z_2$ и $w_2$ нула (у супротном, добија се лоше дефинисани елемент
{\tt $\Absnzv{(0, 0)}$})\footnote{Све функције (укључујући и
  апстрактну функцију $\Absnzv{\_}$) у HOL су тоталне. Ипак, све леме
  о тој функцији које су доказане, садрже један додатни услов, а то је
  да њихов аргумент није $(0, 0)$. Зато, не постоји разлог да
  резонујемо о вредности $\Absnzv{(0, 0)}$ и може се сматрати као лоше
  дефинисана вредност}. Дефиницја је подигнута на ниво количничког
типа

{\tt
\begin{tabbing}
{\bf lift\_definition} $+_{hc}$ :: "complex$_{hc}$ $\Rightarrow$ complex$_{hc}$ $\Rightarrow$ complex$_{hc}$" is plus\_hc\_rep
\end{tabbing}
}
\noindent Ово генерише следећи услов који треба показати $\lbrakk z
\approxhc z'; w \approxhc w'\rbrakk \Longrightarrow z +_{hc} w
\approxhc z' +_{hc} w'$, који се лако доказује анализом случајева да
ли су оба $z_2$ и $w_2$ једнаки нули. Приметимо да због HOL захтева да
све функције буду тоталне, ми не можемо дефинисати функцију само за
добро дефинисане случајеве, и у доказу морамо да посматрамо и лоше
дефинисане случајеве.

Даље, показује се да ова операција проширује уобичајено сабирање
комплексних бројева (операцију $+$ у $\mathbb{C}$).

{\tt
\begin{tabbing}
{\bf lemma} "of\_complex z $+_{hc}$ of\_complex w = of\_complex (z + w)"
\end{tabbing}
}

\noindent Сума обичних комплексних бројева и $\infty_{hc}$ је
$\infty_{hc}$ (ипак, $\infty_{hc} +_{hc} \infty_{hc}$ је лоше
дефинсана).

{\tt
\begin{tabbing}
{\bf lemma} "of\_complex z $+_{hc}$ $\infty_{hc}$ = $\infty_{hc}$"\\
{\bf lemma} "$\infty_{hc}$ $+_{hc}$ of\_complex z = $\infty_{hc}$"
\end{tabbing}
}

Операција $+_{hc}$ је асоцијативна и комутативна, али $\infty_{hc}$
нема инверзни елемент, што прекида лепа алгебарска својства операције
$+$ на $\mathbb{C}$.

Друге аритметичке операције су такође проширене. На најнижем,
репрезентативном нивоу, унарни минус $(z_1, z_2)$ је $(-z_1, z_2)$,
производ $(z_1, z_2)$ и $(w_1, w_2)$ је $(z_1*z_2, w_1*w_2)$, а
реципрочна вредност $(z_1, z_2)$ је $(z_2, z_1)$ -- ове операције су
онда подигнуте на апстрактни количнички тип што производи операције
означене са {\tt uminus}$_{hc}$, $*_{hc}$, и {\tt
  recip}$_{hc}$. Одузимање (означено са $-_{hc}$) је дефинисано
коришћењем $+_{hc}$ и {\tt uminus}$_{hc}$, а дељење (означено са
$:_{hc}$) се дефинише коришћењем $*_{hc}$ и {\tt recip}$_{hc}$. Као и
у случају сабирања, показано је да све ове операције одговарају
обичним операцијама коначне комплексне равни (нпр, {\tt {\bf lemma}
  uminus$_{hc}$ (of\_complex $z$) = of\_complex $(-z)$}). Следеће леме
показују понашање ових операција када се у њима појављује и тачка
бесконачно (приметмо да изрази $0_{hc} *_{hc} \infty_{hc}$,
$\infty_{hc} *_{hc} 0_{hc}$, $0_{hc} :_{hc} 0_{hc}$, и $\infty_{hc}
:_{hc} \infty_{hc}$ су лоше дефинисани).

{\tt
\begin{tabbing}
  {\bf lemma} "uminus$_{hc}$\ $\infty_{hc}$ = $\infty_{hc}$"\\
  {\bf lemma} "recip$_{hc}$\ $\infty_{hc}$ = $0_{hc}$" "recip$_{hc}$\ $0_{hc}$ = $\infty_{hc}$"\\
  {\bf lemma} "$z \neq 0_{hc}$ $\Longrightarrow$ $z$ $*_{hc}$ $\infty_{hc}$ = $\infty_{hc}$ $\ \wedge\ $  $\infty_{hc}$ $*_{hc}$ $z$ = $\infty_{hc}$"\\
  {\bf lemma} "$z \neq 0_{hc}$ $\Longrightarrow$ $z$ $:_{hc}$ $\infty_{hc}$ = $0_{hc}$"\\
  {\bf lemma} "$z \neq \infty_{hc}$ $\Longrightarrow$ $\infty_{hc}$ $:_{hc}$ $z$ = $\infty_{hc}$"
\end{tabbing}
}

Такође, проширен је и комплексни коњугат (на репрезентативном типу
$(z_1, z_2)$ је мапирано на $(\cnj{z_1}, \cnj{z_2})$), што даје
операцију {\tt cnj}$_{hc}$. Веома важна операција у комплексној
геометрији је \emph{инверзија у односу на јединични круг}:

{\tt
\begin{tabbing}
\hspace{5mm}\=\hspace{5mm}\=\hspace{5mm}\=\hspace{5mm}\=\hspace{5mm}\=\kill
{\bf definition} inversion$_{hc}$ :: "complex$_{hc}$ $\Rightarrow$ complex$_{hc}$" {\bf where}\\
\> "inversion$_{hc}$ = cnj$_{hc}$ $\circ$ recip$_{hc}$"
\end{tabbing}
}

Основне особине инверзије се лако доказују.
{\tt
\begin{tabbing}
\hspace{5mm}\=\hspace{5mm}\=\hspace{5mm}\=\hspace{5mm}\=\hspace{5mm}\=\kill
{\bf lemma} "inversion$_{hc}$ $\circ$ inversion$_{hc}$ = $id$"\\
{\bf lemma} "inversion$_{hc}$ $0_{hc}$ = $\infty_{hc}$" "inversion$_{hc}$ $\infty_{hc}$ = $0_{hc}$"
\end{tabbing}
}


\paragraph{Размера и дворазмера.}
Размера и дворазмера су веома важни појмови у пројективној геометрији
и у проширеној комплексној равни (дворазмера се карактерише као
инваријанта Мебијусових трансформација -- основних трансформација у
$\extC$, и могуће је дефинисати праве коришћењем размера и круга
коришћењем дворазмере).

Размера тачака $z$, $v$ и $w$ се обично дефинише као
$\frac{z-v}{z-w}$. Наша дефиниција уводи хомогене координате. {\tt
\begin{tabbing}
\hspace{5mm}\=\hspace{5mm}\=\hspace{5mm}\=\hspace{5mm}\=\hspace{5mm}\=\kill
{\bf definition} ratio\_rep {\bf where} "ratio\_rep z v w =  \\
\>(l\=et ($z_1$, $z_2$) = $\Repnzv{z}$; ($v_1$, $v_2$) = $\Repnzv{v}$; ($w_1$, $w_2$) = $\Repnzv{w}$ \\
\>\>in $\Absnzv{((z_1*v_2 - v_1*z_2)*w_2, (z_1*w_2 - w_1*z_2)*v_2)}$)"\\
{\bf lift\_definition} ratio :: "complex$_{hc}$ $\Rightarrow$ complex$_{hc}$ $\Rightarrow$ complex$_{hc}$ $\Rightarrow$ complex$_{hc}$" \\
\>{\bf is} ratio\_rep
\end{tabbing}
} 
\noindent Приметимо да је ово добро дефинисано у свим случајевима осим
када важи $z=w=v$ или $z=v=\infty_{hc}$ или $z=w=\infty_{hc}$ или
$v=w=\infty_{hc}$ (ипак, у доказима код подизања на количнички тип ови
лоше дефинисани случајеви такође морају бити анализирани). Додатно,
оргинална размера разлика је дефинисана у свим случајевима осим када
$z=w=v$ или $z=\infty_{hc}$ или $v=w=\infty_{hc}$, тако да наша
дефиниција у хомогеним координатама природно проширује оригиналну
дефиницију. Следеће леме показују понашање размере у свим добро
дефинисаним случајевима (одговара оригиналној размери разлика кад год
је она дефинисана). {\tt
\begin{tabbing}
\hspace{5mm}\=\hspace{5mm}\=\hspace{5mm}\=\hspace{5mm}\=\hspace{5mm}\=\kill
{\bf lemma} "$\lbrakk$$z \neq v \vee z \neq w;\ z \neq \infty_{hc};\  v \neq \infty_{hc} \vee w \neq \infty_{hc}$$\rbrakk$ $\Longrightarrow$ \\
\>ratio $z$ $v$ $w$ = ($z$ $-_{hc}$ $v$) $:_{hc}$ ($z$ $-_{hc}$ $w$)"\\
{\bf lemma} $\lbrakk$$v \neq \infty_{hc}$;\ $w \neq \infty_{hc}$$\rbrakk$ $\Longrightarrow$ ratio $\infty_{hc}$ $v$ $w$ = $1_{hc}$\\
{\bf lemma} $\lbrakk$$z \neq \infty_{hc}$;\ $w \neq \infty_{hc}$$\rbrakk$ $\Longrightarrow$ ratio $z$ $\infty_{hc}$ $w$ = $\infty_{hc}$\\
{\bf lemma} $\lbrakk$$z \neq \infty_{hc}$;\ $v \neq \infty_{hc}$$\rbrakk$ $\Longrightarrow$ ratio $z$ $v$ $\infty_{hc}$ = $0_{hc}$
\end{tabbing}
}

\noindent Последње две леме су последице прве леме. Такође, приметимо
да размера не може бити дефинисана на природни начин у случају када су
барем две тачке бесконачно (тако да функција размере остане непрекидна
по свим својим параметрима).

Дворазмера је дефинисана над 4 тачке $(z, u, v, w)$, обично као
$\frac{(z-u)(v-w)}{(z-w)(v-u)}$. Поново, ми је дефинишемо користећи
хомогене координате.

{\tt
\begin{tabbing}
\hspace{3mm}\=\hspace{3mm}\=\hspace{3mm}\=\hspace{3mm}\=\hspace{3mm}\=\kill
{\bf definition} cross\_ratio\_rep {\bf where} "cross\_ratio\_rep z u v w = \\
\>(l\=et \=($z_1$, $z_2$) = \Repnzv{$z$}; ($u_1$, $u_2$) = \Repnzv{$u$};\\
\>\>\> ($v_1$, $v_2$) = \Repnzv{$v$}; ($w_1$, $w_2$) = \Repnzv{$w$} in\\
\>$\Absnzv{(z_1*u_2 - u_1*z_2)*(v_1*w_2 - w_1*v_2), (z_1*w_2 - w_1*z_2)*(v_1*u_2 - u_1*v_2))}$"\\
{\bf lift\_definition} cross\_ratio :: "complex$_{hc}$ $\Rightarrow$ complex$_{hc}$ $\Rightarrow$ \\
\>complex$_{hc}$ $\Rightarrow$ complex$_{hc}$ $\Rightarrow$ complex$_{hc}$" {\bf is} cross\_ratio\_rep
\end{tabbing}
}

Ово је добро дефинисану у свим случајевима осим када $z=u=w$ или
$z=v=w$ или $z=u=v$ или $u=v=w$ (приметимо да бесконачне вредности за
$z$, $u$, $v$ или $w$ су дозвољене, што није случај у оргиналној
формулацији разломка). Нека основна својства дворазмере су дата
следећим лемама.

{\tt
\begin{tabbing}
\hspace{5mm}\=\hspace{5mm}\=\hspace{5mm}\=\hspace{5mm}\=\hspace{5mm}\=\kill
{\bf lemma} "$\lbrakk$$(z \neq u \wedge v \neq w) \vee (z \neq w \wedge u \neq v)$;$z \neq \infty_{hc}$;$u \neq \infty_{hc}$;$v \neq \infty_{hc}$ $w \neq \infty_{hc}$$\rbrakk$\\ 
\> $\Longrightarrow$ cross\_ratio $z$ $u$ $v$ $w$ = $((z-_{hc}u) *_{hc} (v-_{hc}) :_{hc} ((z-_{hc}w) *_{hc} (v-_{hc}u))$"\\
{\bf lemma} "cross\_ratio $z$ $0_{hc}$ $1_{hc}$ $\infty_{hc}$ = $z$"\\
{\bf lemma} "$\lbrakk$ $z_1 \neq z_2$;$z_1 \neq z_3$ $\rbrakk$ $\Longrightarrow$ cross\_ratio $z_1$ $z_1$ $z_2$ $z_3$ = $0_{hc}$"\\
{\bf lemma} "$\lbrakk$ $z_2 \neq z_1$;$z_2 \neq z_3$ $\rbrakk$ $\Longrightarrow$ cross\_ratio $z_2$ $z_1$ $z_2$ $z_3$ = $1_{hc}$"\\
{\bf lemma} "$\lbrakk$ $z_3 \neq z_1$;$z_3 \neq z_2$ $\rbrakk$ $\Longrightarrow$ cross\_ratio $z_3$ $z_1$ $z_2$ $z_3$ = $\infty_{hc}$"
\end{tabbing}
}

\subsubsection{Риманова сфера и стереографска пројекција}
Проширена комплексна раван се може идентификовати са Римановом
(јединичном) сфером коришћењем стереографске пројекције
\cite{needham,schwerdtfeger}. Сфера се пројектује из свог 

сeвeрног пола $N$ на $xOy$ раван (коjу означавамо са
$\mathbb{C}$). Oва проjeкциjа успоставља биjeктивно прeсликавањe
$\stpr{}$ измeђу $\Sigma \setminus N$ и коначнe комплeкснe равни
$\mathbb{C}$. Тачка бeсконачно je дeфинисана као слика од $N$.

У Isabelle/HOL систему, сфeра $\Sigma$ je дeфинисана као нови тип.

{\tt
  \begin{tabbing}
    \hspace{5mm}\=\hspace{5mm}\=\hspace{5mm}\=\hspace{5mm}\=\hspace{5mm}\=\kill
{\bf typedef} riemann\_sphere = "\{$(x, y, z)$::R3\_vec. $x^2+y^2+z^2 = 1$\}"
  \end{tabbing}
}

Као и раниje, ово дeфинишe функциjу {\tt Rep\_riemann\_sphere} (коjа
je означeна са $\Reprs{\_}$) и функциjу {\tt Abs\_riemann\_sphere}
(коjа je означeна са $\Absrs{\_}$) коjа повeзуje тачкe апстрактног
типа ({\tt riemann\_sphere}) и тачкe рeпрeзантативног типа (троjкe
рeалних броjeва). Стeрeографска проjeкциjа сe уводи на слeдeћи начин:

{\tt
\begin{tabbing}
\hspace{4mm}\=\hspace{4mm}\=\hspace{4mm}\=\hspace{4mm}\=\hspace{4mm}\=\kill
{\bf definition} stereographic\_rep :: "riemann\_sphere $\Rightarrow$ C2\_vec$_{\neq 0}$" where \\
 \> "stereographic\_rep $M$ = \\
\>\> (l\=et ($x$, $y$, $z$) = $\Reprs{M}$ \\
\>\>\>  in if ($x$, \= $y$, $z$) $\neq$ (0, 0, 1) then $\Absnzv{(x + i * y,\, 1 - z)}$ \\
\>\>\>  else $\Absnzv{(1,\,0)}$)"\\
{\bf lift\_definition} stereographic :: "riemann\_sphere $\Rightarrow$ complex$_{hc}$" {\bf is}\\
\>stereographic\_rep
\end{tabbing}
}

За свe тачкe, ово je добро дeфинисано (вeктор $(x + i * y, 1 - z)$ je
нeнула jeр $(x, y, z) \neq (0, 0, 1)$, и $(1, 0)$ je очито нeнула).

Инвeрзна стeрeорафска проjeкциjа сe дeфинишe на слeдeћи начин. 

{\tt
\begin{tabbing}
{\bf def}\={\bf inition} inv\_stereographic\_rep :: "C2\_vec$_{\neq 0}$ $\Rightarrow$ riemann\_sphere" {\bf where} \\
  \> "inv\=\_stereographic\_rep $z$ = \\
  \>      \> (l\=et ($z_1$, $z_2$) = $\Repnzv{z}$  \\
  \>      \>   \>in \= if \= $z_2$ = 0 then $\Absrs{(0, 0, 1)}$ \\
  \>      \>   \>    \>else l\=et \= $z$ = $z_1$/$z_2$; $XY$ = (2*$z$)/cor (1+$|z|^2$); $Z$ = ($|z|^2$-1)/(1+$|z|^2$) \\
  \>      \>   \>    \>\> in $\Absrs{(Re\ XY,\ Im\ XY,\ Z)}$)"\\
{\bf lift\_definition} inv\_stereographic :: "complex$_{hc}$ $\Rightarrow$ riemann\_sphere" {\bf is} \\
\>inv\_stereographic\_rep
\end{tabbing}
}
\noindent За свe тачкe, ово je добро дeфинисано (сума квадрата три
координатe je 1 у оба случаjа, па сe можe примeнити функциjа {\tt
  Abs\_riemann\_sphere}).

Вeза измeђу двe функциje je дата слeдeћим лeмама.

{\tt
\begin{tabbing}
\hspace{5mm}\=\hspace{5mm}\=\hspace{5mm}\=\hspace{5mm}\=\hspace{5mm}\=\kill
{\bf lemma} "stereographic $\circ$ inv\_stereographic = id"\\
{\bf lemma} "inv\_stereographic $\circ$ stereographic = id"\\
{\bf lemma} "bij stereographic" "bij inv\_stereographic"\\
\end{tabbing}
}
\noindent Докази нису тeшки, али захтeваjу формализациjу врло
нeзгодних израчунавања.

\paragraph{Тетивно растојање.}
Риманова сфера може бити метрички простор. Најчешћи начин да се уведе
метрички простор је коришћењем \emph{тетивне метрике} -- растојање
ижмеђу две тачке на сфери је дужина тетиве која их спаја.  {\tt
\begin{tabbing}
\hspace{5mm}\=\hspace{5mm}\=\hspace{5mm}\=\hspace{5mm}\=\hspace{5mm}\=\kill
{\bf definition} dist$_{rs}$ :: "riemann\_sphere $\Rightarrow$ riemann\_sphere $\Rightarrow$ real" {\bf where}\\
\>  "dist$_{rs}$ $M_1$ $M_2$ = (l\=et ($x_1$, $y_1$, $z_1$) = $\Reprs{M_1}$; ($x_1$, $y_1$, $z_1$) = $\Reprs{M_2}$\\
\>\>       in norm ($x_1$ - $x_2$, $y_1$ - $y_2$, $z_1$ - $z_2$))"
\end{tabbing}
}

Функција {\tt norm} је уграђена функција и у овом случају она рачуна
Еуклидску векторску норму. Коришћењем (сада већ познате) чињенице да
$\mathbb{R}^3$ је метрички простор (са функциом растојања
$\lambda\ x\ y.\ {\tt norm}(x - y)$), није било тешко показати да је
тип {\tt riemann\_sphere} опремљен са {\tt dist$_{rs}$} метрички
простор, тј. да је он инстанца локала {\tt metric\_space}. Иако је
дефинисана на сфери, тетивна метрика има своју репрезентацију и у
равни.

{\tt
\begin{tabbing}
\hspace{2mm}\=\hspace{5mm}\=\hspace{5mm}\=\hspace{5mm}\=\hspace{5mm}\=\kill
{\bf lemma} {\bf assumes}\\
\>"stereographic $M_1$ = of\_complex $m_1$"  \\
\>"stereographic $M_2$ = of\_complex $m_2$"\\
\>{\bf shows} "dist$_{rs}$ $M_1$ $M_2$ = 2*|$m_1$-$m_2$| / (sqrt (1+|$m_1$|$^2$)*sqrt (1+|$m_2$|$^2$))"\\
{\bf lemma} {\bf assumes} \\
\>"stereographic $M_1$ = $\infty_{hs}$" \\
\>"stereographic $M_2$ = of\_complex $m$"\\
\>{\bf shows} "dist$_{rs}$ $M_1$ $M_2$ = 2 / sqrt (1+|$m$|$^2$)"\\
{\bf lemma} {\bf assumes} "stereographic $M_1$ = of\_complex $m$" \\
\>"stereographic $M_2$ = $\infty_{hs}$"\\
\>{\bf shows} "dist$_{rs}$ $M_1$ $M_2$ = 2 / sqrt (1+|$m$|$^2$)"\\
{\bf lemma} {\bf assumes} "stereographic $M_1$ = $\infty_{hs}$" "stereographic $M_2$ = $\infty_{hs}$"\\
\>{\bf shows} "dist$_{rs}$ $M_1$ $M_2$ = 0"
\end{tabbing}
}

Ове леме праве разлику између коначних и бесконачних тачака, али се
ова анализа случаја може избећи коришћењем хомогених координата.

{\tt
\begin{tabbing}
\hspace{5mm}\=\hspace{5mm}\=\hspace{5mm}\=\hspace{5mm}\=\hspace{5mm}\=\kill
{\bf definition} "$\llangle z, w \rrangle$ = (vec\_cnj $\Repnzv{z}$) $*_{vv}$ ($\Repnzv{w}$)"\\
{\bf definition} "$\llangle z \rrangle$ = sqrt (Re $\llangle z, z\rrangle$)"\\
{\bf definition} "dist\_hc\_rep = 2*sqrt(1 - |$\llangle z,w \rrangle$|$^2$ / ($\llangle z \rrangle^2$ $*$ $\llangle w \rrangle^2$))"\\
{\bf lift\_definition} dist$_{hc}$ :: "complex$_{hc}$ $\Rightarrow$ complex$_{hc}$ $\Rightarrow$ real {\bf is} dist\_hc\_rep\\
{\bf lemma} "dist$_{rs}$ $M_1$ $M_2$ = dist$_{hc}$ (stereographic $M_1$) (stereographic $M_2$)"
\end{tabbing}
}
\noindent Понекад, ова форма се зове Fubini-Study метрика.

Тип {\tt complex$_{hc}$} опремљен са {\tt dist$_{hc}$} метриком је
такође инстанца локала {\tt metric\_space}.  Ово тривијално следи из
последње леме која је повезује са метричким простором на Римановој
сфери. Постоје и директни докази ове чињенице (нпр. Hille \cite{hille}
даје директан доказ захваљујући Shizuo Kakutani, али доказ је
некомплетан јер занемарује могућност да једна тачка буде бесконачно) а
ми смо и те директне доказе формализовали\footnote{Наша формализација
  је започета без анализирања Риманове сфере, тако да смо у почетку
  једино и могли користити директне доказе, али у неком тренутку увели
  смо појам Риманове сфере и то је помогло да се многи докази упросте,
  укључујући и овај.}. Испоставило се да је нека својства лакше
показати на Римановој сфери коришћењем функције {\tt dist$_{rs}$}
(нпр. неједнакост троугла), али нека својста је било лакше показати у
пројекцији коришћењем функције {\tt dist$_{hc}$} (нпр. да је метрички
простор савршен, тј. да нема изолованих тачака), што показује значај
постојања различитих модела за исти концепт.

Коришћењем тетивне метрике у проширеној комплексној равни, и Еуклидске
метрике на сфери у $\mathbb{R}^3$, показано је да су стереографска
пројекција и инверзна стереографска пројекција непрекидне.


{\tt
  \begin{tabbing}
    \hspace{5mm}\=\hspace{5mm}\=\hspace{5mm}\=\hspace{5mm}\=\hspace{5mm}\=\kill
{\bf lemma} \="continuous\_on UNIV stereographic" \\
\> "continuous\_on UNIV inv\_stereographic"
  \end{tabbing}
}


Приметимо да у претходној леми, метрика је имплицитна (у систему
Isabelle/HOL претпоставља се да коришћена метрика је управо она
метрика која је коришћена да се покаже да је дати тип инстанца локала
{\tt metric\_space}).

\subsection{Мебијусове трансформације}
\label{subsec:mobius}
Мебијусове трансформације (које се јо називају и холоморфна, linear
fractional или билинеарна пресликавања) су основне трансформације
проширене комплексне равни. У нашој формализацији оне су уведене
алгебарски. Свака трансформација је представљена регуларном
(не-сингуларном, не-дегенерисаном) $2\times 2$ матрицом која ленеарно
делује на хомогене координате. Како пропорционалне хомогене координате
представљају исту тачку у $\extC$, тако и пропорционалне матрице
представљају исту Мебијусову трансформацију. Поново, формализација се
састоји из три корака коришћењем lifting/transfer пакета. Прво, уводи
се тип регуларних матрица.

{\tt
\begin{tabbing}
{\bf typedef} C2\_mat\_reg = "\{$M$ :: C2\_mat. mat\_det $M$ $\neq$ 0\}"
\end{tabbing}
}

\noindent Функција репрезентације {\tt Rep\_C2\_mat\_reg} ће бити
означена са $\Reprm{\_}$, а апстрактна функција {\tt
  Abs\_C2\_mat\_reg} ће бити означена са $\Absrm{\_}$. Регуларне
матрице формирају групу у односу на множење и она се често назива
\emph{генерална линеарна група} и означава се са $GL(2, \C)$. У неким
случајевима се разматра само њена подгрупа, \emph{специјална линерана
  група}, означена са $SL(2, \C)$, која садржи само оне матрице чија
је детерминанта једнака 1.

\paragraph{Мебијусова група.}
Кажемо да су две регуларне матрице еквивалентне акко су њихове
репрезентације пропорционалне.  {\tt
  \begin{tabbing}
    \hspace{5mm}\=\hspace{5mm}\=\hspace{5mm}\=\hspace{5mm}\=\hspace{5mm}\=\kill
{\bf definition} $\approxrm$ :: "C2\_mat\_reg $\Rightarrow$ C2\_mat\_reg $\Rightarrow$ bool" where  \\
\> "$M_1$ $\approxrm$ $M_2$ $\longleftrightarrow$ ($\exists$ (k::complex). k $\neq$ 0 $\wedge$ $\Reprm{M_2}$ = k *$_{sm}$ $\Reprm{M_1}$)"
  \end{tabbing}
}

\noindent Лако се показује да је ово релација еквиваленције. Елементи
Мебијусове групе се уводе као класа еквиваленције над овом релацијом.

{\tt
\begin{tabbing}
{\bf quotient\_type} mobius = C2\_mat\_reg / $\approxrm$
\end{tabbing}
}

\noindent Понекад ћемо користити помоћни конструктор {\tt mk\_mobius}
који враћа елемент Мебијусове групе (класу еквиваленције) за дата 4
комплексна параметра (што има смисла само када је одговарајућа матрица
регуларна).

Мебијисови елементи формирају групу над композицијом. Ова група се
назива \emph{пројективна генерална линерна група} и означена је са
$PGL(2, \mathbb{C})$. Поново, могу се разматрати само они елементи
$SGL(2, \mathbb{C})$ чија детерминанта је једнака $1$. Композиција
Мебијусових елемената се постиже множењем матрица које их
репрезентују.

{\tt
\begin{tabbing}
{\bf def}\={\bf inition} mobius\_comp\_rep :: "C2\_mat\_reg $\Rightarrow$ C2\_mat\_reg $\Rightarrow$ C2\_mat\_reg" \\
\> {\bf where} "moe\=bius\_comp\_rep $M_1$ $M_2$ = $\Absrm{\Reprm{M_1}\ *_{mm}\ \Reprm{M_2}}$"\\
{\bf lift\_definition} mobius\_comp :: "mobius $\Rightarrow$ mobius $\Rightarrow$ mobius" {\bf is}\\
\>mobius\_comp\_rep
\end{tabbing}
}

\noindent Слично, инверзна Мебијусова трансформација се добија
инверзијом матрице која је редставља.

{\tt
\begin{tabbing}
{\bf def}\={\bf inition} mobius\_inv\_rep :: "C2\_mat\_reg $\Rightarrow$ C2\_mat\_reg" {\bf where} \\
         \>"mobius\_inv\_rep $M$ = $\Absrm{\mathtt{mat\_inv}\ \Reprm{M}}$"\\
{\bf lift\_definition} mobius\_inv :: "mobius $\Rightarrow$ mobius" is "mobius\_inv\_rep"
\end{tabbing}
}
\noindent Коначно, Мебијусова трансформација која је идентитет је
представљена јединичном матрицом.  {\tt
\begin{tabbing}
{\bf definition} mobius\_id\_rep :: "C2\_mat\_reg" {\bf where} "mobius\_id\_rep = $\Absrm{\mathtt{eye}}$"\\
{\bf lift\_definition} mobius\_id :: "mobius" is mobius\_id\_rep
\end{tabbing}
}

Све ове дефиниције увек уводе добро дефинисане објекте (јер је
производ регуларних матрица регуларна матрица, а инверз регуларне
матрице је такође регуларна матрица). Докази који су обавезни да би се
дефиниција могла подићи (нпр. {\tt $M_1$ $\approxrm$ $M_2$
  $\Longrightarrow$ mobius\_inv\_rep $M_1$ $\approxrm$
  mobius\_inv\_rep $M_2$}) се лако показују. Онда, показује се да је
тип {\tt mobius} заједно са овим операцијама инстанца локала {\tt
  group\_add} који је већ уграђен у систем Isabelle/HOL. Зато, ми ћемо
понекад означавати {\tt mobius\_comp} са $+$, {\tt mobius\_inv} са
унарним $-$, и {\tt mobius\_id} са $0$.

\paragraph{Дејство Мебијусове групе.}
Мебијусове трансформације су дате као дејство Мебијусове групе на
тачке проширене комплексне равни (које су дате у хомогеним
координатама).

{\tt
\begin{tabbing}
\hspace{5mm}\=\hspace{5mm}\=\hspace{5mm}\=\hspace{5mm}\=\hspace{5mm}\=\kill
{\bf definition} mobius\_pt\_rep :: "C2\_mat\_reg $\Rightarrow$ C2\_vec$_{\neq 0}$ $\Rightarrow$ C2\_vec$_{\neq 0}$"  \\
\> {\bf where} "moe\=bius\_pt\_rep $M$ $z$ = $\Absnzv{\Reprm{M}\ *_{mv}\ \Repnzv{z}}$"\\
{\bf lift\_definition} mobius\_pt :: "mobius $\Rightarrow$ complex$_{hc}$ $\Rightarrow$ complex$_{hc}$" {\bf is}\\
\> mobius\_pt\_rep
\end{tabbing}
}
\noindent Како производ регуларне матрице и не-нула вектора је увек
не-нула вектор, резултат је увек добро дефинисан. Подизање дефиниција
генерише обавезан доказ {\tt $\lbrakk$$M$ $\approxrm$ $M'$; $z$
  $\approxhc$ $z'$$\rbrakk$ $\Longrightarrow$ mobius\_pt\_rep $M$ $z$
  $\approxhc$ mobius\_pt\_rep $M'$ $z'$} који се прилично лако
показује.

Када се узима у обзир дејство групе на проширену комплексну раван,
онда се може видети да операције групе заиста одговарају композицији
мапирања, инверзном мапирању и идентичном мапирању.
{\tt
\begin{tabbing}
{\bf lemma} "mobius\_pt (mobius\_comp $M_1$ $M_2$) = (mobius\_pt $M_1$) $\circ$ (mobius\_pt $M_2$)"\\
{\bf lemma} "mobius\_pt (mobius\_inv $M$) = inv (mobius\_pt $M$)"\\
{\bf lemma} "mobius\_pt (mobius\_id) = id"
\end{tabbing}
}
Дејство је транзитивно (јер је увек бијективно пресликавање). 
{\tt
\begin{tabbing}
{\bf lemma} "bij (mobius\_pt M)"
\end{tabbing}
}

У класичној литератури Мебијусове трансформације се обично
представљају у форми $\frac{az+b}{cz+d}$, и наредна лема заиста
оправдава и овакав запис (али са специјалним случајем када је $z$
тачка бесконачно).
\begin{tabbing}
{\bf lemma} \={\bf assumes} "mat\_det ($a$, $b$, $c$, $d$) $\neq$ 0" \\
  \>{\bf shows} "moeb\=ius\_pt (mk\_mobius $a$ $b$ $c$ $d$) $z$ =  \\
  \>                    \>(if $z$ \= $\neq$ $\infty_{hc}$ then  \\
  \>                    \>        \>((of\_complex $a$) $*_{hc}$ $z$ $+_{hc}$ (of\_complex $b$)) $:_{hc}$  \\
  \>                    \>        \>((of\_complex $c$) $*_{hc}$ $z$ $+_{hc}$ (of\_complex $d$)) \\
  \>                    \>else (of\_complex $a$) $:_{hc}$ (of\_complex c))"
\end{tabbing}
}

Произвољна трансформација у $\extC$ ће бити звана Мебијусовом
трансформацијом акко је она дејство неког елемента Мебијусове групе.
{\tt
\begin{tabbing}
\hspace{5mm}\=\hspace{5mm}\=\hspace{5mm}\=\hspace{5mm}\=\hspace{5mm}\=\kill
{\bf definition} is\_mobius :: "(complex$_{hc}$ $\Rightarrow$ complex$_{hc}$) $\Rightarrow$ bool" {\bf where}\\
\>"is\_mobius $f$ $\longleftrightarrow$ ($\exists$ $M$. $f$ = mobius\_pt $M$)"
\end{tabbing}
}

Приметимо да већина до сада изнетих резултата зависи од чињенице да је
матрица репрезентације Мебијусове трансформације регуларна -- у
супротном, дејство би било дегенерисано и целу раван $\extC$ би
сликало у једну тачку.

\paragraph{Неке специјалне Мебијусове трансформације.}
Многе трансформације са којима се сусрећемо у геометрији су заправо
специјална врста Мебијусових трансформација. Веома важна погрупа је
група \emph{Еуклидских сличности} (које се још називају и
\emph{интегралне трансформације}). Оне су одређене са два комплексна
параметра (и представљају Мебијусову трансформацију када први од та
два параметра није нула). {\tt
\begin{tabbing}
\hspace{5mm}\=\hspace{5mm}\=\hspace{5mm}\=\hspace{5mm}\=\hspace{5mm}\=\kill
{\bf definition} similarity :: "complex $\Rightarrow$ complex $\Rightarrow$ mobius" where \\
\>"similarity $a$ $b$ = mk\_mobius $a$ $b$ 0 1"
\end{tabbing}
}
\noindent Сличности формирају групу (која се понекад назива и
\emph{параболичка група}).  {\tt
\begin{tabbing}
\hspace{5mm}\=\hspace{5mm}\=\hspace{5mm}\=\hspace{5mm}\=\hspace{5mm}\=\kill
{\bf lemma} "$\lbrakk$$a\neq 0$; $c \neq 0$$\rbrakk$ $\Longrightarrow$ mobius\_comp (similarity $a$ $b$) (similarity $c$ $d$) = \\
\>similarity $(a*c)$ $(a*d+b)$"\\
{\bf lemma} "$a \neq 0$ $\Longrightarrow$ mobius\_inv (similarity $a$ $b$) = similarity $(1/a)$ $(-b/a)$"\\
{\bf lemma} "id\_mobius = similarity $1$ $0$"\\
\end{tabbing}
}

Њихово дејство је линеарна трансформација $\C$, а свака линеарна
трансформација $\C$ која није константна је дејство елемента групе
Еуклидских сличности.  {\tt
\begin{tabbing}
\hspace{5mm}\=\hspace{5mm}\=\hspace{5mm}\=\hspace{5mm}\=\hspace{5mm}\=\kill
{\bf lemma} "$a \neq 0$ $\Longrightarrow$ \=mobius\_pt (similarity $a$ $b$) = \\
\>($\lambda$ $z$. (of\_complex $a$) $*_{hc}$ $z$ $+_{hc}$ (of\_complex $b$))"
\end{tabbing}
}

Еуклидске сличности су једини елементи Мебијусове групе такви да је
тачка $\infty_{hc}$ фиксна тачка.  {\tt
\begin{tabbing}
\hspace{5mm}\=\hspace{5mm}\=\hspace{5mm}\=\hspace{5mm}\=\hspace{5mm}\=\kill
{\bf lemma} "mobius\_pt $M$ $\infty_{hc}$ = $\infty_{hc}$ $\longleftrightarrow$ ($\exists$ $a$ $b$. $a \neq 0$ $\wedge$ $M$ = similarity $a$ $b$)"
\end{tabbing}
}

Ако су и тачка $\infty_{hc}$ и тачка $0_{hc}$ фиксне, онда је то
сличност са коефицијентима $a$ и $b=0$, а дејство је облика {\tt
  $\lambda$ $z$. (of\_complex $a$) $*_{hc}$ $z$}.

{\tt
\begin{tabbing}
  \hspace{5mm}\=\hspace{5mm}\=\hspace{5mm}\=\hspace{5mm}\=\hspace{5mm}\=\kill
  {\bf lemma} "mobius\_pt $M$ $\infty_{hc}$ = $\infty_{hc}$ $\wedge$ mobius\_pt $M$ $0_{hc}$ = $0_{hc}$ $\longleftrightarrow$ \\
\>($\exists$ $a$. $a \neq 0$ $\wedge$ $M$ = similarity $a$ $0$)"
\end{tabbing}
}

Еуклидске сличности укључују танслацију, ротацију и дилатацију и свака
Еуклидска сличност се може добити композијом ова три пресликавања.
{\tt
\begin{tabbing}
\hspace{5mm}\=\hspace{5mm}\=\hspace{5mm}\=\hspace{5mm}\=\hspace{5mm}\=\kill
{\bf definition}\ "translation $v$ \== similarity 1 $v$"\\
{\bf definition}\ "rotation $\phi$ \>= similarity (cis $\phi$) 0"\\
{\bf definition}\ "dilatation $k$ \>= similarity (cor $k$) 0"\\
{\bf lem}\={\bf ma} "$a \neq 0$ $\Longrightarrow$ similarity $a$ $b$ = \\
\>(translation $b$) + (rotation (arg $a$)) + (dilatation $|a|$)"
\end{tabbing}
}

Реципрочна вредност ($1_{hc}:_{hc}z$) је такође Мебијусова трансформација. 
{\tt
\begin{tabbing}
\hspace{5mm}\=\hspace{5mm}\=\hspace{5mm}\=\hspace{5mm}\=\hspace{5mm}\=\kill
{\bf definition} "reciprocation = mk\_mobius (1, 0, 0, 1)"\\
{\bf lemma} "recip$_{hc}$ = mobius\_pt reciprocation"
\end{tabbing}
}
\noindent Са друге стране, инверзија није Мебијусова трансформација
(то је основни пример такозваних анти-Мебијусових трансформација, или
антихоломорфне функције). 

Веома важна чињеница је да свака Мебијусова трансформација се може
добити композицијом Еуклидских сличности и реципрочне функције. Један
од начина како се ово може постићи дат је следећом лемом (када је
$c=0$ је случај Еуклидских сличности и ово је раније већ анализирано).

{\tt
\begin{tabbing}
\hspace{5mm}\=\hspace{5mm}\=\hspace{5mm}\=\hspace{5mm}\=\hspace{5mm}\=\kill
{\bf lemma} \= {\bf assumes} "$c\neq 0$" and "$a*d - b*c \neq 0$"\\
{\bf shows} "mk\_mobius a b c d = \\
\>\>\> translation (a/c) + rotation\_dilatation ((b*c - a*d)/(c*c)) + \\
\>\>\> reciprocal + translation (d/c)"
\end{tabbing}
}

\noindent Декомпозиција је коришћена у многим доказима. Наиме, да би
показали да свака Мебијусова трансформација има неко својство, довољно
је показати да реципрочна функција и Еуклидске сличности заадовољавају
то својство и да композиција чува то својство (обично, најтеже је
показати у случају реципрочне функције, а остала два корака буду
углавном много једноставнија).

{\tt
\begin{tabbing}
\hspace{5mm}\=\hspace{5mm}\=\hspace{5mm}\=\hspace{5mm}\=\hspace{5mm}\=\kill
{\bf lemma} {\bf assumes} "$\bigwedge$ $v$. $P$ (translation $v$)" "$\bigwedge$ $\alpha$. $P$ (rotation $\alpha$)"\\
\> "$\bigwedge$ $k$. $P$ (dilatation $k$)" "$P$ (reciprocation)"\\
\> "$\bigwedge$ $M_1$ $M_2$. $\lbrakk$ $P$ $M_1$; $P$ $M_2$ $\rbrakk$ $\Longrightarrow$ $P$ $(M_1 + M_2)$"\\
\>{\bf shows} "$P$ $M$"
\end{tabbing}
}

\paragraph{Дворазмера као Мебијусова трансформација}
За било које три фиксне тачке $z_1$, $z_2$ и $z_3$, {\tt cross\_ratio
  $z$ $z_1$ $z_2$ $z_3$} се може посматрати као фунцкија једне
променљиве $z$. Следећа лема гарантује да је ова функција Мебијусова
трансформација и да према особина дворазмере она слика $z_1$ у
$0_{hc}$, $z_2$ у $1_{hc}$ и $z_3$ у $\infty_{hc}$.

{\tt
\begin{tabbing}
\hspace{5mm}\=\hspace{5mm}\=\hspace{5mm}\=\hspace{5mm}\=\hspace{5mm}\=\kill
{\bf lemma} "$\lbrakk$ $z_1 \neq z_2; z_1 \neq z_3; z_2 \neq z_3$ $\rbrakk$ $\Longrightarrow$ \\
\>is\_mobius ($\lambda$ $z$. cross\_ratio $z$ $z_1$ $z_2$ $z_3$)"
\end{tabbing}
}

Имајући ово, дворазмера се може користити да се покаже да постоји
Мебијусова трансформација која слика било које три различите тачке
редом у $0_{hc}$, $1_{hc}$ и $\infty_{hc}$. Како Мебијусове
трансформације чине групу, једноставна последица овога је да постоји
Мебијусова трансформација која слика било које три различите тачке у
било које три различите тачке.

{\tt
\begin{tabbing}
\hspace{5mm}\=\hspace{5mm}\=\hspace{5mm}\=\hspace{5mm}\=\hspace{5mm}\=\kill
{\bf lemma} "$\lbrakk$ $z_1 \neq z_2$; $z_1 \neq z_3$; $z_2 \neq z_3$ $\rbrakk$ $\Longrightarrow$ $(\exists$ $M$. mobius\_pt $M$ $z_1$ = $0_{hc}$ $\wedge$ \\
\> mobius\_pt $M$ $z_2$ = $1_{hc}$ $\wedge$ mobius\_pt $M$ $z_3$ = $\infty_{hc}$)"
\end{tabbing}
}

Следећа лема има веома важну примену у даљем развоју теорије јер
омогућава закључивање "без губитка на општости (бгно)"
\cite{wlog}. Наиме, ако Мебијусова трансформација чува неко својство,
онда уместо три произвољне тачке може се посматрати само случај
специјалних тачака $0_{hc}$, $1_{hc}$, и $\infty_{hc}$.

{\tt
\begin{tabbing}
\hspace{5mm}\=\hspace{5mm}\=\hspace{5mm}\=\hspace{5mm}\=\hspace{5mm}\=\kill
{\bf lemma} {\bf assumes} "$P$ $0_{hc}$ $1_{hc}$ $\infty_{hc}$" "$z_1 \neq z_2$" "$z_1 \neq z_3$" "$z_2 \neq z_3$"\\
\>"$\bigwedge$ $M$ $u$ $v$ $w$.\ \=$P$ $u$ $v$ $w$ $\Longrightarrow$ \\
\>\> $P$ (mobius\_pt $M$ $u$) (mobius\_pt $M$ $b$) (mobius\_pt $M$ $c$)"\\
\>{\bf shows} "$P$ $z_1$ $z_2$ $z_3$"
\end{tabbing}
}

Једна од првих примена "бгно" резоновања за Мебијусове трансформације
је у анализи фиксних тачака Мебијусових трансформација. Лако се
показује да једино идентично пресликавање има фиксне тачке $0_{hc}$,
$1_{hc}$, и $\infty_{hc}$. Такође важи да ако Мебијусова
трансформација $M$ има три различите фиксне тачке, онда је она
идентитет, али директан доказ овога се заснива на чињеници да $2\times
2$ матрица има највише два независна сопствена вектора, а овакво
закључивање се лако може избећи коришћењем "бгно" резоновања (како
било које три тачке можемо сликати редом у $0_{hc}$, $1_{hc}$, и
$\infty_{hc}$ неким пресликавањем $M'$, а онда $M'+M-M'$ има ове три
тачке фиксне па мора бити једнако $0$).

{\tt
\begin{tabbing}
\hspace{5mm}\=\hspace{5mm}\=\hspace{5mm}\=\hspace{5mm}\=\hspace{5mm}\=\kill
{\bf lemma} "$\lbrakk$ mobius\_pt $M$ $0_{hs}$ = $0_{hs}$; mobius\_pt $M$ $1_{hs}$ = $1_{hs}$; \\
\>mobius\_pt $M$ $\infty_{hs}$ = $\infty_{hs}$ $\rbrakk$ $\Longrightarrow$ M = id\_mobius"\\
{\bf lemma} "$\lbrakk$ mobius\_pt $M$ $z_1$ = $z_1$; mobius\_pt $M$ $z_2$ = $z_2$; \\
\>mobius\_pt $M$ $z_3$ = $z_3$; $z_1 \neq z_2$; $z_1 \neq z_3$; $z_2 \neq z_3$ $\rbrakk$ $\Longrightarrow$ M = id\_mobius"
\end{tabbing}
}

Последица овога је да постоји јединствена Мебијусова трансформација
која слика три различите тачке у друге три различите тачке (већ је
показано да такво пресликавање постоји, а ако би постојала два таква
пресликавања онда би њихова разлика морала имати три фиксне тачке, што
значи да би била идентитет).

{\tt
\begin{tabbing}
\hspace{3mm}\=\hspace{5mm}\=\hspace{5mm}\=\hspace{5mm}\=\hspace{5mm}\=\kill
{\bf lemma} "$\lbrakk z_1 \neq z_2$; $z_1 \neq z_3$; $z_2 \neq z_3$; $w_1 \neq w_2$; $w_1 \neq w_3$; $w_2 \neq w_3 \rbrakk$ $\Longrightarrow$ $\exists$! $M$.\\ 
\>mobius\_pt $M$ $z_1$ = $w_1$ $\wedge$ mobius\_pt $M$ $z_2$ = $w_2$ $\wedge$ mobius\_pt $M$ $z_3$ = $w_3$"
\end{tabbing}
}

Мебијусове трансформације чувају дворазмеру. Поново, директан доказ би
био компликован, па је елегантан идиректни доказ формализован (у
основи, разлика {\tt $\lambda z$. cross\_ratio $z$ $z_1$ $z_2$ $z_3$}
и $M$ слика ($M$ $z_1$) у $0_{hc}$, ($M$ $z_2$) у $1_{hc}$, и ($M$
$z_3$) у $\infty_{hc}$, па зато мора бити једнака {\tt $\lambda
  z$. cross\_ratio $z$ ($M$ $z_1$) ($M$ $z_2$) ($M$ $z_3$)}, и тврђење
следи замењујући ($M$ $z$) са $z$).

{\tt
\begin{tabbing}
\hspace{3mm}\=\hspace{5mm}\=\hspace{5mm}\=\hspace{5mm}\=\hspace{5mm}\=\kill
{\bf lemma} "$\lbrakk z_1 \neq z_2$; $z_1 \neq z_3$; $z_2 \neq z_3 \rbrakk$ $\Longrightarrow$\\
\> cross\_ratio $z$ $z_1$ $z_2$ $z_3$ = cross\_ratio \=(mobius\_pt $M$ $z$) (mobius\_pt $M$ $z_1$)\\
\>\>(mobius\_pt $M$ $z_2$) (mobius\_pt $M$ $z_3$)"
\end{tabbing}
}


\subsection{Кругоправа}
\label{subsec:circlines}
Веома важно својство проширене комплексне равни је могућност да праве
и кругове посматрамо на униформан начин. Основни објекат је
\emph{уопштен круг} или скраћено \emph{кругоправа}. У нашој
формализацији ми смо пратили приступ који је описао Schwerdtfeger
\cite{schwerdtfeger} и представили смо кругоправе Хермитеовим, не-нула
$2\times 2$ матрицама. У оргиналној формулацији, матрица
$\left(\begin{array}{cc}A & B\\C & D\end{array}\right)$ одговара
  једначини $A*z*\mathtt{cnj}\,z + B*\mathtt{cnj}\,z + C*z + D = 0$,
  где је $C = \mathtt{cnj}\,B$ и $A$ и $D$ су реални (јер је матрица
  Хермитеова).  Кључано је да ова једначина представља праву када је
  $A=0$, а иначе круг.

Поново, наша формализација се састоји из три корака. Прво, уведен је
тип Хермитеових, не-нула матрица.

{\tt
\begin{tabbing}
\hspace{5mm}\=\hspace{5mm}\=\hspace{5mm}\=\hspace{5mm}\=\hspace{5mm}\=\kill
{\bf definition} is\_C2\_mat\_herm :: "C2\_mat $\Rightarrow$ bool" {\bf where}\\
\> "is\_C2\_mat\_herm H $\longleftrightarrow$ hermitean $H$ $\wedge$ $H$ $\neq$ mat\_zero"\\
{\bf typedef} C2\_mat\_herm = "\{$H$ :: C2\_mat. is\_C2\_mat\_herm $H$\}"
\end{tabbing}
}

Функција репрезентације {\tt Rep\_C2\_mat\_herm} ће бити означена са
$\Repcm{\_}$, а апстрактна функција {\tt Abs\_C2\_mat\_herm} ће бити
означена са $\Abscm{\_}$.  Имајући на уму интерпретацију у форми
једначине, јасно је да поново пропорационалне матрице би требало
сматрати еквивалентним. Овог пута, фактор пропорционалности матрица је
реалан не-нула број.

{\tt
\begin{tabbing}
\hspace{5mm}\=\hspace{5mm}\=\hspace{5mm}\=\hspace{5mm}\=\hspace{5mm}\=\kill
{\bf definition} $\approxcm$ :: "C2\_mat\_herm $\Rightarrow$ C2\_mat\_herm $\Rightarrow$ bool" {\bf where}\\
\>"$H_1$ $\approxcm$ $H_2$ $\longleftrightarrow$ ($\exists$ ($k$::real). $k$ $\neq$ $0$ $\wedge$ $\Repcm{H_2}$ = cor $k$ *$_{sm}$ $\Repcm{H_1}$)"
\end{tabbing}
}

Лако се показује да је ово релација еквиваленције, а кругоправе се
дефинишу коришћењем количничке конструкције као класа еквиваленције.

{\tt
\begin{tabbing}
\hspace{5mm}\=\hspace{5mm}\=\hspace{5mm}\=\hspace{5mm}\=\hspace{5mm}\=\kill
{\bf quotient\_type} circline = C2\_mat\_herm / $\approxcm$
\end{tabbing}
}

Помоћни конструктор {\tt mk\_circline} даје кругоправу (класу
еквиваленције) за дата четири комплексна броја $A$, $B$, $C$ и $D$
(под претпоставком да они формирају Хермитеову, не-нула матрицу).

Свака кругоправа одређује одговарајући скуп тачака. Поново, опис који
је дат у хомогеним координатама је нешто бољи него оригинални опис
који је дат за обичне комплексне бројеве. Тачка са хомогеним
координатама $(z_1, z_2)$ ће припадати скупу тачака кругоправе акко
$A*z_1*\mathtt{cnj}\,z_1 + B*\mathtt{cnj}\,z_1*z_2 +
C*z_1*\mathtt{cnj}\,z_2 + D*z_2*\mathtt{cnj}\,z_2 = 0$. Приметимо да
је ово квадратна форма која је одређена вектором хомогених координата
и Хермитеовом матрицом. Зато, скуп тачака на датој кругоправи се
формализује на следећи начин (ми овде уједно дајемо и дефиниције
билинеарне и квадратне форме које се уведене у нашој основној теорији
линеарне алгебре).

{\tt
\begin{tabbing}
\hspace{5mm}\=\hspace{5mm}\=\hspace{5mm}\=\hspace{5mm}\=\hspace{5mm}\=\kill
{\bf definition} "bilinear\_form $H$ $z_1$ $z_2$ = (vec\_cnj $z_1$) $*_{vm}$ H $*_{vv}$ $z_2$"\\
{\bf definition} "quad\_form $H$ $z$ = bilinear\_form $H$ $z$ $z$"\\
{\bf definition} on\_circline\_rep :: "C2\_mat\_herm $\Rightarrow$ C2\_vec$_{\neq 0}$ $\Rightarrow$ bool" {\bf where}\\
\>"on\_circline\_rep $H$ $z$ $\longleftrightarrow$ quad\_form $\Repcm{H}$ $\Repnzv{z}$ = 0"\\
{\bf lift\_definition} on\_circline :: "circline $\Rightarrow$ complex$_{hc}$ $\Rightarrow$ bool" {\bf is}\\
\> on\_circline\_rep\\
{\bf definition} circline\_set :: "complex$_{hc}$ set" {\bf where} \\
\>"circline\_set $H$ = \{$z$. on\_circline $H$ $z$\}"
\end{tabbing}
}

\noindent Подизање дефиниције {\tt on\_circline} ствара тврђење
$\lbrakk H_1 \approxcm H_2$; $z_1 \approxhc z_2 \rbrakk$
$\Longrightarrow$ $\mathtt{on\_circline\_rep}\ H_1\ z_1
\longleftrightarrow \mathtt{on\_circline\_rep}\ H_2\ z_2$ које се лако
показује.


\paragraph{Some special circlines.}
Among all circlines most prominent ones are the unit circle and the
x-axis, and imaginary unit circle.  {\tt
\begin{tabbing}
\hspace{5mm}\=\hspace{5mm}\=\hspace{5mm}\=\hspace{5mm}\=\hspace{5mm}\=\kill
{\bf definition} "unit\_circle\_rep = \Abscm{(1, 0, 0, -1)}"\\
{\bf lift\_definition} unit\_circle :: "circline" {\bf is} unit\_circle\_rep\\
{\bf definition} "x\_axis\_rep = \Abscm{(0, i, -i, 0)}"\\
{\bf lift\_definition} x\_axis :: "circline" {\bf is} x\_axis\_rep\\
{\bf definition} "imag\_unit\_circle\_rep = \Abscm{(1, 0, 0, 1)}"\\
{\bf lift\_definition} imag\_unit\_circle :: "circline" {\bf is} imag\_unit\_circle\_rep
\end{tabbing}
}

It is easy to show some basic properties of these circlines. For example:
{\tt
\begin{tabbing}
\hspace{5mm}\=\hspace{5mm}\=\hspace{5mm}\=\hspace{5mm}\=\hspace{5mm}\=\kill
{\bf lemma} \="$0_{hc} \in $ circline\_set x\_axis" "$1_{hc} \in $ circline\_set x\_axis" \\
\>"$\infty_{hc} \in $ circline\_set x\_axis"
\end{tabbing}
}

\paragraph{Повезаност са правама и круговима у обичној Еуклидској равни}
У проширеној комплексној равни не постоји разлика између појма праве и
појма круга. Ипак, праве могу бити дефнисане као оне кругоправе код
којих матрице имају коефицијент $A = 0$, или, еквивалентно као оне
кругоправе које садрже тачку $\infty_{hc}$.

{\tt
\begin{tabbing}
\hspace{5mm}\=\hspace{5mm}\=\hspace{5mm}\=\hspace{5mm}\=\hspace{5mm}\=\kill
{\bf definition} is\_line\_rep  {\bf where} \\
\>"is\_line\_rep $H$ $\longleftrightarrow$ (let ($A$, $B$, $C$, $D$) = $\Repcm{H}$ in $A = 0$)"\\
{\bf lift\_definition} is\_line :: "circline $\Rightarrow$ bool" {\bf is} is\_line\_rep\\
{\bf definition} is\_circle\_rep  {\bf where} \\
\>"is\_circle\_rep $H$ $\longleftrightarrow$ (let ($A$, $B$, $C$, $D$) = $\Repcm{H}$ in $A \neq 0$)"\\
{\bf lift\_definition} is\_circle :: "circline $\Rightarrow$ bool" {\bf is} is\_circle\_rep\\
{\bf lemma}\ "is\_line $H$ $\longleftrightarrow$ $\neg$ is\_circle $H$"   "is\_line $H$ $\vee$ is\_circle $H$"\\
{\bf lemma} \="is\_line $H$ $\longleftrightarrow$ $\infty_{hc} \in$ circline\_set $H$"\\
\>"is\_circle $H$ $\longleftrightarrow$ $\infty_{hc} \notin$ circline\_set $H$"
\end{tabbing}
}

Сваки Еуклидски круг и Еуклидкса права (у обичној комплексној равни,
коришћењем стандардне, Еуклидске метрике) може бити представљено
коришћењем кругоправе.  {\tt
  \begin{tabbing}
    \hspace{5mm}\=\hspace{5mm}\=\hspace{5mm}\=\hspace{5mm}\=\hspace{5mm}\=\kill
{\bf definition} mk\_circle\_rep $\mu$ $r$ = $\Abscm{(1,\ -\mu,\ -\mathtt{cnj}\ \mu,\ |\mu|^2-(\mathtt{cor}\ r)^2)}$\\
{\bf lift\_definition} mk\_circle :: "complex $\Rightarrow$ real $\Rightarrow$ circline" {\bf is} mk\_circle\_rep\\
{\bf lemma} "$r \ge 0$ $\Longrightarrow$ circline\_set (mk\_circle $\mu$ $r$) = of\_complex ` $\{z.\ |z - \mu| = r\}$"\\
{\bf definition} mk\_line\_rep {\bf where} "mk\_line\_rep $z_1$ $z_2$ = \\
\>  (let $B = i*(z_2-z_1)$ in $\Abscm{(0,\ B,\ \mathtt{cnj}\ B,\ -(B*\mathtt{cnj}\ z_1+\mathtt{cnj}\ B*z_1)}$)"\\
{\bf lift\_definition} mk\_line :: "complex $\Rightarrow$ complex $\Rightarrow$ circline" {\bf is} mk\_line\_rep\\
{\bf lemma} "$z_1 \neq z_2$ $\Longrightarrow$ \\
\> circline\_set (mk\_line $z_1$ $z_2$) - $\{\infty_{hc}\}$ = of\_complex ` \{$z$. collinear $z_1$ $z_2$ $z$\}"
  \end{tabbing}
}

Супротно такође важи, скуп тачака који су одређени кругоправом је увек
или Еуклидски круг или Еуклидска права. Следећа функција одређује
параметре круга или параметре праве (центар и полупречник у случају
круга или две различите тачке у случају праве) за дату кругоправу.
{\tt
  \begin{tabbing}
    \hspace{5mm}\=\hspace{5mm}\=\hspace{5mm}\=\hspace{5mm}\=\hspace{5mm}\=\kill
{\bf definition} euclidean\_circle\_rep {\bf where} "euclidean\_circle\_rep $H$ = \\
\> $($let $(A, B, C, D)$ = $\Repcm{H}$ in ($-B/A$, sqrt$($Re $((B*C - A*D)/(A*A))))$$)$"\\
{\bf lift\_definition} euclidean\_circle :: "circline $\Rightarrow$ complex $\times$ real" {\bf is}\\
\> euclidean\_circle\_rep\\
{\bf definition} euclidean\_line\_rep {\bf where} "euclidean\_line\_rep $H$ = \\
\>(l\=et \=$(A, B, C, D)$ = $\Repcm{H}$; \\
\>\>\>$z_1$ = $-(D*B)/(2*B*C)$;\\
\>\>\>$z_2$ = $z_1$ + $i$ * sgn (if arg $B$ > 0 then $-B$ else $B$)\\
\>\>in ($z_1$, $z_2$))"\\
{\bf lift\_definition} euclidean\_line :: "circline $\Rightarrow$ complex $\times$ complex" {\bf is}\\
\> euclidean\_line\_rep
  \end{tabbing}
}
\noindent Приметимо да нормални вектор праве је вектор који је
ортогоналан на коефицијент $B$ --- у дефиницији друге тачке вектор $B$
мора бити нормализован у намери да би могли да подигнемо дефеницију
(тако да добијене тачке су исте за сваку матрицу која репрезентује
исту кругоправу). Ово даје нешто већи израз $z_2=z_1+i*B$.

Додатно, кардиналност скупа тачака кругоправе зависи од знака израза
$\mathtt{Re} ((B*C - A*D)/(A*A))$. Зато, кругоправе могу бити
класификоване у три категорије у зависности од знака детерминанте
(који је увек реалан број, јер је матрица Хермитеова).

{\tt
\begin{tabbing}
\hspace{5mm}\=\hspace{5mm}\=\hspace{5mm}\=\hspace{5mm}\=\hspace{5mm}\=\kill
{\bf definition} circline\_type\_rep {\bf where}\\
\>"circline\_type\_rep $H$ = sgn (Re (mat\_det ($\Repcm{H}$)))"\\
{\bf lift\_definition} circline\_type :: "circline $\Rightarrow$ real" {\bf is} circline\_type\_rep
\end{tabbing}
}
\noindent Обавезно тврђење $H \approxcm H' \Longrightarrow$ {\tt
  circline\_type\_rep $H$ = circline\_type\_rep $H'$} се лако
показује, јер {\tt Re (mat\_det ($k$ $*_{sm}$ $H$)) = $($Re $k)^2$ *
  Re (mat\_det $H$)} важи за све Хермитеове матрице $H$ и за све $k$
са имагинарним делом $0$.

Сада постаје јасно да скуп тачака на датој кругоправи је празан акко
је тип кругоправе позитиван (ове кругоправе се зову \emph{ имагинарне
  кругоправе}), да садржи само једну тачку акко је тип круг (оне се
зову \emph{тачка кругоправе}) и да је бесконачан акко је тип негативан
(оне се зову \emph{реалне кругоправе}).  Оно што је било изненађујуће
је да се испоставило да је веома тешко показати ово тврђење формално и
било га је могуће показати само када је формализовано дејство Мебијуса
на кругоправе што је омогућило да се користи "бгно"
резоновање. Приметимо да не постоје имагинарне праве јер кад је $A =
0$, онда {\tt mat\_det $H$ $\ge 0$}.

Коначно, веза између реалних кругоправих и Еуклидских прави и кругова
се може успоставити.

{\tt
\begin{tabbing}
\hspace{5mm}\=\hspace{5mm}\=\hspace{5mm}\=\hspace{5mm}\=\hspace{5mm}\=\kill
{\bf lemma}\\
\> assumes "is\_circle H" "($\mu$, $r$) = euclidean\_circle H"\\
\>  shows "circline\_set H = of\_complex ` \{$z.$ $|z-\mu|$ = $r$\}"\\
{\bf lemma}\\
\>  assumes "is\_line $H$" "($z_1$, $z_2$) = euclidean\_line H" "circline\_type $H$ $< 0$"\\
\>  shows "circline\_set $H$ - $\{\infty_{hc}\}$ = of\_complex ` \{$z$. collinear $z_1$ $z_2$ $z$\}"
\end{tabbing}
}
\noindent Приметимо да прва лема такође важи за имагинарни и тачка
круг јер су оба скупа празна. Ипак, друга лема једино важи за реалне
праве јер у случају тачка праве важи да $z_1=z_2$, па је леви скуп
празан, а десни је универзални скуп.

\paragraph{Кругоправе на Римановој сфери.}
Кругоправе у равни одговарају круговима на Римановој сфери, и ми смо
формално показали ову везу. Сваку круг у тродимензионом простору се
може добити као пресек сфере и равни. Успоставили смо један-на-један
пресликавање између кругова на Римановој сфери и равни у
простору. Приметимо и да није неопходно да раван сече сферу и тада
ћемо рећи да она дефинише јединствен имагинаран круг. Веза између
равни у простору и кругоправих у проширеној комплексној равни је
описао Schwerdtfeger \cite{schwerdtfeger}. Ипак, аутор није приметио
да за једну специјалну кругоправу (она чија репрезентативна матрица је
јединична матрица) не постоји раван у $\mathbb{R}^3$ која јој одговара
--- и да би могли да имамо такву раван, потребно је да уместо
посматрања равни у $\mathbb{R}^3$, узмемо у обзир тродимензионални
пројективни простор и коначну хиперраван. Зато, ми дефинишемо раван на
следећи начин (опет у три корака).

{\tt
\begin{tabbing}
\hspace{5mm}\=\hspace{5mm}\=\hspace{5mm}\=\hspace{5mm}\=\hspace{5mm}\=\kill
{\bf typedef} R4\_vec$_{\neq 0}$ = "\{$(a, b, c, d)$ :: R4\_vec. $(a, b, c, d) \neq$ vec\_zero\}"
\end{tabbing}
}

Приметимо да у $\mathbb{R}^3$, један од бројева $a$, $b$, или $c$ ће
бити различит од $0$. Ипак, наша дефиниција дозвољава постојање равни
$(0, 0, 0, d)$ која лежи у бесконачности. Функција репрезентације ће
бити означена са $\Reppl{\_}$, а апстрактна функција ће бити означена
са $\Abspl{\_}$. Поново, две равни су еквивалентне акко су
пропорционалне (овог пута за неки не-нула реални фактор).  {\tt
\begin{tabbing}
\hspace{5mm}\=\hspace{5mm}\=\hspace{5mm}\=\hspace{5mm}\=\hspace{5mm}\=\kill
{\bf definition} $\approxp$ :: "R4\_vec$_{\neq 0}$ $\Rightarrow$ R4\_vec$_{\neq 0}$ $\Rightarrow$ bool" {\bf where}\\
\>"$\alpha_1 \approxp \alpha_2 \longleftrightarrow (\exists k.\ k \neq 0 \wedge \Reppl{\alpha_2} = k*\Reppl{\alpha_1})$"
\end{tabbing}
}

Коначно, равни (кругови који су у њима су добијени пресеком са
Римановом сфером) се дефинишу као класа еквиваленције ове релације.

{\tt
\begin{tabbing}
\hspace{5mm}\=\hspace{5mm}\=\hspace{5mm}\=\hspace{5mm}\=\hspace{5mm}\=\kill
{\bf quotient\_type} plane = R4\_vec$_{\neq 0}$ / $\approxp$
\end{tabbing}
}

Коефицијенти равни дају линеарну једначину а тачка на Римановој сфери
лежи на кругу одређеном са равни акко њена репрезентација задовољава
линеарну једначину.

{\tt
\begin{tabbing}
\hspace{5mm}\=\hspace{5mm}\=\hspace{5mm}\=\hspace{5mm}\=\hspace{5mm}\=\kill
{\bf definition} on\_sphere\_circle\_rep {\bf where}\\
\>"on\_sphere\_circle\_rep $\alpha$ $M$ $\longleftrightarrow$ \\
\>\>(l\=et ($a$, $b$, $c$, $d$) = $\Reppl{\alpha}$; ($X$, $Y$, $Z$) = $\Reprs{M}$\\
\>\>\>  in $a*X + b*Y + c*Z + d = 0$)"\\
{\bf lift\_definition} on\_sphere\_circle :: "plane $\Rightarrow$ riemann\_sphere $\Rightarrow$ bool {\bf is} \\
\>on\_sphere\_circle\_rep\\
{\bf definition} sphere\_circle\_set :: "riemann\_sphere set" {\bf where}\\
\>"sphere\_circle\_set $\alpha$ = $\{A.$ on\_sphere\_circle $\alpha$ $A\}$"
\end{tabbing}
}
\noindent Приметимо да нисмо морали да уведемо тачке у тродимензионом
пројективном простору (и њихове хомогене координате) јер смо ми једино
заинтересовани за тачке на Римановој сфери које нису бесконачне.

Следеће, ми уводимо стереографску и инверзну стереографску пројекцију
између кругова на Римановој сфери и кругова у проширеној комплексној
равни.

{\tt
\begin{tabbing}
\hspace{5mm}\=\hspace{5mm}\=\hspace{5mm}\=\hspace{5mm}\=\hspace{5mm}\=\kill
{\bf definition} stereographic\_circline\_rep {\bf where} \\
\>"stereographic\_circline\_rep $\alpha$  =\\
\>\>(l\=et \=$(a, b, c, d)$ = $\Reppl{\alpha}$; $A=\mathtt{cor}\,((c+d)/2)$; $B=(\mathtt{cor}\,a + i* \mathtt{cor}\,b)/2)$;\\
\>\>\>\>$C=(\mathtt{cor}\,a - i*\mathtt{cor}\,b)/2$; $D=\mathtt{cor}\,((d-c)/2))$\\
\>\>\>in $\Abscm{(A, B, C, D)}$"\\
{\bf lift\_definition} stereographic\_circline :: "plane $\Rightarrow$ circline" {\bf is}\\
\>stereographic\_circline\_rep\\
{\bf definition} inv\_stereographic\_circline\_rep {\bf where} \\
\>"inv\_stereographic\_circline\_rep $H$  =\\
\>\>(l\=et $(A, B, C, D)$ = $\Repcm{H}$\\
\>\>\>in $\Abspl{(\mathtt{Re}(B+C), \mathtt{Re}(i*(C-B)), \mathtt{Re}(A-D), \mathtt{Re}(D+A))}$"\\
{\bf lift\_definition} inv\_stereographic\_circline :: "circline $\Rightarrow$ plane" {\bf is}\\
\>inv\_stereographic\_circline\_rep
\end{tabbing}
}

Ова два пресликавања су бијективна и међусобно инверзна. Пројекција
скупа тачака на кругу на Римановој сфери је управо скуп тачака на
кругоправи која се добија управо уведеном стереографском пројекцијом
круга.  {\tt
\begin{tabbing}
\hspace{5mm}\=\hspace{5mm}\=\hspace{5mm}\=\hspace{5mm}\=\hspace{5mm}\=\kill
{\bf lemma} "stereographic\_circline $\circ$ inv\_stereographic\_circline = id"\\
{\bf lemma} "inv\_stereographic\_circline $\circ$ stereographic\_circline = id"\\
{\bf lemma} "bij stereographic\_circline" "bij inv\_stereographic\_circline"\\
{\bf lemma} "\=stereographic ` sphere\_circle\_set $\alpha$ = \\
\>circline\_set (stereographic\_circline $\alpha$)"
\end{tabbing}
}

\paragraph{Тетивне кругоправе.}
Још једна интересантна чињеница је да су реалне кругоправе запрао
скупови тачака које су на једнаком одстојању од неких датих тачака
(заправо увек постоје тачно две такве тачке), али посматрајући у
тетивној метрици. На Римановој сфери ове две тачке (зваћемо их тетивни
центри) се добијају пресеком сфере и праве која пролази кроз центар
круга и нормална је на раван која садржи тај круг.

Тетивна кругоправа са датом тачком $a$ и полупречником $r$ је одређена
на следећи начин.

{\tt
  \begin{tabbing}
    \hspace{5mm}\=\hspace{5mm}\=\hspace{5mm}\=\hspace{5mm}\=\hspace{5mm}\=\kill
{\bf definition} chordal\_circle\_rep {\bf where} "chordal\_circle\_rep $\mu_c$ $r_c$ = \\
\>  (l\=et \=($\mu_1$, $\mu_2$) = $\Repnzv{\mu_c}$;\\
\>\>\>$A$ = 4*$|\mu_2|^2$ - (cor $r_c$)$^2$*($|\mu_1|^2 + |\mu_2|^2$); $B$ = $-4$*$\mu_1$*cnj $\mu_2$;\\
\>\>\>$C$ = -4*cnj $\mu_1$*$\mu_2$; $D$ = 4*$|\mu_1|^2$ - (cor $r_c$)$^2$*($|\mu_1|^2 + |\mu_2|^2$)\\
\>\>in mk\_circline\_rep $A$ $B$ $C$ $D$)"\\
{\bf lift\_definition} chordal\_circle :: "complex$_{hc}$ $\Rightarrow$ real $\Rightarrow$ circline" {\bf is}\\
\> chordal\_circle\_rep\\
{\bf lemma} "\=$z$ $\in$ circline\_set (chordal\_circle $\mu_c$ $r_c$) $\longleftrightarrow$\\
\>$r_c \ge 0$ $\wedge$ dist$_{hc}$ $z$ $\mu_c$ = $r_c$"
  \end{tabbing}
}

\noindent За дату кругоправу њен центар и радијус се могу одредити
ослањајући се на следеће леме (у зависности да ли коефицинети $B$ и
$C$ у репрезентативној матрици су нула).
{\tt
  \begin{tabbing}
    \hspace{2mm}\=\hspace{5mm}\=\hspace{5mm}\=\hspace{5mm}\=\hspace{5mm}\=\kill
{\bf lemma}\\
\>{\bf assumes} \="is\_C2\_mat\_herm $(A, B, C, D)$" "Re ($A*D$) $<$ $0$" "$B = 0$"\\
\>{\bf sh}\={\bf ows}\\
\>\>"mk\_circline $A$ $B$ $C$ $D$ = chordal\_circle $\infty_{hc}$ sqrt$($Re $((4*A)/(A-D))$$)$"\\
\>\>"mk\_circline $A$ $B$ $C$ $D$ = chordal\_circle $0_{hc}$ sqrt$($Re $((4*D)/(D-A))$$)$"\\
{\bf lemma} {\bf assumes}\\
\> "is\_C2\_mat\_herm $(A, B, C, D)$" "Re (mat\_det $(A, B, C, D)$) $<$ $0$" "$B \neq 0$"\\
\> "$C * \mu_c^2  + (D - A) * \mu_c - B = 0$"  "$r_c$ = sqrt$((4 + \mathtt{Re}((4 * \mu_c/B) * A)) / (1 + \mathtt{Re} (|\mu_c|^2)))$"\\
\>{\bf shows} "mk\_circline $A$ $B$ $C$ $D$ = chordal\_circle (of\_complex $\mu_c$) $r_c$"
  \end{tabbing}
}

Као и у претходним случајевима, може се увести функција која враћа
тетивне параметре (потребно је направити разлику међу случајевима
$B=0$ и $B \neq 0$ и у другом случају је потребно решити квадратну
једначину која описује тетивни центар).

\paragraph{Симетрија.}
Још од античке Грчке, инверзија кругa је посматрана као аналогија
рефлексији праве. У проширеној комплексној равни не постоји суштинска
разлика између кругова и прави, тако да ћемо ми посматрати само једну
врсту релације и за две тачке ћемо рећи да су \emph{симетричне у
  односу на круг} ако се оне сликају једна у другу коришћењем било
рефлексије или инверзије у односу на произвољану праву или круг. Када
смо тражили алгебраску репрезентацију ове релације изненадили смо се
колико је била једноставна и елегантна -- тачке су симетричне акко је
билинеарна форма њиховог репрезентативног вектора и репрезентативне
матрице круга једнака нули.
{\tt
\begin{tabbing}
\hspace{5mm}\=\hspace{5mm}\=\hspace{5mm}\=\hspace{5mm}\=\hspace{5mm}\=\kill
{\bf definition} circline\_symmetric\_rep {\bf where}\\
\>"circline\_symmetric\_rep $z_1$ $z_2$ $H$ $\longleftrightarrow$ bilinear\_form $\Repnzv{z_1}$ $\Repnzv{z_2}$ $\Repcm{H}$ $= 0$"\\
{\bf lift\_definition} circline\_symmetric :: "complex$_{hc}$ $\Rightarrow$ complex$_{hc}$ $\Rightarrow$ \\
\>circline $\Rightarrow$ bool" {\bf is} circline\_symmetric\_rep
\end{tabbing}
}

Посматрајући скуп тачака на кругоправи и поредећи наше две дефиниције,
постаје јасно да тачке на кругоправи су управо оне које су
инваријантне у односу на симетрију у односу на ту кругоправу.
{\tt
\begin{tabbing}
\hspace{5mm}\=\hspace{5mm}\=\hspace{5mm}\=\hspace{5mm}\=\hspace{5mm}\=\kill
{\bf lemma} "on\_circline $H$ $z$ $\longleftrightarrow$ circline\_symmetric $H$ $z$ $z$"
\end{tabbing}
}

\paragraph{Дејство Мебијусових трансформација на кругоправе}
Већ смо видели како Мебијусове трансформације делују на тачке $\extC$.
Оне такође делују и на кругоправе (и дефиниција је изабрана тако да су
два дејства компатибилна). Додатно, дајемо и дефиницију операције
конгруенције две матрице (која је дефинисана у нашој помоћној теорији
линеарне алгебре).

{\tt
\begin{tabbing}
\hspace{5mm}\=\hspace{5mm}\=\hspace{5mm}\=\hspace{5mm}\=\hspace{5mm}\=\kill
{\bf definition} "congruence $M$ $H$ = mat\_adj $M$ $*_{mm}$ $H$ $*_{mm}$ $M$"\\
{\bf definition} mobius\_circline\_rep \\
\>:: "C2\_mat\_reg $\Rightarrow$ C2\_mat\_herm $\Rightarrow$ C2\_mat\_herm" {\bf where}\\
\>"mobius\_circline\_rep $M$ $H$ = $\Abscm{\mathtt{congruence}\ (\mathtt{mat\_inv}\ \Reprm{M})\ \Repcm{H}}$"\\
{\bf lift\_definition} mobius\_circline :: "mobius $\Rightarrow$ circline $\Rightarrow$ circline" {\bf is} \\
\>mobius\_circline\_rep
\end{tabbing}
}

\noindent Својства која има дејство Мебијусових трансформација на кругоправе је врло слично као и код дејства Мебијусових трансформација на тачке. На пример, 

{\tt
\begin{tabbing}
\hspace{5mm}\=\hspace{5mm}\=\hspace{5mm}\=\hspace{5mm}\=\hspace{5mm}\=\kill
{\bf lemma} "\=mobius\_circline (mobius\_comp $M_1$ $M_2$) = \\
\>mobius\_circline $M_1$ $\circ$ mobius\_circline $M_2$"\\
{\bf lemma} "mobius\_circline (mobius\_inv $M$) = inv (mobius\_circline $M$)"\\
{\bf lemma} "mobius\_circline (mobius\_id) = id"\\
{\bf lemma} "inj mobius\_circline"
\end{tabbing}
}

Централна лема у овом одељку прави везу између дејства Мебијусових
трансформација на тачкама и на кругоправама (и што је основно,
показује се да Мебијусове трансформације сликају кругоправе на
кругоправе).

{\tt
\begin{tabbing}
\hspace{5mm}\=\hspace{5mm}\=\hspace{5mm}\=\hspace{5mm}\=\hspace{5mm}\=\kill
{\bf lemma} "\=mobius\_pt $M$ ` circline\_set $H$ = \\
\>circline\_set (mobius\_circline $M$ $H$)"
\end{tabbing}
}

\noindent Поред овога чува се и тип кругоправе (што повлачи, на
пример, да се реалне кругоправе сликају на реалне кругоправе).

{\tt
\begin{tabbing}
\hspace{5mm}\=\hspace{5mm}\=\hspace{5mm}\=\hspace{5mm}\=\hspace{5mm}\=\kill
{\bf lemma} circline\_type (mobius\_circline $M$ $H$) = circline\_type $H$
\end{tabbing}
}

Још једно важно својство (које је нешто општије него претходно
наведено својство) је да симетрија тачака очувана након дејства
Мебијусових трансформација (што је још назива и прицип симетрије).

{\tt
\begin{tabbing}
\hspace{5mm}\=\hspace{5mm}\=\hspace{5mm}\=\hspace{5mm}\=\hspace{5mm}\=\kill
{\bf lemma} \={\bf assumes} "circline\_symmetric $z_1$ $z_2$ $H$"\\
\>{\bf shows} "circline\_symmetric \=(mobius\_pt $M$ $z_1$) (mobius\_pt $M$ $z_2$)\\
\>\>(mobius\_circline $M$ $H$)"
\end{tabbing}
}

Последње две леме су веома важни геометријски резултати, и захваљујући
веома погодне алгебарске репрезентације њих је било прилично лако
показати у нашој формализацији. Оба доказа се заснивају на следећој
једностаној чињениции из линеарне алгебре.  {\tt
\begin{tabbing}
\hspace{5mm}\=\hspace{5mm}\=\hspace{5mm}\=\hspace{5mm}\=\hspace{5mm}\=\kill
{\bf lemma} "\=mat\_det $M$ $\neq$ $0$ $\Longrightarrow$ "bilinear\_form $z_1$ $z_2$ $H$ = \\
\>bilinear\_form ($M *_{mv} z_1$) ($M *_{mv} z_2$) (congruence (mat\_inv $M$) $H$)"
\end{tabbing}
}


\paragraph{Јединственост кругоправе.}
У Еуклидској геометрији добро је позната чињеница да постоји
јединствена права кроз две различите тачке и јединствени круг кроз три
различите тачке. Слични резулатати важе и у $\extC$.  Ипак, да би се
дошло до закључака потребно је извршити анализу случајева према типу
кругоправе.  Кругоправе позитивног типа не садрже тачке па код њих не
постоји јединственост. Кругоправе нула типа садрже једну тачку и за
сваку тачку постоји јединствена кругоправа нула типа која је
садржи. Постоји јединствена кругоправа кроз било које три различите
тачке (и она мора бити негативног типа).
 
{\tt
\begin{tabbing}
\hspace{5mm}\=\hspace{5mm}\=\hspace{5mm}\=\hspace{5mm}\=\hspace{5mm}\=\kill
{\bf lemma} "$\exists !$ $H$. circline\_type $H$ = 0 $\wedge$ $z$ $\in$ circline\_set $H$\\
{\bf lemma} "$\lbrakk z_1 \neq z_2;\ z_1 \neq z_3;\ z_2 \neq z_3\rbrakk$ $\Longrightarrow$ \\
\>$\exists !$ $H$. $z_1 \in $ circline\_set $H$ $\wedge$ $z_2 \in$ circline\_set $H$ $\wedge$ $z_3 \in$ circline\_set $H$"
\end{tabbing}
}

\noindent Веома изненађујуће, ми нисмо успели да докажемо ове леме
директно. Ипак, након примене "бгно" резоновања и након пресликавања
тачака у канонску позицију ($0_{hc}$, $1_{hc}$ и $\infty_{hc}$) добили
смо веома кратак и елегантан доказ (јер је било могуће показати
коришћењем израчунавања да {\tt x-оса} је једина кругоправа кроз ове
три канонске тачке). Како су праве карактеризоване као управу оне
кругоправе које садрже $\infty_{hc}$, постаје јасно да постоји
јединствена права кроз били које две различите коначне тачке.

\paragraph{Сет кардиналности кругоправе.}
Још једна од ствари која се узима "здраво за готово" је кардиналност
кругоправи различитог типа. Већ смо рекли да ови докази захтевају
"бгно" резоновање, али овог пута користили смо другачију врсту "бгно"
резоновања. Испоставља се да је у многим случајевима лакше резоновати
о круговима уколико је њихов ценатр у координатном почетку --- у том
случају, њихова матрица је дијагонална.  Ми смо формализовали
специјалан случај чувеног резултата из ленеарне алгебре да Хермитеова
$2\times2$ матрица је конгруентна са реалном дијагоналном матрицом
(шта више, елементи на дијагонали су реалне сопствене вредности
матрице, а конгруенција је успостављена коришћењем унитарних матрица
--- конгруенција се такође може успоставити коришћењем једноставније
матрице (матрице транслације), али онда она не би имала многа лепа
својства).

{\tt
\begin{tabbing}
\hspace{5mm}\=\hspace{5mm}\=\hspace{5mm}\=\hspace{5mm}\=\hspace{5mm}\=\kill
{\bf lemma} {\bf assumes} "hermitean $H$"\\
\>{\bf shows} "$\exists\,k_1\,k_2\,M$. \=mat\_det $M$ $\neq 0$ $\wedge$ unitary $M$ $\wedge$\\
\>\>congruence $M$ $H$ = (cor $k_1$, 0, 0, cor $k_2$)"
\end{tabbing}
}

Последица је да за сваку кругоправу постоји унитарна Мебијусова
трансформација која слика кругоправу тако да је њен центар у
координатном почетку (заправо, постоје две такве трансформације ако су
сопствене вредности различите). Видећемо да унитарне трансформације
одговарају ротацијама Риманове сфере, тако да последња чињеница има
једноставно геометријско објашњење. Кругоправе се могу
дијагонализовати коришћењем само транслација, али унитарне
трансформације често имају лепша својства.

{\tt
\begin{tabbing}
\hspace{5mm}\=\hspace{5mm}\=\hspace{5mm}\=\hspace{5mm}\=\hspace{5mm}\=\kill
{\bf lemma} "$\exists$ $M$ $H'$. \= unitary\_mobius $M$ $\wedge$ \\
\>mobius\_circline $M$ $H$ = $H'$ $\wedge$ circline\_diag $H'$"\\
{\bf lemma} \={\bf assumes} \="$\bigwedge$ $H'$. circline\_diag $H'$ $\Longrightarrow$ $P\ H$"\\
\>\> "$\bigwedge$ $M$ $H$. $P\ H$ $\Longrightarrow$ $P$ (mobius\_circline $M$ $H$)"\\
\>{\bf shows}\>"$P$ $H$"
\end{tabbing}
} 

\noindent Приметимо да {\tt unitary\_mobius} је предикат који подиже
{\tt unitary} својство са $\mathbb{C}^2$ матрица на {\tt
mobius} тип. Слично, {\tt circline\_diag} подиже услов
дијагоналне матрице на {\tt circline} тип.

Коришћењем овакве врсте "бгно" резоновања постаје прилично јасно како
показати следећу карактеризацију за кардиналност скупа кругоправе.

{\tt
\begin{tabbing}
\hspace{5mm}\=\hspace{5mm}\=\hspace{5mm}\=\hspace{5mm}\=\hspace{5mm}\=\kill
{\bf lemma} "circline\_type $H$ > $0$ $\longleftrightarrow$ circline\_set $H$ = $\{\}$"\\
{\bf lemma} "circline\_type $H$ = $0$ $\longleftrightarrow$ $\exists z$. circline\_set $H$ = $\{z\}$"\\
{\bf lemma} "\=circline\_type $H$ < $0$ $\longleftrightarrow$\\
\> $\exists\,z_1\,z_2\,z_3$. $z_1 \neq z_2$ $\wedge$ $z_1 \neq z_3$ $\wedge$ $z_2 \neq z_3$ $\wedge$ circline\_set $H$ $\supseteq$ $\{z_1, z_2, z_3\}$"\\
\end{tabbing}
}

Важна, нетривијална, последица јединствености кругоправе и
кардиналности скупа кругоправе је да функција {\tt circline\_set} је
ињективна, тј. за сваки непразан скуп тачака кругоправе, постоји
јединствена класа пропорционалних матрица која их све одређује ({\tt
  circline\_set} је празан за све имагинарне кругоправе, што значи да
ово својство не важи када је скуп тачака кругоправе празан).
{\tt
\begin{tabbing}
\hspace{5mm}\=\hspace{5mm}\=\hspace{5mm}\=\hspace{5mm}\=\hspace{5mm}\=\kill
{\bf lemma} "$\lbrakk$ circline\_set $H_1$ = circline\_set $H_2$; circline\_set $H_1$ $\neq \{\}$ $\rbrakk$ $\Longrightarrow$ \\
\>$H_1 = H_2$"
\end{tabbing}
}

\subsection{Оријентисане кругоправе}
\label{subsec:orientation}
У овом одељку ми ћемо описати како је могуће увести оријентацију за
кругоправе. Многи важни појмови зависе од оријентације.  Један од
најважнијих појмова је појам \emph{disc} --- унуштрањост
кругоправе. Слично као што је то био случај код скупа тачака
кругоправе, скуп тачака диска се уводи коришћењем квадратне форме у
чијем изразу се налази матрица кругоправе --- скуп тачка диска
кругоправе је скуп тачака за које важи $A*z*\mathtt{cnj}\,z +
B*\mathtt{cnj}\,z + C*z + D < 0$, при чему је
$\left(\begin{array}{cc}A & B\\C & D\end{array}\right)$ је матрица
  која репрезентује кругоправу. Како скуп тачака диска мора бити
  инваријантан у односу на избор представника, јасно је да матрице
  оријентисане кругоправе су еквивалентне само ако су оне
  пропорционалне у односу на неки реални фактор (подсетимо се да код
  неоријентисаних кругоправих фактор може бити произвољан реалан
  ненула број).

{\tt
\begin{tabbing}
\hspace{5mm}\=\hspace{5mm}\=\hspace{5mm}\=\hspace{5mm}\=\hspace{5mm}\=\kill
{\bf definition} $\approxocm$ :: "C2\_mat\_herm $\Rightarrow$ C2\_mat\_herm $\Rightarrow$ bool" {\bf where}\\
\>"$H_1$ $\approxocm$ $H_2$ $\longleftrightarrow$ ($\exists$ ($k$::real). $k$ $>$ 0 $\wedge$ $\Repcm{H_2}$ = cor $k$ *$_{sm}$ $\Repcm{H_1}$)"
\end{tabbing}
}

Лако се показује да је ово релација еквиваленције, тако да су
кругоправе дефинисане преко количничке конструкције као класе
еквиваленције.

{\tt
\begin{tabbing}
\hspace{5mm}\=\hspace{5mm}\=\hspace{5mm}\=\hspace{5mm}\=\hspace{5mm}\=\kill
{\bf quotient\_type} o\_circline = C2\_mat\_herm / $\approxocm$
\end{tabbing}
}

Сада можемо користити квадратну форму да дефинишемо унутрашњост,
спољашњост и границу оријентисане кругоправе.

{\tt
\begin{tabbing}
\hspace{5mm}\=\hspace{5mm}\=\hspace{5mm}\=\hspace{5mm}\=\hspace{5mm}\=\kill
{\bf definition} on\_o\_circline\_rep :: "C2\_mat\_herm $\Rightarrow$ C2\_vec$_{\neq 0}$ $\Rightarrow$ bool" {\bf where}\\
\>"on\_o\_circline\_rep $H$ $z$ $\longleftrightarrow$ quad\_form $\Repcm{H}$ $\Repnzv{z}$ = 0"\\
{\bf definition} in\_o\_circline\_rep :: "C2\_mat\_herm $\Rightarrow$ C2\_vec$_{\neq 0}$ $\Rightarrow$ bool" {\bf where}\\
\>"in\_o\_circline\_rep $H$ $z$ $\longleftrightarrow$ quad\_form $\Repcm{H}$ $\Repnzv{z}$ < 0"\\
{\bf definition} out\_o\_circline\_rep :: "C2\_mat\_herm $\Rightarrow$ C2\_vec$_{\neq 0}$ $\Rightarrow$ bool" {\bf where}\\
\>"out\_o\_circline\_rep $H$ $z$ $\longleftrightarrow$ quad\_form $\Repcm{H}$ $\Repnzv{z}$ > 0"
\end{tabbing}
}

\noindent Ове дефиниције се подижу на {\tt on\_o\_circline}, {\tt
  in\_o\_circline}, и {\tt out\_o\_circline} (при томе доказујемо
неопходне услове), и, коначно, уводе се следеће три дефиницје.

{\tt
\begin{tabbing}
\hspace{5mm}\=\hspace{5mm}\=\hspace{5mm}\=\hspace{5mm}\=\hspace{5mm}\=\kill
{\bf definition} o\_circline\_set :: "complex$_{hc}$ set" {\bf where} \\
\>"o\_circline\_set $H$ = \{$z$. on\_o\_circline $H$ $z$\}"\\
{\bf definition} disc :: "complex$_{hc}$ set" {\bf where} \\
\>"disc $H$ = \{$z$. in\_o\_circline $H$ $z$\}"\\
{\bf definition} disc\_compl :: "complex$_{hc}$ set" {\bf where} \\
\>"disc\_compl $H$ = \{$z$. out\_o\_circline $H$ $z$\}"
\end{tabbing}
}

Ова три скупа су међусобно дисјунктна и заједно попуњавају целу раван.
{\tt
\begin{tabbing}
\hspace{5mm}\=\hspace{5mm}\=\hspace{5mm}\=\hspace{5mm}\=\hspace{5mm}\=\kill
{\bf lemma} \="disc $H$ $\cap$ disc\_compl $H$ = $\{\}$" \\
\>"disc $H$ $\cap$ o\_circline\_set $H$ = $\{\}$"\\
\>"disc\_compl $H$ $\cap$ o\_circline\_set $H$ = $\{\}$"\\
\>"disc $H$ $\cup$ disc\_compl $H$ $\cup$ o\_circline\_set $H$ = UNIV"
\end{tabbing}
}

За дату оријентисану кругоправу, може се тривијално одредити њен
неоријентисани део, а ове две кругоправе имају исти скуп тачака.
{\tt
  \begin{tabbing}
    \hspace{5mm}\=\hspace{5mm}\=\hspace{5mm}\=\hspace{5mm}\=\hspace{5mm}\=\kill
{\bf lift\_definition} of\_o\_circline ($\ofocircline{\_}$) :: "o\_circline $\Rightarrow$ circline" {\bf is} $id$ \\
{\bf lemma} "circline\_set ($\ofocircline{H}$) = o\_circline\_set $H$"
  \end{tabbing}
}

У {\bf lift\_definition} увели смо скраћени запис (?proveri ovo)
функције {\tt of\_o\_circline}, тако да, на пример, $\ofocircline{H}$
у леми је скраћеница за {\tt of\_o\_circline\ $H$}.

За сваку кругоправу, постоји тачно једна супротно оријентисана
кругоправа
{\tt
  \begin{tabbing}
    \hspace{5mm}\=\hspace{5mm}\=\hspace{5mm}\=\hspace{5mm}\=\hspace{5mm}\=\kill
{\bf definition} "opp\_o\_circline\_rep $H$ = $\Abscm{-1 *_{sm} \Repcm{H}}$"\\
{\bf lift\_definition} opp\_o\_circline ($\oppocircline{\_}$) :: "o\_circline $\Rightarrow$ o\_circline" {\bf is} \\
\>opp\_o\_circline\_rep
  \end{tabbing}
}
\noindent Одређивање супротне кругоправе је идемпотентно јер супротне
кругоправе имају исти скуп тачака, али размењују диск и његов
комплемент.  {\tt
  \begin{tabbing}
    \hspace{5mm}\=\hspace{5mm}\=\hspace{5mm}\=\hspace{5mm}\=\hspace{5mm}\=\kill
{\bf lemma} "$\oppocircline{(\oppocircline{H})} = H$"\\
{\bf lemma} \="o\_circline\_set ($\oppocircline{H}$) = o\_circline\_set $H$"\\
\>"disc ($\oppocircline{H}$) = disc\_compl $H$" "disc\_compl ($\oppocircline{H}$) = disc $H$"
  \end{tabbing}
}

Функције $\ofocircline{\_}$ и {\tt o\_circline\_set} су у одређеном
смислу ињективне.  {\tt
  \begin{tabbing}
    \hspace{5mm}\=\hspace{5mm}\=\hspace{5mm}\=\hspace{5mm}\=\hspace{5mm}\=\kill
{\bf lemma} "$\ofocircline{H_1} = \ofocircline{H_2}$ $\Longrightarrow$ $H_1$ = $H_2$ $\vee$ $H_1$ = $\oppocircline{H_2}$"\\
{\bf lemma} "$\lbrakk$o\_circline\_set $H_1$ = o\_circline\_set $H_2$; o\_circline\_set $H_1$ $\neq$ $\{\}$$\rbrakk$ $\Longrightarrow$ \\
\> $H_1 = H_2$ $\vee$ $H_1 = \oppocircline{H_2}$"
  \end{tabbing}
}

Дата Хермитеова матрица кругоправе представља тачно једну од две
могуће оријентисане кругоправе. Избор шта ћемо звати позитивно
оријентисана кругоправа је произвољан. Ми смо одлучили да пратимо
приступ који је предложио Schwerdtfeger \cite{schwerdtfeger}, где се
користи водећи коефицијент $A$ као први критеријум, који каже да
кругоправe са матрицом у којој важи $A > 0$ се зову позитивно
оријентисане, а ако у матрици важи $A < 0$ онда се зову негативно
оријентисане.  Ипак, Schwerdtfeger није дискутовао још један могући
случај када је $A = 0$ (у случају прави), тако да смо ми морали да
проширимо његову дефиницију да би имали потпуну карактеризацију.

{\tt
  \begin{tabbing}
    \hspace{5mm}\=\hspace{5mm}\=\hspace{5mm}\=\hspace{5mm}\=\hspace{5mm}\=\kill
{\bf definition} "pos\_o\_circline\_rep {\bf where} "pos\_o\_circline\_rep $H$ $\longleftrightarrow$\\
\>(l\=et ($A$, $B$, $C$, $D$) = $\Repcm{H}$\\ 
\>\>  in \=Re $A > 0$ $\vee$ \\
\>\>\>(Re $A = 0$ $\wedge$ (($B \neq 0$ $\wedge$ arg $B > 0$) $\vee$ ($B = 0$ $\wedge$ Re $D > 0$))))"\\
{\bf lift\_definition} pos\_o\_circline :: "o\_circline $\Rightarrow$ bool" {\bf is} pos\_o\_circline\_rep
  \end{tabbing}
}

\noindent Сада, тачно једна од две супротно оријентисане кругоправе је
позитивно оријентисана.

{\tt
  \begin{tabbing}
    \hspace{5mm}\=\hspace{5mm}\=\hspace{5mm}\=\hspace{5mm}\=\hspace{5mm}\=\kill
{\bf lemma} \="pos\_o\_circline $H$ $\vee$ pos\_o\_circline ($\oppocircline{H}$)"\\
\>  "pos\_o\_circline ($\oppocircline{H}$) $\longleftrightarrow$ $\neg$ pos\_o\_circline $H$"
  \end{tabbing}
}

Оријентација кругова је и алгебраски једноставна (посматра се знак
коефицијента $A$) и геометријски природан захваљујући следећој
једнноставној карактеризацији.

{\tt
  \begin{tabbing}
    \hspace{5mm}\=\hspace{5mm}\=\hspace{5mm}\=\hspace{5mm}\=\hspace{5mm}\=\kill
{\bf lemma} "$\infty_h$  $\notin$ o\_circline\_set $H$ $\Longrightarrow$ pos\_o\_circline $H$ $\longleftrightarrow$ $\infty_h$ $\notin$ disc $H$"
  \end{tabbing}
}

\noindent Још једна лепа геометријска карактеризација за позитивну
оријентацију је да Еуклидски центар позитивно оријентисаних Еуклидових
круговова је садржан у њиховом диску.

{\tt
  \begin{tabbing}
    \hspace{5mm}\=\hspace{5mm}\=\hspace{5mm}\=\hspace{5mm}\=\hspace{5mm}\=\kill
{\bf lemma} \={\bf assumes} \="is\_circle ($\ofocircline{H}$)" "circline\_type ($\ofocircline{H}$) < 0"\\
\>\> "($a$, $r$) = euclidean\_circle ($\ofocircline{H}$)"\\
\>{\bf shows} "pos\_oriented H $\longleftrightarrow$ of\_complex a $\in$ disc H" 
  \end{tabbing}
}


\noindent Приметимо да оријентација прави и тачка кругова је вештачки
уведена (само да бисмо имали тотално дефинисану позитивну
оријентацију), и она нема природну геометријску интерпретацију. Ово
прекида непрекидност оријентације и ми мислимо да није могууће увести
оријентацију прави тако да функција оријентације буде свуда
непрекидна. Зато, када у неким наредним лемама будемо говорили о
оријентацији ми ћемо експлицитно искључити случај прави.

Tоталнa карактеризацијa за позитивну оријентацију нам омогућава да
створимо пресликавање из неоријентисаних у оријентисану кругоправу
(добијамо увек позитивно оријентисану кругоправу).

{\tt
  \begin{tabbing}
    \hspace{5mm}\=\hspace{5mm}\=\hspace{5mm}\=\hspace{5mm}\=\hspace{5mm}\=\kill
{\bf definition} of\_circline\_rep :: "C2\_mat\_herm $\Rightarrow$ C2\_mat\_herm" {\bf where}\\
\> "of\_circline\_rep $H$ = (\=if pos\_o\_circline\_rep $H$ then $H$\\
\>\>else opp\_o\_circline\_rep $H$)"\\
{\bf lift\_definition} of\_circline ($\ofcircline{\_}$) :: "circline $\Rightarrow$ o\_circline" {\bf is} of\_circline\_rep
  \end{tabbing}
}
\noindent Показана су бројна својства функције {\tt of\_circline}, а
ми смо овде записали само најзначајнија.

{\tt
  \begin{tabbing}
    \hspace{5mm}\=\hspace{5mm}\=\hspace{5mm}\=\hspace{5mm}\=\hspace{5mm}\=\kill
{\bf lemma} \="o\_circline\_set ($\ofcircline{H}$) = circline\_set $H$"\\
{\bf lemma} \>"pos\_o\_circline ($\ofcircline{H}$)"\\
{\bf lemma} \>"$\ofocircline{(\ofcircline{H})}$ = $H$" "pos\_o\_circline $H$ $\Longrightarrow$ $\ofcircline{(\ofocircline{H})}$ = $H$"\\
{\bf lemma} "$\ofcircline{H_1}$ = $\ofcircline{H_2}$ $\Longrightarrow$ $H_1 = H_2$"
  \end{tabbing}
}


\paragraph{Дејство Мебијусових трансформација на оријентисане кругоправе.} 
На репрезентативном нивоу дејство Мебијусових трансформација на
оријентисане кругоправе је исто као и дејство на неоријентисане
кругоправе.  {\tt
  \begin{tabbing}
    \hspace{5mm}\=\hspace{5mm}\=\hspace{5mm}\=\hspace{5mm}\=\hspace{5mm}\=\kill
{\bf lift\_definition} mobius\_o\_circline :: "mobius $\Rightarrow$ o\_circline $\Rightarrow$ o\_circline" {\bf is} \\
\>mobius\_circline\_rep
  \end{tabbing}
}

\noindent Дејство Мебијуса на (неоријентисане) кругоправе се може
дефинисати коришћењем дефиниције за дејство Мебијуса на оријентисане
кругоправе, али обрнуто не би могло.  {\tt
  \begin{tabbing}
    \hspace{5mm}\=\hspace{5mm}\=\hspace{5mm}\=\hspace{5mm}\=\hspace{5mm}\=\kill
{\bf lemma} "\=mobius\_circline $M$ $H$ = $\ofocircline{(\mathtt{mobius\_o\_circline}\ M\ (\ofcircline{H}))}$"\\
{\bf lemma} "l\=et \=$H_1$ = mobius\_o\_circline $M$ $H$; $H_2$ = $\ofcircline{(\mathtt{mobius\_circline}\ M\ (\ofocircline{H}))}$ \\
\>in  $H_1$ = $H_2$ $\vee$ $H_1$ = $\oppocircline{H_2}$"
  \end{tabbing}
}

\noindent Дејство Мебијусових трансформација на оријетисане кругоправе
има слична својства као и дејство Мебијусових трансформација на
неоријентисане кругоправе. На пример, оне се слажу у погледу инверза
({\tt {\bf lemma} "mobius\_o\_circline (mobius\_inv $M$) = inv
  (mobius\_o\_circline $M$)"}), композиције, трансформације која је
идентитет, оне су обе ињективне ({\tt inj mobius\_circline}), и тако
даље. Централне леме у овом одељку повезују дејство Мебијусових
трансформација на тачке, оријентисане кругоправе и дискове.

{\tt
  \begin{tabbing}
    \hspace{5mm}\=\hspace{5mm}\=\hspace{5mm}\=\hspace{5mm}\=\hspace{5mm}\=\kill
{\bf lemma} "\=mobius\_pt $M$ ` o\_circline\_set H = \\
\>o\_circline\_set (mobius\_o\_circline $M$ $H$)"\\
{\bf lemma} "\=mobius\_pt $M$ ` disc H = disc (mobius\_o\_circline $M$ $H$)"\\
{\bf lemma} "\=mobius\_pt $M$ ` disc\_compl H = disc\_compl (mobius\_o\_circline $M$ $H$)"
  \end{tabbing}
}

Све Еуклидске сличности чувају оријентацију кругоправе.  {\tt
  \begin{tabbing}
    \hspace{5mm}\=\hspace{5mm}\=\hspace{5mm}\=\hspace{5mm}\=\hspace{5mm}\=\kill
{\bf lemma}
  assumes "$a$ $\neq$ $0$" "$M$ = similarity $a$ $b$" "$\infty_{hc}$ $\notin$ o\_circline\_set $H$"\\
\>  shows "pos\_o\_circline $$H$$ $\longleftrightarrow$ pos\_o\_circline (mobius\_o\_circline $M$ $H$)"
  \end{tabbing}
}

\noindent Оријентација слике дате оријентисане кругоправе $H$ након
дате Мебијусове трансформације $M$ зависи од тога да ли пол $M$ (тачка
која слика $M$ у $\infty_{hc}$) лећи на диску или у диску који је
комплементаран са $H$ (ако је у скупу $H$, онда се слика у праву, а у
том случају не дискутујемо оријентацију

{\tt
  \begin{tabbing}
  \hspace{3mm}\=\hspace{5mm}\=\hspace{5mm}\=\hspace{5mm}\=\hspace{5mm}\=\kill
{\bf lemma}\\
\>"$0_{hc}$ $\in$ disc\_compl $H$ $\Longrightarrow$ pos\_o\_circline (mobius\_o\_circline reciprocation $H$)"\\
\>"$0_{hc}$ $\in$ disc $H$ $\Longrightarrow$ $\neg$ pos\_o\_circline (mobius\_o\_circline reciprocation $H$)"\\
{\bf lemma}\\
\>assumes "$M$ = mk\_mobius a b c d" "c $\neq$ 0" "a*d - b*c $\neq$ 0"\\
\>shows \="pole $M$ $\in$ disc $H$ $\longrightarrow$ $\neg$ pos\_o\_circline (mobius\_o\_circline $M$ $H$)"\\
\>\>"pole $M$ $\in$ disc\_compl $H$ $\longrightarrow$ pos\_o\_circline (mobius\_o\_circline $M$ $H$)"
  \end{tabbing}
}

\noindent Приметимо да је ово другачије него што тврди Schwerdtfeger
\cite{schwerdtfeger}: "Реципроцитет чува оријентацију круга који не
садржи 0, али инвертује оријетацију било ког круга који садржи 0 као
унутрашњу тачку. Свака Мебијусова трансформација чува оријентацију
било ког круга који не садржи свој пол. Ако круг садржи свој пол, онда
круг који је слика има супротну оријентацију". Наша формализација
показује да оријентација резултујућег круга не зависи од оријентације
полазног круга (на пример, у случају реципроцитета, оријентација
полазног круга показује релативну позицију круга и тачке бесконачно
што је одређено знаком коефицијента $A$ у репрезентативној матрици и
то је сасвим независно од релативне позиције круга и нула тачке које
су одређене знаком коефицијента $D$ --- ова два коефицијента се
размењују приликом примене трансформације реципроцитет).


\paragraph{Очување угла.}
Мебијусове трансформације су конформно пресликавање, што значи да оне
чувају оријентисане углове међу оријентисаним кругоправама. Ако се
угао дефинише коришћењем чисто алгебарског приступа (пратећи
\cite{schwerdtfeger}), онда је врло лако показати ово својство. Овде
ћемо навести дефинициу и мешовите детерминанте коју смо дефинисали
раније у нашој основној теорији.
 {\tt
  \begin{tabbing}
    \hspace{3mm}\=\hspace{5mm}\=\hspace{5mm}\=\hspace{5mm}\=\hspace{5mm}\=\kill
{\bf fun} mat\_det\_mix :: "C2\_mat $\Rightarrow$ C2\_mat $\Rightarrow$ complex" {\bf where}\\
\> "mat\_det\_mix $(A_1, B_1, C_1, D_1)$ $(A_2, B_2, C_2, D_2)$ =\\
\>\> $A_1*D_2 - B_1*C_2 + A_2*D_1 - B_2*C_1$"\\
{\bf definition} cos\_angle\_rep {\bf where}\\
\>  "cos\_angle\_rep $H_1$ $H_2$ = \=- Re (mat\_det\_mix $\Repcm{H_1}$ $\Repcm{H_2}$) / \\
\> \> 2 * (sqrt (Re (mat\_det $\Repcm{H_1}$ * mat\_det $\Repcm{H_2}$))))"\\
{\bf lift\_definition} cos\_angle :: "o\_circline $\Rightarrow$ o\_circline $\Rightarrow$ complex" {\bf is}\\
\> cos\_angle\_rep\\
{\bf lemma} "cos\_angle $H_1$ $H_2$ = \\
\> cos\_angle (moebius\_o\_circline $M$ $H_1$) (moebius\_o\_circline $M$ $H\_2$)"
  \end{tabbing}
}

Ипак, ова дефиниција није интуитивна, и из педагошких разлога желели
смо да је повежемо са нешто уобичајенијом дефиницијом. Прво,
дефинисали смо угао између два комплексна вектора ($\downharpoonright
\_ \downharpoonleft$ означава функцију за нормализацију угла која је
описана раније).  {\tt
  \begin{tabbing}
    \hspace{5mm}\=\hspace{5mm}\=\hspace{5mm}\=\hspace{5mm}\=\hspace{5mm}\=\kill
{\bf definition} ang\_vec ("$\measuredangle$") {\bf where} "$\measuredangle$ $z_1$ $z_2$ = $\downharpoonright$arg $z_2$ - arg $z_1$$\downharpoonleft$"    
  \end{tabbing}
}

За дати центар $\mu$ обичног Еуклидског круга и тачку $z$ на њему,
дефинишемо тангетни вектор у $z$ као радијус вектор
$\overrightarrow{\mu z}$, ротиран за $\pi/2$, у смеру казаљке на сату
или у супротном смеру у зависности од оријентације.  {\tt
  \begin{tabbing}
    \hspace{5mm}\=\hspace{5mm}\=\hspace{5mm}\=\hspace{5mm}\=\hspace{5mm}\=\kill
{\bf definition} tang\_vec :: "complex $\Rightarrow$ complex $\Rightarrow$ bool $\Rightarrow$ complex" {\bf where}\\
\>"tang\_vec $\mu$ $z$ $p$ = sgn\_bool $p$ * $i$ * ($z$ - $\mu$)"
  \end{tabbing}
}
\noindent У болеан променљивој $p$ енкодира се оријентација круга, а
функција {\tt sgn\_bool $p$} враћа $1$ када је $p$ тачно, а $-1$ када
је $p$ нетачно. Коначно, угао између два оријентисана круга у њиховој
заједничкој тачки $z$ се дефинише као угао између тангентних вектора у
$z$.

{\tt
  \begin{tabbing}
    \hspace{5mm}\=\hspace{5mm}\=\hspace{5mm}\=\hspace{5mm}\=\hspace{5mm}\=\kill
{\bf definition} ang\_circ {\bf where}\\
\> "ang\_circ $z$ $\mu_1$ $\mu_2$ $p_1$ $p_2$ = $\measuredangle$ (tang\_vec $\mu_1$ $z$ $p_1$) (tang\_vec $\mu_2$ $z$ $p_2$)"
  \end{tabbing}
}

\noindent Коначно, веза између алгебарске и геометријске дефиниције
косинуса угла дата је следећом лемом.

{\tt
  \begin{tabbing}
    \hspace{5mm}\=\hspace{5mm}\=\hspace{5mm}\=\hspace{5mm}\=\hspace{5mm}\=\kill
{\bf lemma} {\bf assumes} "is\_circle ($\ofocircline{H_1}$)" "is\_circle ($\ofocircline{H_2}$)"\\
\>\>  "circline\_type ($\ofocircline{H_1}$) $< 0$" "circline\_type ($\ofocircline{H_2}$) $< 0$"\\
\>\>  "($\mu_1$, $r_1$) = euclidean\_circle ($\ofocircline{H_1}$)"\\
\>\>  "($\mu_2$, $r_2$) = euclidean\_circle ($\ofocircline{H_2}$)"\\
\>\>  "of\_complex $z$ $\in$ o\_circline\_set H1 $\cap$ o\_circline\_set H2"\\
\>{\bf shows} "\=cos\_angle $H_1$ $H_2$ = \\
\>\>cos (ang\_circ $z$ $\mu_1$ $\mu_2$ (pos\_o\_circline $H_1$) (pos\_o\_circline $H_2$))"
  \end{tabbing}
}
\noindent Да би доказали ову лему било је неопходно показати закон
косинуса у систему Isabelle/HOL, али се ово показало као веома
једноставан задатак.


\subsection{Неке важне подгрупе Мебијусових трансформација}
\label{subsec:classification}

Већ смо описали параболичку групу (групу Еуклидских сличности), кључну
за Еуклидску геометрију равни. Сада ћемо описати карактеристике две
веома важне подгрупе Мебијусове групе --- групу сферних ротација,
важну за елиптичку планарну геометрију, и групу аутоморфизама диска
која је важна за хиперболичку планарну геометрију.

\paragraph{Ротације сфере.}
Генерална унитарна група, коју означавамо са $GU_2(\mathbb{C})$ је
група која садржи све Мебијусове трансформације које су репрезентоване
уопштеним унитарним матрицама.  {\tt
  \begin{tabbing}
    \hspace{3mm}\=\hspace{5mm}\=\hspace{5mm}\=\hspace{5mm}\=\hspace{5mm}\=\kill
{\bf definition} unitary\_gen {\bf where}\\
\>"unitary\_gen $M$ $\longleftrightarrow$ $(\exists$ $k$::complex. $k \neq 0$ $\wedge$ mat\_adj $M *_{mm} M$ = $k$ $*_{sm}$ eye$)$"
  \end{tabbing}
}

\noindent Иако је у дефиницији дозовљено да $k$ буде комплексан
фактор, испоставља се да је једино могуће да $k$ буде
реалан. Генереализоване унитарне матрице могу бити растављене на
обичне унитарне матрице и позитивне садржаоце (?? ovo mi se nesto ne
svidja, nesto ovde nije bas najbolje) јединичне матрице.

{\tt
  \begin{tabbing}
    \hspace{5mm}\=\hspace{5mm}\=\hspace{5mm}\=\hspace{5mm}\=\hspace{5mm}\=\kill
{\bf definition} unitary {\bf where} "unitary $M$ $\longleftrightarrow$ mat\_adj $M *_{mm} M$ = eye"\\
{\bf lemma} "unitary\_gen $M$ $\longleftrightarrow$ \\
\> $($$\exists\ k\ M'$. $k > 0$ $\wedge$ unitary $M'$ $\wedge$ $M$ = (cor $k$ $*_{sm}$ eye) $*_{mm}$ $M'$)"
  \end{tabbing}
}

Група унитарних матрица је веома важна јер описује све ротације
Риманове сфере (изоморфна је реалној специјалној ортогоналној групи
$SO_3(\mathbb{R})$). Једна карактеризација $GU_2(\mathbb{C})$ у
$\extC$ је да је то група трансформација која које је имагинарни
јединични круг фиксан (ово је круг чија матрица репрезентације је
јединична и налази се у равни у бесконачности).

{\tt
  \begin{tabbing}
    \hspace{5mm}\=\hspace{5mm}\=\hspace{5mm}\=\hspace{5mm}\=\hspace{5mm}\=\kill
{\bf lemma} "mat\_det $(A, B, C, D)$ $\neq 0$ $\Longrightarrow$ unitary\_gen ($A$, $B$, $C$, $D$)  $\longleftrightarrow$\\
\>moebius\_circline (mk\_moebius $A$ $B$ $C$ $D$) imag\_unit\_circle = \\
\>imag\_unit\_circle"
  \end{tabbing}
}


Карактеризација генерализованих унитарних матрица у координатама је
дата са следећом лемом. {\tt
  \begin{tabbing}
    \hspace{5mm}\=\hspace{5mm}\=\hspace{5mm}\=\hspace{5mm}\=\hspace{5mm}\=\kill
{\bf lemma} "unitary\_gen M $\longleftrightarrow$ $($$\exists$ $a$ $b$ $k$.\ let $M' = (a,\,b,\,-\mathtt{cnj}\ b,\,\mathtt{cnj}\ a)$ in \\
\>$k \neq 0$ $\wedge$ mat\_det $M' \neq 0$ $\wedge$ $M = k *_{sm} M'$$)$"
  \end{tabbing}
}

Успут дефинисали смо специјалну унитарну групу $SU_2(\mathbb{C})$,
која садржи генерализоване унитарне матрице са детерминантом једнаком
један (?? ovo za determinantu proveri) (оне се препознају по форми
$(a,\,b,\,-\mathtt{cnj}\ b,\,\mathtt{cnj}\ a)$, без множитеља $k$, и
ми ово користимо да би извели координатну форму генерализованих
унитарних матрица).


\paragraph{Аутоморфизми диска.}
Дуална група претходној групи трансформација је група генерализованих
унитарних матрица чија сигнатура је $1-1$ ($GU_{1,1}(\mathbb{C})$).

{\tt
  \begin{tabbing}
    \hspace{5mm}\=\hspace{5mm}\=\hspace{5mm}\=\hspace{5mm}\=\hspace{5mm}\=\kill
{\bf definition} unitary11 {\bf where}\\
\>"unitary11 $M$ $\longleftrightarrow$ mat\_adj $M *_{mm} (1, 0, 0, -1) *_{mm} M = (1, 0, 0, -1)$"\\
{\bf definition} unitary11\_gen {\bf where}\\
\>"unitary11\_gen $M$ $\longleftrightarrow$ $(\exists$ $k$::complex. $k \neq 0$ $\wedge$\\
\>\>mat\_adj $M$ $*_{mm} (1, 0, 0, -1) *_{mm}$ $M$ = $k$ $*_{sm}$ $(1, 0, 0, -1)$$)$"
  \end{tabbing}
}
\noindent Поново, дефиниција дозвољава комплексан фактор $k$, 
али се показује да једино реални фактори имају смисла.

Карактеризација $GU_{1,1}(\mathbb{C})$ је да она садржи све Мебијусове
трансформације које фиксирају јединични круг.

{\tt
  \begin{tabbing}
    \hspace{5mm}\=\hspace{5mm}\=\hspace{5mm}\=\hspace{5mm}\=\hspace{5mm}\=\kill
{\bf lemma} "mat\_det $(A, B, C, D)$ $\neq 0$ $\Longrightarrow$ unitary11\_gen ($A$, $B$, $C$, $D$)  $\longleftrightarrow$\\
\>moebius\_circline (mk\_moebius $A$ $B$ $C$ $D$) unit\_circle = unit\_circle"
  \end{tabbing}
}

Карактеризација генерализоване унитарне 1-1 матрице у координатама је
дата са следећим лемама.  {\tt
  \begin{tabbing}
    \hspace{5mm}\=\hspace{5mm}\=\hspace{5mm}\=\hspace{5mm}\=\hspace{5mm}\=\kill
{\bf lemma} "unitary11\_gen $M$ $\longleftrightarrow$ $($$\exists$ $a$ $b$ $k$. let $M' = (a,\,b,\,\mathtt{cnj}\ b,\,\mathtt{cnj}\ a)$ in \\
\> $k \neq 0$ $\wedge$ mat\_det $M' \neq 0$ $\wedge$ $(M = k *_{sm} M'$ $\vee$ $M = k *_{sm} (\mathtt{cis}\ pi,\,0,\,0,\,1) *_{sm} M'$$))$\\
{\bf lemma} "unitary11\_gen M $\longleftrightarrow$ $($$\exists$ $a$ $b$ $k$. let $M' = (a,\,b,\,\mathtt{cnj}\ b,\,\mathtt{cnj}\ a)$ in \\
\> $k \neq 0$ $\wedge$ mat\_det $M' \neq 0$ $\wedge$ $M = k *_{sm} M'$ $)$"
  \end{tabbing}
}
\noindent Приметимо да је прва лема садржана у другој леми. Ипак, било
је лакше доказати прву лему и то даје матрице следећег облика $k
*_{sm} (a,\,b,\,-{\tt cnj}\ b,\, -{\tt cnj}\ a)$ --- геометријски,
друга група трансформација комбинује прву групу са додатном централном
симетријом.

Још једна важна карактеризација ових трансформација је коришћењем
такозваног Блашке фактора. Сва трансформација је композиција Блашке
фактора (рефлексије која неку тачку која је на јединичној кружници
слика у нула) и ротације.

{\tt
  \begin{tabbing}
    \hspace{5mm}\=\hspace{5mm}\=\hspace{5mm}\=\hspace{5mm}\=\hspace{5mm}\=\kill
{\bf lemma} \={\bf assumes} \="$k \neq 0$" "$M' = (a,\,b,\,\mathtt{cnj}\ b,\,\mathtt{cnj}\ a)$"\\
\>\>"$M = k *_{sm} M'$" "mat\_det $M' \neq 0$" "$a \neq 0$"\\
\hspace{5mm}\=\kill
\>{\bf shows} "\=$\exists$ $k'$ $\phi$ $a'$. $k' \neq 0$ $\wedge$ $a' * \mathtt{cnj}\ a' \neq 1$ $\wedge$\\
\>\> $M = k' *_{sm} (\mathtt{cis}\ \phi,\,0,\,0,\,1) *_{mm} (1,\,-a',\,-\mathtt{cnj}\ a',\,1)$"
  \end{tabbing}
}
\noindent Изузетак је у случају када $a=0$ и онда уместо Блашке
фактора, користи се реципроцитет (бесконачно замењује $a'$ у
претходној леми).  {\tt
  \begin{tabbing}
    \hspace{5mm}\=\hspace{5mm}\=\hspace{5mm}\=\hspace{5mm}\=\hspace{5mm}\=\kill
{\bf lemma} \={\bf assumes} \="$k \neq 0$" "$M' = (0,\,b,\,\mathtt{cnj}\ b,\,0)$" "$b \neq 0$" "$M = k *_{sm} M'$" \\
\>{\bf shows} "\=$\exists$ $k'$ $\phi$. $k' \neq 0$ $\wedge$ $M = k' *_{sm} (\mathtt{cis}\ \phi,\,0,\,0,\,1) *_{mm} (0,\,1,\,1,\,0)$"
  \end{tabbing}
}

Матрице $GU_{1,1}(\mathbb{C})$ се природно деле у две подгрупе.  Све
трансформације фиксирају јединични круг, али прва подгрупа се састоји
од трансформација које мапирају јединични диск у самог себе (такозвани
\emph{аутоморфизми диска}), док се друга подгрупа састоји из
трансформација које размењују јединични диск и његов комплемент. За
дату матрицу, њена подгрупа се једино може одредити посматрајући знак
детерминанте $M' = (a,\,b,\,\mathtt{cnj}\ b,\,\mathtt{cnj}\ a)$. Ако
је само $M = (a_1, b_1, c_1, d_1)$ дато, а нису дати $M'$, а ни $k$,
онда је критеријум за утврђивање подгрупе вредност $\mathtt{sgn}
(\mathtt{Re}\ ((a_1*d_1)/(b_1*c_1)) - 1)$.

Приметимо да су све важне подгрупе овде описане једино у терминима
алгебре. Формализовали смо и неке геометријске доказе који дају
еквиваленту карактеризацију овима које смо већ описали. Додатно, важи
да су сви аналитички аутоморфизми диска једнаки композицији Блашке
фактора и ротација (ипак, доказ се заснива на математичкој анализи,
принципу максималног модула и Шварцовој леми -- техникама које ми
нисмо узимали у обзир). Чак и слабије тврђење да су сви Мебијусови
аутоморфизми диска ове форме није још формално доказано (кључни корак
је показати да аутоморфизми диска фиксирају јединични круг, а то је
нешто што нисмо могли показати без детаљног испитивања топологије на
чему тренутно радимо).


\section{Дискусија}
\label{sec:discuss}
Визуелно, геометријски аргументи се често користе у доказима у
уџбеницима. Као пример, ми ћемо демонстрирати доказ о очувању својства
угла након примене Мебијусових трансформација на који се често може
наићи у различитим књигама о овој теми (у овом поглављу ми ћемо
пратити приступ Needham \cite{needham} чији циљ није веома формална
књига, али, ипак, овакав начин резоновања присутан је и код многих
других аутора).

Прво важно питање је појам угла. Углови могу бити дефинисани између
оријентисаних или неоријентисаних криви, а и сами углови могу бити
оријентисани или неорјентисани.  Needham дефинише угао између две
криве на следећи начин: " Нека су $S_1$ и $S_2$ криве које се секу у
тачки $z$. Као што је илустровано, ми можемо повући њихове тангенте
$T_1$ и $T_2$ у тачки $z$. Угао између криви $S_1$ и $S_2$ у њиховој
заједничкој тачки $z$ је оштар угао $\alpha$ од $T_1$ до $T_2$. Значи
овај угао $\alpha$ има знак који му је додељен: угао између $S_2$ и
$S_1$ је минус илустровани угао између $S_1$ и $S_2$." То значи да је
угао дефинисан само између неоријентисаних криви (и то је различито у
односу на нашу дефиницију), али сам угао је оријентисан (а то је исто
као и у нашој финалној дефиницији).  У раној фази наше формализације
ми смо дефинисали и користили неоријентисани конвексан и оштар угао
између два вектора.

{\tt
  \begin{tabbing}
    \hspace{5mm}\=\hspace{5mm}\=\hspace{5mm}\=\hspace{5mm}\=\hspace{5mm}\=\kill
{\bf definition} "$\measuredangle_c$" {\bf where} "$\measuredangle_c$ $z_1$ $z_2$ $\equiv$ abs ($\measuredangle$ $z_1$ $z_2$)"\\
{\bf definition} acutize {\bf where} "acutize $\alpha$ = (if $\alpha$ $>$ $\frac{\pi}{2}$ then $\pi$ - $\alpha$ else $\alpha$)"\\
{\bf definition} "$\measuredangle_a$" {\bf where} "$\measuredangle_a$ $z_1$ $z_2$ $\equiv$ acutize ($\measuredangle_c$ $z_1$ $z_2$)"
  \end{tabbing}
}

Како су наше кругоправе оријентисане од старта, ми смо показали да на
оштар угао између два круга не утиче оријентација и да се он може
изразити у терминима три тачке (тачке пресека и тачке које
представљају центар круга).

{\tt
  \begin{tabbing}
    \hspace{5mm}\=\hspace{5mm}\=\hspace{5mm}\=\hspace{5mm}\=\hspace{5mm}\=\kill
{\bf lemma} "$\lbrakk z \neq \mu_1$;$z \neq \mu_2\rbrakk$ $\Longrightarrow$ ang\_circ\_a $z$ $\mu_1$ $\mu_2$ $p_1$ $p_2$ = $\measuredangle_a\ (z - \mu_1)\ (z - \mu_2)$"
\end{tabbing}
}
\noindent Функција {\tt ang\_circ\_a} је дефинисана као оштар угао
између два тангетна вектора (слично {\tt ang\_circ} у нашој коначној
формализацији).

Доказ да Мебијусова трансформација чува угао који стоји у уџбенику
\cite{needham} се ослања на чињеницу да свака Мебијусова
трансформација се може раставити на транслацију, ротацију, дилетацију
и инверзију. Чињеница да транслације, ротације и дилетације чувају
угао је узета као подразумевана и није доказивана (и да будемо искрени
формализације ове чињенице није била тешка када смо успели да све
појмове формално дефинишемо на одговарајући начин). Централни изазов
је показати да инверзија чува углове, тј. да "Инверзија је
антикомфорно пресликавање".  Доказ се заснива на "чињеници да за било
коју дату тачку $z$ која није на кругу инверзије $K$, постоји тачно
један круг који је ортогоналан на $K$ и пролази кроз $z$ у било ком
правцу". Даље, доказ се наставља са "Претпоставимо да две криве $S_1$
и $S_2$ се секу у $z$, и да су њихове тангенте $T_1$ и $T_2$, а угао
између њих је $\alpha$. Да би сазнали шта се дешава са углом након
инверзије у односу на $K$, заменимо $S_1$ и $S_2$ са јединственим
круговима $R_1$ и $R_2$ ортогоналним на $K$ који пролазе кроз $z$ у
истом смеру као што је и смер $S_1$ и $S_2$, тј., круговима чије
тангенте у $z$ су $T_1$ и $T_2$. Како инверзија у односну на $K$ слика
сваки од ових кругова на саме себе нови угао у $\tilde{z}$ је
$-\alpha$. Крај.''

У нашем ранијем покушају ми смо формализовали овај "доказ", али је ово
захтевало веома велику количину уложеног труда у поређену са углађеним
алгебарским доказом у нашој финалној формализацији.  Прво, уџбеник је
често врло непрецизан у томе да ли се користи "комплексна инверзија"
или "геометријска инверзија" (тј. према нашим терминима које смо
раније увели -- да ли се користи реципроцитет или инверзија). У доказу
из уџбеника аутор користи инверзију у односу на произвољан круг $K$,
али је довољно посматрати само реципроцитет (који је увек дат у односу
на јединични круг). Формализација резоновња које је дато у уџбенику је
већ дало прилично велике формуле, и било би још компликованије и
монотоније (ако је уопште и могуће) завршити доказ коришћењем
инверзије у односу на произвољни круг. На пример, једноставан
реципроцитет круга са центром $\mu$ и радујусом $r$ даје круг са
центром $\tilde{\mu} = \mu / \mathtt{cor}\ (|\mu|^2 - r^2)$, и
радијусом $\tilde{r} = r / ||\mu|^2 - r^2|$, и ова веза би била још
комплекснија за произвољну Мебијусову трансформацију, ако би била
записана у координатама, без коришћења појма матрица као што смо ми
радили у нашој главној формализацији.

Формални запис тврђења о очувању угла је следећи.

{\tt
  \begin{tabbing}
    \hspace{5mm}\=\hspace{5mm}\=\hspace{5mm}\=\hspace{5mm}\=\hspace{5mm}\=\kill
{\bf lemma } \\
 \> {\bf assumes} \= "$z$ $\in$ circle $\mu_1$ $r_1$" "$z$ $\in$ circle $\mu_2$ $r_2$"  \\
  \>               \> "inv ` circle $\mu_1$ $r_1$ = circle $\tilde{\mu_1}$ $\tilde{r_1}$"   \\
  \>               \> "inv ` circle $\mu_2$ $r_2$ = circle $\tilde{\mu_2}$ $\tilde{r_2}$"\\
  \> {\bf shows} "ang\_circ\_a $z$ $\mu_1$ $\mu_2$ = ang\_circ\_a $\tilde{z}$ $\tilde{\mu_1}$ $\tilde{\mu_2}$"
  \end{tabbing}
}

Поред тога што недостаје дискусија за броје специјалне случајеве, у
неформалном доказу недостаје и један значајан део.  Наиме, лако је
показати да $\tilde{z}$ је пресек $R_1$ и $R_2$ (то је пресек
$\tilde{S_1}$ и $\tilde{S_2}$, које су слике $S_1$ и $S_2$ након
инверзије), али показати да $R_1$ и $\tilde{S_1}$ и да $R_2$ и
$\tilde{S_2}$ имају исту тангенту у $\tilde{z}$ је захтевало не тако
тривијална израчунавања (тај доказ се заснива на чињеници да су центар
$\mu_i'$ круга $R_i$, центар $\tilde{\mu_i}$ круга $\tilde{S_i}$, и
$\tilde{z}$ колинеарни).

Једноставан аргумент симетрије који каже да су углови између два круга
у њиховим двема различитим тачкама пресека једнаки поново није било
једноставно формализовати.
{\tt
  \begin{tabbing}
    \hspace{5mm}\=\hspace{5mm}\=\hspace{5mm}\=\hspace{5mm}\=\hspace{5mm}\=\kill
{\bf lemma} {\bf assumes} \="$\mu_1$ $\neq$ $\mu_2$" "$r_1$ $>$ 0" "$r_2$ $>$ 0"\\
\>"$\{z_1, z_2\}$ $\subseteq$ circle $\mu_1$ $r_1$ $\cap$ circle $\mu_2$ $r_2$" "$z_1$ $\neq$ $z_2$" \\
    \hspace{5mm}\=\hspace{5mm}\=\hspace{5mm}\=\hspace{5mm}\=\hspace{5mm}\=\kill
  \> {\bf shows} "ang\_circ\_a $z_1$ $\mu_1$ $\mu_2$ = ang\_circ\_a $z_2$ $\mu_1$ $\mu_2$"
  \end{tabbing}
}
\noindent Ми смо показали ову лему тек након примене "бгно" резоновања
и померањем слике тако да центри два круга који се посматрају буду на
$x$-оси.

У доказу смо идентификовали бројне дегенеративне случајеве који су
морали да се анализирају одвојено. Прво смо морали да покажемо да
кругови који се секу имају исти центар (тј. да $\mu_1$$\neq$$\mu_2$)
само ако су они једнаки и онда је оштар угао између њих једнак
$0$. Ако су оба центра колинеарна са пресечном тачком $z$ (тј. ако
важи {\tt collinear $\mu_1$ $\mu_2$ $z$}), два круга се додирују (било
споља или изнутра), и опет је оштар угао једнак $0$.

Постојање круга $R_i$ који је ортогоналан на јединичну кружницу и који
има исту тангенту у датој тачки $z$ као и дати круг са центром $\mu_i$
је дато следећом лемом (што заправо даје центар $\mu_i'$ тог новог
круга).
{\tt
  \begin{tabbing}
    \hspace{5mm}\=\hspace{5mm}\=\hspace{5mm}\=\hspace{5mm}\=\hspace{5mm}\=\kill
{\bf lemma} \\
\> {\bf assumes} \="$\langle$$\mu_i$ - $z$, $z$$\rangle$ $\neq 0$"\\
\>\> "$\mu_i'$ = $z$ + (1 - $z$*$\mathtt{cnj}\ z$) * ($\mu_i$ - $z$) / $(2 * \langle$$\mu_i$ - $z$, $z$$\rangle$)"\\
  \> {\bf shows} "collinear $z$ $\mu_i$ $\mu_i'$" "$z$ $\in$ ortho\_unit\_circ $\mu'_i$" \\
  \end{tabbing}
}
\noindent Аналитички израз је открио још неке дегенеративне
случајеве. Бројилац може бити нула једино ако се круго секу на
јединичној кружници (тј. када је $z*\mathtt{cnj}\ z = 1$). У том
случају, доказ из уџбеника се не може применити јер је $\mu_1' =
\mu_2' = z$, и кругови $R_1$ и $R_2$ се не могу конструисати (они су
празни кругови). Случај када је именилац једнак нули (било за $\mu_1'$
или $\mu_2'$) је такође дегенеративан. Ово се дешава када су вектори
$\mu_i - z$ и $z$ ортогонални. Геометријски, у том случају се круг
$R_i$ дегенерише у праву (што и није проблем у проширеној комплексној
равни, али јесте проблем у поставци која важи у оригиналном доказу
која се налази у обичној комлексној равни). Зато, овај специјалан
случај мора да се анализира одвојено. Тако је наша формална анализа
брзо показала да једноставно тврђење у Needham да "за дату било коју
тачку $z$ која није на кругу инверзије $K$, постоји тачно један круг
који је ортогоналан на $K$ и пролази кроз $z$ у било ком задатом
правцу" није тачна у многим случајевима.


\section{Закључци и даљи рад}
\label{sec:concl}
У овом раду смо показали неке елементе наше формализације геометрије
проширене комплексне равни $\extC$ коришћењем комплексне пројективне
равни, али и Риманове сфере. Формализовали смо аритметичке операције у
$\extC$, размеру и дворазмеру, тетивну метрику у $\extC$, групу
Мебијусових трансформација и њихово дејство на $\extC$, неке њене
специјалне подгрупе (Еуклидске сличности, ротације сфере, аутоморфизме
диска), кругоправе и њихову везу са круговима и правама, тетивном
метриком, Римановом сфером, једниственост кругоправи, дејство
Мебијусових трансформација на кругоправе, типове и кардиналност скупа
кругоправе, оријентисане кругоправе, однос између Мебијусових
трансформација и оријентације, својство очувања угла након дејстав
Мебијусових трансформација итд. Наша тренутна теорија има око 12,000
лина Isabelle/HOL кода (сви докази су структурни и записани су у
језику за доказе Isabelle/Isar и наши ранији покушаји који су смањени
у краће алгебарске доказе нису укључени), око 125 дефиниција и око 800
лема.

Кључан корак у нашој формализацији је била одлука да се користи
алгебарска репрезентација свих важних објеката (вектора хомогених
координата, матрица за Мебијусове трансформације, Хермитеове матрице
за кругоправе итд.). Иако ово није нов приступ (на пример,
Schwerdtfeger's класична књига \cite{schwerdtfeger} прати овај приступ
прилично конзистентно), он ипак није тако уобичајен у литератури (и у
метеријалима курсева који се могу наћи на интернету). Уместо њега
преовладао је геометријски приступ.  Ми смо покушали да пратимо такву
врсту геометријског резоновања у раној фази нашег рада на овој теми,
али смо наишли на бројне потешкоће и нисмо имали много успеха. На
основу овог искуства, закључујемо да увођењем моћне технике ленеарне
алгебре омогућава значајно лакши рад на формализацији него што је то
случај када се користи геометријско резоновање.

Може се дискутовати да ли у неким случајевима геометријски аргументи
дају боље објашњење неких теорема, али када се посматра само
оправданост, алгебраски приступ је јасно супериорнији. Ипак, да би
имали везу са стандардним приступом у коме се користи геометријска
интуиција увели смо неколико додатних дефиниција (које су више
геометријске или више алгебарске) и морали смо показати да су ове
дефиниције еквивалентне. На пример, када је дефиниција угла дата само
коришћењем алгебарских операција на матрицама и њиховим
детерминантама, својство очувања угла је било веома лако показати, али
због образовне сврхе ово постаје значајно једино када се та дефиниција
споји са стандардном дефиницијом угла између криви (тј. њихових
тангетних вектора) --- у супротном, формализација постаје игра са
симболима који немају никакво значење.

Још један важан закључак до ког смо дошли је да у формалним
документима треба што чешће избегавати анализу случајева и екстензије
које ово омогућавају треба што чешће користити (нпр. било је много
боље користити хомогене координате уместо једне одвојене тачке
бесконачне коју би морали засебно да анализирамо у сваком тврђену или
дефиницији; слично, било је много лакше радити са кругоправама него
разликовати случај прави и кругова, итд.). Увођење два модела истог
концепта (на пример, у нашем случају, хомогених координата и Риманове
сфере) такође помаже, јер су неки докази лакши у једном моделу, а неки
у другом.

У принципу наши докази нису дугачки (15-20 линија у просеку). Ипак,
понекад је било потребно изводити веома досадне закључке, поготову
када се пребацивало са реалних на комплексне бројеве и обратно
(коришћењем функција за конверзију {\tt Re} и {\tt cor}). Ове
конверзије се углавном и не појављују у неформалном тексту и добро би
дошла нека аутоматизација оваквог закључивања. Аутоматизација система
Isabell је прилично моћна у резоновању једнакости у којима су обични
комплексни бројеви и ту смо често користили метод {\tt (simp add:
  field\_simps)} (са неким мањим изузецима), али када су у питању
неједнакости, аутоматизација није била добра и много тога смо морали
да показујемо ручно, корак по корак, а оваква тврђења се често
сматрају веома тривијалним у неформалном тексту.

У нашем даљем раду планирамо да користимо ове резулатате у
формализацији не-Еуклидских геометрија и њихових модула (посебно,
сферични модел елиптичке геометрије, Поинкареов диск модел и модел
горње полуравни хиперболичке геометрије).







