\chapter{Формализациjа аналитичке геометриjе}
\label{chapter::analiticka}

% ------------------------------------------------------------------------------
\section{Увод}
% ------------------------------------------------------------------------------

Синтетичка геометрија се обично изучава ригорозно, као пример
ригорозног аксиоматског извођења. Са друге стране, аналитичка
геометрија се углавном изучава неформално. Често се ова два приступа
представљају независно и веза између њих се ретко показује. Oваj рад
покушава да премости и више празнина за које мислимо да тренутно
постоје у формализацији геометрије.

\begin{enumerate}
\item Прво, наш циљ је да формализујемо аналитичку геометрију,
  тј. Де\-ка\-рто\-ву раван у оквиру интерактивног доказивача теорема, са
  ригорозним приступом, али веома блиско стандардном средњошколском
  образовању.  Представићемо добро изграђену формализацију Декартове
  геометрије равни у оквиру система \emph{Isabelle/HOL}.
\item Намеравамо да докажемо да су различите дефиниције основних
  појмова аналитичке геометрије које можемо видети у литератури
  заправо еквивалентне, и да заправо представљају јединствен
  апстрактни ентитет -- Декартову раван.  Дефиниције ћемо преузети из
  стандардних уџбеника.
\item Намеравамо да докажемо да стандардна геометрија координатне равни
  представља модел аксиоматског система Тарског. Наиме, доказаћемо да
  Декартова координатна раван задовољава све аксиоме Тарског.
\item За већину аксиома Хилберта доказаћемо да их задовољава Декартова
  координатна раван.
\item Потом, намеравамо да анализирамо доказе и да упоредимо који од
  два система аксиома, систем аксиома Тарског или Хилбертов систем
  аксиома, је лакши за формализацију.
\end{enumerate}


Поред тога што су многе теореме формализоване и доказане у оквиру
система \emph{Isabelle/HOL}, ми такође дискутујемо и наше искуство у
примени различитих техника за поједностављење доказа.  Најзначајнија
техника је "без губитка на општости" (``бгно''), која прати приступ
Харисона \cite{wlog}, а формална оправданост овог приступа је
постигнута коришћењем различитих изо\-ме\-три\-јских трансформација.



% ------------------------------------------------------------------------------
\section{Формализација геометрије Декартове равни}
\label{sec:cartesian}
% ------------------------------------------------------------------------------

Када се формализује теорија, мора се одлучити који појмови ће бити
осно\-вни, а који појмови ће бити дефинисани помоћу тих основних
појмова. Циљ наше формализације аналитичке геометрије је да успостави
везу са си\-нте\-ти\-чком геометријом, па зато има исте основне
појмове који су дати у си\-нте\-ти\-чком приступу. Свака геометрија
има класу објеката који се називају \emph{тачке}. Неке геометрије (на
пример, Хилбертова) имају и додатни скуп објеката који се називају
\emph{праве}, док неке геометрије (на пример, геометрија Тарског)
праве уопште не разматрају као примитивне објекте.  У неким
геометријама, праве су дефинисани појам, и оне су дефинисане као скуп
тачака.  Ово подразумева рад са теоријом скупова, а многе
аксиоматизације желе то да избегну.  У нашој формализацији
ана\-ли\-ти\-чке геометрије, ми ћемо дефинисати и тачке и праве јер
желимо да омогућимо анализу и геометрије Тарског и геометрије
Хилберта. Основна релација која спаја тачке и праве је релације
\emph{инциденције}, која неформално означава да права садржи тачку
(или дуално да се тачка налази на правoj). Други примитивни појмови (у
већини аксиоматских система) су релација \emph{изме\-ђу} (која дефинише
редослед колинеарних тачака) и релација \emph{подударности}.

Важно је напоменути да су у аналитичкој геометрији многи појмови често
дати у облику дефиниција, а заправо ти појмови су изведени појмови у
си\-нте\-ти\-чкој геометрији. На пример, у књигама за средњу школу
дефинише се да су праве нормалне ако је производ њихових праваца
$-1$. Ипак, ово нарушава везу са синтетичком геометријом (где је
нормалност изведени појам) јер би оваква карактеризација требало да
буде доказана као теорема, а не узета као дефиниција.

\subsection{Тачке у аналитичкој геометрији.}
Тачка у реалној координатној равни је одређена са својим $x$ и $y$
координатама. Зато, тачке су парови реалних бројева ($\mathbb{R}^2$),
што се може лако формализовати у Isabelle/HOL систему са {\tt
  \textbf{type\_synonym}\ point$^{ag}\ =\ "real \times real"$}.

\subsection{Редослед тачака.} Редослед (колинеарних) тачака се
дефинише коришћењем релације \emph{између}. Ово је релација која има
три аргумента, $\mathcal{B}(A, B, C)$ означава да су тачке $A$, $B$, и
$C$ колинеаране и да је тачка $B$ између тачака $A$ и $C$. Ипак, неке
аксиоматизације (на пример, аксиоматизација Тарског) дозвољава случај
када је тачка $B$ једнака тачки $A$ или тачки $C$. Рећи ћемо да је
таква релација\emph{између} {\em инклузивна}, док неке друге
аксиоматизације (на пример, Хилбертова аксиоматизација) не дозвољавају
једнакост тачака и тада кажемо да је релација \emph{између} {\em
  ексклузивна}. У првом случају, тачке $A$, $B$ и $C$ задовољавају
релацију \emph{између} ако постоји реалан број $0 \le k \le 1$ такав
да $\overrightarrow{AB} = k \cdot \overrightarrow{AC}$. Желимо да
избегнемо експлицитно коришћење вектора јер су они чешће изведени, а
ређе примитиван појам у синтетичкој геометрији, тако да релацију
\emph{између} формализујемо у Isabelle/HOL систему на следећи начин:

\selectlanguage{english}
{\tt 
\begin{tabbing}
\hspace{5mm}\=\hspace{5mm}\=\kill
\textbf{definition} "$\agbett{(xa, ya)}{(xb, yb)}{(xc, yc)} \longleftrightarrow$}\\
\>$(\exists (k::real).\ 0 \le k \ \wedge\ k \le 1 \ \wedge$\\
\>\>$(xb - xa) = k \cdot (xc - xa) \ \wedge\ (yb - ya) = k \cdot (yc - ya))$"}
\end{tabbing}
}
\selectlanguage{serbian}

\noindent Ако захтевамо да тачке $A$, $B$ и $C$ буду различите, онда
мора да важи $0 < k < 1$, и релацију ћемо означавати са
$\agbeth{}{}{}$.

\subsection{Подударност.} Релација \emph{подударно} дефинише се на паровима
тачака. Неформално, $\congrt{A}{B}{C}{D}$ означава да је дуж $AB$
\emph{подударана} дужи $CD$. Стандардна метрика у $\mathbb{R}^2$
дефинише да је растојање међу тачкама $A(x_A, y_A)$, $B(x_B, y_B)$
једнако $d(A, B) = \sqrt{(x_B-x_A)^2+(y_B-y_A)^2}$. Квадратно
растојање се дефинише као $\agsqdist{A}{B} =
(x_B-x_A)^2+(y_B-y_A)^2$. Тачке $A$ и $B$ су подударне тачкама $C$ и
$D$ ако и само ако $\agsqdist{A}{B} = \agsqdist{C}{D}$. У
\emph{Isabelle/HOL} систему ово се може формализовати на следећи
начин:

\selectlanguage{english}
{\tt
\begin{tabbing}
\textbf{definition} "$\agsqdist{(x_1, y_1)}{(x_2, y_2)} = (x_2-x_1)\cdot (x_2-x_1)+(y_2-y_1)\cdot (y_2-y_1)$"\\
\textbf{definition} "$\agcongr{A_1}{B_1}{A_2}{B_2} \longleftrightarrow \agsqdist{A_1}{B_1} = \agsqdist{A_2}{B_2}$"
\end{tabbing}
}
\selectlanguage{serbian}

\subsection{Права и инциденција.}

\paragraph{Једначина праве.}
Праве у Декартовој координатној равни се обично пре\-дста\-вља\-ју
једначинама облика $Ax + By + C = 0$, па тако тројка $(A, B, C) \in
\mathbb{R}^3$ означава праву. Ипак, тројке у којима је $A = 0$ и $B =
0$ морају бити изузете јер не представљају исправну једначину
праве. Такође, једначине $Ax + By + C = 0$ и $kAx + kBy + kC = 0$, за
реално $k \neq 0$, означавају исту праву. Зато права не може бити
дефинисана коришћењем само једне једначине, већ права мора бити
дефинисана као класа једначина које имају пропорционалне
коефицијенте. Формализација у систему \emph{Isabelle/HOL} се састоји
из неколико корака. Прво, дефинише се домен валидних тројки који су
коефицијенти једначине.  
\selectlanguage{english} {\tt
\begin{tabbing}
\textbf{typedef} line\_coeffs$^{ag}$ $=$ \\
\hspace{5mm}"$\{((A::real), (B::real), (C::real)).\ A \neq 0 \vee B \neq 0\}$"
\end{tabbing}
}
\selectlanguage{serbian}
\noindent Када је овај тип дефинисан, функција
{\tt Rep\_line\_coeffs} ($\RepRt{\_}$) конвертује апстрактне вредности овог типа
у њихове конкретне репрезентације (тројке реалних бројева), а функција
{\tt Abs\_line\_coeffs} ($\AbsRt{\_}$) конвертује (валидне) тројке у вредности
које припадају овом типу.

Две тројке су еквивалентне ако и само ако су пропорционалне.
\selectlanguage{english}
{\tt
\begin{tabbing}
\hspace{5mm}\=\kill
\textbf{definition} "$l_1 \approx^{ag} l_2$ $\longleftrightarrow$} \\
\>  $(\exists\ A_1\,B_1\,C_1\,A_2\,B_2\,C_2.$\\
\>  $\RepRt{l_1} = (A_1, B_1, C_1)) \ \wedge\ \RepRt{l_2} = (A_2, B_2, C_2)\ \wedge$\\
\>  $(\exists k.\ k \neq 0 \,\wedge\, A_2 = k\cdot A_1 \,\wedge\,  B_2 = k\cdot B_1\,\wedge\,C_2 = k\cdot C_1))$"}
\end{tabbing}
}
\selectlanguage{serbian}
\noindent Потом је доказано да је ово релација
еквиваленције. Дефиниција за тип праве користи подршку за количничке
типове и количничке дефиниције. Значи права (тип {\tt line$^{ag}$})
се дефинише коришћењем {\tt quotient\_type} команде као класа
еквиваленције над релацијом $\approx^{ag}$.

Да би избегли коришћење теорије скупова, геометријске аксиоматизације
које експлицитно разматрају праве користе релацију инциденције. Ако се
користи претходна дефиниција за праву, онда се проверавање инциденције
своди на израчунавање да ли тачка $(x, y)$ задовољава једначину праве
$A\cdot x + B\cdot y + C = 0$, за неке коефицијенте $A$, $B$, и $C$
који су представници класе.
\selectlanguage{english}
{\tt
\begin{tabbing}
\hspace{5mm}\=\hspace{5mm}\=\kill
\textbf{definition} "ag\_in\_h\ $(x, y)\ l \longleftrightarrow$}\\
\>$(\exists\ A\ B\ C.\ \RepRt{l} = (A,\ B,\ C) \,\wedge\,  (A\cdot x + B\cdot y + C = 0))$"}
\end{tabbing}
}
\selectlanguage{serbian}

Ипак, да би доказали да је релација заснована на представницима класе
добро дефинисана, мора бити доказано да ако се изаберу други
представници класе, рецимо $A'$, $B'$, и $C'$ (који су пропорционални
са $A$, $B$, и $C$), онда важи $A'\cdot x + B'\cdot y + C = 0$. Зато
ми у нашој Isabelle/HOL формализацији користимо пакет који подржава
рад са количничким типовима ({\tt quotient package}). Онда се
$\aginh{A}{l}$ дефинише коришћењем {\tt \textbf{quotient\_definition}}
која се заснива на релацији {\tt ag\_in\_h}. Лема добре дефинисаности
је
\selectlanguage{english} {\tt
\begin{tabbing}
\hspace{5mm}\=\hspace{5mm}\=\kill
\textbf{lemma} \\
\>\textbf{shows} "$l \approx l' \Longrightarrow$ {\tt ag\_in\_h}$\ P\ l =$ {\tt ag\_in\_h}$\ P\ l'$"
\end{tabbing}
}
\selectlanguage{serbian}


\paragraph{Афина дефиниција.}
У афиној геометрији, права се дефинише помоћу фиксне тачке и
вектора. Као и тачка, вектор такође може бити записан као пар реалних
бројева на следећи начин: {\tt \textbf{type\_synonym}\ vec$^{ag}\ =\ "real
  \times real"$}. Вектори дефинисани на овај начин чине векторски
простор (са природно дефинисаним векторским збиром и скаларним
производом). Тачке и вектори се могу сабирати као $(x, y) + (v_x, v_y)
= (x + v_x, y + v_y)$. Зато, права се записује као тачка и вектор који
је различит од нуле:
\selectlanguage{english}
{\tt
\begin{tabbing}
\textbf{typedef} line\_point\_vec$^{ag} =$"${(p::$point$^{ag}, v::$vec$^{ag}).\ v \neq (0, 0)\}$"
\end{tabbing}
}
\selectlanguage{serbian}

Ипак, различите тачке и вектори могу заправо одређивати једну те исту
праву, па конструкција са количничким типом опет мора бити коришћена.
\selectlanguage{english}
{\tt
\begin{tabbing}
\hspace{5mm}\=\\
\textbf{definition} "$l_1 \approx^{ag} l_2 \longleftrightarrow (\exists\,p_1\,v_1\,p_2\,v_2.$}\\
\>$\RepRt{l_1} = (p_1, v_1) \,\wedge\,  \RepRt{l_2} = (p_2, v_2) \,\wedge$\\
\>$(\exists k m.\ v_1 = k\cdot v_2 \,\wedge\, p_2 = p_1 + m\cdot v_1))$"}
\end{tabbing}
}
\selectlanguage{serbian}
\noindent Доказује се да је ово заиста релација еквиваленције. Тада је
могуће дефинисати тип којим се представљају праве ({\tt line$^{ag}$})
као класа еквиваленције над релацијом $\approx^{ag}$ коришћењем
команде {\tt quotient\_type}.

Ову дефиницију је било могуће задати и коришћењем детерминанти (чиме
би се избегло одређивање коефицијената $k$ и $m$). Услови би били
$|v_1, v_2| = 0$ и $|v_1, p_2 - p_1| = 0$.

Након што се докаже добра дефинисаност, инциденција се дефинише на
начин који можете видети у наставку (поново се уопштава подизањем на
виши ниво) коришћењем количничког пакета.  
\selectlanguage{english}
{\tt
\begin{tabbing}
\hspace{5mm}\=\hspace{5mm}\=\kill
\textbf{definiton} "ag\_in\_h$\,p\,l\longleftrightarrow\,(\exists\,p_0\,v_0.\, \RepRt{l} = (p_0, v_0) \,\wedge\,  (\exists k.\,p = p_0 + k \cdot v_0))$"
\end{tabbing}
}
\selectlanguage{serbian}
Још једна могућа дефиниција праве је класа еквиваленције парова
различитих тачака. Ми нисмо формализовали овај приступ јер је
тривијално изоморфан са афином дефиницијом (разлика тачака је вектор
који се појављује у афиној дефиницији).

\subsection{Изометрије}

У синтетичкој геометрији изометрије се уводе коришћењем
дефиниције. Рефлексије могу прве да се дефинишу, а онда се друге
изометрије могу дефинисати као композиција рефлексија. Ипак, у нашој
формализацији Декартове равни, изометрије се користе само као помоћно
средство да упросте наше доказе (што ће бити додатно појашњено у
одељку \ref{sec:iso}). Зато ми нисмо били заинтересовани да дефинишемо
изометрије као примитивне појмове (као што су тачке и
подударност) него смо представили аналитичке дефиниције и
доказали својства која су потребна за касније доказе.

Транслација је дефинисана преко датог вектора (који није експлицитно
дефинисан, већ је представљен као пар реалних бројева). Формална
дефиниција у \emph{Isabelle/HOL} систему је једноставна.
\selectlanguage{english}
{\tt
\begin{tabbing}
\textbf{definiton} "transp$\ag{(v_1, v_2)}{(x_1, x_2)} = (v_1 + x_1, v_2 + x_2)$"
\end{tabbing}
}
\selectlanguage{serbian}
Ротација је параметризована за реални параметар $\alpha$ (који
представља угао ротације), а ми само посматрамо ротације око
координатног почетка (остале ротације могу се добити као композиција
транслације и ротације око координатног почетка).  Користимо основна
правила тригонометрије да би добили следећу формалну дефиницију у
систему \emph{Isabelle/HOL}.
\selectlanguage{english}
{\tt
\begin{tabbing}
\textbf{definition} "rotp$\ag{\alpha}{(x, y)} = ((\cos \alpha)\cdot x - (\sin
\alpha)\cdot y , (\sin \alpha)\cdot x + (\cos \alpha)\cdot y)$"
\end{tabbing}
}
\selectlanguage{serbian}
Такође, централна симетрија се лако дефинише коришћењем координата
тачке:
\selectlanguage{english}
 {\tt
\begin{tabbing}
\textbf{definiton} "symp$\ag{(x, y)} = (-x, -y)$"
\end{tabbing}
}
\selectlanguage{serbian}
Важна особина свих изометрија је својство инваријантности, тј.  оне
чувају основне релације (као што су \emph{између} и \emph{подударно}).
\selectlanguage{english}
{\tt
\begin{tabbing}
\hspace{5mm}\=\kill
\textbf{lemma} "$\agbett{A}{B}{C} \longleftrightarrow \agbett{($transp$\ag{v}{A})}{($transp$\ag{v}{B})}{($transp$\ag{v}{C})}$"\\
\textbf{lemma} "$\agcongr{A}{B}{C}{D} \longleftrightarrow$}\\
\> $\agcongr{($transp$\ag{v}{A})}{($transp$\ag{v}{B})}{($transp$\ag{v}{C})}{($transp$\ag{v}{D})}$"}\\
\textbf{lemma} "$\agbett{A}{B}{C} \longleftrightarrow \agbett{($rotp$\ag{\alpha}{A})}{($rotp$\ag{\alpha}{B})}{($rotp$\ag{\alpha}{C})}$"\\
\textbf{lemma} "$\agcongr{A}{B}{C}{D} \longleftrightarrow$}\\
\> $\agcongr{($rotp$\ag{\alpha}{A})}{($rotp$\ag{\alpha}{B})}{($rotp$\ag{\alpha}{C})}{($rotp$\ag{\alpha}{D})}$"}\\
\textbf{lemma} "$\agbett{A}{B}{C} \longleftrightarrow \agbett{($symp$\ag{A})}{($symp$\ag{B})}{($symp$\ag{C})}$"\\
\textbf{lemma} "$\agcongr{A}{B}{C}{D} \longleftrightarrow
   \agcongr{($symp$\ag{A})}{($symp$\ag{B})}{($symp$\ag{C})}{($symp$\ag{D})}$"
\end{tabbing}
}
\selectlanguage{serbian}
Изометрије се пре свега користе да трансформишу тачку у њену канонску
позицију (обично транслацијом тачке на $y$-осу).  Следеће леме
показују да је то могуће учинити.
\selectlanguage{english} {\tt
\begin{tabbing}
\hspace{5mm}\=\kill
\textbf{lemma} "$\exists v.\ $transp$\ag{v}{P} = (0, 0)$"\\
\textbf{lemma} "$\exists \alpha.\ $rotp$\ag{\alpha}{P} = (0, p)$"\\
\textbf{lemma} "$\agbett{(0, 0)}{P_1}{P_2} \longrightarrow$}\\
\> $\exists \alpha\ p_1\ p_2.\ $rotp$\ag{\alpha}{P_1} = (0, p_1)\ \wedge\ $rotp$\ag{\alpha}{P_2} = (0, p_2)$"}
\end{tabbing}
}
\selectlanguage{serbian}
Изометријске трансформације праве се дефинишу коришћењем
изо\-ме\-три\-јских трансформација над тачкама (права се трансформише тако
што се тра\-нсфо\-рмишу две њене произвољне тачке).

%------------------------------------------------------------------------------
\subsection{Коришћење изометријских трансформација}
\label{sec:iso}
% ------------------------------------------------------------------------------
Једна од најважнијих техника која је коришћена за упрошћавање
фо\-рма\-ли\-за\-ције ослањала се на коришћење изометријских
трансформација. Ми ћемо покушати да представимо мотивациони разлог за
примену изометрија на следећем, једноставном примеру.

Лема која се често користи у доказима је да ако $\agbett{A}{X}{B}$ и
$\agbett{A}{B}{Y}$ онда важи $\agbett{X}{B}{Y}$. Представићемо како се
ова лема може доказати без коришћења изометријских трансформација и
како се доказује када се користе изометријске трансформације. Чак и на
овом једноставном примеру, ако применимо директан доказ, без коришћења
изометријских трансформација, алгебарски рачун постаје превише
комплексан.

Нека важи $A=(x_A, y_A)$, $B=(x_B, y_B)$, и $X=(x_X, y_X)$.  Како
$\agbett{A}{X}{B}$ важи, постоји реалан број $k_1$, $0 \le k_1 \le 1$,
такав да $(x_X - x_A) = k_1 \cdot (x_B - x_A)$, и $(y_X - y_A) = k_1
\cdot (y_B - y_A)$. Слично, како $\agbett{A}{B}{Y}$ важи, постоји
реалан број $k_2$, $0 \le k_2 \le 1$, такав да $(x_B - x_A) = k_2
\cdot (x_Y - x_A)$, и $(y_B - y_A) = k_2 \cdot (y_Y - y_A)$. Онда,
може се дефинисати реалан број $k$ са $(k_2 - k_2\cdot k_1) /
(1-k_2\cdot k_1).$ Ако $X\neq B$, онда коришћењем комплексних
алгебарских трансформација, може се доказати да $0 \le k \le 1$, и да
$(x_B - x_X) = k \cdot (x_Y - x_X)$, и $(y_B - y_X) = k \cdot (y_Y -
y_X)$, и зато $\agbett{X}{B}{Y}$ важи. Дегенерисани случај $X=B$
тривијално важи.

Ипак, ако применимо изометријске трансформације, онда можемо
претпоставити да $A=(0, 0)$, $B=(0, y_B)$, и $X=(0, y_X)$, и да $0 \le
y_X \le y_B$. Случај када је $y_B = 0$ тривијално важи. У супротном,
$x_Y = 0$ и $0 \le y_B \le y_Y$. Зато, $y_X \le y_B \le y_Y$, и
тврђење важи. Приметимо да у овом случају нису биле потребне велике
алгебарске трансформације и доказ се ослања на једноставне особине
транзитивности релације $\le$.

Поредећи претходна два доказа, можемо да видимо како примена
изометријских трансформација значајно упрошћава потребна израчунавања
и скраћује доказе.

Како је ова техника доста коришћена у нашој формализацији, важно је
продискутовати који је најбољи начин да се формулишу одговарајуће леме
које оправдавају употребу ове технике и покушати што више
аутоматизовати коришћење ове технике. Ми смо применили приступ који је
предложио Харисон \cite{wlog}.

Својство $P$ је инваријантно под трансформацијом $t$ акко својство $P$
важи за било које тачке које се добију трансформацијом $t$ од тачака
за које је својство $P$ важило.
\selectlanguage{english}
{\tt
\begin{tabbing}
\textbf{definiton} "inv$\ P\ t \longleftrightarrow (\forall\ A\ B\ C.\ P\ A\ B\ C
\longleftrightarrow P\ (t A)\ (t B)\ (t C))$"
\end{tabbing}
}
\selectlanguage{serbian}
Тада, следећа лема се може користити да сведемо тврђење које важи за
било које тачке које су колинеарне на тврђење за које разматрамо само
тачке на $y$-оси (можемо изабрати и $x$-осу уколико нам тако више
одговара).
\selectlanguage{english}
{\tt
\begin{tabbing}
\hspace{5mm}\=\kill
\textbf{lemma}\\
\>\textbf{assumes} \="$\forall\ y_B\ y_C.\ 0 \le y_B \ \wedge\  y_B \le y_C \longrightarrow P\ (0, 0)\ (0, y_B)\ (0, y_C)$"\\
\>\>       "$\forall\,v.\ $inv$\ P\ ($transp$\ag{v}{})$" "$\forall\,\alpha.\ $inv$\ P\ ($rotp$\ag{\alpha}{})$"\\
\>\>       "inv$\ P\ ($symp$\ag{})$"\\
\>\textbf{shows}\>"$\forall\,A\,B\,C.\ \agbett{A}{B}{C} \longrightarrow\ P\ A\ B\ C$"
\end{tabbing}
} \selectlanguage{serbian} Доказ да је неко тврђење инваријантно у
односу на изометријску трансформацију највише се ослања на леме у
којима се доказује да су релација \emph{између} и релација
\emph{подударности} инваријантне у односу на изометријске
трансформације.

% ------------------------------------------------------------------------------
\section{Модел аксиоматског система Тарског}
\label{sec:tarski}
% ------------------------------------------------------------------------------

Прво ћемо навести аксиоме Тарског. Постоје две основне релације:
\begin{itemize}
\item Релација \emph{између}: $Bxyz$ означава да је $y$ \emph{између}
  $x$ и $z$.
\item Релација \emph{подударно}: $xy \equiv vw$ означава да је $xy$
  \emph{подударно} $vw$, односно да је дужина дужи $xy$ једнака дужини
  дужи $vw$.
\end{itemize}

\noindent \textbf{Аксиоме подударности}

\begin{description}
\item[A1] Рефлексивност подударности: $xy \equiv yx$.
\item[A2] Идентитет подударности: $xy \equiv zz \to x = y$.
\item[A3] Транзитивност подударности: $(xy \equiv zu \land xy \equiv vw) \to zu \equiv vw$.
\end{description}

\noindent \textbf{Аксиоме распореда}

\begin{description}
\item[A4] Идентитет релације \emph{између}: $Bxyx \to x = y$.
\item[A5] Пашова аксиома: $(Bxuz \land Byvz) \to \exists a. (Buay \land Bvax)$.
\item[A6] Аксиома непрекидности: \\
  $\exists a\forall x\forall y. [(\phi(x) \land \psi(y)) \to Baxy] \to \exists b\forall x\forall y.[(\phi(x) \land \psi(y)) \to Bxby]$.
\item[A7] Аксиома доње димензије: $\exists a\exists b\exists c.[\neg Babc \land \neg Bbca \land \neg Bcab]$.
\end{description}

\noindent \textbf{Аксиоме подударности и распореда}

\begin{description}
\item[A8] Аксиома горње димензије: \\
  $(xu \equiv xv \land yu \equiv yv \land zu \equiv zv \land u \neq v) \to (Bxyz \lor Byzx \lor Bzxy)$.
\item[A9] Еуклидова аксиома има три варијанте:\\
  I: $((Bxyw \land xy \equiv yw) \land (Bxuv \land xu \equiv uv) \land (Byuz \land yu \equiv zu)) \to yz \equiv vw$. \\
  II: $Bxyz \lor Byzx \lor Bzxy \lor \exists a. (xa \equiv ya \land xa \equiv za)$. \\
  III: $(Bxuv \land Byuz \land x \neq u) \to \exists a\exists b. (Bxya \land Bxzb \land Bavb)$.
\item[A10] Аксиома пет дужи: \\
  $(x \neq y \land Bxyz \land Bx'y'z' \land xy \equiv x'y' \land yz \equiv y'z' \land xu \equiv x'u' \land yu \equiv y'u') \to zu \equiv z'u'$.
\item[A11] Конструкција дужи: $\exists z. Bxyz \land yz \equiv ab$.
\end{description}  

Наш циљ у овом поглављу је да докажемо да наше дефиниције Декартове
координатне равни задовољавају све аксиоме геометрије Тарског
\cite{tarski}.  Основни појмови у геометрији Тарског су само три појма
- тачке, (инклузивна) релација \emph{између} (означена са
$\bett{A}{B}{C}$) и релација \emph{подударности} (коју означавамо са
$\congrt{A}{B}{C}{D}$). У геометрији Тарског праве нису експлицитно
дефинисане и колинеарност се дефинише коришћењем релације
\emph{између} \\ 
\selectlanguage{english} {\tt
\begin{center}
\textbf{definition} "$\colint{A}{B}{C} \longleftrightarrow \bett{A}{B}{C} \vee \bett{B}{C}{A} \vee \bett{C}{A}{B}$"
\end{center}
}
\selectlanguage{serbian}

\subsection{Аксиоме подударности.}

Прве три аксиоме Тарског представљају основна својства подударности.
{\tt
\begin{tabbing}
\hspace{5mm}\=\kill
\textbf{lemma} "$\congrt{A}{B}{B}{A}$"\\
\textbf{lemma} "$\congrt{A}{B}{C}{C} \longrightarrow\ A = B$"\\
\textbf{lemma} "$\congrt{A}{B}{C}{D} \wedge\ \congrt{A}{B}{E}{F}\ \longrightarrow \congrt{C}{D}{E}{F}$"
\end{tabbing}
}
\vspace{-2mm}

Желимо да докажемо да наша релација $\agcongr{}{}{}{}$ задовољава
својства релације $\congrt{}{}{}{}$ која је апстрактно задана са
претходним аксиомама (тј. да дате аксиоме важе у нашем моделу
Декартове координатне равни). У нашој формализацији, аксиоме
геометрије Тарског су формулисане коришћењем локала ({\tt
  \textbf{locale}}, и доказано је да координатна раван представља
интерпретацију тог дефинисаног локала. Како је ово техничка страна
формализације у \emph{Isabelle/HOL} систему, ми је нећемо овде
дискутовати у више детаља (погледати одељак \ref{locales_Isabelle}).
На пример, за прву аксиому, доказ се своди на доказивање тврђења
\mbox{$\agcongr{A}{B}{B}{A}$}. Докази су праволинијски и готово
аутоматски (поједностављивањем након развијања дефиниција).

\subsection{Аксиоме распореда.}


\paragraph{Идентитет у релацији \emph{између}.}

Прва аксиома (инклузивне) релације \emph{између} даје једно њено
једноставно својство и, за наш модел, доказује се готово аутоматски.
\selectlanguage{english} {\tt
\begin{tabbing}
\hspace{5mm}\=\kill
\textbf{lemma} "$\bett{A}{B}{A} \longrightarrow A = B$"
\end{tabbing}
}
\selectlanguage{serbian}

\paragraph{Пашова аксиома.}

Следећа аксиома је Пашова аксиома:
\selectlanguage{english}
{\tt
\begin{tabbing}
\hspace{5mm}\=\kill
\textbf{lemma} "$\bett{A}{P}{C} \wedge \bett{B}{Q}{C} \longrightarrow (\exists X.\ (\bett{P}{X}{B} \wedge \bett{Q}{X}{A}))$"
\end{tabbing}
}
\selectlanguage{serbian}

Под претпоставком да су све тачке које се помињу у аксиоми различите и
да нису све тачке колинеарне, слика која одговара аксиоми је:
\begin{center}
\input{ax_t_5.tkz}
\end{center}


Пре него што дамо доказ да у нашем моделу Декартове координатне равни
важи ова аксиома, желимо да продискутујемо нека питања која се односе
на геометрију Тарског и која су се показала важним за свеукупну
организацију нашег доказа. Последња верзија аксиоматског система
Тарског је направљена да буде минимална (садржи само 11 аксиома), и
централне аксиоме које описују релацију \emph{између} су идентитет
релације \emph{између} и Пашова аксиома. У формализацији геометрије
Тарског (\cite{narboux}) сва остала елементарна својства ове релације
се изводе из ове две аксиоме. На пример, да би се извела симетричност
релације \emph{између} (и.е., $\bett{A}{B}{C} \longrightarrow
\bett{C}{B}{A}$), Пашова аксиома се примењује на тројке $ABC$ и $BCC$
и тада се добија тачка $X$ тако да важи $\bett{C}{X}{A}$ и
$\bett{B}{X}{B}$, и онда према првој аксиоми, $X=B$ и
$\bett{C}{B}{A}$. Ипак, према нашем искуству, у намери да докажемо да
је наша Декартова координатна раван модел аксиома Тарског (поготово за
Пашову аксиому), потребно је да већ имамо доказане неке њене последице
(као што су симетричност и транзитивност). Додајмо да су раније
варијанте аксиоматског система Тарског имале више аксиома, а ова
својства су управо била нека од тих додатних аксиома. Такође, својство
симетрије је једноставније својство него Пашова аксиома (на пример,
оно укључује само тачке које леже на правој, док у аксиоми Паша имамо
тачке које леже у равни и не морају бити колинеарне). Додатно,
претходни доказ користи веома суптилна својства која зависе од тога
како је Пашова аксиома формулисана. На пример, ако се у њеном закључку
користи $\bett{B}{X}{P}$ и $\bett{A}{X}{Q}$ уместо $\bett{P}{X}{B}$ и
$\bett{Q}{X}{A}$, онда доказ не може да се изведе. Зато смо закључили
да би добар приступ био да директно докажемо да нека елементарна
својства (као што су симетрија и транзитивност) релације \emph{између}
важе у моделу, а онда да користимо ове чињенице у доказу много
комплексније Пашове аксиоме.  \selectlanguage{english} {\tt
\begin{tabbing}
\hspace{5mm}\=\kill
\>\textbf{lemma} "$\agbett{A}{A}{B}$"\\
\>\textbf{lemma} "$\agbett{A}{B}{C} \longrightarrow \agbett{C}{B}{A}$"\\
\>\textbf{lemma} "$\agbett{A}{X}{B}\ \wedge\ \agbett{A}{B}{Y} \longrightarrow \agbett{X}{B}{Y}$"\\
\>\textbf{lemma} "$\agbett{A}{X}{B}\ \wedge\ \agbett{A}{B}{Y} \longrightarrow \agbett{A}{X}{Y}$"
\end{tabbing}
}
\selectlanguage{serbian}
Пре него што наставимо са доказом да наша Декартова координатна раван
у потпуности задовољава Пашову аксиому, потребно је анализирати
неколико дегенерисаних случајева. Прва група дегенерисаних случајева
настаје када су неке од тачака у конструкцији једнаке. На пример,
$\bett{A}{P}{C}$ дозвољава да $A=P=C$, или $A=P\neq C$, или $A\neq
P=C$ или $A \neq P \neq C$. Директан приступ би био да се сваки од
ових случајева посебно анализира. Међутим, бољи приступ је да се
пажљиво анализира претпоставка и да се одреди који од случајева су
суштински различити. Испоставља се да су само два различита случаја
битна. Ако је $P=C$, онда је $Q$ тражена тачка. Ако је $Q=C$, онда је
$P$ тражена тачка. Следећа група дегенерисаних случајева настаје када
су све тачке колинеарне. У овом случају важи, или $\bett{A}{B}{C}$ или
$\bett{B}{A}{C}$ или $\bett{B}{C}{A}$. У првом случају $B$ је тражена
тачка, у другом случају $A$ је тражена тачка, а у трећем случају $P$
је тражена тачка.

Приметимо да се сви дегенерисани случајеви Пашове аксиоме директно
доказују коришћењем елементарних својстава и да у овим случајевима
није било потребно користити координатна израчунавања. Ово сугерише да
су дегенерисани случајеви Пашове аксиоме еквивалентни конјукцији датих
својстава. Додатно, ово сугерише да ако се промени аксиоматизација
Тарског тако да укључује ова елементарна својства, онда се Пашова
аксиома може ослабити тако да садржи само централни случај
неколинеарних, различитих тачака.


Коначно, остаје да се докаже централни случај. У том случају,
коришћене су алгебарске трансформације да се израчунају координате
тачке $X$ и да се докаже претпоставка. Да би се упростио доказ,
коришћене су изометрије, као што је описано у одељку
\ref{sec:iso}. Почетна конфигурација је трансформисана тако да $A$
постаје координатни почетак, односно $(0, 0)$, да $P = (0, y_P)$ и $C
= (0, y_C)$ леже на позитивном делу $y$-осе. Нека је $B=(x_B, y_B)$,
$Q=(x_Q, y_Q)$ и $X = (x_X, y_X)$. Како $\bett{A}{P}{C}$ важи, постоји
реалан број $k_3$, $0 \le k_3 \le 1$, такав да $y_P = k_3\cdot y_C$.
Слично, како $\bett{B}{Q}{C}$ важи, постоји реалан број $k_4$, $0 \le
k_4 \le 1$, такав да $(x_B - x_A) = k_2 \cdot (x_Y - x_A)$, и $x_Q -
x_B = -k_4\cdot x_B$ и $y_Q - y_B = k_4\cdot (y_C - y_B)$. Онда,
можемо дефинисати реалан број $k_1 = \frac{k_3\cdot (1 - k_4)}{k_4 +
  k_3 - k_3\cdot k_4}.$ Како за $A$, $P$ и $C$ важи $A \neq P \neq C$
и тачке нису колинеарне (јер посматрамо само централни, недегенерисани
случај), онда, коришћењем директних алгебарских израчунавања, може
бити доказано да $0 \le k_1 \le 1$, и да $ x_X = k_1 \cdot x_B$, и
$y_X - y_P = k_1\cdot (y_B - y_P)$, и зато $\bett{P}{X}{B}$
важи. Слично, можемо дефинисати реалан број $k_2 = \frac{k_4\cdot (1 -
  k_3)}{k_4 + k_3 - k_3\cdot k_4}$ и доказати да $0 \le k_2 \le 1$ и
да важи следеће: $x_X - x_Q = -k_2\cdot x_Q$ и $y_X - y_Q = - k_2\cdot
y_Q$ и према томе $\bett{Q}{X}{A}$ важи. Из ова два закључка ми смо
одредили тачку $X$.


\paragraph{Аксиома ниже димензије.}
Следећа лема каже да постоје 3 неколинеарне тачке, што представља
аксиому ниже димензије у аксиоматици Тарског. Зато сваки модел ових
аксиома мора имати димензију већу од 1.  
\selectlanguage{english} {\tt
\begin{tabbing}
\textbf{lemma} "$\exists\ A\ B\ C.\ \neg\ \colint{A}{B}{C}$"
\end{tabbing}
}
\selectlanguage{serbian}
\noindent У нашој Декартовој равни тривијално важи (нпр. $(0, 0)$,
$(0, 1)$, и $(1, 0)$ су неколинеарне).

\paragraph{Аксиома (схема) непрекидности.}

Аксиома непрекидности Тарског је у ствари конструкција Дедекиндовог
пресека. Интуитивно, ако су све тачке скупа тачака са једне стране у
односу на тачке које припадају другом скупу тачака, онда постоји
тачка која је између та два скупа. Оригинална аксиоматизација Тарског
је дефинисана у оквиру логике првог реда и скупови нису експлицитно
познати у оквиру формализације Тарског. Зато, уместо да користи
скупове тачака, Тарски користи предикате логике првог реда, $\phi$ и
$\psi$.

$$(\exists a.\ \forall x.\ \forall y.\ \phi\ x \wedge \psi\ y \longrightarrow \bett{a}{x}{y}) \longrightarrow (\exists b.\ \forall x.\ \forall y.\ \phi\ x \wedge \psi\ y \longrightarrow \bett{x}{b}{y})$$

Ипак, формулација ове леме у оквиру логике вишег реда система
\emph{Isabelle/HOL} не ограничава предикате $\phi$ и $\psi$ да буду
предикати логике првог реда. Зато, строго гледано, наша формализација
аксиоматског система Тарског у оквиру система \emph{Isabelle/HOL} даје
другачију геометрију у односу на оригиналну геометрију Тарског.
\selectlanguage{english} {\tt
\begin{tabbing}
\hspace{5mm}\=\kill
\textbf{lemma}\\
\>\textbf{assumes} "$\exists a.\ \forall x.\ \forall y.\ \phi\ x \wedge \psi\ y \longrightarrow \agbett{a}{x}{y}$"\\
\>\textbf{shows} "$\exists b.\ \forall x.\ \forall y.\ \phi\ x \wedge \psi\ y \longrightarrow \agbett{x}{b}{y}$"
\end{tabbing}
}
\selectlanguage{serbian}
Међутим, испоставља се да је могуће доказати да Декартова координатна
раван такође задовољава строжију варијанту аксиоме (без ограничавања
да предикати $\phi$ и $\psi$ су предикати логике првог реда). Ако је
један скуп празан, тврђење тривијално важи. Ако скупови имају
заједничку тачку, онда је та тачка уједно и тражена тачка. У другим
случајевим, примењујемо изометријске трансформације тако да све тачке
из оба скупа леже на позитивном делу $y$-осе. Онда, доказ тврђења се
своди на доказивање следећег:
\selectlanguage{english}
{\tt
\begin{tabbing}
\hspace{5mm}\=\kill
\textbf{lemma}\\
\>\textbf{assumes}\\
\>"$P = \{x.\ x\ge 0 \wedge \phi(0, x)\}$" "$Q = \{y.\ y\ge 0\wedge \psi(0, y)\}$"\\
\>"$\neg (\exists b.\ b \in P \wedge b \in Q)$" "$\exists x_0.\ x_0 \in P$" "$\exists y_0.\ y_0 \in Q$"\\
\>"$\forall x \in P.\ \forall y \in Q.\ \agbett{(0, 0)}{(0, x)}{(0, y)}$"\\
\>\textbf{shows}\\
\>"$\exists b.\ \forall x \in P.\ \forall y \in Q.\ \agbett{(0, x)}{(0, b)}{(0, y)}$"
\end{tabbing}
} \selectlanguage{serbian} Доказивање овог тврђења захтева коришћење
нетривијалних особина реалних бројева, пре свега, њихову
потпуност. Потпуност реалних бројева у систему \emph{Isabelle/HOL} је
формализована следећом теоремом (супремум, особина најмање горње
границе):

\selectlanguage{english}
{\tt 
\begin{tabbing}
\hspace{15mm}\=\kill
\textbf{lemma} "$(\exists x.\ x \in P) \wedge (\exists y.\ \forall x\in P.\ x < y) \longrightarrow$} \\
\> $\exists S.\ (\forall y.\ (\exists x\in P.\ y < x) \leftrightarrow y < S)$"}
\end{tabbing}
}
\selectlanguage{serbian}

Скуп $P$ задовољава својство супремума. Заиста, како, по претпоставци,
$P$ и $Q$ немају заједнички елемент, а из претпоставке следи да
$\forall x \in P.\ \forall y \in Q.\ x < y$, тако да је било који
елемент из $Q$ горња граница за $P$. По претпоставци, $P$ и $Q$ су
непразни, тако да постоји елемент $b$ такав да $\forall x \in P.\ x
\leq b$ и $\forall y \in Q.\ b \leq y$, а то заправо значи да теорема
важи.

\subsection{Аксиоме подударности и распореда.}
\paragraph{Аксиома горње димензије.}
Три тачке које су на истом одстојању од две различите тачке леже на
истој правој. Зато, сваки модел ових аксиома мора имати димензију мању
од 3.
\selectlanguage{english}
{\tt
\begin{tabbing}
\hspace{5mm}\=\kill
\textbf{lemma} "$\congrt{A}{P}{A}{Q} \,\wedge\, \congrt{B}{P}{B}{Q} \,\wedge\, \congrt{C}{P}{C}{Q} \,\wedge\,  P \neq Q \ \longrightarrow \colint{A}{B}{C}$"
\end{tabbing}
}
\selectlanguage{serbian}
\begin{center}
\input{ax_t_10.tkz}
\end{center}

Ово тврђење је било лако доказати анализом различитих случајева и
коришћењем алгебарских трансформација. Није било потребно користити
изометријске трансформације.

\paragraph{Аксиома конструкције дужи.}
\selectlanguage{english}
{\tt
\begin{tabbing}
\hspace{5mm}\=\kill
\textbf{lemma} "$\exists E.\ \bett{A}{B}{E}\ \wedge\ \congrt{B}{E}{C}{D}$"
\end{tabbing}
} \selectlanguage{serbian} 
Доказ да наш модел Декартове координатне равни задовољава ову аксиому
је једноставан и почиње трансформацијом тачака тако да тачка $A$
постаје координатни почетак, а тачка $B$ лежи на позитивном делу
$y$-осе. Онда $A = (0, 0)$ и $B = (0, b)$, $b \ge 0$. Нека $d =
\sqrt{\agsqdist{C}{D}}$. Онда $E = (0, b + d)$.

\paragraph{Аксиома пет дужи.}
\selectlanguage{english}
{\tt
\begin{tabbing}
\hspace{5mm}\=assumes\ \=\kill
\textbf{lemma} "$\congrt{A}{B}{A'}{B'} \,\wedge\, \congrt{B}{C}{B'}{C'}  \,\wedge\,  \congrt{A}{D}{A'}{D'}  \,\wedge\,  \congrt{B}{D}{B'}{D'} \,\wedge\,$}\\
\>$\bett{A}{B}{C} \,\wedge\, \bett{A'}{B'}{C'} \,\wedge\, A \neq B \longrightarrow  \congrt{C}{D}{C'}{D'}$"}
\end{tabbing}
}
\selectlanguage{serbian}
Доказ да наш модел задовољава ову аксиому је прилично директан, али
захтева компликована израчунавања. Да бисмо упростили доказ, тачке $A$,
$B$ и $C$ су трансформисане тако да леже на позитивном делу
$y$-осе. Како су у израчунавањима потребни квадратни корени, није било
могуће користити аутоматизацију као у претходним доказима и многи
ситни кораци су морали бити исписани ручно.

\paragraph{Еуклидова аксиома.}
\selectlanguage{english}
{\tt
\begin{tabbing}
\hspace{10mm}\=assumes\ \=\kill
\textbf{lemma} "$\bett{A}{D}{T} \,\wedge\, \bett{B}{D}{C} \,\wedge\, A \neq D \ \longrightarrow\ $}\\
\>$(\exists X Y.\ (\bett{A}{B}{X}\ \wedge\ \bett{A}{C}{Y}\ \wedge\ \bett{X}{T}{Y}))$"}
\end{tabbing}
}
\selectlanguage{serbian}
Одговарајућа слика када су све тачке различите:

\begin{center}
\input{ax_t_6.tkz}
\end{center}

У доказу овог тврђења коришћене су изометријске трансформације. Тачке
$A$, $D$ и $T$ су пресликане редом у тачке $(0, 0)$, $(d, 0)$ и $(t,
0)$, односно у тачке на $y$-оси. Потом су анализирани дегенерисани
случајеви, односно случајеви када су неке од тачака једнаке или када
су све тачке колинеарне. У дегенерисаним случајевима, одређивање тачака
$X$ и $Y$ није представљало потешкоћу јер углавном су оне неке од
датих тачака, односно неке од тачака $A$, $B$, $C$, $D$ или
$T$. Рецимо, уколико су тачке колинеарне и ако важи $\bett{A}{C}{T}$,
онда је тачка $X$ заправо тачка $B$, а тачка $Y$ је тачка $T$.

Доказивање општег случаја захтева доста алгебарских израчунавања. Пре
свега, потребно је одредити координате тачака $X$ и $Y$, а потом на
основу тих координата одредити три коефицијента који представљају
однос међу тачкама, односно, први коефицијент представља однос међу
тачкама $A$, $B$ и $X$, други међу тачкама $A$, $C$ и $Y$, а трећи
међу тачкама $X$, $T$ и $Y$. Да бисмо доказали да ове тачке заиста
задовољавају релацију $\agbett{\_}{\_}{\_}$, потребно је доказати да
се сваки од три одређена коефицијента се налази у интервалу $[0, 1]$,
односно $0 \le k_i \le 1$, при чему $i = 1, 2, 3$. Иако је доказ ове
чињенице директан, потребно је доста израчунавања, а због знака $\le$
није могуће користити аутоматизацију већ је морало да се доста корака
доказује ручно.

\newpage

% ------------------------------------------------------------------------------
\section{Геометрија Хилберта}
\label{sec:hilbert}
% ------------------------------------------------------------------------------

Прво ћемо навести све аксиоме Хилберта.

\noindent \textbf{Аксиоме инциденције}
\begin{description}
\item[I--1] За две тачке $A$, $B$ постоји увек права $a$ која припада
  свакој од двеју тачака $A$, $B$.
\item[I--2] За две тачке $A$, $B$ не постоји више од једне праве која
  би припадала свакој од двеју тачака $A$, $B$.
\item[I--3] На правој постоје увек најмање две тачке. Постоје најмање
  три тачке које не леже на једној правој.
\item[I--4] Ма за које три тачке $A$, $B$, $C$, које не леже на истој
  правој, постоји увек раван $\alpha$ која припада свакој од ове три
  тачке $A$, $B$, $C$. За сваку раван увек постоји тачка која јој
  припада.
\item[I--5] За ма које три тачке $A$, $B$, $C$, које не леже на истој
  правој не постоји више од једне равни која припада свакој од ових
  трију тачака $A$, $B$, $C$.
\item[I--6] Ако две тачке $A$, $B$ праве $a$ леже у равни $\alpha$,
  онда свака тачка праве $a$ лежи у равни $\alpha$.
\item[I--7] Ако две равни $\alpha$, $\beta$ имају заједничку тачку
  $A$, онда оне имају најмање још једну заједничку тачку $B$.
\item[I--8] Постоје најмање четири тачке које не леже у једној равни.
\end{description}

\noindent \textbf{Аксиоме распореда}

\begin{description}
\item[II--1] Ако тачка $B$ лежи између тачака $A$ и $C$, онда су $A$,
  $B$, $C$ три различите тачке праве и $B$ лежи такође између $C$ и
  $A$.
\item[II--2] За две тачке $A$ и $C$ увек постоји најмање једна тачка
  $B$ на правој $AB$, тако да $C$ лежи између $C$ и $A$.
\item[II--3] Од ма којих трију тачака праве не постоји више од једне
  која лежи између оне друге две.
\item[II--4] Нека су $A$, $B$, $C$ три тачке које не леже на једној
  правој и нека је $a$ права у равни $ABC$ која не пролази ни кроз
  једну од тих тачака; ако дата права пролази кроз једну од тачака
  дужи $AB$, она мора пролазити кроз једну од тачака дужи $AC$, или
  тачака дужи $BC$.
\end{description}  

\noindent \textbf{Аксиоме подударности}

\begin{description}
\item[III--1] Ако су $A$, $B$ две тачке на правој $a$ и ако је, даље,
  $A'$ тачка на истој или на другој правој $a'$, онда се може увек
  наћи таква тачка $B'$ праве $a'$ на датој страни од тачке $A'$, да
  дуж $AB$ буде подударна или једнака дужи $A'B'$, што означавамо на
  следећи начин: $AB \equiv A'B'$.
\item[III--2] Ако су дужи $A'B'$ и $A''B''$ подударне једној истој
  дужи $AB$, биће и дуж $A'B'$ подударна дужи $A''B''$.
\item[III--3] Нека су $AB$ и $BC$ две дужи на правој $a$ без
  заједничких тачака и нека су, даље, $A'B'$ и $B'C'$ две дужи на
  истој правој $a$ или на некој другој правој $a'$ које исто тако
  немају заједничких тачака; ако је тада $AB \equiv A'B'$ и $BC \equiv
  B'C'$, биће увек и $AC \equiv A'C'$.
\item[III--4] Прво ћемо дати дефиницију угла, а потом и
  аксиому. \\
  \textbf{Дефиниција:} Нека је $\alpha$ произвољна раван, а $h$ и $k$
  нека су ма које различите полуправе које излазе из тачке $O$ у равни
  $\alpha$ и припадају разним правама. Систем од две полуправе $h$,
  $k$ назваћемо \emph{углом} и означаваћемо га са $\angle(h, k)$ или
  са $\angle(k, h)$.\\
  Аксиома: Нека је дат угао $\angle(h, k)$ у равни $\alpha$ и права
  $a'$ у равни $\alpha'$ као одређена страна равни $\alpha'$ према
  правој $a'$. Нека $h'$ означава полуправу праве $a'$ која полази из
  тачке $O'$; онда у равни $\alpha'$ постоји једна и само једна
  полуправа $k'$ тако да је угао $\angle(h, k)$ подударан или једнак
  углу $\angle(h', k')$ и у исто време све унутрашње тачке угла
  $\angle(h', k')$ налазе се на датој страни од праве $a'$, што ћемо
  означити на овај начин $\angle(h, k) \equiv \angle(h', k')$. Сваки
  је угао подударан самом себи.
\end{description}

\noindent \textbf{Аксиома паралелности}

\begin{description}
\item[IV] Еуклидова аксиома: Нека је $a$ произвољна права и $A$ тачка
  ван $a$: тада постоји у равни, одређеној правом $a$ и тачком $A$,
  највише једна права која пролази кроз $A$ и не пресеца $a$.
\end{description}

\noindent \textbf{Аксиоме непрекидности}

\begin{description}
\item[V--1] Архимедова аксиома: Ако су $AB$ и $CD$ ма које две дужи,
  онда постоји такав број $n$, да када се дуж $CD$ пренесе $n$ од $A$
  једно за другим по полуправој која пролази кроз тачку $B$ прелази се
  преко тачке $B$.
\item[V--2] Аксиома линеарне потпуности: Систем тачака неке праве са
  својим релацијама распореда и подударности не може се тако
  проширити, да остану очуване релације које постоје између претходних
  елемената као и основне особине линеарног распореда и подударности
  које проистичу из аксиома I--III, и аксиоме V--1.
\end{description}  

Циљ у овом одељку је да докажемо да наше дефиниције Декартовог
координатног система задовољавају аксиоме Хилбертове
геометрије. Основни објекти у Хилбертовој планарној геометрији су
тачке, праве, релација \emph{између} (означена са $\beth{A}{B}{C}$) и
релација подударности (означена са $\congrh{A}{B}{C}$).


У оригиналној Хилбертовој аксиоматизацији \cite{hilbert} неке
претпоставке се имплицитно подразумевају у односу на контекст у коме
су дате. На пример, ако је речено \emph{``постоје две тачке``}, то
увек значи постоје две различите тачке. Без ове претпоставке нека
тврђења не важе (нпр.~релација \emph{између} не важи ако су тачке
једнаке).


\subsection{Аксиоме инциденције}

Прве две аксиоме су формализоване коришћењем само једног тврђења.
\selectlanguage{english}
{\tt
\begin{tabbing}
\textbf{lemma} "$A \neq B \longrightarrow \exists!\ l.\ \inh{A}{l} \,\wedge\, \inh{B}{l}$"
\end{tabbing}
}
\selectlanguage{serbian}

Последња аксиома ове групе је формализована коришћењем два одвојена
тврђења.
\selectlanguage{english}
{\tt
\begin{tabbing}
\textbf{lemma} "$\exists A B.\ A \neq B \,\wedge\,\inh{A}{l} \,\wedge\, \inh{B}{l}$"\\
\textbf{lemma} "$\exists A B C.\ \neg\ \colinh{A}{B}{C}$"
\end{tabbing}
}
\selectlanguage{serbian}
\noindent Релација колинеарности $\mathcal{C}_h$ (која је коришћена у
претходној дефиницији) се дефинише на следећи начин:

\selectlanguage{english}
{\tt
\textbf{definition} "$\colinh{A}{B}{C} \longleftrightarrow \exists
l.\ \inh{A}{l}\ \wedge\ \inh{B}{l}\ \wedge \ \inh{C}{l}.$"
}
\selectlanguage{serbian}

Наравно, ми желимо да докажемо да наш модел (са нашим дефинициjама
тачке, праве и основних релациjа (подударно, између)) задовољава ове 
аксиоме. На пример, ово значи да ми треба да докажемо:

\vspace{-3mm}
{\tt
\begin{tabbing}
\textbf{lemma} "$A \neq B \longrightarrow \exists l.\ \aginh{A}{l} \wedge
\aginh{B}{l}.$"
\end{tabbing}
}

Докази ових лема су тривијални и углавном су добијени развијањем
дефиниција и онда коришћењем аутоматског доказивања (коришћењем методе
Гребнерових база).

\subsection{Аксиоме поретка}
Аксиоме поретка описују својства (ексклузивне) релације \emph{између}.
\selectlanguage{english}
{\tt
\begin{tabbing}
\hspace{5mm}\=assumes\ \=\kill
\textbf{lemma} "$\beth{A}{B}{C} \longrightarrow A \neq B\,\wedge\,A \neq C\,\wedge\,B \neq C\,\wedge\,\colinh{A}{B}{C}\,\wedge\,\beth{C}{B}{A}$"\\
\textbf{lemma} "$A \neq C \longrightarrow \exists B.\ \beth{A}{C}{B}$"\\
\textbf{lemma} "$\inh{A}{l}\,\wedge\,\inh{B}{l}\,\wedge\, \inh{C}{l}\,\wedge\,A \neq B\,\wedge\,B \neq C\,\wedge\,A \neq C \longrightarrow$}\\
\> $(\beth{A}{B}{C}\ \wedge\ \neg \beth{B}{C}{A}\ \wedge\ \neg \beth{C}{A}{B}) \ \vee$ \\
\> $(\neg\beth{A}{B}{C}\ \wedge\ \beth{B}{C}{A}\ \wedge\  \neg \beth{C}{A}{B}) \ \vee$\\
\>$(\neg\beth{A}{B}{C}\ \wedge\ \neg \beth{B}{C}{A}\ \wedge\ \beth{C}{A}{B})$"}
\end{tabbing}
}
\selectlanguage{serbian}

Докази да релације $\agcongr{}{}{}$, $\aginh{}{}{}$, и $\agbeth{}{}{}$
задовољавају ове аксиоме су једноставни и углавном су изведени
одмотавањем дефиниција и коришћењем аутоматизације.

\paragraph{Пашова аксиома.}

\selectlanguage{english}
{\tt
\begin{tabbing}
\hspace{10mm}\=assumes\ \=\kill
\textbf{lemma} "}$A \neq B\,\wedge\,B \neq C\,\wedge\,C \neq A\,\wedge\,\beth{A}{P}{B}\,\wedge$\\
\>$\,\inh{P}{l}\,\wedge\,\neg \inh{C}{l}\,\wedge\,\neg \inh{A}{l}\,\wedge\,\neg \inh{B}{l}h\ \longrightarrow$\\
\>\>$\exists Q.\ (\beth{A}{Q}{C}\  \wedge\ \inh{Q}{l})\ \vee\
               (\beth{B}{Q}{C}\  \wedge\  \inh{Q}{l})$"}
\end{tabbing}
}
\selectlanguage{serbian}

\begin{center}
\input{ax_h_Pasch.tkz}
\end{center}

У оригиналној Пашовој аксиоми постоји још једна претпоставка -- тачке
$A$, $B$ и $C$ нису колинеарне, тако да је аксиома формулисана само за
централни, недегенерисани случај. Ипак, у нашем моделу тврђење
тривијално важи ако оне јесу колинеарне, тако да смо ми доказали да
наш модел задовољава и централни случај и дегенерисани случај када су
тачке колинеарне. Приметимо да због својстава Хилбертове релације
\emph{између}, претпоставка да су тачке различите не може бити
изостављена.

Доказ користи стандардне технике. Прво, користе се изометријске
трансформације да транслирају тачке на $y$-оси, тако да $A = (0, 0)$,
$B = (x_B, 0)$ и $P = (x_P, 0)$. Нека је $C = (x_C, y_C)$ и
$\RepRt{l} = (l_A, l_B, l_C)$. У зависности у
којим дужима се тражена тачка налази, имамо два велика различита
случаја. Коришћењем својства $\beth{A}{P}{B}$ доказује се да важи
$l_A\cdot y_B \neq 0$ и онда можемо одредити два коефицијента $k_1 =
\frac{-l_C}{l_A\cdot y_B}$ и $k_2 = \frac{l_A\cdot y_B + l_C}{l_A\cdot
  y_B}$.  Даље, доказује се да важи $0 < k_1 < 1$ или $0 < k_2 <
1$. Коришћењем $0 < k_1 < 1$, тачка $Q = (x_Q, y_Q)$ је одређена са
$x_Q = k_1\cdot x_C$ и $y_Q = k_1\cdot y_C$, па зато $\beth{A}{Q}{C}$
важи. У другом случају, када друго својство важи, тачка $Q=(x_q, y_q)$
је одређена са $x_Q = k_2\cdot (x_C - x_B) + x_B$ и $y_Q = k_2\cdot
y_C$, па зато $\bett{B}{Q}{C}$ важи.

\subsection{Аксиоме подударности}
Прва аксиома подударности омогућава конструисање подударних дужи на
датој правој. У Хилбертовој књизи "Основи геометрије" \cite{hilbert}
аксиома се формулише на следећи начин: \emph{"Ако су $A$ и $B$ две
  тачке на правој $a$, $A'$ је тачка на истој или другој правој $a'$
  онда је увек могуће одредити тачку $B'$ на датој страни праве $a'$ у
  односу на тачку $A'$ такву да је дуж $AB$ подударна дужи $A'B'$."}
Ипак, у нашој формализацији део \emph{"на датој страни"} је промењен и
уместо једне одређене су две тачке (приметимо да је ово имплицитно и
речено у оригиналној аксиоми).

\selectlanguage{english}
{\tt
\begin{tabbing}
\hspace{5mm}\=assumes\ \=\kill
\textbf{lemma} "}$A \neq B\,\wedge\,\inh{A}{l}\,\wedge\,\inh{B}{l}\,\wedge\,\inh{A'}{l'}\ \longrightarrow$\\
\> $\exists B'\, C'.\ \inh{B'}{l'}\,\wedge\,\inh{C'}{l'}\,\wedge\,\beth{C'}{A'}{B'}\,\wedge\,\congrh{A}{B}{A'}{B'}\,\wedge\,\congrh{A}{B}{A'}{C'}$"}
\end{tabbing}
}
\selectlanguage{serbian}

Доказ да ова аксиома важи у нашем моделу Декартове координатне равни,
почиње са изометријским трансформацијама тако да $A'$ постаје $(0, 0)$
и $l'$ постаје $x$-оса. Тада је прилично једноставно одредити две
тачке на $x$-оси тако што одредимо координате ових тачака користећи
услов да $\agsqdist{}{}$ (квадратно растојање) између било које од њих
и тачке $A'$ је исто као и $\agsqdist{A}{B}$.

Следеће две аксиоме су директно доказане одвијањем одговарајућих
дефиниција и применом алгебарских трансформација и метода Гребнерових
база.

\selectlanguage{english}
{\tt
\begin{tabbing}
\hspace{20mm}\=assumes\ \=\kill
\textbf{lemma} "$\congrh{A}{B}{A'}{B'}\,\wedge\,\congrh{A}{B}{A''}{B''}\ \longrightarrow\ \congrh{A'}{B'}{A''}{B''}$"\\
\textbf{lemma} "$\beth{A}{B}{C}\,\wedge\,\beth{A'}{B'}{C'}\,\wedge\,\congrh{A}{B}{A'}{B'}\,\wedge\,\congrh{B}{C}{B'}{C'} \ \longrightarrow$}\\ 
\>$\congrh{A}{C}{A'}{C'}$"}
\end{tabbing}
}
\selectlanguage{serbian}

Следеће три аксиоме подударности у Хилбертовој аксиоматизацији су o
појму угла, а ми у оквиру ове тезе нећемо разматрати формализацију
угла.

За разлику од осталих аксиома које су формулисане на основу основних
појмова (тачка, права, \emph{припада}, \emph{подударно},
\emph{између}), у овим аксиомама се јавља потреба коришћења уведених
појмова (нпр. углова, полуправих, подударности углова, припадности
полуправе правој, $\ldots$). Све ове помоћне дефиниције постају
саставни део аксиоматског система и он постаје непотребно велики.

Појам угла је сам по себи веома широк и заправо се под једним
неформалним термином крију различити математички појмови. Угао се може
дефинисати преко две полуправе које се секу, али и преко две праве
које се секу. када се фиксирају полуправе, остаје питање који од два
дела равни се подразумева под углом -- да ли угао увек мора бити
конвексан или су те полуправе уређен пар, па се подразумева
оријентисани угао од прве праве ка другој. Проблем са Хилбертовим
текстом је што иако се може закључити која је мотивација датих
дефиниција, на пуно места текст је непрецизан и потребно је извршити
одређена прецизирања да би се добила формална дефиниција (а то се може
урадити на различите начине и самим тим аксиоматске основе могу бити
различите). На пример, није јасно шта значи да тачка припада углу --
да ли је то скуп тачака или посебан тип за који је потребно увести
нову релацију инциденције. Потом, формално су уведене полуправе, али
не и критеријуму када две полуправе сачињавају праву, нити када
полуправа припада правој, што представља проблем јер се и ти
критеријуми користе (на пример, када се уводе сумплементарни углови).

\subsection{Аксиома паралелности}

\selectlanguage{english}
{\tt
\begin{tabbing}
\hspace{5mm}\=assumes\ \=\kill
\textbf{lemma} "$\neg \inh{P}{l} \ \longrightarrow\ \exists!\,l'.\ \inh{P}{l'}\ \wedge\ \neg (\exists\ P_1.\ \inh{P_1}{l}\ \wedge\  \inh{P_1}{l'})$"
\end{tabbing}
}
\selectlanguage{serbian}

Доказ ове аксиоме састоји се из два дела. Прво је доказано да таква
права постоји а потом да је она јединствена. Доказивање постојања је
учињено одређивањем коефицијената тражене праве. Нека је $P = (x_P,
y_P)$ и $\RepRt{l} = (l_A, l_B, l_C)$. Онда су коефицијенти тражене
праве $(l_A, l_B, -l_A\cdot x_P - l_B\cdot y_P)$. У другом делу
доказа, полази се од претпоставке да постоје две праве које
задовољавају услов $\inh{P}{l'}\ \wedge\ \neg
(\exists\ P_1.\ \inh{P_1}{l}\ \wedge\ \inh{P_1}{l'})$. Доказано је да
су њихови коефицијенти пропорционални па су самим тим и праве једнаке.

\subsection{Аксиомe непрекидности}

\paragraph{Архимедова аксиома.}
Нека је $A_1$ нека тачка на правој између случајно изабраних тачака $A$
и $B$. Нека су тачке $A_2, A_3, A_4, \ldots$ такве да $A_1$ лежи
између тачке $A$ и $A_2$, $A_2$ између $A_1$ и $A_3$, $A_3$ између
$A_2$ и $A_4$ итд. Додатно, нека су дужи $AA_1, A_1A_2, A_2A_3,
A_3A_4, \ldots$ једнаки међусобно. Онда, у овом низу тачака, увек
постоји тачка $A_n$ таква да $B$ лежи између $A$ и $A_n$.

Прилично је тешко репрезентовати низ тачака на начин како је то задато
у аксиоми и наше решење је било да користимо листу. Прво, дефинишемо
листу такву да су сваке четири узастопне тачке подударне, а за сваке
три узастопне тачке важи релација \emph{између}.

\selectlanguage{english}
{\tt
\begin{tabbing}
\hspace{5mm}\=assumes\ \=\kill
\textbf{definition} \\
\> "}congruentl $l \longrightarrow$ length\ $l \ge 3\ \wedge$\\
\>\>  $\forall i.\ 0 \le i\ \wedge\ i+2 <$ length\ $l \longrightarrow$ \\
\>\>  $\congrh{(l\ !\ i)}{(l\ !\ (i+1))}{(l\ !\ (i+1))}{(l\ !\ (i+2))}\ \wedge $\\
\>\>  $\beth{(l\ !\ i)}{(l\ !\ (i+1))}{(l\ !\ (i+2))}$"}
\end{tabbing}
}
\selectlanguage{serbian}

Са оваквом дефиницијом, аксиома је мало трансформисана, али и даље са
истим значењем, и она каже да постоји низ тачака са својствима која су
горе поменута таква да за барем једну тачку $A'$ из дате листе важи
$\bett{A}{B}{A'}$. У \emph{Isabelle/HOL} систему ово је формализовано
на следећи начин:

\selectlanguage{english}
{\tt
\begin{tabbing}
\hspace{5mm}\=\kill
\textbf{lemma} "}$\beth{A}{A_1}{B}\ \longrightarrow$\\
\> $(\exists l.\ $congruentl $(A\ \#\ A1\ \#\ l)\ \wedge\ (\exists i.\ \beth{A}{B}{(l\ !\ i)}))$"}
\end{tabbing}
}
\selectlanguage{serbian}

Главна идеја овог доказа је у тврђењима $\agsqdist{A}{A'} >
\agsqdist{A}{B}$ и $\agsqdist{A}{A'} = t\cdot
\agsqdist{A}{A_1}$. Зато, у првом делу доказа одредимо $t$ такво да
$t\cdot \agsqdist{A}{A_1} > \agsqdist{A}{B}$ важи. Ово се постиже
применом Архимедовог правила за реалне бројеве. Даље, доказано је да
постоји листа $l$ таква да {\tt congruentl} $l$ важи, да је та листа
дужа од $t$, и таква да су њена прва два елемента $A$ и $A_1$. Ово је
урађено индукцијом по параметру $t$. База индукције, када је $t = 0$
тривијално важи. У индукционом кораку, листа је проширена са једном
тачком таквом да важи релација подударности за њу и последње три тачке
листе и да важи релација \emph{између} за последња два елемента листе
и додату тачку. Коришћењем ових услова, координате нове тачке се лако
одређују алгебарским израчунавањима. Када је једном конструисана,
листа задовољава услове аксиоме, што се лако доказује у последњим
корацима доказа. У доказу се користе неке додатне леме које углавном
служе да се опишу својства листе која задовољава услов {\tt
  congruentl} $l$.

% ------------------------------------------------------------------------------
\section{Завршна разматрања}
\label{sec:concl}
% ------------------------------------------------------------------------------

У овој тези ми смо представили добро изграђену формализацију Декартове
геометрије равни у оквиру система \emph{Isabelle/HOL}. Дато је
неколико различитих дефиниција Декартове координатне равни и доказано
је да су све дефиниције еквивалентне. Дефиниције су преузете из
стандардних уџбеника.  Међутим, да би их исказали у формалном окружењу
асистента за доказивање теорема, било је потребно подићи ниво
ригорозности. На пример, када дефинишемо праве преко једначина, неки
уџбеници помињу да различите једначине репрезентују исту праву ако су
њихови коефицијенти ``пропорционални'', док неки други уџбеници често
ово важно тврђење и не наведу. У текстовима се обично не помињу
конструкције као што су релација еквиваленције и класа еквиваленције
које су у основи наше формалне дефиниције.

Формално је доказано да Декартова координатна раван задовољава све
аксиоме Тарског и већину аксиома Хилберта (укључујући и аксиому
непрекидности).  Доказ да наша дефиниција Декартове координатне равни
задовољава све аксиоме Хилберта је тема за наредни рад јер смо
констатовали да формулација аксиоме комплетности и аксиома у којима се
помиње изведени појам угла представљају велики изазов за формализацију
јер су у Хилбертовом тексту задате веома непрецизно.

Наше искуство је да доказивање да наш модел задовољава једноставне
Хилбертове аксиоме лакше него доказивање да модел задовољава аксиоме
Тарског. Разлог за ово највише лежи у дефиницији релације
\emph{између}. Наиме, Тарски дозвољава да тачке које су у релацији
\emph{између} буду једнаке. Ово је разлог за постојање бројних
дегенерисаних случајева који морају да се анализирају посебно што
додатно усложњава расуђивање и доказе. Међутим, Хилбертове аксиоме су
формулисане коришћењем неких изведених појмова (нпр. углова) што
представља проблем за нашу формализацију.

Чињеница да је аналитичка геометрија модел синтетичке геометрије се
често подразумева као једна једноставна чињеница. Ипак, наше искуство
показује да, иако концептуално једноставан, доказ ове чињенице захтева
прилично комплексна израчунавања и веома је захтеван за
формализацију. Испоставља се да је најважнија техника коришћена да се
упросте докази ``без губитка на општости'' и коришћење изометријских
трансформација. На пример, прво смо покушали да докажемо централни
случај Пашове аксиоме без примене изометријских трансформација. Иако
би требало да буде могуће извести такав доказ, израчунавања која су се
појавила су била толико комплексна да ми нисмо успели да завршимо
доказ. После примене изометријских трансформација, израчунавања су и
даље била нетривијална, али ипак, ми смо успели да завршимо овај
доказ. Треба имати на уму да смо морали да се често користимо ручним
израчунавањима јер чак и моћна тактика која се заснива на Гребенеровим
базама није успела да аутоматски упрости алгебарске изразе. Из овог
експеримента са Пашовом аксиомом, закључили смо колики је значај
изометријских трансформација и следећа тврђења нисмо ни покушавали да
доказујемо директно.

Наша формализација аналитичке геометрије се заснива на аксиомама
реалних бројева и у многим доказима су коришћена својства реалних
бројева. Многа својства важе за било које поље бројева (и тактика
заснована на Гребенеровим базама је такође била успешна и у том
случају). Међутим, да би доказали аксиому непрекидности користили смо
својство супремума, које не важи у произвољном пољу. У нашем даљем
раду, планирамо да изградимо аналитичку геометрију без коришћења
аксиома реалних бројева, тј. да дефинишемо аналитичку геометрију у
оквиру аксиоматизације Тарског или Хилберта. Заједно са овим радом, то
би омогућило дубљу анализу неких теоријских својстава модела
геометрије. На пример, желимо да докажемо категоричност и система
аксиома Тарског и система аксиома Хилберта (и да докажемо да су сви
модели изоморфни и еквивалентни Декартовој координатној равни).

