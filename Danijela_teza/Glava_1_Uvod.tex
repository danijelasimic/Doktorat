\chapter{Увод}

\section{Мотивација и циљ тезе}

У класичној математици постоји много различитих геометријских
теорија. Такође, разли\-чи\-та су и гледишта шта се сматра стандардном
(eуклидском) геометријом. Пoнекад, геоме\-три\-ја се дефинише као
независна формална теорија, а понекад као специфични модел. Наравно,
везе између различитих заснивања геометрије су јаке. На пример, може
се доказати да Дека\-рто\-ва раван представља модел формалних
аксиоматских теорија еуклидске геометрије.

Традиционална eуклидска (синтетичка) геометрија је још од античке Грчке
заснована на често малом скупу основних појмова (на пример, тачке,
праве, основних геометријских релација попут инциденције, подударности
итд.) и на скупу аксиома које имплицитно дефинишу ове основне
појмове. Иако су Еуклидови ,,Елементи'' један од најутицајних радова
из математике, поставило се озбиљно питање да ли је систем аксиома,
теорема и лема којима се геометрија описује заиста
прецизан. Испоставило се да су нађене грешке у доказима, а и да су
неки докази били непотпуни јер су имали имплицитне претпоставке
настале због погрешне интуиције или погрешног позивања на слике
(дијаграме). Ове празнине су утицале на појаву других аксиоматских
система чији је циљ био да дају формалну, прецизнију аксиоматизацију
Еуклидове геометрије. Најважнији од тих система су Хилбертов систем
аксиома и систем аксиома Тарског.

\emph{Хилбертов систем} уводи три основна појма (тачка, права и
раван), 6 релација и 20 аксиома подељених по групама. Хилберт је желео
да направи систем који је прецизнији од Еуклидовог, у којем ништа није
остављено интуицији. Овакав приступ је повећао ниво ригорозности не
само у геометрији, него у другим областима математике.

\emph{Систем Тарског} је мањи, уводи један основни појам (тачка), 2
релације и 11 аксиома. Његова основна предност у односу на Хилбертов
систем је у његовој једноставности. Са друге стране, систем Тарског
уводи појам праве као скупа тачака што доста отежава резоновање јер
захтева да се у доказима користи теорија скупова.

Једнo од најзначајнијих открића у математици, које датира из XVII
века, јесте Декартово откриће координатног система, које је омогућило
да се алгебарским изразима представе геометријске фигуре. То је довело
до рада на новој математичкој области која је названа \emph{аналитичка
  геометрија}.

У математичком образовању у средњим школама и на факултетима често се
демонстрирају оба приступа у геометрији (аналитички и
синтетички). Ипак, док се синтетички приступ предаје као ригорозан
систем (са намером да се демонстрира формалан, аксиоматски приступ
изградње математичких теорија), аналитичка геометрија се показује
много мање формално.  Такође, ова два приступа се уводе независно и
веза између њих се ретко формално показује у оквиру стандардног
наставног плана.

Иако се појам сферне геометрије појавио још у старој Грчкој, озбиљније
истраживање неeуклидских геометрија (сферне, хиперболичке и др.) је
започето 1829.~године са радом Лобачевског. Ипак, са њиховим
интензивнијим истраживањем се почело тек пола века касније. Оно што је
највише утицало на ову промену јесте откриће комплексних бројева
крајем XVIII века. Комплексни бројеви су представљали значајну алатку
за истраживање особина објеката у различитим геометријама. Заменом
Декартове координатне равни комплексном равни добијају се
је\-дно\-став\-ни\-је формуле које описују геометријске објекте. Након
Гаусовe теорије о закривљеним површинама и Римановог рада о
многострукостима, геометрија Лобачевског добија на значају. Ипак,
највећи утицај има рад Белтрамија који доказује да дводимензионална
неeуклидска геометрија није ништа друго до геометрија неке површи
константне негативне кривине. Хиперболичка геометрија се изучава кроз
многе њене моделе. Уводи се појам пројективног диск модела који Клајн
касније популаризује. Поeнкаре посматра полуравански модел који су
предложили Лиувил и Белтрами и пре свега изучава изометрије
хиперболичке равни које чувају оријентацију. Данас се те
трансформације и у ширем контексту изучавају у оквиру Мебијусових
трансформација.

%proveriti kod Lucica -- Poinkare ili Poenkare

Потреба за ригорозним заснивањем математике постоји веома дуго и са
развојем ма\-те\-ма\-ти\-ке повећавао се и степен ригорозности. Међу
наукама, математика се издваја својим прецизним језиком и јасним
правилима аргументовања, тј. извођења. Ова чињеница омогућава да се
тачност ма\-те\-ма\-ти\-чких тврђења аргументују формалним извођењима,
тј. доказима. Још у седамнаестом веку, постојала је идеја да мора
постајати неки општи језик којим би се могла записати
ма\-те\-ма\-ти\-чка тврђења и општи систем правила за извођење. Један
од најзначајних напредака у математици почетком двадесетог века било је
откриће да се математички аргументи могу представити у формалним
аксиоматским системима на такав начин да се њихова исправност може
једноставно испитати коришћењем једноставних механичких
правила. Генерално, математика се могла формализовати коришћењем
аксиоматске теорије скупова, теорије ти\-по\-ва, логике вишег реда и
слично. Математички доказ је ригорозан ако може бити записан у некој
формалној логици као низ закључака који су изведени применом јасно
дефинисаних правила.

Често, механички проверени докази попуњавају празнине које постоје у
дефиницијама и доказима и упућују на дубљу анализу теме која се
изучава. У историји математике постоји пуно контроверзи око
исправности математичких доказа. Године 1935.~Лекат је објавио књигу о
грешкама које су до 1900.~године направили познати математичари. Поред
грешака, често се дешавало да математичари нису умели да одреде да ли
је неки доказ исправан или не и дешавало се да се у потпуности верује
да је доказ тачан ако га је објавио познати математичар, као Гаус или
Коши, и њихови докази нису подлегали дубљој критици. У деветнаестом
веку докази постају све комплекснији и математичари почињу да све више
истичу важност ригорозности доказа. Математичари се свакодневно
сусрећу са прескоченим корацима у доказима, са непрецизним
дефиницијама, са хипотезама и претпоставкама које недостају. Понекад
грешке у доказима не буду примећене веома дуго. На пример, први доказ
теореме o обојивости графа са четири боје је имао грешку која је
уочена тек десет година касније. Иако је грешке углавном лако
исправити, има случајева када је то веома тешко. На пример,
1980.~године објављено је да је завршена класификација простих
коначних група, али је примећено да постоји пропуст у једној од класа
и исправка тог пропуста објављена је тек 2001.~године, а доказ је имао
1221 страну \cite{aschbacher2004classification}.  Додатно, често се
дешава да се одређени делови доказа никада не прикажу, често уз
реченицу ,,специјалан случај се тривијално доказује'' при чему се
дешава да за тај специјалан случај тврђење не важи или га није
тривијално доказати. Поред овога, понекад је потребно много времена да
би се неки доказ проверио. На пример, доказ Кеплерове хипотезе коју је
саставио Томас Хејлс има 300 страна и 12 рецензената су провели четири
године у анализи доказа и коначно су написали да су 99\% сигурни да је
доказ исправан.


Mноги научници су сматрали да је потпуна формализација математике
недостижни идеал. Са појавом рачунара настала је могућност генерисања
машински проверљивих доказа. Тако су се појавили системи за формално
доказивање теорема. Постоје системи који омогућавају потпуно
ау\-то\-мат\-ску конструкцију доказа и они користе технике попут SAT
решавача, технике презаписивања, резолуцију, алгебарске
доказиваче. Иако је изградња система за потпуно аутоматско доказивање
теорема важан подухват, постоје, за сада, мале реалне могућности да се
направи систем који заиста аутоматски доказује компликована
ма\-те\-ма\-ти\-чка тврђења.

Зато је посебан акценат на системима који се заснивају на интеракцији
корисника и рачунара. Такви системи су полуаутоматски и у процесу
формалног доказивања теорема од стране корисника (често информатичар
и/или математичар) помажу тако што контролишу исправност доказа и,
колико је то могуће, проналазе аутоматске доказе. Ови {\em
  интерактивни доказивачи} се називају и {\em асистенти за доказивање
  теорема}. Данас постоји мноштво интерактивних доказивача:
\emph{Isabelle}, \emph{Isabelle/HOL}, \emph{Coq}, \emph{HOL Light},
\emph{PVS} и други. Посебно се истичу \emph{Isabelle/HOL} и \emph{Coq}
као системи са великим бројем корисника који су током година развили
велики скуп библиотека са формално доказаним теоријама које је могуће
даље надограђивати. Асистенти за доказивање теорема се користе у
различитим областима. Пре свега могу се користити за формалну
верификацију рачунарских програма. Поред тога, значајна примена је и у
образовању. Помажу развој и продубљивање математичког знања.

% метод  метода
% женски род, заменити где треба

Интересовање за аутоматско доказивање у геометрији постоји још
одавно. Један од првих аутоматских доказивача теорема уопште био је
аутоматски доказивач за геометрију. Тарски је развио алгебарску методу
за доказивање теорема еуклидске геометрије, али је она била
неупотребљива за компликоване теореме. Највећи напредак је направљен
тек средином двадесетог века када је Ву предложио своју алгебарску
методу за доказивање теорема у еуклидској геометрији. Његовом методом
могле су се доказати и веома комплексне теореме. Још једна алгебарска
метода која се развила у исто време је метода Гребнерових база. Ови
методи имају алгебарски, тј. аналитички приступ у доказивању и
заснивају се на репрезентацији тачака коришћењем
координата. Мо\-де\-рни доказивачи теорема који се заснивају на овим
методама могу да докажу стотине нетривијалних теорема. Ипак, велика
мана ових система је што не производе класичне доказе, већ само
пропратне аргументе који нису читљиви. Деведесетих година XX-ог века
постојало је више покушаја да се овај проблем реши и развијене су нове
методе засноване на аксиоматизацији синтетичке геометрије -- метода
површина, метода пуног угла, итд. Ипак, њихова главна мана је што су
далеко мање ефикасни у односу на алгебарске методе. Већина система са
аналитичким приступом за доказивање теорема се користи као софтвер
којем се верује иако нису формално верификовани. Да би се повећала
њихова поузданост потребно их је повезати са модерним интерактивним
доказивачима теорема и то је могуће учинити на два начина -- њиховом
имплементацијом у оквиру интерактивног доказивача теорема и
доказивањем њихове исправности или коришћењем интерактивних доказивача
да провере њиховe сертификате. Неколико корака у овом правцу је већ
направљено \cite{wucoq,thedu}.

Примена система за аутоматско доказивање теорема у геометрији је
велика. На пример могу се користити у образовању. Поред тога, користе
се у научним областима као што су роботика, биологија, препознавање
слика и другим.


\section{Доприноси тезе}

Овај рад покушава да премости неколико празнина за које мислимо да
тренутно постоје у формализацији геометрије.

%% integirisati sa tackom 2 sa spiska i dodati u uvod u kompleksnu
Због своје важности, геометрија комплексних бројева је добро описана у
литератури. Постоје многи уџбеници који описују ову област са много
детаља (током нашег рада, ми смо интензивно користили уџбенике које су
писали Нидам (енг.~\emph{Needham}) \cite{needham} и Швердфегер
(нем.~\emph{Schwerdtfeger}) \cite{schwerdtfeger}).  Такође, постоји
велики избор материјала за ову област (слајдова, белешки, приручника)
који су доступни на вебу. Ипак, ми нисмо упознати да постоји
формализација ове области и у овом раду, ми представљамо наше потпуно
формално, механички проверено представљање геометрије комплексне равни
које је, према нашем сазнању, прво такве врсте.

Додатно, ми сматрамо да је једнако (или чак више) важно искуство које
смо стекли приликом различитих покушаја да достигнемо коначни циљ од
коначног резултата. Наиме, постоји много различитих начина на које је
област изложена у литератури. На пример, Нидам \cite{needham} и
Швердфегер \cite{schwerdtfeger}, представљају два врло различита
начина приказивања исте приче --- један приступ је више геометријски
оријентисан, док је други више алгебарски оријентисан. Наше искуство
показује да је избор правог приступа важан корак у остваривању циља да
формализација буде спроводљива у оквиру асистента за доказивање
теорема --- и показало се да што је више приступ алгебарски
оријентисан, то је формализација једноставнија, лепша, флексибилнија и
робуснија.


У оквиру рада на докторској тези, формализована је аналитичка
геометрија Декартове равни, геометрија комплексне равни, дат је део
формализације Поeнкареовог диск модела, дата је формална анализа
алгебарских метода и систем за аутоматско доказивање у
стереометрији. У наставку текста набројани су основи доприноси тезе:

\begin{itemize}
\item Формализована је аналитичка геометрија тј. Декартова раван у
  оквиру система за интерактивно доказивање теорема. Пре\-дста\-вљена
  је добро изграђена формализација Декартове геометрије равни у оквиру
  система \emph{Isabelle/HOL}. Дато је неколико различитих дефиниција
  Декартове координатне равни и доказано је да су све дефиниције
  еквивалентне. Дефиниције су преузете из стандардних уџбеника, али је
  подигнут ниво ригорозности.  На пример, у текстовима се обично не
  помињу појмови као што су релација еквиваленције и класа
  еквиваленције које ће морати да буду уведене у формалним
  дефиницијама. Формално је доказано да Декартова координатна раван
  задовољава све аксиоме Тарског и већину аксиома Хилберта (укључујући
  и аксиому непрекидности). Анализирани су докази и који од два
  система аксиома је лакши за формализацију. Формално је доказано да
  је аналитичка геометрија модел синтетичке геометрије и анализирано
  је колико су докази заиста једноставни.

\item Коначни резултат нашег рада је добро развијена теорија проширене
  комплексне равни (дате као комплексна пројективна права, али и као
  Риманова сфера), њених објеката (кругови и праве) и њених
  трансформација (на пример, инверзија или Мебијусових
  трансформација). Ова формализација може да служи као веома важан
  блок за изградњу будућих формалних модела различитих геометрија
  (нпр, наша мотивација за овај рад је била управо у покушају да се
  формализује Поeнкареов диск модел хиперболичке геометрије). Већина
  концепата које смо формализовали већ је описана у литератури (иако
  је постојало много детаља које смо морали да попунимо јер их нисмо
  нашли у литератури који смо разматрали). Ипак, наш рад је захтевао
  обједињавање различитих извора у једну јединствену, формалну
  репрезентацију и пребацивање у један јединствен језик узевши да су
  описи били првобитно дати на много различитих начина. На пример, чак
  и у оквиру истог уџбеника, без икаквог формалног оправдања, аутори
  често лако прелазе из једне поставке у другу (рецимо, из обичне
  комплексне равни у проширену комплексну раван), прелазе између
  геометријског и алгебарског представљања, често користе многе
  недоказане, нетривијалне чињенице (посматрајући их као део
  математичког ``фолклора'') и др. Један од наших најзначајнијих
  доприноса је управо расветљавање ових непрецизности и креирање
  униформног, јасног и самосталног материјала.

\item Извршена је формализација шест аксиома Тарског у оквиру
  Поeнкареовог диск модела. Дата је дефиниција релације \emph{између}
  и доказана су нека њена основна својства у оквиру Поeнкареовог диск
  модела.

\item У циљу изградње формално верификованог система за аутоматско
  доказивање у геометрији који користи метода Гребнерових база или
  Вуову методу, имплементиран је корак превођења планиметријских
  тврђења у алгебарску форму у оквиру система \emph{Isabelle/HOL}.
  Поред тога, направљен је и прототип система за доказивање тврђења у
  стереометрији (на чијој се верификацији и даље ради). Први корак је
  представити стереометријске објекте и тврђења у одговарајућем облику
  коришћењем полинома. Направљен је софтвер који омогућава опис
  стереометријских конструкција и тврђења у једноставном облику који
  је разумљив човеку, а потом превођење тог записа у систем полинома
  на који се примењује Вуова метода или метода Гребнерових
  база. Систем је тестиран кроз неколико различитих задатака из
  уџбеника за средње школе и факултетe и задатака са математичких
  такмичења, и анализирана је ефикасност оваквог приступа.
\end{itemize}


\section{Организација тезе}

Остатак тезе организован је на следећи начин. Глава
\ref{chapter::isabelle} садржи преглед развоја интерактивних
доказивача теорема. Дате су теоријске основе и описани основни
принципи технологије интерактивних доказивача теорема. Описанe су
главне карактеристике система \emph{Isabelle/HOL} и дати су бројни
примери. У глави \ref{chapter::pregled_oblasti} наведени су неки
најзначајнији радови у формализацији геометрије. Такође, представљени
су и радови из аутоматског доказивања у геометрији. У глави
\ref{chapter::analiticka} дати су резултати формализације аналитичке
геометрије и доказано је да она представља модел аксиома Тарског
(поглавље \ref{sec:tarski}) и модел аксиома Хилберта (поглавље
\ref{sec:hilbert}). Глава \ref{chapter::hiperbolicka} приказује
формализацију проширене комплексне равни, описане су трансформације
проширене комплексне равни и доказано је да неке аксиоме Тарског важе
у Поeнкареовом диск моделу. У Глави \ref{chapter::algMetodi}
представљени су алгебарски методи за аутоматско доказивање у
геометрији, дата је формализација алгебризације геометријских тврђења
за планарну геометрију и предложена је алгебризација геометријских
тврђења у стереометрији. Предложени систем је тестиран над више
примера и представљени су резултати. Најзад, у глави
\ref{chapter::zakljucak} сумирани су закључци и наведени неки правци
будућег рада.
