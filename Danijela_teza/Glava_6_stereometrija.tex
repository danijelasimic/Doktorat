\chapter[Aлгeбарски методи и стереометрија]{Формална анализа алгебарских метода и њихове примене на проблеме у стереометрији}
\chaptermark{Aлгeбарски методи и стереометрија}
\label{chapter::algMetodi}

% ------------------------------------------------------------------------------
\section{Увод}
% ------------------------------------------------------------------------------

Објекти и релације еуклидске геометрије могу бити описани коришћењем
полинома. Додатно, свака геометријска конструкција може бити изражена
скупом полинома, а многа геометријска тврђења могу бити доказана
кoришћењем алгебарских метода као што су Гребнерове базе или Вуов
метод над скупом полинома. Описаћемо имплементацију алгоритма у
систему \emph{Isabelle/HOL} који као улазне податке прихвата термове
који описују геометријску конструкцију и враћа одговарајући скуп
полинома. Даљи циљ је примена метода Гребнерових база у оквиру система
\emph{Isabelle/HOL} над генерисаним полиномима у намери да се докаже
исправност тврђења.

Главна идеја је да се повеже аутоматско и формално доказивање у
геометрији. И даље не постоји јединствен, нити верификован алгоритам
који трансформише геометријску конструкцију у скуп полинома. Обично,
превођење у полиноме се ради ад-хок методама и не постоји формална
веза између добијених полинома и датих геометријских објеката. Са
формално верификованим методом превођења овај проблем би био решен и
у оквиру ове тезе биће представљени кораци у том правцу.

\section{Алгебарски методи у геометрији}

\subsection{Превођење геометријских тврђења у алгебарску форму}

Алгебарски методи се користе у аутоматском доказивању у геометрији за
теореме конструктивног типа, тј. тврђења о геометријским објектима
које су добијене током геометријске конструкције. Уводе (симболичке)
координате за геометријске објекте (тачке, и понекад праве) који се
јављају у конструкцији и геометријске конструкције и тврђења
изражавају као алгебарске једначине у којима се јављају уведене
координате. Тиме се уместо разматрања еуклидске геометрије прелази на
разматрање у Декартовој координатној равни, тј. једном моделу
еуклидске геометрије. Након увођења координата користе се алгебарске
технике да докажу да тврђење следи из конструкције.

Пре него што је могуће применити алгебарске методе, геометријско
тврђење мора бити записано у алгебарској форми, као скуп
полиномијалних једнакости (односно полинома), при чему се претпоставља
да су све полиномијалне једнакости облика $f(x) = 0$.  Процедуре за
алгебризацију најчешће уводе нове симболичке променљиве за координате
тачке и уводе полиномијане једнакости који карактеришу сваки
конструктивни корак и свако тврђење које је потребно доказати. Иако је
могуће и за праве које се појављују у конструкцији увести непознате
коефицијенте као симболичке променљиве, уобичајено је да се избегава
такав приступ и користе се само тачке (а праве су имплицитно
задате). Свака конструкција почиње скупом слободних тачака и током
конструкције се уводе зависне тачке. У неким случајевима је могуће
зависне тачке бирати са неким степеном слободе (на пример, избор
произвољне тачке на датој правoj).

Свака тачка добија координате које су репрезентоване симболичким
променљивима. Најчешће се слободне променљиве означавају са $u_i$ ($i
= 0, 1, 2, \ldots$), а зависне променљиве се означавају са $x_i$ ($i =
0, 1, 2, \ldots$). Ако је тачка зависна, онда ће све њене координате
бити зависне променљиве.  Ако је тачка слободна, онда су њене
координате такође слободне. Ако је тачка зависна, али са неким
степеном слободе, онда једна координата у дводимензионалном случају
или две координате у тродимензионом случају координате могу бити
слободне. Ипак, ако постоји избор коју координату изабрати за зависну,
а коју за слободну, то питање није тривијално и захтева посебну
пажњу. На пример, у Декартовој координатној равни, ако је тачка $A$
произвољна тачка праве $l$, једна од њених координата може бити
слободна, а једна зависна и може бити израчуната на основу ограничења
која важе за праву. Ипак, ако је права $l$ паралелна са $x$-осом, онда
$y$ координата тачке $A$ не може бити слободна. Слично, ако је $l$
паралелна са $y$-осом онда $x$ координата тачке $A$ не може бити
слободна.

Геометријска ограничења која важе за тачке могу се формулисати у виду
алгебарских ограничења над координатама тачке (тј. као полиномијалне
једначине над уведеним симболичким координатама). На пример,
претпоставимо да су симболичке координате за тачку $A$ $(a^x, a^y)$,
за тачку $B$ $(b^x, b^y)$, а за тачку $C$ $(c^x, c^y)$. Чињеница
да је тачка $A$ средишња тачка интервала $BC$ одговара алгебарским
условима $2\cdot a^x = b^x + c^x$ и $2\cdot a^y = b^y + c^y$. Услов да
су тачке $A$, $B$ и $C$ колинеарне одговара алгебарском услову $(a^x -
b^x )(b^y - c^y) = (a^y - b^y )(b^x - c^x )$. Слични услови се могу
формулисати и за друге основне геометријске релације (паралелне праве,
нормалне праве, симетрала дужи итд.).


\subsubsection{Примери}

\begin{primer}
\label{primer1}
Дат је троугао $ABC$. Нека је $B_1$ средишња тачка дужи $AC$, $C_1$
нека је средишња тачка интервала $AB$. Доказати да је средња линија
$B_1C_1$ паралелна страници троугла $BC$.

\begin{figure}[hb]
\begin{center}
\input{srednja_linija.tkz}
\end{center}
\caption{Теорема о средњој линији троугла} \label{fig:srednja_linija}
\end{figure}


Бирамо координатни систем тако да темена фигуре буду у канонском
положају, те је стога први корак поставити координате за свако теме
троугла. Тачке $A$, $B$ и $C$ су слободне тачке, те им додељујемо
координате $A(u_0, u_1)$, $B(u_2, u_3)$ и $C(u_4, u_5)$. Тачке $B_1$ и
$C_1$ су зависне, те им додељујемо координате $B_1(x_0, x_1)$ и
$C_1(x_2, x_3)$. Како је $B_1$ средишња тачка дужи $AC$, важи
$$2\cdot x_0 - u_0 - u_4 = 0$$ 
$$2\cdot x_1 - u_1 - u_5 = 0$$ Полином са леве страна прве једначине
означићемо са $f_1$, а полином са леве стране друге једначине
означићемо са $f_2$.

Тачка $C_1$ је средишња тачка дужи $AB$, па важи:
$$2\cdot x_2 - u_0 - u_2 = 0$$
$$2\cdot x_3 - u_1 - u_3 = 0$$ Полином са леве страна прве једначине
означићемо са $f_3$, а полином са леве стране друге једначине
означићемо са $f_4$.

Са овe четири једначине дат је опис конструкције. Кажемо да полиноми
$f_1$, $f_2$, $f_3$ и $f_4$ припадају скупу--конструкције.

Потребно је доказати да је $BC$ паралелно са $B_1C_1$, односно,
записано полиномима, потребно је да важи:
$$(x_2 - x_0)(u_5 - u_3) - (x_3 - x_1)(u_4 - u_2) = 0$$ Полином са
леве стране једначине означићемо са $g$ и то је полином тврђења
(припада скупу--тврђења).

Потребно је доказати да свака $n$-торка која анулира полиноме
конструкције такође анулира и полиноме тврђења, тј. потребно је
доказати да
\begin{tabbing}
$\forall\ u_0\ u_1\ u_2\ u_3\ u_4\ u_5\ x_0\ x_1\ x_2\ $ \= $x_3 \in \mathbb{R}.\ 2\cdot x_0 - u_0 - u_4 = 0 \land 2\cdot x_1 - u_1 - u_5 = 0$ \\
                                                       \> $\land\ 2\cdot x_2 - u_0 - u_2 = 0 \land 2\cdot x_3 - u_1 - u_3 = 0$ \\
                                                        \>$\Longrightarrow (x_2 - x_0)(u_5 - u_3) - (x_3 - x_1)(u_4 - u_2) = 0$
\end{tabbing}
\end{primer}

Приметимо да у претходном примеру услов да је $ABC$ троугао није
преведен у услов да су тачке $A$, $B$ и $C$ међусобно
различите. Додатно, услов да је $B_1C_1$ паралелно са $BC$ је дато
условом једначином $(x_2 - x_0)(u_5 - u_3) - (x_3 - x_1)(u_4 - u_2) =
0$. Са друге стране, ова алгебарска једначина је једнака једном
слабијем услову, наиме услову: $B \equiv C$ или $B_1 \equiv C_1$ или
$B_1C_1$ паралелно са $BC$. Преведено у геометријски запис, тврђење
које је доказано алгебарском методом је следеће:

{\em Нека је $B_1$ средишња тачка дужи $AC$ и $C_1$ нека је
  средишња тачка интервала $AB$. Онда је $B_1C_1$ паралелно са $BC$
  или је $B$ идентично тачки $C$ или је тачка $B_1$ идентична тачки
  $C_1$.}

Како $B_1 \not\equiv C_1$ следи из $B \not\equiv C$, претходно тврђење
је еквивалентно са:

{\em Нека су $B$ и $C$ две различите тачке. Нека је $B_1$ средишња
  тачка дужи $AC$ и $C_1$ нека је средишња тачка дужи $AB$. Онда је
  дуж $B_1C_1$ паралелна са $BC$.}

Овај пример показује да превођење геометријског тврђења у алгебарски
запис и обратно захтева да се обрати пажња на многе детаље. У већини
система, хипотеза облика $AB || CD$ се обично записује једначином
облика $(b^x - a^x )(d^y - c^y ) = (d^x - c^x )(b^y - a^y )$, чак се
ова једначина користи као дефиниција за $AB || CD$. Ипак, овакав
приступ раскида везу са синтетичком геометријом.

У зависности од конструкције и тврђења може бити много полинома који
припадају скупу--конструкције или скупу--тврђења. Ова два скупа су
веома важна за метод Гребнерових база (или за Вуов метод) и касније
биће детаљније појашњена њихова улога.

Може се доказати да је већина геометријских својстава инваријантна у
односу на изометријске трансформације \cite{wucoq, wlog}. Ако су $P_1$
и $P_2$ две слободне тачке, увек постоји изометрија (прецизније,
композиција осних рефлексија) која слика $P_1$ у тачку $(0, 0)$ (тј. у
координатни почетак), а тачку $P_2$ у тачку на $x$-оси (или у тачку на
$y$-оси). Зато се без губитка на општости може претпоставити да
слободна тачка има координате $(0, 0)$, док друга тачка има координате
$(u_0, 0)$ или $(0, u_0)$ (иако су оба избора коректна, у неким
случајевима избор може утицати на ефикасност, или, у случају
једноставног Вуовог метода, избор може утицати на могућност
доказивања). Примена овог закључка може значајно утицати на обим посла
који има алгебарска метода. Уз то, постоје хеуристике (са циљем да
побољшају ефикасност) за избор која од слободних тачака је
најпогоднија за ове специјалне координате.
 
\begin{primer}
Без губитка на општости у Примеру \ref{primer1}, тачкама $B$ $C$ могу
бити додељене координате $B(0, 0)$ и $C(u_4, 0)$. Тада се алгебарско
тврђење које треба доказати своди на:
\begin{tabbing}
$\forall\ u_0\ u_1\ u_4\ x_0\ x_1\ x_2\ x_3 \in \mathbb{R}.\ $ \= $2\cdot x_0 - u_0 - u_4 = 0 \land 2\cdot x_1 - u_1 = 0$ \\
                                                             \> $\land\ 2\cdot x_2 - u_0 = 0 \land 2\cdot x_3 - u_1 = 0$ \\
                                                             \> $\Longrightarrow (x_2 - x_0 ) \cdot 0 - (x_3 - x_1 )\cdot u_4 = 0$
\end{tabbing}
што тривијално следи ($x_3 − x_1 = 0$ следи из $2\cdot x_1 - u_1 = 0$
и $2\cdot x_3 - u_1 = 0$).
\end{primer}

\subsection{Алгебарски алгоритми}

Када се геометријско тврђење преведе у алгебарску форму, могуће је
применити алгебарски метод за доказивање теорема. Алгебарски
доказивачи теорема користе специфичан алгоритам над системом
полинома. Ако су $f_1, \ldots, f_k$ полиноми скупа--конструкције, а
$g_1, \ldots, g_l$ полиноми који су добијени из тврђења, онда се
доказивање теореме своди на проверу да ли је за свако $g_i$ испуњено:
$$\forall v_1, \ldots, v_n \in \mathbb{R} \bigwedge_{i=1}^k f_i(v_1,
\ldots, v_n) = 0 \Longrightarrow g_i(v_1, \ldots, v_n) = 0$$ Тарски је
приметио да се коришћењем елиминације квантификатора за реалне бројеве
може доћи до доказа. Али у пракси, тешко је доказати нетривијална
математичка тврђења на овај начин. Зато се примењује другачији
приступ. Главна идеја, коју је предложио Ву 1978. године је да велики
број геометријских тврђења, које се формулишу као универзална
алгебарска тврђења у терминима координата, су такође тачна и за
комплексне вредности координата. Уместо проверавања полинома над
реалним бројевима, користи се поље комплексних бројева и посматра се
следећа претпоставка \footnote{Наравно, постоји проблем некомплетности
  метода (у односу на геометрију) јер у неким случајевима тврђење важи
  у $\mathbb{R}$, али метод не успева да докаже због контрапримера
  који важе у $\mathbb{C}$.  }:
\begin{align}
\forall v_1, \ldots, v_n \in \mathbb{C} \bigwedge_{i=1}^k f_i(v_1,
\ldots, v_n) = 0 \Longrightarrow g_i(v_1, \ldots, v_n) = 0 \label{jednacina}
\end{align}
Ово је
тачно ако $g$ припада идеалу $I = \langle f_1, \ldots f_k \rangle$
који је генерисан над полиномима $f_i$ ($i = 1, \ldots k$), тј. када
постоји цео број $r$ и полиноми $h_1, \ldots h_l$ такви да $g_i^r =
\sum_{i=1}^k h_if_i$. Хилбертова (Nullstellensatz) теорема тврди да
ако је поље алгебарски затворено (а $\mathbb{C}$ јесте) онда је
обрнуто такође тачно.

Два најзначајнија алгебарска метода користе врсту еуклидског дељења да
провере исправност претпоставке дате у \ref{jednacina}. Бухбергеров
алгоритам трансформише полазни скуп у \emph{Гребнерову базу} у којој
алгоритам дељења се може ефикасно употребити, а Вуов метод користи
\emph{псеудо--дељење} које на неки начин имитира еуклидско дељење.

\subsection{Вуов метод}

Главна операција над полиномима у Вуовом методу је псеудо--дељење које
када се примени на два полинома $p(v_1, \ldots, v_n)$ и $q(v_1,
\ldots, v_n)$ производи декомпозицију
$$c_m\ p = tq + r$$ при чему је $c(v_1 ,\ldots , v_{n-1})$ водећи
коефицијент у полиному $q$ уз променљиву $v_n$, $m$ је број ненула
коефицијената полинома $p$, $t(v_1, \ldots, v_n)$ је псеудо--количник,
$r(v_1, \ldots, v_n)$ је псеудо--остатак, степен $v_n$ у $r$ је мањи
него у $q$. Како важи $r = c_m\ p - tq$, јасно је да $r$ припада
идеалу генерисаном над полиномима $p$ и $q$. 

Први корак (једноставног) Вуовог метода \cite{chou1988mechanical}
користи псеудо--дељење да трансформише конструисани систем полинома
($\bigwedge_{i=1}^k f_i$) у троугаону форму, тј. у систем једначина у
коме свака наредна једначина у систему уводи тачно једну нову зависну
променљиву. Након тога, коначни остатак се рачуна псеудо--дељењем
полинома тврђења ($g_i$) са сваким полиномом троугаоног система.

Вуов метод у свом најједноставнијем облику омогућава израчунавање
полинома $c, h_1, \ldots, h_k$ и $r$ таквих да важи
$$cg_i = \sum_{i=1}^k h_if_i + r$$

Ако је коначни остатак $r$ једнак нули, онда се сматра да је
претпоставка доказана. Једноставан Вуов метод није комплетан (у
алгебарском смислу). Комплекснија и комплетна верзија метода користи
растуће ланце који се разматрају у оквиру Рит-Вуовог принципа.

% ------------------------------------------------------------------------------
\subsection{Припадност идеалу, Гребнерове базе, пример примене Гребнеровог метода на конкретан проблем}
% ------------------------------------------------------------------------------

У овом одељку су дате основне дефиниције и теореме које представљају
математичку основу за овај рад. Прва дефиниција је дефиниција проблема
који треба решити. Друга дефиниција дефинише Гребнерове базе, алатку
која даје одговор на проблем припадности идеалу. Коначно, дата је и
теорема која спаја ове две дефиниције.

\begin{definition}[Припадност идеалу]
  За дате $f, f_1, \ldots, f_k \in K[X_1, \ldots, X_n]$ где су $f,
  f_1, \ldots, f_k$ полиноми, а $K[X_1, \ldots, X_n]$ је прстен
  полинома над $K$, и $\langle f_1, \ldots f_k \rangle$ је идеал
  генерисан са $f_1, \ldots f_k$, тада је проблем припадности идеалу
  проблем одређивања да ли је $f \in \langle f_1, \ldots f_k \rangle$
  задовољено.
\end{definition}

\begin{definition}
  Нека $I = \langle f_1, \ldots, f_n \rangle$ је идеал генерисан
  коначним скупом полинома.  $G$ je \textbf{Гребенерова база} идеала
  $I$ ако и само ако је дељење полинома са више променљивих (означено
  са $\to_G$) било ког полинома у идеалу $I$ са $G$ даје 0.
\end{definition}

\begin{theorem}
Проблем $f \in I$ је еквивалентан $f \stackrel{*}{\to}_G 0$.
\end{theorem}

Оно што је заправо речено овом теоремом је да ако постоји скуп
полинома конструкције и ако су сви полиноми једнаки нули, онда је
могуће одредити Гребнерову базу за овај скуп и ако је могуће доказати
(коришћењем полинома Гребнерове базе) да су сви полиноми тврђења једнаки
нули онда је геометријско тврђење тачно. За израчунавање Гребнерове
базе користи се \emph{Бухбергеров алгоритам} који се лако може
имплементирати. Данас постоје многе хеуристике које омогућавају да
израчунавања буду бржа, али ми ћемо га представити у његовој основном
облику:

\begin{definition}
\textbf{S-полином} над полиномима $f_i$ и $f_j$, означен са $S(f_i,
f_j)$ се израчунава на следећи начин:
\begin{description}
\item[1)] $m = GCD(H(f_i), H(f_j))$ 
\item[2)] $m = m_i * H(f_i)$ при чему је $H(f_i)$ водећи моном у $f_i$ 
\item[3)] $m = m_j * H(f_j)$  при чему је $H(f_j)$ водећи моном у $f_j$
\item[4)] $S(f_i, f_j) = m_i*f_i - m_j*f_j$ 
\end{description}
\end{definition}

За скуп полинома $\{f_i, ..., f_j\}$ Бухбергеров алгоритам се састоји
из следећих корака:
\begin{description}
\item[1)] S-полином се одређује за свака два полинома из скупа чији
  водећи мономи нису узајамно прости и новодобијени полиноми се додају
  у скуп.
\item[2)] Понавља се први корак док год има полинома који могу бити
  додати.
\end{description}

Метод Гребнерових база је већ имплементиран у систему \emph{Isabelle/HOL} и
може се користити на следећи начин:
\selectlanguage{english}
{\tt
\begin{tabbing}
\textbf{lem}\=\textbf{ma} "}$\lbrakk -2\cdot u_1 + x_1 + x_2 =$ ($0$::real); $-2\cdot v_1 + y_1 + y_2$ = $0 \rbrakk \Longrightarrow$ \\
            \> $x_1\cdot v_1 - x_1\cdot y_2 + u_1\cdot y_2 - v_1\cdot x_2 - y_1\cdot u_1 + y_1\cdot x_2 = 0$"} \\
\textbf{by algebra}
\end{tabbing}
}
\selectlanguage{serbian}

при чему је метод Гребнерових база позван са {\tt \textbf{by
    algebra}}.

\begin{primer}
Доказати да се симетрале страница троугла секу
  у једној тачки.
\begin{figure}[hb]
\begin{center}
\input{simetrale_stranica.tkz}
\end{center}
\caption{Теорема о симетралама страница троугла} \label{fig:simetrale_stranica}
\end{figure}


Бирамо координатни систем тако да темена фигуре буду у канонском
положају. То значи да је први корак поставити координате троугла тако
што пoставимо тачку $A$ у координатни почетак, односно $A = (0,
0)$. Потом, можемо одабрати да су координате тачке $B = (b^x, 0)$, a
тачке $C = (c^x, c^y)$ (тврђење је потребно доказати за било које
координате тачака, али се лако може доказати да је могуће било које
координате транслирати у ове које су специјално одабране и олакшавају
даља израчунавања). Тачке $A$, $B$ и $C$ су слободне и њихове
координате би могли означити са $u_i$ ($i = 1, 2, 3$), али да би лакше
пратили ознаке у овом примеру, означили смо их другачије. Сада се
геометријска конструкција преводи у скуп полинома.

Прво записујемо хипотезу да се симетрале страница $AB$ и $BC$ секу у
тачки $O_1 = (o^x_{1}, o^y_{1})$. Како су симетрале потпуно одређене тачкама
$A, B, C$ и $O_1$, добијају се следеће једначине:

\medskip
\begin{center}
 \begin{tabular}{rr}
$f_1:$ & $o^x_{1} - \frac{b^x}{2} = 0$\\
{} & {}\\
$f_2:$ & $\frac{b^x - c^x}{c^y}\cdot o^x_{1} - o^y_{1} + \frac{{c^y}^2 - {b^x}^2 + {c^x}^2}{2\cdot c^y} = 0$
\end{tabular}
\end{center}

\medskip

Сада посматрамо другу хипотезу -- симетрале страница $AB$ и $AC$ се
секу у тачки $O_2 = (o^x_{2}, o^y_{2})$. Добијамо две нове једначине:

\medskip
\begin{center}
 \begin{tabular}{rr}
$f_1':$ & $o^x_{2} - \frac{b^x}{2} = 0$\\
{} & {}\\
$f_3:$ & $\frac{c^x}{c^y}\cdot o^x_{2} + o^y_{2} - \frac{c^y}{2} - \frac{{c^x}^2}{2\cdot c^y} = 0$
\end{tabular}
\end{center}

\medskip
Значи, имамо скуп полинома:
\begin{align}
 G = &\{f_1, f_2, f'_1, f_3\} \nonumber
\end{align}

Овим полиномима је заправо описана конструкција. Сада је потребно
доказати да важи $O_1 = O_2$, тј. $(o^x_{1}, o^y_{1}) = (o^x_{2},
o^y_{2})$. То значи да је потребно одредити Гребнерову базу $G'$ скупа
$G$ и циљ је доказати $o^x_{1} - o^x_{2} \to_{G'} = 0$ и $o^y_{1} -
o^y_{2} \to_{G'} = 0$. Са ова два полинома (лева страна датих
једначина) дати су полиноми тврђења.

У намери да се израчуна Гребнерова база скупа $G$ користи се
Бухбергеров алгоритам и добијени резултат је:

\begin{align}
  G' = & \{f_1, f_2, f_3, f_4, f_5, f_6\} = \{ o^x_{1} - \frac{b^x}{2},\quad  o^x_{2} - \frac{b^x}{2},\nonumber \\
  &\frac{b^x-c^x}{c^y}\cdot o^x_{1} - o^y_{1} + \frac{{c^y}^2 - {b^x}^2 + {c^x}^2}{2\cdot c^y},\quad \frac{c^x}{c^y}\cdot o^x_{2} + o^y_{2} - \frac{c^y}{2} - \frac{{c^x}^2}{2\cdot c^y}, \nonumber \\
  &-o^y_{1} + \frac{c^y}{2} + \frac{{c^x}^2}{2\cdot c^y} - \frac{c^x\cdot b^x}{2\cdot
    c^y},\quad o^y_{2} - \frac{c^y}{2} - \frac{{c^x}^2}{2\cdot c^y} + \frac{c^x\cdot
    c}{2\cdot c^y} \nonumber \}
\end{align}

Коришћењем овог скупа, $o^x_{1} - o^x_{2}  \stackrel{*}{\to}_{G'} 0$ се лако може доказати. Заиста,
$$o^x_{1} - o^x_{2} \quad \to_{f_1}\quad  -o^x_{2} + \frac{b^x}{2} \quad\to_{f_3}\quad -o^x_{2} + \frac{b^x}{2} + o^x_{2} - \frac{b^x}{2} = 0\;.$$

Слично се може доказати $o^y_{1} - o^y_{2} \stackrel{*}{\to}_{G'} 0$,
$$o^y_{1} - o^y_{2} \quad \to_{f_5} \quad -o^y_{2} + \frac{c^y}{2} + \frac{{c^x}^2}{2c^y} - \frac{c^xb^x}{2c^y} \quad \to_{f_6}$$ 
$$\to_{f_6} \quad -o^y_{2} + \frac{c^y}{2} + \frac{{c^x}^2}{2c^y} - \frac{c^xb^x}{2c^y} + o^y_{2} - \frac{c^y}{2} - \frac{{c^x}^2}{2c^y} - \frac{c^xb^x}{2c^y} = 0\;.$$
\end{primer}


% ------------------------------------------------------------------------------
\section{Формална анализа алгебарских метода у систему \emph{Isabelle/HOL}}
% ------------------------------------------------------------------------------

% ------------------------------------------------------------------------------
\subsection{Репрезентација планиметријских конструкција коришћењем термова}
% ------------------------------------------------------------------------------

Прво, било је потребно представити геометријску конструкцију на
одговарајући начин тако да се лако може аутоматски обрадити, тј. да се
може користити у оквиру нашег алгоритма. Зато су геометријске
конструкције репрезентоване термовима. Тренутно, постоје два типа
објеката -- тачке и праве. Додатно, геометријска тврђења су
репрезентована коришћењем термова. У систему \emph{Isabelle/HOL}
одговарајући типови се дефинишу на следећи начин:

\begin{small}
\selectlanguage{english} {\tt
  \begin{tabbing}
    \hspace{5mm}\=\hspace{5mm}\=\hspace{5mm}\=\hspace{5mm}\=\hspace{5mm}\=\kill
\textbf{data}\=\textbf{type} \\
    \> point = Point id  $|$ Intersect line line $|$ Midpoint point point \\ 

\textbf{and} line = \= Line point point  $|$ Bisector point point \\
          \>\>\>     $|$ Normal line point $|$ Parallel line point \\

\textbf{and} statement = \= IsIncident point line \\
          \>\>\> $|$ IsEqualp point point  $|$ IsEquall line line \\
          \>\>\> $|$ IsParallel line line  $|$ IsNormal line line \\
          \>\>\> $|$ IsCongruent point point point point \\
          \>\>\> $|$ IsBisector line point point
\end{tabbing}
}
\selectlanguage{serbian}
\end{small}

Као што можемо видети, тачка може бити дата својим идентификатором или
може бити конструисана као пресек две праве или као средиште дужи
одређене двема тачкама. Слично, права може бити конструисана као права
која је одређена двема тачкама или као симетрала дате дужи
итд. Такође, постоје и различита тврђења. На пример, {\tt IsIncident
  point line} означава тврђење да тачка припада правој, а {\tt IsEqualp
  point point} означава да су две тачке једнаке. \\

\begin{primer}
Терм {\tt IsIncident (Point 1) (Line (Point 1) (Point 2))} означава
тврђење да тачка припада правој која је одређена том тачком и још
једном датом тачком.
\end{primer}

\newpage

\begin{primer}
Терм
\begin{small}
\selectlanguage{english}
\emph{
{\tt
  \begin{tabbing}
    \hspace{5mm}\=\hspace{5mm}\=\hspace{5mm}\=\hspace{5mm}\=\hspace{5mm}\=\kill
\textbf{let} \= $c$ = Bisector (Point $A$) (Point $B$); \\
   \> $b$ = Bisector (Point $A$) (Point $C$); \\
   \> $a$ = Bisector (Point $B$) (Point $C$); \\
   \> $O_1$ = Intersect $a$ $b$; \\
   \> $O_2$ = Intersect $a$ $c$ \textbf{in} \\
   \> IsEqualp $O_1$ $O_2$
\end{tabbing}
}
}
\selectlanguage{serbian}
\end{small}
је пример који је описан раније -- симетрале страница се секу у једној
тачки.
\end{primer}

Синтетичким термовима који служе за репрезентацију геометријских
тврђења може бити дата различита семантика интерпретирањем у
различитим моделима геометрије (на пример, Декартова координатна
раван, геометрија Хилберта, геометрија Тарског). Коришћењем система
модула у систему \emph{Isabelle/HOL} ({\tt \textbf{locales}}) избегава
се понављање дефиниција. Зато је дефинисан модуо {\tt
  AbstractGeometry} који садржи примитивне релације (на пример,
релацију инциденције, релацију \emph{између}, релацију
\emph{подударно}) и дефинише њихова својства. Изведени концепти се
могу дефинисати једино у оквиру овог локала. На пример, изведени појам
колинеарност се своди на примитиван појам инциденције --- кажемо да су
три тачке колинеарне акко постоји права којој припадају све три
тачке. Различите геометрије могу интерпретирати овај модуо и
(апстрактне) дефиниције изведених појмова се пренесе у те геометрије.

\begin{sloppypar}
Семантика термова (у апстрактној геометрији) је дата функцијама
\mbox{{\tt point\_interp}}, {\tt line\_interp} и {\tt
  statement\_interp} чији улазни подаци (редом) су {\tt point}, {\tt
  line} и {\tt statement}, а повратна вредност су редом (апстрактна)
тачка, (апстрактна) права или вредност Boolean. Како је апстрактна
интерпретација термова јединствено одређена само ако су слободне тачке
фиксне, све ове функције имају и додатни аргумент --- функцију која
пресликава индексе слободних тачака у тачке.
\end{sloppypar}

Тврђења се интерпретирају коришћењем примитивних релација апстрактне
геометрије, док се конструкције своде на примитивне релације
коришћењем Хилбертовог $\varepsilon$ оператора ({\tt \textbf{SOME}} у
систему \emph{Isabelle/HOL}). На пример: {\tt
\begin{tabbing}
sta\=tement\_interp (Incident $p$ $l$) $fp$ = \\
   \>incident (point\_interp $p$ $fp$) (line\_interp $l$ $fp$) \\
point\_interp (Intersection $l_1$ $l_2$) $fp$ = \\
   \>(\textbf{SOME} $P$. \= incident $P$ (line\_interp $l_1$ $fp$) \\
   \>                    \> $\land$ incident $P$ (line\_interp $l_2$ $fp$))
\end{tabbing}
}

Тврђење (записано термом) је исправно у (апстрактној) геометрији ако
је тачно за све интерпретације (за било који избор слободних тачака).

\selectlanguage{english}
{\tt 
\begin{tabbing}
\textbf{defi}\=\textbf{nition} (\textbf{in} AbstractGeometry) valid :: "statement\_term => bool" \\
  \textbf{where} \\
  \> "valid $stmt$ = (\textbf{ALL} $fp$. statement\_interp $stmt$ $fp$)"
\end{tabbing}
}
\selectlanguage{serbian}

Сви појмови се подижу на ниво конкретног геометријског модела (на
пример, Декартова координатна раван, геометрија Хилберта, геометрија
Тарског) када се докаже да су интерпретација модула {\tt
  AbstractGeometry}.

Након што је дефинисана репрезентација геометријских конструкција
коришћењем термова, следећи корак је превести термове у скупове
полинома тако да се на њих може применити метод Гребнерових база или
Вуов метод.


\subsection{Алгебризација планиметријских термова}

Алгоритам се користи да трансформише репрезентацију геометријске
конструкције и тврђења из записа коришћењем термова у одговарајуће
полиноме. Алгоритам је рекурзиван и његовом применом се добијају два
скупа. Први скуп је скуп полинома који репрезентују геометријску
конструкцију и зато се зове \emph{скуп--конструкција}. Други скуп је
скуп полинома који репрезентују тврђења и њега зовемо
\emph{скуп--тврђења}. Метод Гребнерових база се заснива на доказивању
да се сваки полином из скупа--тврђења може свести на нула коришћењем
Гребнерове базе скупа--конструкција.

Алгоритам рекурзивно обрађује дати терм и за сваки непознати објекат
уводи нове координате. Такође, истовремено, додају се нови полиноми у
одговарајуће скупове који се заснивају на идентитетима аналитичке
геометрије. У сваком тренутку чувају се подаци о тренутном стању,
односно о симболичким координатама које су до тог тренутка уведене
(што су заправо подтермови полазног терма самог тврђења).


Као пример, доказаћемо кораке алгоритма за тврђење {\tt IsIncident
  point\_t line\_t} при чему {\tt point\_t} и {\tt line\_t} могу бити
произвољни, комплексни термови за тачку и праву. Кораци су следећи:
\begin{itemize}
\item додају се нове променљиве $x_0$ и $y_0$. Ове променљиве су
  непознате координате за тачку $O$ која је дата термом {\tt point\_t}
  --- $O(x_0, y_0)$
\item додају се променљиве $a_0$, $b_0$, и $c_0$ које представљају
  непознате коефицијенте за праву $p$ која је дата термом {\tt
    line\_t} --- $p = a_0\cdot x + b_0\cdot y + c_0$.
\item позива се функција {\tt point\_poly(point\_t, $x_0$, $y_0$)} која
  конструише полиноме који спајају променљиве $x_0$ и $y_0$ са термом
  {\tt point\_t}.
\item позива се функција {\tt line\_poly(line\_t, $a_0$, $b_0$)} која
  конструише полиноме који спајају променљиве $a_0$, $b_0$ и $c_0$ са
  термом {\tt line\_t}.
\item додаје се полином $a_0\cdot x_0 + b_0\cdot y_0 + c_0$ у
  скуп--тврђења.
\end{itemize}

Као илустрација у наставку се може видети део кода у систему
\emph{Isabelle/HOL} који имплементира овај корак превођења.  
\begin{small}
\selectlanguage{english} {\tt
  \begin{tabbing}
   alg\=brize (IsIncident $p$ $l$) ==  \\
      \>"}\=\textbf{let} \= $x$ = point\_id\_x 0; $y$ = point\_id\_y 0; \\
         \>\>\> $a$ = line\_id\_a 0;  $b$ = line\_id\_b 0; $c$ = line\_id\_c 0; \\
         \>\>\> ($s'$, $pp$) = point\_poly $p$ $x$ $y$ (| maxp = 0, maxl = 0 |); \\
         \>\>\> (\_, $lp$) = line\_poly $l$ $a$ $b$ $s'$  \textbf{in} \\
         \>\>   (sup $pp$ $lp$, \\
         \>\> Fset.Set[poly\_of (\=PSum [PMult[PVar $a$, PVar $x$], \\ 
         \>\>\>                  PMult[PVar $b$, PVar $y$]])])"}
\end{tabbing}
}
\selectlanguage{serbian}
\end{small}

За репрезентацију полинома коришћена је \emph{Isabelle/HOL} теорија
\emph{Executable Multivariate Polynomials}
\cite{sternagel2013executable}.

\begin{sloppypar}
Као што се може приметити постоје две нове функције \mbox{{\tt
    point\_poly}} и \mbox{{\tt line\_poly}} које имају два аргумента
--- термове и две променљиве. Ове функције су узајамно рекурзивне и
користе се да се одреде полиноми скупа--конструкција. Демонстрираћемо
како функционишу на следећем примеру --- {\tt Intersect line1\_t
  line2\_t}. Као и раније, термови {\tt line1\_t} и {\tt line2\_t}
репрезентују линије и могу бити произвољно комплексни. Терм примера
репрезентује тачку и потребно је одредити полиноме који одређују ту
тачку. Кораци алгоритма у овом примеру су следећи:
\end{sloppypar}
\begin{itemize}
\item додајемо променљиве $a_1$ и $b_1$ који су коефицијенти праве ($p
  = a_1\cdot x + b_1\cdot y + 1$) која је дата термом {\tt line1\_t}
\item додајемо променљиве $a_2$ i $b_2$ који су коефицијенти праве ($p
  = a_2\cdot x + b_2\cdot y + 1$) која је дата термом {\tt line2\_t}
 \item позива се функција {\tt line\_poly(line\_t, $a_1$, $b_1$)}
 \item позива се функција {\tt line\_poly(line\_t, $a_2$, $b_2$)}
 \item додају се полиноми $x\cdot(b_2\cdot a_1 - a_2\cdot b_1) - b_1 +
   b_2$ и $y\cdot(b_2\cdot a_1 - a_2\cdot b_1) - (a_2 - a_1)$ у
   скуп--конструкције.
\end{itemize}
Ови полиноми су добијени коришћењем геометријских једнакости тако да
дато геометријско својство важи.

Описани алгоритам се може оптимизовати и могуће је додати још
геометријских објеката (кругови, елипсе итд.) и геометријских тврђења.

% ------------------------------------------------------------------------------
\subsection{Доказивање исправности}
% ------------------------------------------------------------------------------

Главна идеја је да се аутоматски метод за доказивање теорема у
геометрији формално верификује. То значи да је главни део нашег рада
да се докаже исправност алгоритма превођења. У намери да се то
уради, биће коришћена аналитичка геометрија као веза између синтетичке
геометрије и алгебре. Потребно је доказати да све што се докаже
коришћењем алгебарских метода такође важи у свим моделима синтетичке
геометрије. Са друге стране, још је потребно доказати да је аналитичка
геометрија модел синтетичке геометрије и још даље, да су сви модели
изоморфни, тј. да све што важи у нашем моделу такође важи и у другим
моделима.

Формализацију везе између синтетичке и аналитичке геометрије, односно,
доказ да је аналитичка геометрија модел геометрије Тарског и
геометрије Хилберта смо пoказали и дискутовали раније, у поглављу
\ref{chapter::analiticka}.


\paragraph{Веза између аналитичке геометрије и алгебре.}
Централна теорема коју смо формално доказали је да ако су сви полиноми
тврђења нула кад год су и полиноми конструкције нула (тј. ако сви
полиноми тврђења припадају идеалу генерисаном над полиномима
конструкције), онда је тврђење исправно у аналитичкој
геометрији. Односно, ако је неко тврђење доказано методом Гребнерових
база оно заиста важи и у аналитичкој геометрији. Формалније записано:
\[
(\forall (u, x))(\forall g\in G)( (\forall f\in F. f(u,x) = 0) \Rightarrow g(u,x) = 0) \Rightarrow \textrm{геометријско тврђење} \nonumber
\]
при чему је $F(u, x)$ скуп--конструкције, а $G(u, x)$ је
скуп--тврђења. Када кажемо {\em геометријско тврђење} мислимо на
тврђење у аналитичкој геометрији јер су алгебарски методи повезани са
аналитичком геометријом. Први део је доказати да важи $(\forall f \in
F)(\forall (u,x))f(u,x) = 0$. Други део је доказати да ако је доказано
$(\forall g\in G)(\forall f\in F. f(u,x) = 0) \Rightarrow g(u,x) = 0)$
онда геометријско тврђење важи у аналитичкој геометрији. 

Запис овог тврђења у систему \emph{Isabelle/HOL} је:
\selectlanguage{english}
{\tt
\begin{tabbing}
\textbf{theorem} "}\textbf{let} ($cp$, $sp$) = algebrize term \textbf{in} \\
(\textbf{ALL} $ass$\=. (\= (\textbf{ALL} $p$ : $cp$. eval\_poly $ass$ $p$ = 0) $\longrightarrow$ \\
     \>                 \> (\textbf{ALL} $p$ : $sp$. eval\_poly $ass$ $p$ = 0)) $\longrightarrow$ \\
     \> AnalyticGeometry.valid $s$)"}
\end{tabbing}
}
\selectlanguage{serbian}

 Доказ се изводи коришћењем индукције у систему
 \emph{Isabelle/HOL}. Посматрајмо следећи пример:

{\tt In (Midpoint (Point 0) (Point 1)) (Line (Point 0) (Point 1))}\\

{\tt Point 0} и {\tt Point 1} добијају фиксне координате $(p_0^x,
p_0^y)$ и $(p_1^x, p_1^y)$. Потом {\tt Midpoint (Point 0) (Point 1)}
добија координате $(x_1, y_1)$ и оне су зависне променљиве и зависе од
{\tt Point 0} и {\tt Point 1}. На исти начин додељујемо координате
$(a_1, b_1, c_1)$ за {\tt Line (Point 0) (Point 1)} $(a_1, b_1, c_1)$
(то су опет зависне променљиве које зависе од већ датих тачака {\tt
  Point 0} и {\tt Point 1}). За доказ исправности потребно је доказати
да важи:
\begin{align}
 2x_1 = p_0^x + p_1^x \qquad a_1(p_1^xp_0^y - p_1^yp_0^x) - c_1(p_1^y - p_0^y) = 0 \nonumber \\ 
 2y_1 = p_0^y + p_1^y \qquad b_1(p_1^xp_0^y - p_1^yp_0^x) + c_1(p_1^x - p_0^x) = 0 \nonumber \\
 \textrm{закључак је дат у форми једначине: } \qquad a_1x_1 + b_1y_1 + c_1 = 0 \nonumber
\end{align}
Ово се лако доказује коришћењем идентитета у аналитичкој геометрији.

\paragraph{Доказивање алгебарског тврђења.}
На основу претходног доказивање геометријског тврђења своди се на
доказивање алгебарског тврђења
$$(\forall (u, x))(\forall g\in G)( (\forall f\in F. f(u,x) = 0)
\Rightarrow g(u,x) = 0).$$ У пракси се ово ради коришћењем екстерних
алгебарских доказивача заснованих на Вуовој методи или на методи
Гребнерових база. Иако у систему \emph{Isabelle/HOL} постоји подршка
за Гребнерове базе комплекснија тврђења су ван домашаја те методе. Ако
се жели висок степен поузданости онда је неопходно извршити одређену
проверу алгебарског доказивача или барем резултата који су од њега
добијени. Овај други приступ заснован је на провери сертификата и
описан је у радовима \cite{grobnercoq, narboux2015towards}. Наиме,
приликом доказивања алгебарског тврђења, у виду матрице памте се
трансформације које је направио алгебарски доказивач. Потом се у
оквиру асистента за доказивање теорема \emph{Coq} проверава да ли се
применом те матрице трансформација на дате полазне полиноме заиста
добија нула полином. Ми се у оквиру ове тезе нисмо бавили оваквим
аспектима повезивања доказивача и асистента за доказивање теорема.

%prica o kategoricnosti geometrija
\paragraph{Веза између синтетичке геометрије и алгебре.} 
Уколико покажемо да тврђење важи у Декартовој координатној равни,
односно у аналитичкој геометрији, да ли из тога следи да се тврђење
може доказати из аксиома Хилберта или из аксиома Тарског? Постоји
тврђење да су сви модели аксиома Хилберта међусобно изоморфни. Слично
тврђење постоји и за моделе аксиома Тарског. Поред тога, показано је и
да су аксиоматски системи Тарског и Хилберта потпуно дедуктивни, што
значи да свако тачно тврђење у моделу може да се докаже из аксиома. На
основу тога знамо да ако алгебарски доказивач покаже одговарајућу везу
између полинома, онда то тврђење важи у Декартовој равни (аналитичкој
геометрији), па самим тим у свим осталим моделима геометрије, што
повлачи да се то тврђење може доказати из, на пример, Хилбертових
аксиома. Формализација ових мета--теоретских особина геометрије је
веома захтеван подухват и било би потребно формализовати појам
доказивости у оквиру геометрије Хилберта или у оквиру геометрије
Тарског. Ипак, ова теза прави одређене кораке у том смеру, а то је
формалан доказ да Декартова раван преставља модел геометрије Хилберта
и модел геометрије Тарског.


% ------------------------------------------------------------------------------
\section{Примена алгебарских метода на проблеме у стереометриjи}
% ------------------------------------------------------------------------------

Коришћење алгебарских доказивача се интензивно користи у планиметрији и
постоји много система посвећених овом проблему. То је разлог зашто смо
у претходном поглављу разматрали теоријске аспекте алгебарских
доказивача. Са друге стране, коришћење алгебарских доказивача у
стереометрији није значајно истраживано и према нашем сазнању, не
постоји потпуно развијен аутоматски систем за коришћење алгебарских
доказивача за доказивање у стереометрији. Стога нам је циљ у овом
поглављу да дизајнирамо систем за запис и трансформацију геометријских
тврђења на начин погодан за примену у оквиру алгебарских
доказивача. Теоријска анализа таквог стереометријског система, слична
оној каква је урађена за планиметријске доказиваче, остављена је за
даљи рад.

\paragraph{Аутоматско доказивање у стереометрији -- досадашњи резултати.}
Чу и сарадници су представили метод запремине за решавање проблема у
стереометрији \cite{volumemethod}.  То је полу--алгебарски метод који
је проширење методе површине за стереометрију. Хипотезе се могу
конструктивно представити, а закључци су полиномијалне једначине које
садрже неколико геометријских величина, као што су однос запремина,
однос дужи, однос површина и Питагорине разлике. Кључна идеја метода
је да елиминише тачке из закључка геометријског тврђења коришћењем
неколико основних својстава запремине.


Главна мотивација за наш рад потекла је из интересантног рада који
представља неколико примера алгебарског доказивања у стереометрији
\cite{shao2016challenging}. У раду се посматрају задаци из
стереометрије са Олимпијских такмичења из математике. Они у раду
представљају три различита проблема и дају полиноме које су извели на
папиру и који описују посматране проблеме. Коришћењем ова три примера
они показују да се алгебарски методи могу користити за доказивање у
стереометрији. За сваки пример користили су три различита метода:
метод карактеристичног скупа \cite{wu2007mathematics,
  wang1998decomposing, gao1991computations, chen2002projection}, метод
Гребнерових база \cite{cox1992ideals, kutzler1986application,
  stifter1993geometry, chou1987characteristic} и метод вектора
\cite{lord1985method}. Методи се пореде и закључак је да метод вектора
даје бољи геометријски доказ, али формуле могу бити дуге и незгодне за
манипулацију и израчунавање. Ипак, они не нуде неки систематичан начин
како се геометријска тврђења могу представити полиномима.

Према нашем досадашњем знању, не постоје радови који описују начин
примене Вуове методе или методе Гребнерових база на проблеме из
стереометрије.

\paragraph{Динамички геометријски софтвер.}
Последњих неколико година, рачунари и технологија се интензивно користе
и мењају начин како се предаје геометрија.  Динамички геометријски
системи као што су \emph{GeoGebra}
 \footnote{\url{https://www.geogebra.org/}},
 \emph{Cinderella} \footnote{\url{https://www.cinderella.de/tiki-index.php}},
 \emph{Geometer's
   Sketchpad} \footnote{\url{http://www.dynamicgeometry.com/}},
 \emph{Cabri} \footnote{\url{http://www.cabri.com/}},
 \emph{Eukleides} \footnote{\url{http://www.eukleides.org/}} се данас
 често користе у свим нивоима образовања. Студенти користе овакве
 системе да би изводили геометријске конструкције и дијаграме које
 могу да мењају променом слободних тачака. Такви динамички дијаграми
 су бољи него статичке слике јер померање слободних тачака може да
 пружи додатни увид у проблем и да открије дегенерисане случајеве и
 да помогне студентима да утврде да ли је нешто тачно ако и само ако
 је неки специјални размештај тачака задат (на пример, неко својство
 може бити тачно само ако је нека тачка између неке друге две тачке, a
 нетачно је ако то није случај, неко својство може бити тачно само за
 оштре, али не и за тупе углове, итд.).

Интензивним мењањем дијаграма померајући слободне тачке, студент може
бити прилично сигуран да ли је својство тачно у општем случају
(тј. тачно у готово свим случајевима, осим у малој групи дегенерисаних
случајева), али ипак, то не можемо сматрати доказом и овакав приступ
је подложан грешкама. Зато, у скорије време, динамички геометријски
системи су проширени аутоматским системима за доказивање, који
аутоматски могу доказати тврђење о конструисаним објектима
\cite{geogebra-provers}. Такви системи су најчешће алгебарски
(операције се изводе над симболичким координатама геометријских
објеката).

Можда још значајна употреба динамичког геометријског софтвера може
бити за тродимензионални простор у коме је често тешко голим оком
одредити неко својство. Ово је најчешће стога што се тродимензиони
простор посматра као дводимензиона пројекција, па самим тим мере и
односи нису у складу са стварним дијаграмом. Неки системи су почели да
развијају подршку за тродимензионе конструкције. У најновијој верзији
система \emph{GeoGebra} развијена је подршка за динамичку
тродимензионалну геометрију и
графику\footnote{\url{https://wiki.geogebra.org/en/3D_Graphics_View}}. Могуће
је креирати и интерактивно мењати тродимензионалне објекте као што су
тачке, праве, полигони, сфере, као и тродимензионе цртеже
функција. Ипак, овај систем не подржава доказивање тврђења о
тродимензионалним објектима.

Такође, постоји и додатак за систем \emph{Cinderella},
\emph{Cindy3D} \footnote{\url{http://gagern.github.io/Cindy3D/}}. Могуће
је цртати објекте коришћењем команди и формула које их описују.

Већина истраживања како у динамичким геометријским системима, као и у
аутоматским доказивачима теорема је посвећена само дводимензионалној
Еуглидској геометрији (планарној геометрији). Иако постоји неколико
покушаја да се примене алгебарски доказивачи теорема на
тродимензионалну eуклидску просторну геометрију (стереометрију), ми
нисмо нашли да постоји детаљни опис ових метода, нити јавно доступних
аутоматских доказивача за стереометрију. У овом раду ми истражујемо и
поредимо неколико приступа како се алгебарски доказивачи засновани на
Вуовој методи и методи Гребнерових база могу применити на проблеме из
стереометрије. Нудимо један систем за стереометрију који може да
доказује тврђења о својствима конструисаних објеката. Такође,
анализирамо корпус проблема из стереометрије и оцењујемо коришћене
методе. Дискутујемо о изазовима и могућим применама у пољу предавања
геометрије.

\subsection{Aлгебризација геометријских релација у стереометрији}
\label{polynomials_6glava}

Да бисмо могли да применимо алгебарске методе прво се мора омогућити
репрезентација различитих геометријских релација између објеката у
стереометрији коришћењем полиномијалних једнакости над њиховим
координатама. У овом поглављу даћемо примере како се то може учинити
за најчешће релације међу објектима.

Постоји више приступа за запис релација као полиномијалних једнакости
и прво питање које се поставља је који објекти у тродимензионом
простору се сматрају основним. У првом приступу за који смо се
одлучили, сви објекти су дефинисани коришћењем тачака (на пример,
праве се дефинишу преко две различите тачке, равни се дефинишу преко
три различите, неколинеарне тачке итд.~). Једине променљиве које се
користе у свим полиномима су координате тачака. У другом приступу, све
врсте објекта се представљају коришћењем њихових сопствених координата
(на пример, права се дефинише координатама једне своје тачке и
координатама вектора правца, а раван се дефинише помоћу коефицијената
једначине равни -- координате вектора нормале на раван и њена
удаљеност од координатног почетка). Полиноми укључују све ове
координатне променљиве. Показаћемо како енкодирати релације коришћењем
оба приступа и упоредићемо њихову ефикасност.

\subsubsection{Основни појмови коришћени у конструкцији полинома}

Већина релација се изражава коришћењем истог скупа појмова које ћемо
овде увести.

Сваки објекат је представњен неком $n$--торком параметара (а видећемо
да то могу бити и симболичке и нумеричке вредности).

Тачке имају три параметра, означена са $({[\_]}^x, {[\_]}^y,
{[\_]}^z)$ који репрезентују њене координате. Свака тачка је задата
или својим симболичким или нумеричким координатама. За сваку
новоуведену тачку се додељују нове симболичке координате.

Праве се представљају различито у зависности од коришћеног приступа. У
првом приступу, права је задата са две различите тачке и са шесторком
која представља координате тих тачака. За праву $p$, прва тачка ће
бити означена са $p_A$, а друга тачка ће бити означена са $p_B$.

У другом приступу, права је дата датом тачком $A$ и датим вектором
$v$. Вектор праве $p$ ће бити означен са $\overrightarrow{p_v}$, а
тачка праве $p$ ће бити означена са $p_A$. Зато, као и у првом
приступу, права ће и у другом приступу имати шест параметара који су
означени са $({[\_]}^{v_x}, {[\_]}^{v_y}, {[\_]}^{v_z}, {[\_]}^{A_x},
{[\_]}^{A_y}, {[\_]}^{A_z})$, који репрезентују праву дату једначином:
$$x = k\cdot[\_]^{v_x} + {[\_]}^{A_x}\ \ y = k\cdot[\_]^{A_y} +
{[\_]}^{A_y}\ \ z = k\cdot[\_]^{v_z} + {[\_]}^{A_z}.$$

У претходној једначини, $k$ означава размеру праве, али ова
информација се не чува међу параметрима праве (а то је поменута
шесторка). У неким полиномима ће бити потребно користити параметар
размере праве, али ће се посматрати као нова симболичка променљива.

Равни се такође могу различито представити у зависности од коришћеног
приступа. У првом приступу, раван се задаје са три неколинеарне тачке,
односно са деветорком њихових координата.  Прва тачка равни $\pi$ ће
бити означена са $\pi_A$, друга тачка ће бити означена са $\pi_B$, а
трећа тачка ће бити означена са $\pi_C$.

У другом приступу, равни су одређене нормалним вектором $v$ и додатним
параметром $d$ (померај у односу на координатни почетак). Вектор равни
$\pi$ ће бити означен са $\overrightarrow{\pi_v}$, а слободан
параметар равни ће бити означен са ${[\_]}^{d}$. Стога ће раван имати
само четири параметра, означена са $({[\_]}^{v_x}, {[\_]}^{v_y},
{[\_]}^{v_z}, {[\_]}^{d})$, који репрезентују раван дату следећом
једначином:
$${[\_]}^{v_x}\cdot x + {[\_]}^{v_y}\cdot y + {[\_]}^{v_z}\cdot z +
{[\_]}^{d} = 0.$$

Вектор одређен двема тачкама $A = (a^x, a^y, a^z)$ и $B = (b^x, b^y,
b^z)$ је $\overrightarrow{AB} = (b^x- a^x, b^y - a^y, b^z -
a^z)$. Стандардни појмови скаларног производа, векторског производа и
мешовитог производа се могу применити над векторима. Скаларни производ
вектора $v = (v^x, v^y, v^z)$ и $u = (u^x, u^y, u^z)$ је $v\cdot u =
v^x\cdot u^x+ v^y\cdot u^y + v^z\cdot u^z$, њихов векторски производ
је одређен матрицом:
$$ v\times u = \left|\begin{array}{ccc} \overrightarrow{i} & \overrightarrow{j} & \overrightarrow{k} \\ 
                       v^x& v^y & v^z \\
                       u^x& u^y & u^z \\
\end{array}\right|,$$
а мешовити производ са вектором $w = (w^x, w^y, w^z)$ је једнак
$v\cdot (u \times w)$, и одређен је матрицом:
$$\left|\begin{array}{ccc} v^x& v^y & v^z \\ u^x& u^y
  & u^z \\ w^x& w^y & w^z \\
\end{array}\right|.$$

\subsubsection{Репрезентaција стереометријских релација}

У овом поглављу су дати полиноми који аритметички описују релације над
конструисаним објектима (на пример, две тачке су једнаке, две линије
су паралелне, две равни су нормалне). Свака релација уводи
полиномијална ограничења над координатама објеката који учествују у
релацији и у зависности од приступа могу бити различити.

Улазни параметри дате релације су параметри свих објеката који су
укључени у ту релацију. На пример, за релацију \mbox{{\tt congruent}
  $A$ $B$ $C$ $D$} улаз су четири тачке, $A$, $B$, $C$, and $D$,
односно њихове симболичке координате: $(a^x, a^y, a^z)$, $(b^x, b^y,
b^z)$, $(c^x, c^y, c^z)$ и $(d^x, d^y, d^z)$.

\begin{description}
% ------------------------
\item[$\triangleright$] {\tt congruent} $A$ $B$ $C$ $D$

  {\em Опис:} Дужи $AB$ и $CD$ су подударне.

  %{\em Input:} Points $A$, $B$, $C$, and $D$.

{\em Полиноми:} \\ $\overrightarrow{AB} \cdot \overrightarrow{AB} =
\overrightarrow{CD} \cdot \overrightarrow{CD}$. \\ Приметимо да из
претходног израза можемо добити полиномијалну једнакост $poly = 0$,
при чему је

\begin{tabbing}
$poly = $ \= $({a^x} - {b^x})^2 + ({a^y} - {b^y})^2 + ({a^z} - {b^z})^2 -$ \\ 
          \> $({c^x} - {d^x})^2 - ({c^y} - {d^y})^2 - ({c^z} - {d^z})^2$
\end{tabbing}

{\em Објашњење:} Квадрати растојања између $A$ и $B$ мора бити једнако
квадрату растојања између $C$ и $D$.

% ------------------------
\item[$\triangleright$] {\tt segments\_in\_ratio} $A$ $B$ $C$ $D$
  $m$ $n$ 

  {\em Опис:} Дужина дужи $AB$ и $CD$ су у датом односу $\frac{m}{n}$,
  тј.~ $\frac{|AB|}{|CD|} = \frac{m}{n}$.

{\em Полиноми:} \\
$n^2 \cdot \overrightarrow{AB} \cdot \overrightarrow{AB} = m^2 \cdot \overrightarrow{CD} \cdot \overrightarrow{CD}$. \\


{\em Објашњење:} Квадрати растојања између $A$ и $B$ и између $C$ и
$D$ морају бити у односу $\frac{m^2}{n^2}$. Приметимо да се ово своди
на подударност када је $m = n$.

% ------------------------
\item[$\triangleright$] {\tt is\_midpoint} $M$ $A$ $B$

{\em Опис:} Проверава да ли је тачка $M$ средња тачка дужи одређене
тачкама $A$ и $B$.

{\em Полиноми:} $\overrightarrow{MA} = \overrightarrow{MB}$.
Приметимо да ова једнакост даје три различита полинома $poly_1 = 0$,
$poly_2 = 0$, $poly_3 = 0$:

\begin{tabbing}
$poly_1 = $ \= $2m^x - a^x - b^x$ \\ 
$poly_2 = $ \= $2m^y - a^y - b^y$ \\ 
$poly_3 = $ \= $2m^z - a^z - b^z$
\end{tabbing}

% ------------------------
\item[$\triangleright$] {\tt point\_segment\_ratio} $M$ $A$ $B$ $p$ $q$

{\em Опис:} Проверава да ли тачка $M$ дели дуж одређену тачкама $A$ и
$B$ у односу који је одређен са $p$ и $q$, тј. $\frac{|MA|}{|MB|} =
\frac{p}{q}$.

{\em Полиноми:} Изводе се три полинома из:
$q\cdot \overrightarrow{MA} = p\cdot \overrightarrow{MB}$.

{\em Објашњење:} Приметимо да {\tt is\_midpoint $M$ $A$ $B$} може
такође бити записано коришћењем овог правила, на следећи начин {\tt
  point\_segment\_ratio} $M$ $A$ $B$ $1$ $1$.


% ------------------------
\item[$\triangleright$] {\tt equal\_points} $A$ $B$

{\em Опис:} Проверава да ли две тачке $A$ и $B$ имају исте координате.

{\em Полиноми:} Изводе се три полинома из израза $\overrightarrow{AB}
= 0$.


% ------------------------
\item[$\triangleright$] {\tt translate} $A$ $O$ $v$

  {\em Опис:} Проверава да ли је тачка $O$ једнака тачки која се
  добија транслирањем тачке point $A$ за вектор $(v^x, v^y, v^z)$.

  {\em Полиноми:} Полиноми су изведени из $\overrightarrow{AO} = (v^x,
  v^y, v^z)$


% ------------------------
\item[$\triangleright$] {\tt orthogonal\_4points} $A$ $B$ $C$ $D$

{\em Опис:} Проверава да ли је права одређена са тачкама $A$ и $B$
нормална на праву одређену са тачкама $C$ и $D$.

{\em Полиноми:} $\overrightarrow{AB} \cdot \overrightarrow{CD} = 0$


% ------------------------
\item[$\triangleright$] {\tt orthogonal\_lines} $p$ $q$
% Dodati pricu o proveri mimoilaznosti

{\em Опис:} Две праве, $p$ и $q$ су нормалне.

{\em Полиноми:} Ако је коришћен први приступ, онда су праве задате
тачкама $(p_A, p_B)$ и $(q_A, q_B)$, па се то своди на претходни
случај и полином је $\overrightarrow{p_Ap_B} \cdot
\overrightarrow{q_Aq_B} = 0$. Ако се користи други приступ, као улаз
задат је вектор правца правих и полином је $\overrightarrow{p_v} \cdot
\overrightarrow{q_v} = 0.$

% ------------------------
\item[$\triangleright$] {\tt incident} $A$ $p$

{\em Опис:} Проверава да ли тачка $A$ припада правoj $p$.

{\em Полиноми:} Ако је коришћен први приступ, онда се изводе три
полинома из једнакости $\overrightarrow{p_Ap_B} \times
\overrightarrow{Ap_A} = 0$. Ако је коришћен други приступ, онда се
изводе три полинома из једнакости $\overrightarrow{p_v} \times
\overrightarrow{Ap_A} = 0.$

% ------------------------
\item[$\triangleright$] {\tt parallel\_lines} $p$ $q$

{\em Опис:} Проверава да ли су две праве, $p$ и $q$ паралелне.

{\em Полиноми:} Изводе се три полинома из $\overrightarrow{p_Ap_B}
\times \overrightarrow{q_Aq_B}$ или из $\overrightarrow{p_v} \times
\overrightarrow{q_v}$ у зависности од приступа.

% ------------------------
\item[$\triangleright$] {\tt line\_orth\_plane\_4points} $l$ $A$ $B$ $C$ $D$

{\em Опис:} Проверава да ли je правa $l$ која садржи тачку $D$
нормална на раван одређеном тачкама $A$, $B$ и $C$.

{\em Полиноми:} Изводе се три полинома из $\overrightarrow{l_v} =
\overrightarrow{AB}\times \overrightarrow{AC}$ и три полинома из
једнакости $\overrightarrow{l_v} \times \overrightarrow{Dl_A} = 0.$.

% ------------------------
\item[$\triangleright$] {\tt parallel\_planes} $\alpha$ $\beta$

{\em Опис:} Проверава да ли су две равни $\alpha$ и $\beta$ паралелне.

{\em Полиноми:} Ако се користи други приступ, изводе се три полинома из 
$\overrightarrow{\alpha_v} \times \overrightarrow{\beta_v} = 0.$

% Videti da li je moguce dobiti malo manje polinome ako se prvi pristup
% ne primenjuje na isti nacin kao drugi.

% Ovde raspisati da bi se videlo kako su ovi polinomi orgromni i ruzni
Ако се користи први приступ изводе се полиноми из једнакости:
$$\overrightarrow{\beta_A\beta_B}\cdot \overrightarrow{\alpha_A\alpha_C} \times \overrightarrow{\alpha_B\alpha_A} = 0$$
$$\overrightarrow{\beta_A\beta_C}\cdot \overrightarrow{\alpha_A\alpha_C} \times \overrightarrow{\alpha_A\alpha_B} = 0$$

%$$(\overrightarrow{\beta_A\beta_C}\times\overrightarrow{\beta_A\beta_B}) \times (\overrightarrow{\alpha_A\alpha_C} \times \overrightarrow{\alpha_A\alpha_B}) = 0$$


% ------------------------
\item[$\triangleright$] {\tt orthogonal\_planes} $\alpha$ $\beta$

{\em Опис:} Проверава да ли су две равни, $\alpha$ и $\beta$ нормалне.

{\em Полиноми:} 
Ако се користи први приступ полином се изводи из: 
$(\overrightarrow{\alpha_A\alpha_B} \times \overrightarrow{\alpha_A\alpha_B}) \cdot (\overrightarrow{\beta_A\beta_B} \times \overrightarrow{\beta_A\beta_B}) = 0$.

Ако се користи други приступ, полином се изводи из:
$\overrightarrow{\alpha_v} \cdot \overrightarrow{\beta_v} = 0$.

% ------------------------
\item[$\triangleright$] {\tt point\_in\_plane} $A$ $\pi$

{\em Опис:} Проверава да ли тачка $A$ припада равни $\pi$.

{\em Полиноми:} Ако је коришћен први приступ, онда се полином добија из:
$$\overrightarrow{\pi_AA}\cdot (\overrightarrow{\pi_A\pi_B} \times \overrightarrow{\pi_A\pi_C}) = 0.$$
Ако је коришћен други приступ, полином се добија из једнакости:
$\overrightarrow{\pi_v} \cdot \overrightarrow{A} + \pi^{d} = 0$.

% ------------------------
\item[$\triangleright$] {\tt parallel\_line\_plane} $p$ $\alpha$

{\em Опис:} Проверава да ли су права $p$ и раван $\alpha$ паралелни.

{\em Полиноми:} 
Ако је коришћен први приступ, једнакост је:
$\overrightarrow{p_Ap_B} \cdot (\overrightarrow{\alpha_A\alpha_B \times \alpha_A\alpha_C}) = 0$.

Ако је коришћен други приступ, полином се добија из једнакости:
$\overrightarrow{p_v} \cdot \overrightarrow{\alpha_v} = 0$.

{\em Примедба:} Ова релација важи и у случају када права припада равни.

% ------------------------
\item[$\triangleright$] {\tt orthogonal\_line\_plane} $p$ $\alpha$

{\em Опис:} Проверава да ли су права $p$ и раван $\alpha$ нормални.

{\em Полиноми:} Ако је коришћен први приступ, полиноми се изводе из две једнакости:
$\overrightarrow{p_Ap_B} \cdot \overrightarrow{\alpha_A\alpha_B} = 0$ and
$\overrightarrow{p_Ap_B} \cdot \overrightarrow{\alpha_A\alpha_C} = 0$.

Ако је коришћен други приступ, три полинома се изводе из једнакости:
$\overrightarrow{p_v} \times \overrightarrow{\alpha_v} = 0$.

% ------------------------
% Dodati pravila za uglove 
\item[$\triangleright$] {\tt equal\_angles} $A$ $O$ $B$ $C$ $K$ $D$

{\em Опис:} Проверава да ли су два угла $\angle AOB$ и $\angle
CKD$ једнаки.

{\em Полиноми:} Ова релација се може изразити коришћењем
тригонометрије. Ипак, треба имати на уму да коришћењем тригонометрије,
услов је ослабљен јер се пореде косинуси углова, а као што је познато,
косинус тупог угла и косинус оштрог угла могу бити исти, а углови
(јасно) нису једнаки. Полиноми се изводе из
$$\cos^2{\angle AOB} = \cos^2{\angle CKD}.$$
А косинус угла се може одредити на следећи начин:
$$\cos^2{\angle AOB} = \frac{(\overrightarrow{AO}\cdot
  \overrightarrow{BO})^2}{|AO|^2|BO|^2}$$ при чему је $|AO|^2 =
\overrightarrow{AO}\cdot \overrightarrow{AO}$.  Једнакост за
$\cos^2{\angle CKD}$ је слична. Коначно, након неколико једноставних
алгебарских операција, полином релације се може извести из једнакости
$$(\overrightarrow{AO}\cdot  \overrightarrow{BO})^2|CK|^2|DK|^2 = (\overrightarrow{CK}\cdot  \overrightarrow{DK})^2|AO|^2|BO|^2.$$

Ипак, као што ћемо видети у наредном поглављу, овако задат полином је
веома комплексан и било је потребно раставити га на једноставније да
би могао ефикасно да се користи у доказивачима теорема.


\paragraph{Тела.}
\label{solids_6glava}
Тела се задају коришћењем релација које важе за њихова темена и за
њихове странице. Одлучили смо да подржимо само она тела која се налазе
у неком \emph{канонском} положају --- на пример, при дефинисању коцке,
једно теме се налази у координатном почетку, а друга три темена се
налазе на координатним осама ($x$--оси, $y$--оси и $z$--оси). Ипак,
овакав приступ има мане. На пример, није могуће задати више од једне
коцке коришћењем елементарне наредбе за задавање коцке. Темена других
коцки на слици које нису у канонском положају морају се задати
коришћењем релација које смо представили изнад. Ипак, са друге стране,
у геометријским проблемима која се сусрећу у збиркама најчешће постоји
само једно слободно тело, и без губитка на општости се може
претпоставити да је оно у канонском положају. Када се у тексту задатка
уводе друга тела, она су обично зависна у односу на већ задато
слободно тело, па су њихова темена у некој релацији са већ датим
објектима. Зато, могућност задавања само канонских објеката за већину
задатака није представљао проблем.

% ------------------------
\item[$\triangleright$] {\tt make\_cube} $A$ $B$ $C$ $D$ $A_1$ $B_1$ $C_1$ $D_1$

  {\em Опис:} Коцка у канонском положају, са дужином странице једнакој
  $1$.

  {\em Објекти и параметри:} Тачке $A(0, 0, 0)$, $B(1, 0, 0)$, $C(1,
  1, 0)$, $D(0, 1, 0)$, $A_1(0, 0, 1)$, $B_1(1, 0, 1)$, $C_1(1, 1, 1)$
  и $D(0, 1, 1)$.

  {\em Полиноми:} Не генеришу се полиноми. 

  {\em Објашњење:} Како је коцка у канонском положају, не уводе се
  нове симболичке променљиве.

% ------------------------
\item[$\triangleright$] {\tt make\_tetrahedron} $A$ $B$ $C$ $D$
\label{tetraedar_6glava}

  {\em Опис: } Тетраедар у канонском положају. 

  {\em Објекти и параметри:} Темена тетраедра имају координате $A(0,
  0, 0)$, $B(1, 0, 0)$, $C(c^x, c^y, 0)$ и $D(c^x, d^y, d^z)$, при
  чему се уводе четири нова параметра.

  \begin{tabbing}
    {\em Полиноми:} \= $poly_1 = 2\cdot c^x - 1$ \\
    \> $poly_2 = 2\cdot {c^y}^2 - 3$ \\
    \> $poly_3 = 3\cdot d^y - c^y$ \\
    \> $poly_4 = 3\cdot {d^z}^2 - 2$
  \end{tabbing}
  
  {\em Објашњење:} $c^x = \frac{1}{2}$, $c^y = \frac{\sqrt{3}}{2}$,
  $d^y = \frac{\sqrt{3}}{6} = \frac{c^y}{3}$, $d^z =
  \frac{\sqrt{2}}{\sqrt{3}}$. Приметимо да сви објекти имају или
  симболичке или бројевне параметре. Коефицијенти полинома морају увек
  бити цели бројеви, па се ирационалне вредности (као и разломци)
  морају увести коришћењем полинома.

% ------------------------
\item[$\triangleright$] {\tt make\_pyramid\_4side} $A$ $B$ $C$ $D$ $S$

  {\em Опис:} Правилна пирамида у канонском положају -- основа
  пирамиде је јединични квадрат у $xOy$ равни, висина пирамиде није
  фиксирана, а стране пирамиде се једнаке дужине.
  
  {\em Објекти и параметри:} Тачке $A(0, 0, 0)$, $B(1, 0, 0)$, $C(1,
  1, 0)$, $D(0, 1, 0)$ и $S(s^x, s^y, s^z)$, са три нове симболичке
  променљиве $s^x$, $s^y$ и $s^z$.
  
  \begin{tabbing}
    {\em Полиноми:} \= $poly_1 = 2\cdot s^x - 1$ \\
                   \> $poly_2 = 2\cdot s^y - 1$
  \end{tabbing}

  {\em Објашњење:} Пројекција врха пирамиде је $(s^x, s^y, 0)$ и она
  лежи у центру јединичног квадрата, тако да $s^x = s^y =
  \frac{1}{2}$. Приметимо да $s^z$ није ограничено.

% ------------------------
\item[$\triangleright$] {\tt make\_square} $A$ $B$ $C$ $D$

  {\em Опис:} Задаје се квадрат у канонском положају, односно квадрат
  у равни $xOy$, чије једно теме је у координатном почетку, а друга
  два темена на координатним осама и дужина странице је једнака $1$.

  {\em Објекти и параметри:} Тачке $A(0, 0, 0)$, $B(1, 0, 0)$,
  $C(1, 1, 0)$, $D(0, 1, 0)$.


  {\em Полиноми:} Не креирају се нови полиноми.

% ------------------------
\item[$\triangleright$] {\tt equilateral\_triangle} $A$ $B$ $C$

  {\em Опис:} Једнакостранични троугао у канонском положају -- налази
  се у $xOy$ равни, једна тачка је у координатном почетку, а друга
  тачка је на $x$--оси.

  {\em Објекти и параметри:} Тачке $A(0, 0, 0)$, $B(1, 0, 0)$,
  $C(c^x, c^y, 0)$, са два нова параметра $c^x$ и $c^y$.

\begin{tabbing}
{\em Полиноми:} \= $poly_1 = 2\cdot c^x - 1$ \\
                   \> $poly_2 = 4\cdot c^y - 3$
\end{tabbing}


% ------------------------
\item[$\triangleright$] {\tt regular\_hexagon} $A_1$ $A_2$ $A_3$ $A_4$ $A_5$ $A_6$

  {\em Опис:} Правилни шестоугаоник у канонском положају -- налази се
  у $xOy$ равни, једна тачка је у координатном почетку, а друга тачка
  је на $x$--оси.

  {\em Објекти и параметри:} Тачке $A_1(0, 0, 0)$, $A_2(1, 0, 0)$,
  $A_3(a_3^x, a_3^y, 0)$, $A_4(1, a_4^y, 0)$, $A_5(0, a_4^y, 0)$ и
  $A_6(a_6^x, a_3^y, 0)$, са четири нова параметра $a_3^x$, $a_3^y$,
  $a_4^y$ и $a_6^x$.

\begin{tabbing}
{\em Полиноми:} \= $poly_1 = 2\cdot a_3^x - 3$ \\
                   \> $poly_2 = 4(a_3^y)^2 - 3$ \\
                   \> $poly_3 = a_4^y - 3$ \\
                   \> $poly_4 = 2a_6^x - 1$
\end{tabbing}
\end{description}

\subsection{Упрошћавање полинома}
\label{simplification_6glava}

Да бисмо могли да тестирамо предложену алгебризацију геометријских
релација користили смо две алатке, \emph{GeoProver} и
\emph{Mathematica}, метод Гребнерових база. Први проблем на који смо
наишли је временско и просторно ограничење које се дешавало у бројним
ситуацијама због комплексности скупа полинома. Разлог за ово је велики
број променљивих и велики број полинома који су у процесу
алгебризације креирани. Треба имати на уму да су полиноми у
стереометрији доста комплекснији од одговарајућих полинома у
планарној геометрији. Зато смо морали да упростимо скуп добијених
полинома.

Користили смо приступ о којем смо раније доста писали -- Харисонов
приступ без губитка на општости \cite{wlog} тако што смо бирали
погодни координатни систем јер избор координатног система може
значајно да смањи комплексност полинома.


За три независно задате тачке, $A$, $B$ и $C$, могуће је изабрати
њихове координате на такав начин да је $A(0, 0, 0)$ у координатном
почетку, $B(0, 0, b^z)$ се налази на $z$--оси, а $C(0, c^y, c^z)$ лежи
у $yOz$ равни. Са овим избором координата, број променљивих је смањен
за шест, а одговарајуће нуле значајно поједностављују полиноме. Овај
приступ је често коришћен у алгебарским методима. Не утиче на општост
тврђења и оправданост коришћења овог приступа је у чињеници да
транслације и ротације могу бити коришћење да трансформишу тачке у
њихове канонске положаје. Транслације и ротације су изометрије што
значи да чувају растојање, али и бројне геометријске релације као што
је инциденција, нормалност, паралелност, меру угла и друге.

Без примене овог метода, чак и најједноставнија тврђења не могу бити
доказана. Избором погодних координата значајно се повећава
једноставност система полинома и тиме алгебарски доказивачи су знатно
ефикаснији. Даље, може се десити да су неки полиноми вишак јер постану
једнаки $0$ и онда је само потребно избрисати их из система. Додатно,
неки полиноми постану полиноми који имају само једну променљиву на
неки степен (имају само један моном) и ти полиноми исто могу бити
избрисани, а одговарајућа променљива се може поставити на нула. 

Поред поменутог, треба још имати на уму да доказивач \emph{GeoProver}
не може да трансформише систем полинома који садржи једнакост $0 = 0$
у троугаони систем. Са друге страна, метод Гребнерових база
имплементиран у систему \emph{Mathematica} није имао проблема приликом
доказивања уколико су се у систему налазили полиноми $0 = 0$ или
$c_i\cdot x_i^s = 0$.

Раније смо већ видели како се погодно могу задати тела. О томе смо
писали у одељку \ref{solids_6glava} и управо избор да тела ставимо у
канонски положај значајно утиче на упрошћавање полинома.

Надаље, применили смо још један метод за симплификацију и то у
случају када имамо полиноме облика $c_i\cdot x_i - c_j\cdot x_j =
0$ или $c_i\cdot x_i + c_j\cdot x_j = 0$. Ако без губитка на
општости претпоставимо да $j < i$, заменимо свако појављивање $x_i$ са
$x_j$ у првом случају, односно свако појављивање $x_i$ са $-x_j$, онда
можемо и поменуте полиноме избрисати из система. Овим се смањује и број
полинома и број променљивих.

Један од најкомплекснијих полинома који се могу добити током
алгебризације је полином који се добија од релације "једнаки
углови". Приликом доказивања тврђења о једнакости углова, доказивач
\emph{GeoProver} је достигао просторни лимит већ након неколико
корака. Са друге стране, метод Гребенерових база из система
\emph{Mathematica} је радио неколико сати након чега смо решили да
прекинемо доказивање (а нисмо добили одговор да ли је тврђење
тачно). Потом је полином за једнакост углова подељен у више мањих,
једноставнијих полинома:
\begin{tabbing}
$scalar_1 =$ \=  $\overrightarrow{AO} \cdot \overrightarrow{BO}$  \\
$scalar_2 =$ \> $\overrightarrow{CK} \cdot \overrightarrow{DK}$ \\
$dist_{CK} =$   \> $|CK|$ \\
$dist_{DK} =$   \> $|DK|$ \\
$dist_{AO} =$   \> $|AO|$ \\
$dist_{BO} =$   \> $|BO|$ \\
$poly =$ \> $scalar_1*scalar_1*dist_{CK}*dist_{DK}$ \\
         \> $- scalar_2*scalar_2*dist_{AO}*dist_{BO}$
\end{tabbing}

Иако смо овим повећали скуп полинома, оба доказивача нису имала
проблема приликом доказивања истог тврђења и оба су доказивање
завршили у кратком временском периоду, око једне секунде.

\subsection{Додатно подешавање полинома за Вуов метод}
\label{presekPravih_6glava}

Најинтересантнији пример приликом задавања релација које задовољава
тачка је случај пресека правих. У тродимезионом простору немају све
праве пресек, неке су паралелне, али неке могу бити и
мимоилазне. Погледајмо релацију и полиноме које она генерише.

{\tt intersection\_lines} $A$ $l_1$ $l_2$

{\em Опис:} Тачка $A$ је пресек две дате праве $l_1$ и $l_2$.

{\em Улаз:} Две дате праве $l_1$ и $l_2$ задате својим параметрима
(посматрамо само други приступ, слична ситуација је и у првом
приступу) $l_1(l_1^{v_x}, l_1^{v_y}, l_1^{v_z}, l_1^{p_x}, l_1^{p_y},
l_1^{p_z})$ и $l_2(l_2^{v_x}, l_2^{v_y}, l_2^{v_z}, l_2^{p_x}
l_2^{p_y}, l_2^{p_z})$. 

{\em Нови објекти и параметри:} Тачка $A$ са симболичким координатама
$(a^x, a^y, a^z)$. Параметри $k_1$ и $k_2$ који редом представљају
размере правих $l_1$ и $l_2$.

{\em Полиноми:} Полиноми се генеришу коришћењем правила:
{\tt intersection\_lines} $A$ $l_1$ $l_2$ = {\tt incident} $A$ $l_1$ и {\tt incident} $A$ $l_2$

\begin{tabbing}
{\em Полиноми:} \= $poly_1 = a^x - k_1\cdot l_1^{v_x} - l_1^{p_x}$ \\
                   \> $poly_2 = a^y - k_1\cdot l_1^{v_y} - l_1^{p_y}$ \\
                   \> $poly_3 = a^z - k_1\cdot l_1^{v_z} - l_1^{p_z}$ \\
                   \> $poly_4 = a^x - k_2\cdot l_2^{v_x} - l_2^{p_x}$ \\
                   \> $poly_5 = a^y - k_2\cdot l_2^{v_y} - l_2^{p_y}$ \\
                   \> $poly_6 = a^z - k_2\cdot l_2^{v_z} - l_2^{p_z}$
\end{tabbing}

{\em Објашњење:} Како тачка $A$ припада обема правама $l_1$ (која је
дата тачком point $\overrightarrow{l_1^p}$ и вектором
$\overrightarrow{l_1^v}$) и $l_2$ (дата тачком
$\overrightarrow{l_2^p}$ и вектором $\overrightarrow{l_2^v}$), онда
тачка $A$ мора да задовољи њихове параметарске једначине, односно,
мора да важи
$\overrightarrow{A} = \overrightarrow{l_1^p} + k_1 \cdot
\overrightarrow{l_1^v}$
и
$\overrightarrow{A} = \overrightarrow{l_2^p} + k_2 \cdot
\overrightarrow{l_2^v}$.

Оно што је важно је да ови полиноми користе две нове променљиве $k_1$
и $k_2$ које редом представљају размере правих $l_1$ и $l_2$. То значи
да је укупан број нових променљивих пет, а постоји шест полинома које
ова релација генерише. Ако су свих шест полинома укључени у скуп
полинома, онда није могуће одредити еквивалентни троугаони систем јер
има више полинома него непознатих променљивих. Довољно је пет полинома
да би се одредило решење система, односно вредности свих параметара, а
преостали, шести полином је у функцији оправдавања решења. Односно,
уколико су тачке мимоилазне, за израчунате вредности шести полином
неће бити једнак нули, иако су остали полиноми једнаки нули. Било који
од задатих шест полинома може бити искључен из система, али мора бити
проверен коришћењем доказивача да ли је и он нула (ако није, онда
тврђење у општем случају не важи).

Ипак, у зависности од система, треба бити пажљив приликом избора који
полином избацити из скупа полинома. На пример, нека је $(1, 1, 0)$
вектор праве $l_1$, а тачка која јој припада је $(0, 0, 0)$ и нека је
$(0, 0, 1)$ вектор праве $l_2$, а права садржи тачку $(1, 1,
1)$. Тада, систем добијених полинома је:
\begin{tabbing}
{\em Polynomials:} \= $poly_1 = a^x - k_1$ \\
                   \> $poly_2 = a^y - k_1$ \\
                   \> $poly_3 = a^z$ \\
                   \> $poly_4 = a^x - 1$ \\
                   \> $poly_5 = a^y - 1$ \\
                   \> $poly_6 = a^z - k_2 - 1$
\end{tabbing}

Као што се може приметити, једини полином који има променљиву $k_2$ је
последњи полином. Зато, последњи полином се не сме избацити из скупа
полинома, а могуће решење је избацити полином $poly_5$. Иако се све
сличне специфичне ситуације могу лако детектовати, имплементација
избора који полином избацити, а које полиноме задржати је прилично
незгодна јер је потребно много \emph{if--else} испитивања.

Сличан проблем је и код релације {\tt intersection\_4points} $M$ $A$
$B$ $C$ $D$ и на истоветан начин се и овај проблем решава.

\subsection{Експерименти}

У овом поглављу ћемо представити резултате тестирања алгебризације
коришћењем система \emph{GeoProver} и Гребнерових база имплементираних
у систему \emph{Mathematica}. Применили смо методе на двадесет и пет
проблема из збирке задатака "Збирка задатака из геометрије простора за
припрему пријемног испита на Архитектонском факултету"
\cite{zbirkaArhitektura}, на три задатка из "Збирке задатака из
геометрије" \cite{janicic1997zbirka} и на проблем једнакости углова
представљен у раду о примени аутоматских доказивача на проблеме са
Олимпијада из математике \cite{shao2016challenging}.

\paragraph{Пример нормала тетраедра.} \emph{Нека је $ABCS$ тетраедар и нека 
су $h_a$, $h_b$, $h_c$ и $h_s$ редом висине из темена тетраедра $A$,
$B$, $C$ и $S$ на одговарајуће наспрамне странице и нека је $H$ пресек
$h_a$ и $h_b$. Тада $H$ припада и висинама $h_s$ и $h_c$. }

\begin{figure}[hb]
\begin{center}
\input{presek_tetraedar.tkz}
\end{center}
\caption{Нормале тетраедра се секу у истој тачки}
\end{figure}

Тврђење се може записати на следећи начин:

\begin{footnotesize}
{\tt
\begin{tabbing}
make\_tetrahedron $A$ $B$ $C$ $S$ \ \ \ \ \ \ \ \ \ \ \ \ \= \\ \\

plane\_points $\alpha_1$ $A$ $B$ $C$ \> plane\_points $\alpha_2$ $B$ $C$ $S$ \\
plane\_points $\alpha_3$ $A$ $C$ $S$ \> plane\_points $\alpha_4$ $A$ $B$ $S$ \\ \\

orthogonal\_line\_plane $h_1$ $\alpha_1$ $S$ \> orthogonal\_line\_plane $h_2$ $\alpha_2$ $A$ \\
orthogonal\_line\_plane $h_3$ $\alpha_3$ $B$ \> orthogonal\_line\_plane $h_4$ $\alpha_4$ $C$ \\ \\

intersection\_lines $H_1$ $h_1$ $h_2$ \> intersection\_lines $H_2$ $h_2$ $h_3$ \\
intersection\_lines $H_3$ $h_2$ $h_4$ \\ \\

equal\_points $H_1$ $H_2$ \\
equal\_points $H_1$ $H_3$ \\
\end{tabbing}
}
\end{footnotesize}

Како је тетраедар у канонском положају према самој конструкцији,
координате тачака су $A(0, 0, 0)$, $B(0, 0, 1)$, $C(x_1, x_2, 0)$ и
$S(x_3, x_4, x_5)$ при чему су $x_1$, $x_2$, $x_3$, $x_4$ и $x_5$
зависне променљиве и налазе се у полиномима који описују тетраедар
(погледати у ранијем поглављу полиноме за тетраедар
\ref{tetraedar_6glava}).

Када се изврши алгебризација коришћењем другог приступа, добија се
$38$ полинома који описују релације међу објектима, $6$ полинома за
које треба доказати да су једнаки нули (односно ти полиноми су
полиноми који изражавају својство да се висине секу у једној тачки) и
$41$ променљива. Потом иде поступак поједностављивања и анализе
добијених полинома. Број полинома који описују релације се смањи на
$29$, a и број променљивих се смањи на $29$. Број полинома за које
треба доказати да су нула је $9$. Може бити чудно да се број полинома
за које треба доказати да су нула увећао, али то увећање се добија
због пресека правих. Наиме, тачке $H_1$, $H_2$ и $H_3$ (за које треба
доказати да су једна иста тачка) се добијају као пресек правих,
односно одговарајућих висина. Како за пресек правих треба доказати да
оне нису мимоилазне, већ да се заиста и секу, за један или више
полинома пресека је потребно доказати да они под датим условима су
такође нула. О овом проблему је говорено раније
\ref{presekPravih_6glava}. Полиноми добијени након поједностављивања
су знатно краћи и најчешће се састоје од само два монома (у почетном
скупу су били знатно комплекснији и у просеку су се састојали од $7$
монома). Оба алгебарска доказивача (\emph{GeoProver} и метод
Гребнерових база имплементираних у систему \emph{Mathematica}) су били
успешни у доказивању да су свих девет полинома једнаки нули. Просечно
време потребно за доказ у систему \emph{GeoProver} је $0.0862$
секунде.

Када се изврши алгебризација коришћењем првог приступа, добија се
$34$ полинома који описују релације међу објектима, $6$ полинома за
које треба доказати да су једнаки нули и $33$ променљиве. Након
поједностављивања добија се $22$ полинома који описују релације међу
објектима, $9$ полинома за које треба доказати да су једнаки нули
(исти разлог као и у другом приступу) и $25$ различитих
променљивих. Полиноми су за нијансу комплекснији него у првом приступу
и у просеку имају $4$ монома. Оба алгебарска доказивача
(\emph{GeoProver} и метод Гребнерових база имплементираних у систему
\emph{Mathematica}) су били успешни у доказивању да су свих девет
полинома једнаки нули. Просечно време потребно за доказ у систему
\emph{GeoProver} је $0.1132$ секунде. \\

\begin{table}[ht]
\begin{center}
\begin{tabular}{L{2cm}||R{2cm}|R{2cm}|R{2cm}|R{2cm}|R{1.5cm}}
\                   &  број полинома & број полинома доказа & просечан број монома & број променљивих & време \\
\hline
\hline
\textbf{први приступ} & $22$ & $9$ & $4$ & $25$ & $0.1132s$ \\
\hline
\textbf{други приступ} & $29$ & $9$ & $7$ & $22$ & $0.0862s$
\end{tabular}
\caption{Упоредни приказ успешности алгебризације у односу на изабрани приступ}
\end{center}
\end{table}

\paragraph{Пример једнаких углова.} \emph{Нека је $ABCD$ тетраедар и нека 
је тачка $O$ центар описаног круга тетраедра $ABCD$. Нека су тачке
$K$, $L$ и $M$ редом средине страница $AB$, $BC$ и $CA$. Доказати да
је $\angle KOL = \angle LOM = \angle MOK$.}

\begin{figure}[hb]
\begin{center}
\input{jednaki_uglovi.tkz}
\end{center}
\caption{Тврђење о једнаким угловима између датих дужи}
\end{figure}

Тврђење се може записати на следећи начин:

\begin{footnotesize}
{\tt
\begin{tabbing}
make\_tetrahedron $A$ $B$ $C$ $D$ \ \ \ \ \ \ \ \ \ \ \ \ \ \ \ \ \ \ \= \\ \\

line\_orth\_plane\_4points $l_1$ $A$ $B$ $C$ $D$ \\
line\_orth\_plane\_4points $l_2$ $A$ $C$ $D$ $B$ \\ \\

intersection\_lines $O$ $l_1$ $l_2$ \\ \\

midpoint $K$ $A$ $B$ \\
midpoint $L$ $B$ $C$ \\
midpoint $M$ $C$ $A$ \\ \\

equal\_angles  $K$ $O$ $L$ $L$ $O$ $M$
\end{tabbing}
}
\end{footnotesize}

Као и у претходном примеру тетраедар је у канонском положају према
самој конструкцији, координате тачака су $A(0, 0, 0)$, $B(0, 0, 1)$,
$C(x_1, x_2, 0)$ и $S(x_3, x_4, x_5)$ при чему су $x_1$, $x_2$, $x_3$,
$x_4$ и $x_5$ зависне променљиве и налазе се у полиномима који описују
тетраедар (погледати у ранијем поглављу полиноме за тетраедар
\ref{tetraedar_6glava}).

Коришћењем првог приступа добија се $31$ полином који представља
релације међу објектима, $1$ полином за који је потребно доказати да
је једнак нули и $36$ променљивих. Након поједностављивања, добија се
$4$ полинома за које треба доказати да су једнаки нули ($3$ полинома
служе као провера да ли се праве заиста секу, а један полином служи за
проверу углова), $24$ који описују релације међу објектима и $18$
различитих променљивих. Међу добијеним полиномима, $15$ полинома је
прилично једноставно и састоје се од два монома. Ипак, преостали
полиноми су комплексни и састоје се у просеку од осам монома. Метод
Гребнерових база је био успешан у доказивању да су свих девет полинома
једнаки нули. Систем \emph{GeoProver} није био успешан и након $0.481$
секунде пријавио је грешку да је досегао предвиђени меморијски лимит
због полинома који има 3339 монома (енг.~ \emph{Space limit exceeded
  in pseudo division. Obtained polynomial with 3339 terms}). Просечно
време потребно за доказ да су полиноми који проверавају да ли има
пресека једнаки нули у систему \emph{GeoProver} је $0.377$ секунду.

Коришћењем другог приступа, број добијених полинома је $28$ и један
полином за који је потребно доказати да је нула, број променљивих је
$32$. Након сређивања и анализе полинома пресека добија се $24$
полинома и два полинома за које је потребно доказати да су нула и $24$
различите променљиве. У овом приступу $18$ полинома је било
једноставно и састојало се из два монома, док је преосталих $6$
полинома комплексније и састоје се у просеку од $8$ монома. Оба
алгебарска доказивача (\emph{GeoProver} и метод Гребнерових база
имплементираних у систему \emph{Mathematica}) су били успешни у
доказивању да су свих девет полинома једнаки нули. Време потребно за
доказ да је полином који проверава да ли има пресека једнак нули је у
систему \emph{GeoProver} једнака $0.081$ секунду, а време за доказ да
је полином који проверава једнакост углова једнак нули у систему
\emph{GeoProver} је $0.835$ секунде.

\begin{table}[hb]
\begin{center}
\begin{tabular}{L{2cm}||R{2cm}|R{2cm}|R{2cm}|R{2cm}|R{1.5cm}}
\                   &  број полинома & број полинома доказа & просечан број монома & број променљивих & време \\
\hline
\hline
\textbf{први приступ} & $24$ & $4$ & $7.2$ & $18$ & \emph{Меморијски лимит} \\
\hline
\textbf{други приступ} & $24$ & $2$ & $3.5$ & $24$ & $0.835s$
\end{tabular}
\caption{Упоредни приказ успешности алгебризације у односу на изабрани приступ}
\end{center}
\end{table}


Оно што се може приметити након ова два једноставна примера је да први
приступ генерише мање променљивих, али да други приступ генерише
једноставније полиноме. Како ће се показати даљим тестирањем,
коришћеним алгебарским доказивачима више одговара када су полиноми
једноставнији и зато су били успешнији када је коришћен други приступ.


\paragraph{Резултати методе Гребнерових база имплементираних у систему \emph{Mathematica}.}
Метод Гребнерових база када је коришћен први приступ над $29$ проблема
је био успешан $23$ пута и није докaзао тврђења након $5$ минута за $6$
посматраних проблема.

Када је коришћен метод Гребнерових база али када је алгебризација
рађена другим приступом, метод Гребнерових база је био успешан у свих
$29$ посматраних проблема.


\paragraph{Резултати система \emph{GeoProver}.}
Систем \emph{GeoProver} када је коришћен први приступ је био успешан
само $13$ пута. У различитим проблемима долазило је до различитих
грешака због којих није успешно доказ завршен: достигнут је временски
лимит (енг.~\emph{Time limit reached.}), достигнут је меморијски лимит
(енг.~\emph{Space limit reached.}) или се догодила генерална грешка
(енг.~\emph{General error occured.}). Последња грешка означава да
доказивач није био у могућности да направи троугаони систем
једначина. То на даље значи да је потребно додатно истражити ову
ситуацију и потенцијално изменити улазне полиноме да би се ова грешка
избегла. Ипак, за сада нисмо у могућности да овакво понашање спречимо
јер ако се током триангулације појави полином $0=0$ доказивач ће
пријавити грешку. У почетном систему је лако могуће регистровати да ли
такав полином постоји, али није лако могуће одредити да ли ће се у
процесу триангулације такав полином створити јер је потребно испитати
односе међу датим променљивима и задатим полиномима, што је доста
комплексно питање.

Коришћењем другог приступа у алгебризацији, систем \emph{GeoProver} је
био нешто успешнији, од $29$ посматраних проблема, систем је био
успешан $22$ пута. Грешке због којих није успео да заврши доказе су:
временски лимит је досегнут или се догодила генерална грешка.

\begin{table}[hb]
\begin{center}
\begin{tabular}{L{2cm}||R{2cm}|R{2cm}|R{2cm}|R{2cm}}
\                      & \emph{GeoProver} успех & \emph{GeoProver} неуспех & \emph{Гребнерове базе} успех & \emph{Гребнерове базе} неуспех \\
\hline
\hline
\textbf{први приступ}  &  $13$                  & $16$                     &  $23$                        & $6$  \\
\hline
\textbf{други приступ} &  $22$                  & $7$                      &  $29$                        & $0$ \\
\end{tabular}

\caption{Упоредни приказ успешности доказивача и приступа алгебризације}
\end{center}
\end{table}

%TODO: tabela sa brojevima iz teksta
%SAGE -- grobner basis: za dalji rad: miltipolynomial ideal

\subsection{Закључак}

Према нашем знању не постоји јавно доступан аутоматски доказивач за
стереометрију. Алгебарски доказивачи (као што је Вуов метод или метод
Гребнерових база) могу да доказују и тврђења у стереометрији, али је
потребно та тврђења представити у полиномијалном облику. 

У оквиру ове тезе изучавали смо како се може извршити алгебризација
геометријских тврђења. Поступак алгебризације објеката и релација у
стереометрији је могуће учинити на више начина и ми смо у овом раду
представили два приступа. Први приступ уводи само координате тачака, а
други приступ у једначине полинома уводи и координате правих и равни
које учествују у геометријском тврђењу.

Поредили смо ова два приступа над истим скупом проблема и покретали
смо два алгебарска доказивача, доказивач заснован на методи
Гребнерових база и доказивач заснован на Вуовoј методи. Један од првих
закључака је да Вуова метода захтева скуп полинома који се могу
трансформисати у троугаони систем. Показало се да је понекад тај услов
тешко испунити, а посебну потешкоћу је правила релација пресека две
праве. Иако се овај проблем може решити, много погоднији за рад је
метод Гребнерових база који овај захтев нема. Друга важна особина ове
аутоматизације је била ефикасност доказивања у зависности од изабраног
метода алгебризације. Тестирањем се показало да је боље када су
полиноми једноставни без обзира на број полинома и број
променљивих. Уколико су полиноми комплексни, онда су алгебарски
доказивачи били мање успешни. Стога се други приступ показао као бољи
јер се коришћењем тог приступа добијају једноставнији полиноми.

Једна од важних тема код алгебарских доказивача јесу услови
недегенерисаности и у даљем раду би требало испитати како дати
геометријску интерпретацију добијеним остацима полинома. Поред тога,
пожељно је проширити систем тако да може да обухвати и обла тела
(сфере, купе и ваљкове) и како одредити однос пресека правих са овим
телима јер пресека може бити више.

У даљем раду циљ је направити заокружен систем за ефикасно
визуелизовање и доказивање проблема у стереометрији. Као први корак би
требало повезати направљени доказивач са динамичким геометријским
софтвером који би истовремено успешно визуелизовао и доказивао
геометријска тврђења. Још један занимљив корак би могао да буде
трансформација геометријских тврђења са природног језика у термовску
репрезентацију која би послужила за исцртавање и доказивање овог
геометријског тврђења.

