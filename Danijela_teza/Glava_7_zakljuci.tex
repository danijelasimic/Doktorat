\chapter{Закључци и даљи рад}
\label{chapter::zakljucak}

\section{Закључци}

У овој тези представили смо формализацију Декартове координатне равни
у оквиру система \emph{Isabelle/HOL}. Дато је неколико различитих
дефиниција Декартове координатне равни и доказано је да су све
дефиниције еквивалентне. Дефиниције су преузете из стандардних
уџбеника, али је било потребно подићи ниво ригорозности. Формално је
доказано да Декартова координатна раван задовољава све аксиоме Тарског
и већину аксиома Хилберта (укључујући и аксиому
непрекидности). Показало се да, иако је већина тврђења једноставна,
доказ тих тврђења захтева комплексна израчунавања и веома је захтеван
за формализацију. Зато се веома често користи техника без губитка на
општости која се заснива на изометријским трансформацијама.

У оквиру формализације хиперболичке геометрије представили смо
формализацију геометрије проширене комплексне равни $\extC$ коришћењем
комплексне пројективне равни, али и Риманове сфере. Формализовали смо
аритметичке операције у $\extC$, размеру и дворазмеру, тетивну метрику
у $\extC$, групу Мебијусових трансформација и њихово дејство на
$\extC$, неке њене специјалне подгрупе (eуклидске сличности, ротације
сфере, аутоморфизме диска), кругоправе и њихову везу са круговима и
правама, тетивном метриком, Римановом сфером, јединственост
кругоправи, дејство Мебијусових трансформација на кругоправе, типове и
кардиналност скупа кругоправе, оријентисане кругоправе, однос између
Мебијусових трансформација и оријентације, својство очувања угла након
дејства Мебијусових трансформација итд. Кључан корак је био да се
користи алгебарска репрезентација свих важних објеката (вектора
хомогених координата, матрица за Мебијусове трансформације, хермитске
матрице за кругоправе итд.). Показало се да је алгебарски приступ
далеко супериорнији у доказивању у односу на геометријски
приступ. Релацију између у Поeнкареовом диск моделу смо дефинисали
коришћењем алгебарског приступа и показано је да шест аксиома Тарског
важе у Поeнкареовом диск моделу и да Еуклидова аксиома паралелности не
важи.

Када је у питању формализација алгебарских доказивача у геометрији
извршена је формализација алгебризације тврђења у планарној
геометрији. Алгебарски доказивач заснован на методи Гребнерових база
је већ формализован у оквиру система \emph{Isabelle/HOL}. У оквиру ове
тезе изучавали смо како се може извршити алгебризација геометријских
тврђења задатих у тродимензионалном простору. Поступак алгебризације
објеката и релација у стереометрији је могуће учинити на више начина и
ми смо у овој тези представили два приступа. Први приступ уводи само
координате тачака, а други приступ у једначине полинома уводи и
координате правих и равни које учествују у геометријском
тврђењу. Поредили смо ова два приступа над истим скупом проблема и
покретали смо два алгебарска доказивача, доказивач заснован на методи
Гребнерових база и доказивач заснован на Вуовој методи. Тестирањем се
показало да је доказивање ефикасније када су полиноми једноставни без
обзира на број полинома и број променљивих. Уколико су полиноми
комплексни, онда су алгебарски доказивачи били мање успешни. Стога се
други приступ у формализацији показао као бољи, јер се коришћењем тог
приступа добијају једноставнији полиноми.


\section{Даљи рад}

У оквиру формализације аналитичке геометрије рад се може наставити на
формализацији да наша дефиниција Декартове координатне равани
задовољава све аксиоме Хилберта, односно увести појам угла, показати
потребна својства и онда доказати аксиому комплетности. У нашем даљем
раду планирамо да дефинишемо аналитичку геометрију у оквиру
аксиоматизације Тарског или Хилберта. То би омогућило да докажемо
категоричност и система аксиома Тарског и система аксиома Хилберта (и
да докажемо да су сви модели изоморфни и еквивалентни Декартовој
координатној равни). Поред бројних примена, комплексна геометрија је
веома важна у физици и у многим областима астрофизике се користе
закључци и тврђења која управо потичу из хиперболичке геометрије. Било
би интересантно испитивати и формализовати та тврђења и закључке и
повезати са нашом постојећом формализацијом. Додатно, могуће је
проширити анализу Мебијусових трансформација и показати да одређене
класе ових трансформација задовољавају нека интересантна својства. У
даљем раду на аутоматизацији стереометрије циљ је направити један
заокружен систем за ефикасно визуелизовање и доказивање проблема у
стереометрији. Као први корак би требало повезати направљени доказивач
са динамичким геометријским софтвером који би истовремено успешно
визуелизовао и доказивао геометријска тврђења. Други корак би могао да
буде трансформација геометријских тврђења са природног језика у
термовску репрезентацију која би послужила за исцртавање и доказивање
овог геометријског тврђења. Додатно, ради прецизности и тачности,
алгебарски доказивач за стереометрију би било могуће формализовати
(као што је то урађено за планарну геометрију) и тако показати његову
исправност.

