\documentclass{article}

\begin{document}

\section{Dokaziva\v ci teorema -- motivacija}

\subsection{Frame 1}
\begin{itemize}
\item Aksiomatski dokaziva\v ci -- dat skup aksioma i primenom logike (metod rezolucije, tabloa, prirodna dedukcija) dolazi se do nekih zaklju\v caka. \\
      Vidjeno u radu Vesne i Sane. (i tu izvrsena provera u okviru sistema Isabelle/HOL)\\
      Velika je mana \v sto nisu efikasni.
      
\item Algebarski dokaziva\v ci -- mogu dokazati veliki broj teorema. Problem je \v sto dokazi nisu \v citljivi. Zasnivaju se na analiti\v ckom pristipu u dokazima, tj.
      na reprezentaciji ta\v caka kori\v s\'cenjem koordinata. Prolem je i \v sto nismo "sigurni" u dobijeni dokaz.
  
\end{itemize}

\subsection{Frame 2}

Prica preuzeta od Bezema: http://dream.inf.ed.ac.uk/events/adg2012/uploads/talks/ADG2012-Beeson.pdf

Ideja je da se ispri\v ca motivacija za ovaj doktorat, otkuda sve to.


Svaku granu grafa objasniti:

\begin{itemize}
\item {\bf Geometrijska teorema -- Geometrijski dokaz} Bilo bi idealno da idemo ovim putem. Od teoreme ka dokazu, a pri tome koristimo geometrijske alatke (aksiome, 
      prirodnu dedukciju itd.) \\
      Veliki je probelm \v sto ovo uop\v ste nije jednostavno uraditi. Naime, problem mo\v ze biti te\v zak i nedokaziv za \v coveka.
      Tu u pri\v cu dolaze ra\v cunari. \\
      Ali, ako koristimo aksiomatske dokaziva\v ce -- ne moramo biti uspe\v sni (oni nisu ba\v s najusp[e\v sniji u dokazivanju). \\
      Zato idemo ZAOBILAZNIM putem (to je deo grafa koji ide preko Algebraskih dokaziva\v ca).
      
\item {\bf Geometrijska teorema -- Algebarska Transformacija} Ovde imamo zapravo dve stvari o kojima treba voditi ra\v cuna:
  \begin{enumerate}
  \item Prebacujemo sve u analiti\v cku geometriju -- za\v sto je to ok? Kako smemo da prebacimo u analiti\v cku geometriju?
  \item Za\v sto su ok konstrukcijski polinomi? -- da li ne\v sto gubimo ili dobijamo prebaciju\'ci u polinome, da li su oni zaista korektni?
  \end{enumerate}

\item {\bf Algebarska transformacija -- Algebarski dokaz} Grebnerove baze ve\'c formalizovane u okviru sistema Isabelle/HOL

\item {\bf Algebarski dokaz -- Geometrijski dokaz} Da li je dokaz zaista dokaz? Za\v sto mo\v zemo da vratimo nazad? Za\v sto ne\v sto 
      \v sto va\v zi u analiti\v ckoj geometriji va\v zi i u sinteti\v ckoj? Kako ide formalizacija: pokazati da se u sinteti\v ckoj geometriji
      mo\v ze izgraditi Dekartova ravan.
     
\end{itemize}

\section{Formalizacija analiti\v cke geometrije}

\subsection{Frame 3}

U klasi\v cnoj matematici postoji mnogo razli\v citih geometrija. Tako\d e,
razli\v cita su gledi\v sta \v sta se smatra standardnom (Euklidskom) geometrijom.
Ponekad, geometrija se defini\v se kao nezavisna formalna teorija, a ponekad
kao specifi\v can model. Naravno, veze izme\d u razli\v citih  zasnivanja geometrije 
su jake.
 
Tradicionalna Euklidska (sinteti\v cka) geometrija, koja datira jo\v s od 
anti\v cke Gr\v cke, je geometrija zasnovana na \v cesto malom skupu osnovnih
pojmova (na primer, ta\v cke, linije, relacija podudarnosti, \ldots) i na 
skupu aksioma koje implicitno defini\v su osnovne pojmove.
Kori\v s\'cenjem aksioma, teorema, lema i logi\v ckih argumenata mogu\'ce je
izvoditi nove zaklju\v cke. Iako su Euklidovi "Elementi" jedan od
najuticajnih radova iz matematike, postavilo se ozbiljno pitanje da li
sistem aksioma, teorema i lema kojima se geometrija opisuje zaista
precizan. Ispostavilo se da su nadjene gre\v ske u dokazima u tekstu,
a i da su neki dokazi bili nekompletni jer su imali implicitne
pretpostavke nastale zbog intuicije ili posmatranja slike. Ove
praznine su uticale na pojavu drugih aksioma{t}{s}kih sistema \v ciji je
cilj bio da daju formalnu aksiomatizaciju Euklidove
geometrije. Najva\v zniji su Hilbertov sistem aksioma, sistem aksioma 
Tarskog i najmodernija varijanta -- Avigadov sistem aksioma.

Hilbertov sistem se sastoji iz tri osnovna pojma (ta\v cka, prava i
ravan), 6 predikata i 20 aksioma podeljenih po grupama. Hilbert nije
\v zeleo ni\v sta da ostavo intuiciji, ve\' c je i najo\v ciglednija
tvrdjenja zapisivao kao aksiome i leme. Ovakav pristup je pove\' cao
nivo rigoroznosti ne samo u geometriji, nego u celoj matematici. 

Sistem Tarskog je manji, sastoji se od jednog osnovnog pojma (ta\v
cka), 2 predikata i 11 aksioma i njegova osnovna prednost u odnosu na
Hilbertov sistem je u njegovoj jednostavnosti. Sa druge strane, sistem
Tarskog uvodi pojam linije kao skup ta\v caka \v, a takav pristup
dosta ote\v zava rezonovanje jer zahteva da se u dokazima teorema i
lema koristi kompleksna teorija skupova.

Jеdnо оd naјzna\v caјniјih оtkri\'ca u matеmatici, kоје datira iz XVII
vеka, јеstе Dеkartоvо оtkrićе kооrdinatnоg sistеma i оnо је
оmоgu\'cilо da sе algеbarskim izrazima prеdstavе gеоmеtriјski оblici. Tо
је dоvеlо dо rada na nоvој matеmatičkој оblasti kојa sе zоvе
\emph{analiti\v cka gеоmеtriјa}. Оna је pоslužila da spојi gеоmеtriјu i
algеbru i bila је vеоma važna za оtkrićе bеskоna\v cnоsti i matеmati\v ckе
analizе.
----------------

Potreba za rigoroznim zasnivanjem matematike postoji veoma dugo i sa
razvojem matematike pove-\'cavao se i stepen rigoroznosti. Sa pojavom
ra\v cunara pojavila se mogu\'cnost ma\v sinski proverivih
dokaza. Tako su se pojavili sistemi za formalno dokazivanje
teorema. \v Cеstо, mеhani\v cki prоvеrеni dоkazi pоpunjavaјu prazninе kоје
pоstоје u dеfiniciјama i dоkazima i upućuјu na dublju analizu tеmе kојa
sе izu\v cava. 

Pоstоje sistemi kоji оmоgu\'cavaju pоtpunо autоma{t}{s}ku
prоveru dоkaza i оni su naj\v ce\v s\'ce kоriste SАT re\v
sava\v ce ili tehnike prezapisivanja. Sistemi kоji se zasnivaju na
lоgikama vi\v seg reda su pоluautоma{t}{s}ki i u prоcesu fоrmalnоg
dоkazivanja tеоrеma оd strane kоrisnika (\v cestо prоgramer i/ili
matemati\v car) pоma\v zu takо \v stо kоntrоli\v su ispravnоst dоkaza i,
kоlikо је tо mоgu\'cе, prоnalaze autоma{t}{s}ke dоkaze. Оvi pоluautоmatski
dоkaziva\v ci se nazivaju i {\em asistenti za dоkazivanje teоrema}. 

Danas pоstојi mnоgо asistеnata za dоkazivanjе tеоrеma: Isabelle,
Isabelle/HОL, Cоq, HОL Light, PVS i drugi. Pоsеbnо sе isti\v cu 
Isabelle/HОL i Cоq kaо sistеmi sa vе\'cim brојеm kоrisnika kојi
su tоkоm gоdina razvili vеliki skup bibliоtеka sa fоrmalnо dоkazanim
tеоriјama kоје је mоgu\'cе daljе nadоgradjivati. Аsistеnti za dоkazivanjе
tеоrеma sе kоristе u razli\v citim оblastima. Prе svеga, isti\v cе sе
primеna u оbrazоvanju. Pоrеd tоga, mоgu sе kоristiti i za fоrmalnu
vеrifikaciјu ra\v cunarskih prоgrama. Pоma\v zu razvој i prоdubljivanjе
matеmatičkоg znanja.

Postoji veliki broj formalizacija fragmenata razli\v citih geometrija
u okviru asistenata za dokazivanje teorema. Delovi Hilberove knjige
"Osnove geometrije" su formalizovani u Isabelle-u i 
Coq-u. U okviru sistema Coq je formalizovana geometrija
Tarskog, konstruktivna geometrija, projektivna geometrija, geometrija
lenjira i \v sestara i druge. 

\subsection{Frame 5-6}

Manje-vi\v se kratko re\'ci ono \v sto pi\v se na slajdovima bez osvrtanja na detalje.
Posebno se osvrnuti na {\bf wlog} kao na veoma va\v znu tehniku.

\section{Formalizing Complex Plane Geometry}

\subsection{Frame 7}

\v Zeleli smo da pro\v sirimo istra\v zivanje i da dodamo formalizaciju Hiperboli\v cke geometrije.
Krenuli smo od Poinkareovog disk modela i \v zeleli smo da formalizujemo da je on
model geometrije Loba\v cevskog (va\v ze sve aksiome osim aksiome paralelenosti).

Poincareov disk model predstavljamo jednim jedini\v cnim krugom.

\subsection{Frame 8}
Linija je normalna na jedini\v cni krug.

To mo\v ze biti prava koja prolazi kroz centar.

\subsection{Frame 9}

Ili mo\v ze bit krug (deo kruga, luk) koji je normalan na jedini\v cni krug.

\subsection{Frame 10}
Ono \v sto nam je interesantno je da posmatramo relaciju izme\d u.

\subsection{Frame 11}

Jedan od na\v cina da se defini\v se izme\d u bi mogao da bude: preslikamo ta\v cke na x-osu i onda gledamo u kakvom su odnosu.

Postoji vi\v se problema: \\
-- transformacija koja vr\v si preslikavanja ne sme da menja redosled ta\v 
   caka, odnosno mora da ga \v cuva \\
-- voleli bismo da je ta transformacija jednostavna, ali je kompleksna \\
-- treba ispitati da li se nakon transformacije ta\v cke nalaze na x-osi ili ne. 
   One koje nisu na x-osi nisu ni na liniji odre\d enoj ovim ta\v ckama (malo 
detaljnije ovo objasniti) \\
-- konacno, kako da povezemo ako imamo vise trojki ta\v caka i za njih hocemo da ispitujemo between; kada imamo jedan, onda ga preslikamo i gledamo x-osu, ali ako imamo 6 ta\v caka,
   kako preslikati svih 6 na x-osu i posmatrati svojsta? Ovaj problem zna\v cajno ote\v zava dokaze.
   
\subsection{Frame 12}   

Problem sa ovim res\v senjem: \\
-- kompleksna definicija (koja razlikuje pravu od kruga) \\
-- te\v sko pokazati da izmoterija \v cuva between \\
-- te\v sko uspostaviti vezu izme\d u ove dve definicije \\

Trebala nam je definicija koja \'ce biti jedinstvena. Sistem u kome mo\v zemo jednostavno definisati pravu, i u kome mo\v zemo na jedinstven na\v cin
definisati between. 

Zato, okre\'cemo se kompleksnoj analizi i algebri koje nude brojna re\v senja. Ipak, da bi ovo razvili, bilo je potrebno razviti veliku teoriju u Isabelle/HOL sistemu

\subsection{Frame 13}

Iako se pojam sferi\v cne geometrije pojavio jo\v s u staroj Gr\v
ckoj, ozbiljnije istra\v zivanje ne-Euklidske geometrije je zapo\v
ceto 1829. godina sa radom Loba\v cevskog. Iako je Loba\v cevski
intenzivno istra\v zivao ne-Euklidsku geometriju i poku\v savao da sa
njom opi\v se realan svet, ostali nau\v cnici nisu bili toliko
zainteresovani za ovu oblast i proslo je pola veka pre nego \v sto se
krenulo sa intenzivnijim istra\v zivanjem.  Ono \v sto je najvi\v se
uticalo na ovu promenu jeste otkri\'ce kompleksnih brojeva krajem
{\lat XVIII} veka. Kompleksni brojevi su predstavljali zna\v cajanu
alatku za istra\v zivanje osobina objekata u razli\v citim
geometrijama. Zamenom Dekartove koordinatne ravni sa kompleksnom ravni
dobijaju se jednostavnije formule koje opisuju geometrijske
objekte. Nakon Gausovog teorije o zakrivljenim povr\v sinama (???? {\lat
curved surfaces}) i Rimanovog rada o zakrivljenim (??{\lat manifolds})
geometrija Loba\v cevskog dobija na zna\v cajnosti. Ipak, najve\' ci
uticaj ima rad Beltramija koji pokazuje da dvodimenzionalna
ne-Euklidska {\lat kako se pise ne-euklidska} geometrija je ni\v sta
drugo do izu\v cavanje odgovaraju\' ce povr\v sine konstantne
negativne krive ??? {\lat constant negative curvature}. Uvodi i pojam
{\em projektivnog disk modela} koji je kasnije popularizovan od strane
Klajna. Poinkare posmatra model poluravni {\lat half-plane} koji su
predlo\v zili {\lat (??? Louville i Beltrami -- kako se oni prevode)} i pre  
svega izu\v cava
izometrije hiperboli\v cke ravni koje \v cuvaju orijentaciju. Danas se
te transformaciju naj\v ce\v s\'ce nazivaju Mebijusove transformacije.

Iako postoji dosta literature na polju kompleksne geometrije, mi nismo na\v sli 
neku njenu formalizaciju.

Potreba za formalizacijom: puno gre\v saka, netrivijalnih zaklju\v caka koji  
nisu pokazani, potom i sitne gre\v ske u samim dokazima, recimo zanemaruju se 
neki slu\v cajevi i uop\v ste se ne pokazuje da tvr\d enje va\v zi, tj. ne va\v 
zi i u tom specijalnom slu\v caju. \v Cesto u okviru iste knjige su postojale 
nedoslednosti u pojmovima koji su kori\v s\'ceni, recimo lako se prelazilo sa 
geometrijskog na alegbarski pojam a da pri tome nije pokazano, tj. opravdano da 
tako ne\v sto sme. 
Tako\d e, va\v zno je napomenitu da je va\v zno i iskustvo koje smo imali tokom 
formalizacije. U knjigama postoje dva pristupa: jedan je vi\v se geometrijski, 
a drugi je vi\v se algebarski. Pokazuje se da je vrlo va\v zno koji se pristup 
izabere da bi uop\v ste mogla uspe\v sno da se izvr\v si formalizacija.

\subsection{frame 18}

Uvodimo homogene koordinate koje su posebno va\v zne i zna\v cajno olak\v 
savaju ra\v cun i reprezenatciju (zapis raznih formula).

Objasniti da je na\v s tip zapravo klasa ekvivalencije skupa ne-nula vektora 
koji su nadom onom relacijom $\approx$

\subsection{Frame 19}

Prokometarisati da za aritmeti\v cke operacije defin\v semo i ono $\infty \cdot 
0$ i sli\v cno i da to moramo da defnisi\v semo iako to nije dobro definisano.

\subsection{Frame 20}

Rimanova sfera se mo\v ze postovetiti sa pro\v sirenom kompleksnom ravni kori\v 
s\'cenjem stereografske projekcije. Ono \v sto je va\v zno je da je Rimanova 
sfera metri\v cki prostor i da je rastojanje me\d u ta\v ckama zapravo du\v 
zina tetive izme\d u njih.

\subsection{Frame 21}

Ono \v sto pi\v se na slajdu. Dodati da je Mebijusov identitet u stvari 
jedini\v cna matrica.

\subsection{Frame 22-23}

Ono \v sto pi\v se.

\subsection{Frame 24}

Ponoviti pri\v cu o klasi ekvivalencije.

Spomenitu postojanje specijalnih krugova -- unit circle, prva kroz koordinatni 
po\v cetak. Spomenuti da ta prava sadr\v zi beskona\v cno. Euklidske prave su 
one kad circline sadr\v zi beskona\v cno. A svaka Euklidska prava i krug mogu 
biti reprezentovani kao circline.

*Kako se ka\v ze na srpskom Hermitean (na\'ci)?

\subsection{Frame 25}

Ispri\v cati da se osobine \v cuvaju.

Spomenuti svojstvo jedinstevnosti uop\v stenog kruga.

\subsection{Frame 26}

Spomenuti da Moebijusova transformacija uop\v stenog kruga ima sli\v cna 
svojstva kao i Moebijusova transformacija ta\v cke (navesti neka svojstva). 
Glavna stvar je naglasiti da Mebijusova transformacija \v cuva circline, tj. 
slika circline u circline (realan u realan). Naglasiti da veza se pravi izme\d 
u mebijusa nad ta\v ckama i mebijusa nad circline. Tako\d e re\'ci da se 
simetri\v cne ta\v cke \v cuvaju pod mebijusom.

\subsection{Frame 27}

Kod o\v cuvanja ugla ispri\v cati da postoje dva pristupa kako ugao da 
predstavimo. Jedan pristup je geometrijski: ugao izme\d u tangenti u ta\v cki 
preseka. Drugi pristup je alegebarski: ona formula. Iako je prvi vizuelno lep\v 
si i lak\v si za razumevanje, drugi je mnogo jednostavniji za analizu i 
pokazivanje svojstava. Ista\'ci da smo mi oba pristupa formalizovali i pokazali 
smo da oni jesu ekvivalenti.

\section{Primena algebarskih metoda u stereometriji}

\subsection{Frame 28}
Dodati pri\v cu da treba izvr\v siti formalnu analizu. Bi\'ce napravljen sistem 
i primenjen na razli\v cite probleme. Formalna analiza transformacija.

\end{document}

































