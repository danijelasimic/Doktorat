\documentclass{article}

\usepackage[utf8]{inputenc}
\usepackage[serbian]{babel}
\usepackage{amsmath}
\usepackage{amsthm}
\usepackage{graphicx}
\usepackage{float}
\usepackage{cirilica}
\usepackage{pgf}
\usepackage{wrapfig}
\usepackage{gclc}

\usepackage{fullpage}
\usepackage{hyperref}
\usepackage{enumitem}

\usepackage{amssymb}
\usepackage{stmaryrd}
\usepackage{tikz}

//sve reference ujednaciti

\begin{document}
       УНИВЕРЗИТЕТ У БЕОГРАДУ  \hfill{У Београду, \_ 2015.} 
       
       МАТЕМАТИЧКИ ФАКУЛТЕТ
 

\begin{center}
 {\small Катедри за рачунарство и информатику } \\
 
 {\bf Пријава теме за израду докторске дисертације}
\end{center}

  Молим да ми одобрите израду докторске дисертације под насловом "Формализација
  различитих модела геометрије". Тему предлажем у договору 
  са ментором др Филипом Марићем. 
  
  \vspace{1cm}
  
  \hfill{Подносилац пријаве} 
  
  \vspace{0.5cm}
  
  \hfill{\underline{\hspace{ 3in}}} 
  
  \hfill{Данијела Симић} 
  
  \vspace{1cm}
  
  \hfill{Ментор:} 
  
  \hfill{др Филип Марић}

  
  \newpage
  
\begin{center}
{\bf Наставно-научном већу Математичког факултета\\
Универзитета у Београду}

\vspace{1cm}

{\bf Пријава теме за израду докторске тезе \\
     кандидата Данијеле Симић (бивше Петровић):}

\vspace{0.5cm}     

{\bf Формализација раличитих модела геометрије i primene}
\end{center}

\section{Подаци о кандидату}
\subsection{Биографија}
   Данијела Симић је рођена 26.09.1986. године у Ваљеву. Основну школу 
   Жикица Јовановић Шпанац завршила је као ђак генерације и вуковац. 
   Потом је уписала Ваљевску гимназију, коју је такође завршила као ђак 
   генерације и вуковац. Током школовања учествовала је на бројним такмичењима
   при чему се посебно истичу резултати и награде са републичких и савезних 
   такмичења из физике и програмирања. Године 2005. уписала је Математички
   факултет, уневерзитета у Београду, смер Рачунарство и Информатика. 
   Студије је завршила 2009. године са просечном оценом 9.86. Исте године
   уписала је докторске студије на смеру Информатика. Положила је све испите 
   на докторским студијама са просечном оценом 10.
   
   Запослена је на Математичком факултету, Универзитета у Београду од 2009.
   године и држала је вежбе из следећих предмета:
   \begin{itemize}
   \item Програмирање 1
   \item Програмирање 2
   \item Увод у организацију рачунара
   \item Вештачка интелигенција
   \end{itemize}
   
\subsection{Списак научних радова}

{\em Објављени радови}
\begin{itemize}
\item {\lat Filip Marić, Ivan Petrović, Danijela Petrović, Predrag Janičić: \href{http://eptcs.web.cse.unsw.edu.au/paper.cgi?THedu11.4}{Formalization and Implementation of Algebraic Methods in Geometry} (\href{http://about.eptcs.org/}{Electronic Proceedings in Theoretical Computer Science} 79, pp. 63–81.)}
\item {\lat Filip Mari\'c, Danijela Petrovi\'c: \href{http://argo.matf.bg.ac.rs/publications/2014/moebius.pdf}{Formalizing Complex Plane Geometry} (\href{http://link.springer.com/article/10.1007/s10472-014-9436-4}{Annals of Mathematics and Artificial Intelligence}, November 2014)}
\item {\lat Danijela Simi\'c: \href{http://argo.matf.bg.ac.rs/publications/2014/small-step-verification.pdf}{Using Small-Step Refinement for Algorithm Verification in Computer Science Education} (accepted for publish in \href{https://www.fose1.plymouth.ac.uk/mathematics\_education/field\%20of\%20work/IJTME/}{The International Journal for Technology in Mathematics Education}, Volume 22, Number 4 (December 2015)}
\end{itemize}

\noindent {\em Саопштења на научним скуповима}
\begin{itemize}
\item {\lat Danijela Petrovi\'c, Filip Mari\'c: \href{http://poincare.matf.bg.ac.rs/~danijela/publications/ADG2012.pdf}{Formalizing Analytic Geometries} (\href{http://dream.inf.ed.ac.uk/events/adg2012/}{Automated Deduction in Geometry 2012})}
\item {\lat Danijela Petrovi\'c: \href{http://poincare.matf.bg.ac.rs/~danijela/publications/text.pdf}{Using Small-Step Refinement for Algorithm Verification in Computer Science Education} (\href{http://www.uc.pt/en/congressos/thedu/thedu14}{ThEdu'14})}
\end{itemize}

\subsection{Конференције и летње школе}
\begin{itemize}
\item {\lat Automated Deduction in Geometry, Automated Deduction in Geometry} 2012, Единбург, Велика Британија, 17.09. -- 19.09.2012.
\item {\lat European Summer School in Logic, Language and Information}, Љубљана, Словенија, 01.08. -- 12.08.2011
\item Vише пута презентовала и учествовала у организацији радионица ARGO групе
\end{itemize}

\section{Преглед области тезе и постојећих резултата}

\subsection{Formalno dokazivanje teorema}

У класичној математици постоји много различитих геометрија. Такође,
различита су и гледишта шта се сматра стандардном (Еуклидском)
геометријом. Пoнекад, геометрија се дефинише као независна формална
теорија, а понекад као специфични модел. Наравно, везе између
различтих заснивања геометрије су јаке. На пример, може се показати да
Декартова раван представља модел формалних теорија геометрије.

Традиционална Еукидска (синетичка) геометрија, која датира још од
античке Грчке, је геометрија заснована на често малом скупу основних
појмова (на пример, тачке, линије, релација подударности, \ldots) и на
скупу аксиома које имплицитно дефинишу ове основне појмове.  Kori\v
s\'cenjem aksioma, teorema, lema i logi\v ckih argumenata mogu\'ce je
izvoditi nove zaklju\v cke. Iako su Euklidovi "Elementi" jedan od
najuticajnih radova iz matematike, postavilo se ozbiljno pitanje da li
sistem aksioma, teorema i lema kojima se geometrija opisuje zaista
precizan. Ispostavilo se da su na\dj ene gre\v ske u dokazima u tekstu,
a i da su neki dokazi bili nekompletni jer su imali implicitne
pretpostavke nastale zbog intuicije ili posmatranja slike. Ove
praznine su uticale na pojavu drugih aksioma{t}{s}kih sistema \v ciji je
cilj bio da daju formalnu aksiomatizaciju Euklidove
geometrije. Nајважнији су Хилбертов систем аксиома, систем аксиома
Тарског и најмодернија варијанта -- Авигадов систем аксиома.

Hilbertov sistem se sastoji iz tri osnovna pojma (ta\v cka, prava i
ravan), 6 predikata i 20 aksioma podeljenih po grupama. Hilbert nije
\v zeleo ni\v sta da ostavo intuiciji, ve\' c je i najo\v ciglednija
tvr\dj enja zapisivao kao aksiome i leme. Ovakav pristup je pove\' cao
nivo rigoroznosti ne samo u geometriji, nego u celoj matematici. 

Sistem Tarskog je manji, sastoji se od jednog osnovnog pojma (ta\v
cka), 2 predikata i 11 aksioma i njegova osnovna prednost u odnosu na
Hilbertov sistem je u njegovoj jednostavnosti. Sa druge strane, sistem
Tarskog uvodi pojam linije kao skup ta\v caka \v, a takav pristup
dosta ote\v zava rezonovanje jer zahteva da se u dokazima teorema i
lema koristi kompleksna teorija skupova.

Једнo од најзначајнијих открића у математици, које датира из {\lat
  XVII} века, јесте Декартово откриће координатног система и оно је
омогућило да се алгебарским изразима представе геометријски облици. То
је довело до рада на новој математичкој области која се зове
\emph{аналитичка геометрија}. Она је послужила да споји геометрију и
алгебру и била је веома важна за откриће бесконачности и математичке
анализе.

Iako se pojam sferi\v cne geometrije pojavio jo\v s u staroj Gr\v
ckoj, ozbiljnije istra\v zivanje ne-Euklidske geometrije je zapo\v
ceto 1829. godina sa radom Loba\v cevskog. Iako je Loba\v cevski
intenzivno istra\v zivao ne-Euklidsku geometriju i poku\v savao da sa
njom opi\v se realan svet, ostali nau\v cnici nisu bili toliko
zainteresovani za ovu oblast i proslo je pola veka pre nego \v sto se
krenulo sa intenzivnijim istra\v zivanjem.  Ono \v sto je najvi\v se
uticalo na ovu promenu jeste otkri\'ce kompleksnih brojeva krajem
{\lat XVIII} veka. Kompleksni brojevi su predstavljali zna\v cajanu
alatku za istra\v zivanje osobina objekata u razli\v citim
geometrijama. Zamenom Dekartove koordinatne ravni sa kompleksnom ravni
dobijaju se jednostavnije formule koje opisuju geometrijske
objekte. Nakon Gausovog teorije o zakrivljenim povr\v sinama (???? {\lat
curved surfaces}) i Rimanovog rada o zakrivljenim (??{\lat manifolds})
geometrija Loba\v cevskog dobija na zna\v cajnosti. Ipak, najve\' ci
uticaj ima rad Beltramija koji pokazuje da dvodimenzionalna
ne-Euklidska {\lat kako se pise ne-euklidska} geometrija je ni\v sta
drugo do izu\v cavanje odgovaraju\' ce povr\v sine konstantne
negativne krive ??? {\lat constant negative curvature}. Uvodi i pojam
{\em projektivnog disk modela} koji je kasnije popularizovan od strane
Klajna. Poinkare posmatra model poluravni {\lat half-plane} koji su
predlo\v zili {\lat (??? Louville i Beltrami -- kako se oni prevode)} i pre svega izu\v cava
izometrije hiperboli\v cke ravni koje \v cuvaju orijentaciju. Danas se
te transformaciju naj\v ce\v s\'ce nazivaju Mebijusove transformacije
{\lat (???? $PSL_2R$ -- da li ovo pominjati)}.

//dodati reference za tarskog i hilberta

\begin{itemize}
\item {\lat Tristan Needham. {\em Visual Complex Analysis.} Oxford University Press, 1998.}

\item {\lat Hans Schwerdtfeger. {\em Geometry of Complex Numbers.} Dover Books on Mathematics. Dover
Publications, 1979.}
\end{itemize}

\subsection{Interaktivno dokazivanje teorema uz pomo\'c\ ra\v cunara}

//prosiriti pricu (pre svega u smislu motivacije)
//veliki rezultati is skorijih rezultata
//naci kod Filipa na sajtu (ima rad pregled) pa odatle izvuci

Potreba za rigoroznim zasnivanjem matematike postoji veoma dugo i sa
razvojem matematike pove-\'cavao se i stepen rigoroznosti. Sa pojavom
ra\v cunara pojavila se mogu\'cnost ma\v sinski proverivih
dokaza. Tako su se pojavili sistemi za formalno dokazivanje
teorema. Често, механички проверени докази попуњавају празнине које
постоје у дефиницијама и доказима и упућују на дубљу анализу теме која
се изучава. Postoje sistemi koji omogu\'cavaju potpuno automa{t}{s}ku
proveru dokaza i oni su naj\v ce\v s\'ce koriste {\lat SAT} re\v
sava\v ce ili tehnike prezapisivanja. Sistemi koji se zasnivaju na
logikama vi\v seg reda su poluautoma{t}{s}ki i u procesu формалног
доказивања теорема od strane korisnika (\v cesto programer i/ili
matemati\v car) poma\v zu тако што контролишу исправност доказа и,
колико је то могуће, pronalaze automa{t}\phantom {s}ke доказe. Ovi poluautomatski
dokaziva\v ci se nazivaju i {\em asistenti za dokazivanje teorema}. 

Данас постоји много асистената за доказивање теорема: {\lat Isabelle,
Isabelle/HOL, Coq, HOL Light, PVS} и други. Посебно се истичу {\lat
Isabelle/HOL} и {\lat Coq} као системи са већим бројем корисника који
су током година развили велики скуп библиотека са формално доказаним
теоријама које је могуће даље надограђивати. Асистенти за доказивање
теорема се користе у различитим областима. Пре свега, истиче се
примена у образовању. Поред тога, могу се користити и за формалну
верификацију рачунарских програма. Помажу развој и продубљивање
математичког знања.

Postoji veliki broj formalizacija fragmenata razli\v citih geometrija
u okviru asistenata za dokazivanje teorema. Delovi Hilberove knjige
"Osnove geometrije" su formalizovani u {\lat Isabelle}-u i {\lat
Coq}-u. U okviru sistema {\lat Coq} je formalizovana geometrija
Tarskog, konstruktivna geometrija, projektivna geometrija, geometrija
lenjira i \v sestara i druge. 

Jедни од најзначајних радова iz formalizacije geometrije су:
\begin{itemize}
\item {\lat Laura Meikle and Jacques Fleuriot. {\em Formalizing Hilberts Grundlagen in Isabelle/Isar.}
In Theorem Proving in Higher Order Logics, volume 2758 of Lecture Notes in Computer
Science. Springer, 2003.}

\item {\lat Phil Scott. {\em Mechanising Hilberts Foundations of Geometry in Isabelle.} Master’s thesis,
University of Edinburgh, 2008.}

\item {\lat Julien Narboux. {\em Mechanical Theorem Proving in Tarski’s Geometry.} In Automated De-
duction in Geometry, volume 4869 of Lecture Notes in Computer Science. Springer, 2007.}

\item {\lat Gilles Kahn. {\em Constructive geometry according to Jan von Plato.} Coq contribution, Coq
V5.10, 1995.}

\item {\lat Fr\'ed\'erique Guilhot. {\em Formalisation en Coq et visualisation d’un cours de g\'eom\'etrie pour
le lyc\'ee.} Technique et Science Informatiques, 24(9), 2005.}

\item Nicolas Magaud, Julien Narboux, and Pascal Schreck. {\em Formalizing Projective Plane ge-
ometry in Coq.} In Automated Deduction in Geometry, volume 6301 of Lecture Notes in
Computer Science. Springer, 2011.
\end{itemize}

//dodati lokalne radove, recimo nesto Pedjino

Формализација sistema Tarski za не-Еуклидске геометрије (Клаин-Белтрами модел) урађена је у раду:
\begin{itemize}
\item {\lat Timothy James McKenzie Makarios. {\em A mechanical verification of the independence of
Tarski’s Euclidean axiom.} Master’s thesis, Victoria University of Wellington, 2012.}
\end{itemize}

Формализација комплексне анализе може се видети у радовима:
\begin{itemize}
\item {\lat Robert Milewski. {\em Fundamental theorem of algebra.} Formalized Mathematics, 9(3), 2001.}

\item {\lat Herman Geuvers, Freek Wiedijk, and Jan Zwanenburg. {\em A Constructive Proof of the
Fundamental Theorem of Algebra without Using the Rationals.} In Types for Proofs and
Programs, volume 2277 of Lecture Notes in Computer Science. Springer, 2002.}
\end{itemize}

\subsection{Automa{t}{s}ko dokazivanje teorema}

Jedan od prvih automa{t}{s}kih dokaziva\v ca teorema bio je automa{t}{s}ki
dokaziva\v c\ za geometriju. Interesovanje za automa{t}{s}ko dokazivanje u
geometriji postoji jo\v s \ odavno, od vremena Tarskog. On je razvio
algebarsku metodu za Euklidsku geometriju, ali je ta metoda bila
neupotrebljiva za komlikovanije teoreme. Ipak, najve\'ci napredak je
napravljen tek sredinom {\lat XX}-og veka kada je Vu predlo\v zio
svoju algebarsku metodu za dokazivanje teorema u Euklidskoj
geometriji. Njegova metoda je mogla da doka\v ze i veoma kompleksne
teoreme. Jo\v s jedna algebarska metoda koja se razvila u isto vreme
je metoda Grebnerovih baza. Ovi metodi imaju analiti\v cki pristup u
dokazivanju i zasnivaju se репрезентацији тачака коришћењем
координата. Модерни доказивачи теорема који се заснивају на овим
методима су коришћени да се докажу стотине нетривијалних
теорема. Ipak, velika mana ovih sistema je \v sto proizvode dokaze
koji nisu \v citljivi. Devedesetih godina {\lat XX}-og veka postojalo
je vi\v se poku\v saja da se ovaj problem re\v si i razvijene su nove
metode zasnovane na aksiomatizaciji sinteti\v cke geometrije -- metoda
povr\v si (??{\lat area method}), metoda punog ugla (?? {\lat full
angle method}). Ipak, njihova glavna mana je \v sto su daleko manje
efikasni u odnosu na algebarske metode.

Primena sistema za automa{t}{s}ko dokazivanje teorema u geometriji je
velika. Pre svega ovi sistemi se mogu koristiti u obrazovanju. Pored
toga koriste se u na\v cnim oblastima kao \v sto su robotika,
biologija, prepoznavanje slika i druge.

Zna\v cajni radovi iz ove oblasti su:
\begin{itemize}
\item {\lat Wen-Ts\"un Wu.  {\em On the decision problem and the  mechanization  of  theorem  proving  in  elementary  geometry.}
Scientia  Sinica ,  21:157--179, 1978}

\item {\lat Deepak Kapur. {\em Using Gr\"obner bases to reason about geometry problems.} Journal of Symbolic Computation, 2(4):399--408, 1986}

\item {\lat  Shang-Ching Chou, Xiao-Shan Gao, and Jing-Zhong  Zhang. {\em Machine  Proofs  in  Geometry.} World Scientific, Singapore, 1994}
\end{itemize}

\section{Циљеви тезе}

\subsection{Формализација аналитичке геометрије Dekartove ravni}

Синтетичка геометрија се обично изучава ригорозно, као пример
ригорозног аксиоматског извођења. Са друге стране, аналитичка
геометрија се углавном изучава неформално. Често се ова два приступа
представљају независно и веза између њих се ретко показује.

Циљ тезе је да повеже ове празнине. Први циљ је да се формализује
аналитичка геометрија у оквиру асистента за доказивање теорема.
Потом, циљ је да се покаже да je анлитичка геометрија модел више
аксиоматских система, тј. система аксиома Тарског и система аксиома
Хилберта.

\subsection{Формализација геометрије комплексне равни}

Постоји јако пуно радова и књига које описују геометрију
комплексне равни. Циљ 
тезе је да представи потпуно развијену теорију проширене 
комплексне равни, њених објеката (линија и кругова) и
њених трансформација (Мебијусове трансформације). Иако је већина 
појмова већ описана у различитим књигама, циљ тезе је да 
споји бројне приступе у један униформни приступ у коме ће бити коришћен
јединствен и прецизан језик за описивање појмова. 
Наиме, чак и у оквиру исте књиге
дешава се да аутори користе исти назив за различите појмове. 
Поред тога, циљ је анализирати и формално показати све случајеве
који често остану недовољно истражени јер их више раличитих аутора 
сматра тривијалним. Пре свега, интересантни резултати се очекују 
у оквиру испитивања Мебијусових трансформација и како оне утичу на
линије, кругове, углове, релације међу тачкама, оријентацију, 
унутрашњост и спољашњост диска.

Поред тога, биће формализована два приступа: геометријски приступ и
алгебарски приступ. Намеће се питање да ли избор приступа утиче на
ефикасност формалног доказивања у оквиру асистента за доказивање
теорема.

\subsection{Формализација Поинкареовог диск модела}

Циљ тезе је да формализује Поинкарев диск модел. Иако постоји много
радова који упућују да би формализација Поинкареовог диск модела могла
бити тривијална, не постоји ни један рад, нити књига који ту
формализацију заиста и представљају. Пре свега, циљ је правилно
дефинисати релацију "између" за три тачке Поинкареове диск равни. Иако
једна од основних релација, њену дефиницију нисмо пронашли до сада,
барем не у истакнутој литератури из ове области.  Потом, циљ је
анализирати како Мебијусове трансформације утичу на релацију "између"
и доказати да Poinkareov disk model je model geometrije Loba\v
cevskog.

\subsection{Formalna analiza algebarskih metoda i razmatranje njihove primene na probleme u stereometriji}

Већина система са аналитичким приступом за доказивање теорема се
користи као софтвер којем се верује иако нису повезани са модерним
интерактивним доказивачима теорема. Да би се повећала њихова
поузданост потребно их је повезати са модерним интерактивним
доказивачима теорема и то је могуће учинити на два начина -- њиховом
имплементацијом у оквиру интерактивног доказивача теорема и
показивањем њихиове исправности или коришћењем интерактивних
доказивача да провере њихова тврђења. Неколико корака у овом правцу је
већ направљено.

Cilj teze je da dopuni ova istra\v zivanja i da ponudi formalno
verifikovan sistem za automa{t}{s}ko dokazivanje u geometriji koji koristi
metod Grebnerovih baza.


\section{Прелиминарни садржај тезе}

Теза ће се састојати из 4 велике целина подељених на мање делове при
чему се неки од делова могу мењати у зависности од тока истраживања:

\begin{enumerate}
\item \underline{Увод.} У овом поглављу биће описани основни појмови и главни циљеви тезе.

\item \underline{Асистенти за формално доказивање.} Биће описани асистени за формално доказивање
      теорема са нагласком на систему {\lat Isabelle/HOL} који ће бити коришћен у раду. 
      
\item \underline{Различити притупи и тренутни резултати у формализацији геометрије.}
      \begin{enumerate}[label*=\arabic*.]
      \item {\em Аутоматско доказивање у геометрији.} Ово поглавље ће представити раличите приступе 
            у овој области и биће наведени најзначајнији резултати.
      \item {\em Интерактивно доказивање у геометрији.}  Биће представљени различити приступи у 
           интерактивном доказивању у геометрији и биће истакнути сви значајни резултати из стално 
           растућег скупа нових радова и истраживања.
      \end{enumerate}

\item \underline{Формализација аналитичке геометрије.}
      \begin{enumerate}[label*=\arabic*.]
      \item {\em Модел Тарског.} Биће описана формализација аналитичке геометрије у моделу Тарског. Биће
            представњене дефиниције појмова и докази аксиома Тарског у аналитичког геометрији.
      \item {\em Модел Хилберта.} Биће представљене дефиниције појмова и докази аксиома Хилберта у
            аналитичкој геометрији.
      \end{enumerate}
      
\item \underline{Формализација хиперболичке геометрије.}
      \begin{enumerate}[label*=\arabic*.]
      \item {\em Формализација геометрије комплексне равни.} Биће приказана формализација многих појмова комплексне геометрије: Мебијусове
            трансформације, круг, линија, угао итд. У оквиру овог поглавља биће анализирани различити приступи у формализацији.
      \item {\em Формализација Поинкареовог диск модела.} У овом поглављу биће анализиране аксиоме Тарског у Поинкареовом диск моделу.
      \end{enumerate}

\item \underline{Formalna analiza algebarskih metoda i razmatranje njihove primene na probleme u stereometriji}
      \begin{enumerate}[label*=\arabic*.]
      \item {\em Formalna analiza metode Grebnerovih baza u sistemu {\lat Isabelle/HOL}.} Ovo poglavlje je posve\'ceno analizi polinoma kojima se predstavljaju geometrijske
            konstrukcije i tvr\dj enja u okviru sistema {\lat Isabelle/HOL}.
      \item {\em Primena algebarskih metoda na probleme u stereometriji.} Bi\'ce analizirana primena algebarskih metoda na zadatke iz stereometrije.
      \end{enumerate}
\end{enumerate}


\section{Досадашњи резулатати кандидата}

Формализација аналитичке геометрије је заршена. Показано је да је она модел
за две аксиоматски засноване геометрије, геометрије Тарског и геометрије 
Хилберта. Ови резултати су приказани у раду:
\begin{itemize}
\item {\lat Petrovi\'c, Danijela, and Filip Mari\'c. "Formalizing Analytic Geometries." {\em Paper presented at ADG 2012: The 9th International Workshop on Automated Deduction in Geometry, held on September 17--19, 2012 at the University of Edinburgh.}}
\end{itemize}

Поред овог рада, урађена је и формализација геометрије комплексне
равни. Дефинисани су основни појмови и показане су многе особине
Мебијусових трансформација. Такође, истражено је како Мебијусове
трансформације утичу на објекте комплексне равни, као што су линија,
круг, угао, оријентација. Резултати овог истраживања приказани су у
раду:
\begin{itemize}
\item {\lat Mari\' c, Filip, and Danijela Petrovi\' c. "Formalizing complex plane geometry." {\em Annals of Mathematics and Artificial Intelligence}: 1-38.}
\end{itemize}
\end{document} 




________________________________________________________________________________________________________


